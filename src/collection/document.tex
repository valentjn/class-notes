\hypersetup{pdftitle={Gesammelte Vorlesungsnotizen}}

\automark[chapter]{part}
\ihead{\leftmark}
\ifoot{\vspace{-1.5mm}\rightmark}

\thispagestyle{empty}
\vspace*{1em}

{%
  \usekomafont{disposition}\huge%
  Gesammelte Vorlesungsnotizen%
}
\vspace*{1em}

\emph{Julian \name{Valentin}}

\vspace*{1em}

Diese Vorlesungsnotizen entstanden als Mitschrieb in Vorlesungen
an der Universität Stuttgart in den Jahren 2009 bis 2014.
Sie dienten hauptsächlich als Lernhilfe für mich;
aus Zeitgründen fehlen viele Skizzen und mathematische Beweise.
Die Notizen werden trotzdem in der Hoffnung zur Verfügung gestellt,
dass sie jemandem einen Nutzen bringen.
Fehler können unter \url{https://github.com/valentjn/class-notes} gemeldet werden.
Die \LaTeX{}-Umsetzung der Notizen steht unter CC BY-SA 4.0
(\url{https://creativecommons.org/licenses/by-sa/4.0/}).

Die Vorlesungsnotizen behandeln Themen aus
der \emph{Analysis} (Vorlesungen I bis VI),
der \emph{Algebra} (VII bis IX),
der \emph{Topologie} (X),
der \emph{angewandten Mathematik} (XI bis XIII),
der \emph{Numerik} (XIV bis XIX),
der \emph{Informatik} (XX bis XXVIII) und
\emph{verschiedenen anderen Gebieten} (XXIX bis XXXI).

{%
  % only include parts in list of parts
  \setcounter{tocdepth}{\parttocdepth}

  \RedeclareSectionCommand[
    % use dotted lines for list of parts, make font smaller
    tocstyle=dottedtocline,
    % make space for Roman part numbers larger
    tocnumwidth=3.8em,
  ]{part}

  % make space for Roman part numbers larger
  \makeatletter
  \renewcommand*{\@pnumwidth}{2.7em}
  \renewcommand*{\@tocrmarg}{2.7em}
  \makeatother

  % list of parts
  \renewcommand*{\contentsname}{Vorlesungen}
  \tableofcontents%
}

\pagebreak

{%
  % set marks manually
  \manualmark{}
  \ihead{Gesammelte Vorlesungsnotizen}
  \ifoot{\vspace{-1.5mm}Inhaltsverzeichnis}

  % use a chapter heading for main table of contents
  \unsettoc{toc}{leveldown}

  % include everything up to subsubsections in main table of contents
  % (and part-wise tables of contents)
  \setcounter{tocdepth}{\subsubsectiontocdepth}

  % main table of contents
  \tableofcontents%
}

\pagebreak

% use a section for table of contents (otherwise there is a page break before it)
\setuptoc{toc}{leveldown}

\input{collectionInclude}
