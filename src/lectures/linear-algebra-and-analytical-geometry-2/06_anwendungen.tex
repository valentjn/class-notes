\section{%
    Anwendungen%
}

\subsection{%
    Endlich erzeugte \name{abel}sche Gruppen%
}

\begin{Satz}{zyklische Gruppen sind genau die zyklischen $\integer$-Moduln} \\
    Seien $G$ eine Gruppe und $x \in G$.
    Dann ist $\aufspann{x} = \{x^i \;|\; i \in \integer\}$
    eine abelsche Untergruppe von $G$, die \begriff{von $x$ erzeugte zyklische
    Untergruppe von $G$}.
    Die Abbildung $\rho: \integer \rightarrow \aufspann{x}$, $i \mapsto x^i$
    ist ein Gruppenepimorphismus von $(\integer, +)$ auf
    $(\aufspann{x}, \cdot)$.
    Ist $\ker \rho = (0)$, so ist $\integer \cong \aufspann{x}$ und die Ordnung
    $|\aufspann{x}|$ (d.\,h. die Anzahl der Elemente von $\aufspann{x}$) ist
    abzählbar unendlich. \\
    Ist $\ker \rho \not= (0)$ und $n \in \natural$ minimal mit $x^n = 1$, dann
    ist $\aufspann{x} = \{1, x, \dotsc, x^{n-1}\}$ und $|\aufspann{x}| = n$.
    Die Ordnung $|\aufspann{x}|$ von $\aufspann{x}$ heißt
    \begriff{Ordnung $|x|$ von $x$}.
    Ist $n = |x| \in \natural$, so ist
    $\aufspann{x} \cong (\integer/n\integer, +)$. \\
    Daher sind die zyklischen Gruppen genau die zyklischen $\integer$-Moduln
    und $(\integer, +)$ ist die einzige unendliche zykliche Gruppe.
\end{Satz}

\begin{Satz}{Klassifikation der endlich erzeugten abelschen Gruppen} \\
    Seien $q_1, \dotsc, q_k \in \natural$ (Elementarteiler) mit
    $q_k \;|\; \dotsb \;|\; q_1 \in \natural$ und $\alpha \in \natural_0$. \\
    Sei $M(q_1, \dotsc, q_k, \alpha) := C_{q_1} \times \dotsb \times C_{q_k}
    \times C_\infty \times \overset{\alpha\text{-mal}}{\dotsb} \times
    C_\infty$
    mit $C_n := (\integer/n\integer, +)$ und $C_\infty := (\integer, +)$. \\
    Dann ist $\{M(q_1, \dotsc, q_k, \alpha) \;|\;
    k \in \natural_0,\;
    q_1, \dotsc, q_k \in \natural,\;
    q_k \;|\; \dotsb \;|\; q_1,\;
    \alpha \in \natural_0\}$
    eine vollständige Liste paarweise nicht-isomorpher, endlich erzeugter
    abelscher Gruppen.
    Für $\alpha = 0$ erhält man mit
    $M(q_1, \dotsc, q_k) := M(q_1, \dotsc, q_k, \alpha)$ und
    $\{M(q_1, \dotsc, q_k) \;|\;
    k \in \natural_0,\;
    q_1, \dotsc, q_k \in \natural,\;
    q_k \;|\; \dotsb \;|\; q_1\}$
    eine vollständige Liste paarweise nicht-isomorpher, endlicher
    abelscher Gruppen. \\
    Dabei ist $|M(q_1, \dotsc, q_k)| = q_1 \dotsb q_k \in \natural$. \\
    Seien $p_1, \dotsc, p_k \in \natural$ Primzahlen,
    $e_1^{(i)} \ge \dotsb \ge e_n^{(i)} \ge 0$ ganze Zahlen für
    $i = 1, \dotsc, k$,
    $\mi{e}_i = (e_1^{(i)}, \dotsc, e_n^{(i)})$ und $\alpha \in \natural_0$.
    Dann erhält man durch \\
    $M(p_1, \mi{e}_1, \dotsb, p_k, \mi{e}_k, \alpha) = C_{p_1^{e_1^{(1)}}}
    \times \dotsb \times C_{p_1^{e_n^{(1)}}} \times \dotsb \times
    C_{p_k^{e_1^{(k)}}} \times \dotsb \times C_{p_k^{e_n^{(k)}}} \times
    C_\infty \times \overset{\alpha\text{-mal}}{\dotsb} \times C_\infty$
    eine vollständige Liste paarweise nicht-isomorpher, endlich erzeugter
    abelscher Gruppen.
\end{Satz}

\begin{Bem}
    Beispielsweise gibt es bis auf Isomorphie sieben abelsche Gruppen $A$
    mit $|A| = 32$ ($C_{32}$, $C_{16} \times C_2$, $C_8 \times C_4$,
    $C_8 \times C_2 \times C_2$, $C_4 \times C_4 \times 2$,
    $C_4 \times C_2 \times C_2 \times C_2$,
    $C_2 \times C_2 \times C_2 \times C_2 \times C_2$), aber nur eine
    mit $|A| = 15$ ($C_{15}$).
\end{Bem}

\begin{Bem}
    Das Wiedererkennungsproblem für abelsche Gruppen ist schwierig zu lösen,
    betrachtet man z.\,B. die $\integer$-Moduln
    $M_1 = \integer/2\integer \oplus \integer$ und
    $M_2 = \integer \oplus \integer$.
    Es gilt $M_1 = \aufspann{(1 + 2\integer, 1), (0, 1)}$ und
    $M_2 = \aufspann{(1, 0), (0, 1)}$, jedoch ist die Ordnungen aller
    Elemente der beiden Erzeugendensysteme $\infty$.
    Aus der Ordnung der Elemente von einem Erzeugendensystem kann man also
    nicht auf die abelsche Gruppe schließen.
\end{Bem}

\begin{Satz}{$\rational \otimes_\integer A = (0)$}
    Sei $A$ eine endliche abelsche Gruppe.
    Dann ist $\rational \otimes_\integer A = (0)$.
\end{Satz}

\begin{Satz}{$\rational \otimes_\integer \mathcal{F}$ als Vektorraum}
    Sei $\mathcal{F}$ ein freier $\integer$-Modul. \\
    Dann ist $\rational \otimes_\integer \mathcal{F}$ ein $n$-dimensionaler
    $\rational$-Vektorraum.
\end{Satz}

\begin{Satz}{Rangbestimmung des freien Anteils} \\
    Seien $M$ eine endlich erzeugte abelsche Gruppe und
    $n = \dim_\rational \rational \otimes_\integer M$. \\
    Dann ist $M = T(M) \oplus \mathcal{F}$, wobei der freie Anteil
    $\mathcal{F}$ von $M$ vom Rang $n$ ist.
\end{Satz}

\begin{Satz}{Anzahl abelscher Gruppen}
    Sei $k \in \natural$.
    Dann gibt es nur endlich viele paarweise nicht-isomorphe
    abelsche Gruppen $A$ mit $|A| = k$.
    Ist $k$ multiplizitätenfrei (in der Primfaktorzerlegung kommt jede
    Primzahl nur mit Exponent $1$ vor), so gibt es bis auf
    Isomorphie genau eine abelsche Gruppe $A$ mit $|A| = k$, nämlich die
    zyklische Gruppe $\integer/k\integer$ der Ordnung $k$.
\end{Satz}

\begin{Satz}{Kriterium für abelsche Gruppe zyklisch}
    Sei $A$ eine abelsche Gruppe.
    Dann ist $A$ zyklisch genau dann, wenn $A$ für jeden Teiler $d$ von $|A|$
    genau eine Untergruppe der Ordnung $d$ besitzt.
\end{Satz}

\subsection{%
    Die kanonisch rationale Form%
}

\begin{Bem}
    Seien $K$ ein Körper, $V$ ein endlich-dimensionaler $K$-Vektorraum und
    $f \in \End_K(V)$.
    Dann kann man den $K[t]$-Modul $V_f = V$ betrachtet, wobei die
    $K[t]$-Operation gegeben ist durch $p(t) \cdot v = (p(f))(v)$.
    Für das Verschwindungsideal $\mathcal{I}_f$ folgt sofort
    $\mathcal{I}_f = \ann_{K[t]}(V_f)$ sowie $\ordnung(V_f) = \mu_f(t)$.
\end{Bem}

\begin{Lemma}{$V_f$ e.e. Torsionsmodul}
    $V_f$ ist endlich-erzeugter Torsionsmodul.
\end{Lemma}

\begin{Lemma}{Unterraum von $V$ $f$-invariant $\Leftrightarrow$
              Unterraum Untermodul von $V_f$}
    Sei $U \ur V$ ein Untervektorraum.
    Dann ist $U$ $f$-invariant genau dann, wenn $U$ ein $K[t]$-Untermodul
    von $V_f$ ist.
\end{Lemma}

\begin{Satz}{$V_f$ und $V_g$ isomorph für $f, g$ konjugiert} \\
    Seien $f, g \in \End_K(V)$ konjugiert, d.\,h. es gibt ein
    $d \in \Aut_K(V)$ mit $f = d^{-1} g d$. \\
    Dann ist $d: V_f \rightarrow V_g$ ein $K[t]$-Isomorphismus und daher ist
    $V_f \cong V_g$ (als $K[t]$-Moduln).
\end{Satz}

\begin{Bem}
    Also ist die Modulstruktur von $V$ als $K[t]$-Modul für konjugierte
    Endomorphismen gleich, d.\,h. $V_f$ und $V_g$ sind zum selben Prototyp
    aus der obigen Liste isomorph.
    Weiter unten wird gezeigt:
    Dieser Prototyp bestimmt eine kanonisch rationale
    $K$-Basis von $V$, sodass konjugierte Endomorphismen dieselbe
    kanonisch rationale Form haben.
    Analog gilt dies für ähnliche Matrizen.
    Weil jede Matrix bzw. jeder Endomorphismus zu ihrer kanonisch rationalen
    ähnlich bzw. konjugiert ist, sind dann Matrizen/Endomorphismen mit der
    gleichen kanonisch rationalen Form ähnlich/konjugiert.
\end{Bem}

\begin{Kor}
    Seien $A, B \in M_n(K)$.
    Dann sind $A$ und $B$ ähnlich genau dann, wenn $A$ und $B$ dieselbe
    kanonisch rationale Form haben.
\end{Kor}

\begin{Bem}
    Das Minimalpolynom $\mu_f(t)$ als normiertes Polynom als Produkt normierter
    irreduzibler Polynome $\mu_f(t) = p_1(t)^{\nu_1} \dotsm p_k(t)^{\nu_k}$
    ($p_1, \dotsc, p_k \in K[t]$ paarweise verschieden, irreduzibel, normiert)
    dargestellt werden.
    So erhält man die Primärkomponentenzerlegung von
    $V_f = M_{p_1} \oplus \dotsb \oplus M_{p_k}$
    wegen $\ordnung(V_f) = \mu_f(t)$.
    Die Primärkomponenten kann man folgendermaßen ausrechnen:
\end{Bem}

\begin{Satz}{Primärkomponenten von $V_f$}
    Sei $\mu_f(t) = p_1(t)^{\nu_1} \dotsm p_k(t)^{\nu_k}$ wie eben. \\
    Dann ist $\ker(p_i(f)^{\nu_i-1}) \lneqq M_{p_i} =
    \ker(p_i(f)^{\nu_i}) \ur V$ für $i = 1, \dotsc, k$.
\end{Satz}

\begin{Satz}{Bestimmung der $\nu_i$}
    Die aufsteigende Kette
    $\ker p_i(f) \subseteq \dotsb \subseteq \ker p_i(f)^j \subseteq \dotsb$
    wird wegen $\dim_K V$ stationär.
    Sei $m$ die kleinste natürliche Zahl, sodass
    $\ker(p_i(f)^m) = \ker(p_i(f)^{m+1})$ ist.
    Dann ist $m = \nu_i$.
\end{Satz}

\begin{Lemma}{Primärkomponente von $t - \lambda$ ist verallg. Eigenraum}\\
    Seien $\mu_f(t) = p_1(t)^{\nu_1} \dotsm p_k(t)^{\nu_k}$ wie eben
    und $\lambda \in K$. \\
    Ist $p_i(t) = t - \lambda$ ein lineares Polynom,
    so ist $M_{p_i} = \mathcal{V}_f(\lambda)$.
\end{Lemma}

\begin{Satz}{Basis des zyklischen $K[t]$-Moduls}
    Seien $p \in K[t]$ ein Polynom mit $\deg p = n$ und \\
    $C_p = K[t]/K[t]p$ der zyklische $K[t]$-Modul.
    Dann ist $\dim_K(C_p) = n$ und $\{\nk{1}, \nk{t}, \dotsc, \nk{t}^{n-1}\}$
    ist $K$-Basis von $C_p$ (als $K$-Vektorraum), wobei
    $\nk{t}^i = t^i + K[t]p \in C_p$ ist.
\end{Satz}

\begin{Satz}{von $v$ erzeugter zyklischer Untermodul $K[t] \cdot v$}
    Sei $v \in V_f$.
    Dann ist der von $v$ erzeugte zyklische $K[t]$-Untermodul $K[t] \cdot v$
    der von $v$ erzeugte $f$-zyklische Unterraum von $V$. \\
    Dieser ist $f$-invariant und $f_{\aufspann{v}}$ sei die Einschränkung
    $f|_{K[t]v}$ von $f$ auf $K[t]v$. \\
    Sei $\mu_{f_{\aufspann{v}}}(t) = p(t) = \alpha_0 + \dotsb +
    \alpha_{k-1} t^{k-1} + t^k$ das normierte Minimalpolynom von
    $f_{\aufspann{v}}$. \\
    Dann ist $\ordnung(v) = p(t)$ und
    $\basis{B} = \{v, f(v), \dotsc, f^{k-1}(v)\}$ ist $K$-Basis von $K[t]v$. \\
    Die Matrix $\hommatrix{f_{\aufspann{v}}}{B}{B}$ ist die
    $k \times k$-Begleitmatrix von $p(t)$.
\end{Satz}

\pagebreak

\begin{Satz}{kanonisch rationale Form I}
    Seien $V$ ein $K$-Vektorraum mit $\dim_K V = n$, $f \in \End_K(V)$ und
    $\mu_f(t) = p_1(t)^{\nu_1} \dotsm p_k(t)^{\nu_k}$ die Primfaktorzerlegung
    von $\mu_f(t)$ in $K[t]$ mit paarweise verschiedenen, irreduziblen,
    normierten Polynomen.
    Sei außerdem $(p) \in M_{\deg p \times \deg p}(K)$ die Begleitmatrix von
    $p \in K[t]$. \\
    Dann gibt es eine Basis $\basis{B}$ von $V$ und natürliche Zahlen
    $e_1^{(i)} \ge \dotsb \ge e_m^{(i)} \ge 0$, $i = 1, \dotsc, m$,
    $m \in \natural$, sodass die $n \times n$-Matrix $\hommatrix{f}{B}{B}$
    die Blockdiagonalform \\
    $\diag\left\{\left(p_1^{e_1^{(1)}}\right), \dotsc,
    \left(p_1^{e_m^{(1)}}\right), \dotsc,
    \left(p_k^{e_1^{(k)}}\right), \dotsc,
    \left(p_k^{e_m^{(k)}}\right)\right\}$ hat, wobei
    $\sum_{i=1}^k \sum_{j=1}^m e_j^{(i)} \deg p_i = n$ ist. \\
    Diese Form heißt \begriff{kanonisch rationale Form} oder auch
    \begriff{\name{Frobenius}-Normalform} von $f$. \\
    Für $n \times n$-Matrizen ist sie analog definiert.
\end{Satz}

\begin{Satz}{kanonisch rationale Form II}
    $V$ hat eine $K$-Basis, sodass
    $\hommatrix{f}{B}{B} = \diag\{(q_1), \dotsc, (q_s)\}$ ist mit
    $q_i = p_1^{e_i^{(1)}} \dotsm p_k^{e_i^{(k)}}$, wobei
    $\mu_f(t) = q_1 = p_1^{\nu_1} \dotsm p_k^{\nu_k}$ ist mit
    $\nu_i = e_i^{(1)}$.
    Zusätzlich ist dann
    $\chi_f(t) = p_1^{e_1^{(1)} + \dotsb + e_m^{(1)}} \dotsm
    p_k^{e_1^{(k)} + \dotsb + e_m^{(k)}}$.
    Insbesondere ist $\mu_f(t) = \mu_f(t)$ genau dann, wenn $V_f$ nur einen
    Elementarteiler hat, d.\,h. zyklischer $K[t]$-Modul ist.
\end{Satz}

\pagebreak
