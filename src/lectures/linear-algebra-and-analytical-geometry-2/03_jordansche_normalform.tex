\section{%
    Die \textsc{Jordan}sche Normalform%
}

\subsection{%
    Der Satz von \textsc{Cayley}-\textsc{Hamilton}%
}

\begin{Satz}{Teilen von charakteristischen Polynomen}
    Seien $V$ ein endlich-dimensionaler $K$-Vektorraum, $f \in \End_K(V)$,
    $U$ ein $f$-invarianter Unterraum und $\widehat{f}$ die Einschränkung
    von $f$ auf $U$. \\
    Dann teilt das charakteristische Polynom der Einschränkung $\widehat{f}$
    das von $f$:
    $\chi_{\widehat{f}}(t) \;|\; \chi_f(t)$.
\end{Satz}

\begin{Bem}
    Man kann Endomorphismen in Polynome über $K$ einsetzen und erhält
    wieder Endomorphismen:
    Ist $p(t) = \sum \alpha_i t^i \in K[t]$ und $f \in \End_K(V)$,
    so ist $p(f) = \sum \alpha_i f^i \in \End_K(V)$. \\
    Für $p(t), q(t) \in K[t]$ gilt $(pq)(f) = p(f) \circ q(f)$.
\end{Bem}

\begin{Def}{zyklischer Unterraum}
    Sei $x \in V$. \\
    Dann heißt $W = \aufspann{x, f(x), f^2(x), \dotsc}$
    der \begriff{von $x$ erzeugte $f$-zyklische Unterraum} von $V$.
\end{Def}

\begin{Lemma}{über zyklische Unterräume}
    Es gilt $W = \{(p(f))(x) \;|\; p \in K[t]\}$. \\
    Der von $x$ erzeugte $f$-zyklische Unterraum $W$ ist $f$-invariant. \\
    $W$ ist der kleinste $f$-invariante Unterraum von $V$, der $x$ enthält.
\end{Lemma}

\begin{Satz}{Basis des zyklischen Unterraums}
    Seien $f \in \End_K(V)$, $W$ der von $x \in V$ erzeugte $f$-zyklische
    Unterraum von $V$ und $k = \dim_K W \ge 1$ (d.\,h. $x \not= 0$). \\
    Dann ist $\basis{B}_W = (x, f(x), f^2(x), \dotsc, f^{k-1}(x))$ eine
    Basis von $W$.
\end{Satz}

\begin{Bem}
    Es gibt $\alpha_0, \dotsc, \alpha_{k-1} \in K$, sodass
    $f^k(x) = -\alpha_0 x - \alpha_1 f(x) - \dotsb -
    \alpha_{k-1} f^{k-1}(x)$. \\
    Ist $\widetilde{f} = f|_W$, so ist
    $\matrixm_{\widetilde{f}}(\basis{B}_W) =$
    \matrixsize{$\begin{pmatrix}0 & \cdots & 0 & -\alpha_0 \\
    1 & & 0 & -\alpha_1 \\ \vdots & \ddots & \vdots & \vdots \\
    0 & \cdots & 1 & -\alpha_{k-1}\end{pmatrix}$} \\
    die \begriff{Begleitmatrix} des Polynoms
    $p(t) = \alpha_0 + \alpha_1 t + \dotsb + \alpha_{k-1} t^{k-1} + t^k$.
\end{Bem}

\begin{Satz}{charakteristisches Polynom der Einschränkung}
    Seien die Bezeichnungen wie eben und
    $f^k(x) = -\alpha_0 x - \alpha_1 f(x) - \dotsb - \alpha_{k-1}(x) f^{k-1}$.
    Dann ist das charakteristische Polynom von $\widetilde{f} = f|_W$
    gegeben durch
    $\chi_{\widetilde{f}}(t) = \alpha_0 + \alpha_1 t + \dotsb +
    \alpha_{k-1} t^{k-1} + t^k$.
\end{Satz}

\begin{Def}{erfüllt}
    Seien $f \in \End_K(V)$ und $p(t) \in K[t]$.
    Dann \begriff{erfüllt $f$ das Polynom $p(t)$}, falls
    $p(f) \equiv 0$.
\end{Def}

\begin{Satz}{\textsc{Cayley}-\textsc{Hamilton}}
    Seien $f \in \End_K(V)$ und $V$ endlich-dimensional. \\
    Dann erfüllt $f$ sein charakteristisches Polynom $\chi_f(t)$.
\end{Satz}

\pagebreak

\subsection{%
    Verallgemeinerte Eigenräume%
}

\begin{Def}{\textsc{Jordan}-Block/-Form}
    \matrixsize{$J_\lambda(k) = \begin{pmatrix}\lambda & 1 & 0 & \cdots & 0 \\
    0 & \lambda & 1 & & 0 \\ \vdots & & \ddots & \ddots & \vdots \\
    0 & 0 & & \lambda & 1 \\ 0 & 0 & \cdots & 0 & \lambda\end{pmatrix}$},
    \qquad
    \matrixsize{$\begin{pmatrix}J_1 & 0 & \cdots & 0 & 0 \\
    0 & J_2 & & 0 & 0 \\ \vdots & & \ddots & & \vdots \\
    0 & & & J_{k-1} & 0 \\ 0 & 0 & \cdots & 0 & J_k\end{pmatrix}$} \\
    Eine $k \times k$-Matrix der Form $J_\lambda(k)$
    heißt \begriff{\textsc{Jordan}-Block}. \\
    Eine Matrix heißt in
    \begriff{\textsc{Jordan}-Form} oder
    \begriff{\textsc{Jordan}sche Normalform}, wenn sie in der Form
    wie oben rechts ist mit $J_i = J_{\lambda_i}(k_i)$ für $i = 1, \dotsc, k$,
    wobei die $\lambda_i$ die (nicht notwendigerweise verschiedenen) Eigenwerte
    von $A$ sind und $k_i \in \natural$ ist.
\end{Def}

\begin{Def}{\textsc{Jordan}-Basis}
    Seien $V$ ein endlich-dimensionaler Vektorraum und $f \in \End_K(V)$,
    wobei das charakteristische Polynom $\chi_f(t)$ in Linearfaktoren
    zerfällt. \\
    Eine \begriff{\textsc{Jordan}-Basis} von $f$ ist eine Basis
    $\basis{B}_f$ von $V$, sodass $\matrixm_f(\basis{B}_f)$ in Jordanform
    ist.
\end{Def}

\begin{Def}{verallgemeinerter Eigenraum}
    Seien $f \in \End_K(V)$, $V$ endlich-dimensional und $\lambda \in K$.
    Dann ist $\ker(f - \ell_\lambda) \ur \ker(f - \ell_\lambda)^2 \ur \dotsb
    \ur \ker(f - \ell_\lambda)^i \ur \dotsb$ eine aufsteigende Kette
    von Unterräumen von $V$, die terminiert
    (d.\,h. es gibt $k \in \natural$, sodass
    $\ker(f - \ell_\lambda)^{n+i} = \ker(f - \ell_\lambda)^n$ für alle
    $i \in \natural$).
    Daher ist $\mathcal{V}_\lambda(f) = \bigcup_{i=1}^\infty
    \ker(f - \ell_\lambda)^i$ ein wohldefinierter Unterraum von $V$.
    $\mathcal{V}_\lambda(f)$ heißt \begriff{verallgemeinerter Eigenraum}
    zum Eigenwert $\lambda$ von $f$ und seine Elemente heißen
    \begriff{verallgemeinerte Eigenvektoren} von $f$.
    Also gilt $\mathcal{V}_\lambda(f) = \{v \in V \;|\;
    \exists_{p \in \natural}\; (f - \ell_\lambda)^p(v) = 0\}$. \\
    Analog kann man auch für quadratische Matrizen $\mathcal{V}_\lambda(A)$
    definieren.
\end{Def}

\begin{Bem}
    Sei $\matrixm_f(\basis{B}_f) = J_\lambda(n)$.
    Dann ist $V_\lambda(f)$ ein- und $\mathcal{V}_\lambda(f)$ $n$-dimensional.
    Ist $\basis{B}_f = (v_1, \dotsc, v_n)$, so ist $v_1 \in V_\lambda(f)$ der
    bis auf skalare Vielfache eindeutig bestimmte Eigenvektor von $f$ mit
    Eigenwert $\lambda$ und $\basis{B}_f$ ist die \begriff{zyklische Basis} des
    von $v_n$ erzeugten $f - \ell_\lambda$-zyklischen Unterraums von $V$.
\end{Bem}

\begin{Satz}{$\mathcal{V}_\lambda(f)$ ist $f$-invarianter Unterraum}
    Sei $\lambda$ ein Eigenwert von $f \in \End_K(V)$. \\
    Dann ist $\mathcal{V}_\lambda(f)$ ein $f$-invarianter Unterraum von $V$,
    der den Eigenraum $V_\lambda(f)$ enthält.
\end{Satz}

\begin{Def}{Zykel}
    Seien $\lambda$ ein Eigenwert von $f \in \End_K(V)$, $v$ ein
    verallgemeinerter Eigenvektor zu $\lambda$
    (d.\,h. $v \in \mathcal{V}_\lambda(f)$)
    und $p \in \natural$ die kleinste
    natürliche Zahl, sodass $(f - \ell_\lambda)^p(v) = 0$. \\
    Dann ist $\basis{B} = ((f - \ell_\lambda)^{p-1}(v),
    (f - \ell_\lambda)^{p-2}(v), \dotsc, (f - \ell_\lambda)(v), v)$
    eine Basis des von $v$ erzeugten $f - \ell_\lambda$-zyklischen Unterraums
    von $V$. \\
    $\basis{B}$ ist \begriff{der von $v$ erzeugte Zykel verallgemeinerter
    Eigenvektoren von $f$} oder kurz \begriff{$\lambda$-Zykel von $f$}.
    $v$ heißt der \begriff{Anfangsvektor} und $(f - \ell_\lambda)^{p-1}(v)$
    der \begriff{Endvektor} des Zykels.
\end{Def}

\begin{Satz}{Eigenschaften von Anfangs-/Endvektor}
    Sei $\basis{B}$ ein $\lambda$-Zykel von $f$. \\
    Dann ist $\basis{B}$ eine Basis des vom Anfangsvektor erzeugten
    $f - \ell_\lambda$-zyklischen Unterraums $W$ von $V$ und dieser
    ist $f$-invariant.
    Die Einschränkung von $f$ auf $W$ besitzt genau einen eindimensionalen
    Eigenraum und dieser wird vom Endvektor des Zykels $\basis{B}$ erzeugt.
    Es gilt $\enmatrix{f|_W}{B} = J_\lambda(p)$.
\end{Satz}

\begin{Satz}{Jordanbasis $\;\Leftrightarrow\;$
             disjunkte Vereinigung von Zykeln}
    Sei $\basis{B}$ eine geordnete Basis von $V$. \\
    Dann ist $\basis{B}$ eine Jordanbasis von $f$ genau dann, wenn
    $\basis{B}$ eine disjunkte Vereinigung von Zykeln verallgemeinerter
    Eigenvektoren von $f$ ist.
\end{Satz}

\begin{Satz}{$V$ ist direkte Summe der verallgemeinerten Eigenräume}
    Sei $f \in \End_K(V)$, wobei $\chi_f(t)$ in Linearfaktoren zerfällt.
    Dann ist $V$ die direkte Summe der verallgemeinerten Eigenräume
    $V = \bigoplus_{\lambda} \mathcal{V}_\lambda(f)$, wobei $\lambda$
    die Menge der Eigenwerte von $f$ durchläuft.
\end{Satz}

\begin{Kor}
    Seien $\lambda_1, \dotsc, \lambda_k$ die paarweise verschiedenen
    Eigenwerte von $f$,
    $\basis{B}_i$ eine Basis von $\mathcal{V}_{\lambda_i}(f)$,
    $\basis{B} = \bigcup_{i=1}^k \basis{B}_i$ und
    $f_i$ die Einschränkung von $f$ auf $\mathcal{V}_{\lambda_{i}}(f)$. \\
    Dann ist $\enmatrix{f}{B} =$ \matrixsize{%
    $\begin{pmatrix}A_1 & & 0 \\ & \ddots & \\ 0 & & A_k\end{pmatrix}$},
    wobei $A_i = \matrixm_{f_i}(\basis{B}_i)$ ist.
\end{Kor}

\subsection{%
    Die \textsc{Jordan}sche Normalform: Algorithmus%
}

\begin{Bem}
    Im Folgenden wird versucht, ein Algorithmus zur Bestimmung der
    JNF und der zugehörigen Jordanbasis eines
    Endomorphismus bzw. einer Matrix zu finden, wobei immer vorausgesetzt wird,
    dass das charakteristische Polynom vollständig in Linearfaktoren
    zerfällt. \\
    Zur Einfachheit kann dank obiger Folgerung angenommen werden, dass
    $\chi_f(t) = (t - \lambda)^n$, d.\,h. $f$ besitzt genau einen
    Eigenwert $\lambda$ mit Vielfachheit $n$.
\end{Bem}

\begin{Lemma}{Kern-Dimensionen eines Jordanblocks}
    Sei $J = J_\lambda(k)$ ein Jordanblock. \\
    Dann ist $\dim_K \ker(J - \lambda E)^i = i$ für $i = 1, \dotsc, k$ und
    $\dim_K \ker(J - \lambda E)^i = k$ für $i > k$.
\end{Lemma}

\begin{Lemma}{Bestimmung der Anzahl und
              Größen der Jordanblöcke einer Matrix} \\
    Seien $A$ eine Matrix in Blockdiagonalform, deren $s$ Diagonalblöcke
    Jordanblöcke $J_i = J_\lambda(i)$ sind ($\lambda \in K$ fest),
    sowie $n_i = \dim_K \ker(A - \lambda E)^i$ und $r \in \natural$,
    sodass $n_{r - 1} < n_r = n_{r+1}$. \\
    Sei außerdem $k_i \in \natural_0$ die Anzahl der vorkommenden Kästchen
    $J_i$. \\
    Dann ist $n_1 = k_1 + k_2 + k_3 + \dotsb + k_r$, \qquad
    $n_2 = n_1 + k_2 + k_3 + \dotsb + k_r$, \\
    $n_3 = n_2 + k_3 + \dotsb + k_r$, \quad \dots, \quad
    $n_r = n_{r-1} + k_r$. \\
    Insbesondere ist $n_i - n_{i-1} = k_i + k_{i+1} + \dotsb + k_r$ für
    $i = 2, \dotsc, r$. \\
    Daher lassen sich die $k_i$ rekursiv aus den $n_j$ ausrechnen.
\end{Lemma}

\begin{Prozedur}{Bestimmung der Jordanschen Normalform (1)} \\
    Sei $A \in M_n(K)$, sodass $\chi_A(t)$ in Linearfaktoren zerfällt. \\
    Dann kann folgendermaßen die Jordansche Normalform von $A$ bestimmt werden:
    \begin{enumerate}
        \item Man ermittelt die Eigenwerte von $A$.
        Für jeden Eigenwert $\lambda \in K$ von $A$ werden die folgenden
        Schritte durchgeführt:

        \item Man berechnet $n_i = \dim_K \ker(A - \lambda E)^i$ für
        $i = 1, 2, \dotsc$.
        Beim ersten $r$ mit $n_r = n_{r+1}$ bricht man ab, denn
        die Dimensionen bleiben dann konstant.

        \item Man berechnet $l_i = n_i - n_{i-1}$ für $i = 1, \dotsc, r$,
        wobei $n_0 = 0$.

        \item Man berechnet $k_i = l_i - l_{i+1}$ für $i = 1, \dotsc, r$,
        wobei $l_{r+1} = 0$.

        \item Der Block der Jordanform von $A$, der zum Eigenwert $\lambda$
        korrespondiert, ist die Blockdiagonalmatrix, bei der
        $J_\lambda(i)$ genau $k_i$-mal als Diagonalblock auftritt.
    \end{enumerate}
\end{Prozedur}

\begin{Prozedur}{Bestimmung der Jordanschen Normalform (2)} \\
    Gegeben seien die $n_i$ wie eben.
    Man malt ein Diagramm aus Kreuzen in der Ebene in einem Gitter
    und zwar in die erste Zeile $l_1 = n_1$ Kreuze, in die zweite
    $l_2 = n_2 - n_1$ und in die $i$-te Zeile
    $l_i = n_i - n_{i-1}$ Kreuze. \\
    Wegen $l_i = k_i + k_{i+1} + \dotsb + k_r$ erhält man eine abfallende
    Folge natürlicher Zahlen, die sich mit $l_1 + l_2 + \dotsb + l_r =
    (n_1 - 0) + (n_2 - n_1) + \dotsb + (n_r - n_{r-1}) = n_r$ gerade zu
    $n_r = \dim_K \mathcal{V}_\lambda(A)$ aufsummieren. \\
    Die Spalten des entstehenden Diagramms geben dann gerade
    die $\lambda$-Zyklen wieder:
    Eine Spalte mit $k$ Kreuzen entspricht einem Jordanblock $J_\lambda(k)$
    der Größe $k$ von $A$. \\
    Das Diagramm heißt \begriff{\textsc{Young}-Diagramm}
    zur Partition $l_1 \ge \dotsb \ge l_r$ von $n_r$ oder
    $\lambda$-Diagramm von $A$ und wird mit $\mathfrak{D}_\lambda$
    bezeichnet. \\
    Im Diagramm entsprechen den untersten/obersten Spitzen der Spalten
    die Anfangs-/Endvektoren der $\lambda$-Zykeln.
\end{Prozedur}

\begin{Def}{linear unabhängig modulo $U$}
    Seien $U \ur V$ und $y_1, \dotsc, y_s \in V$.
    Dann sind die $y_i$ linear unabhängig modulo $U$, falls die
    Nebenklassen $y_1 + U, \dotsc, y_s + U$ in $V/U$ linear unabhängig sind,
    d.\,h. ist $\sum_{i=1}^s \lambda_i y_i \in U$ mit
    $\lambda_1, \dotsc, \lambda_s \in K$, dann ist
    $\lambda_1 = \dotsb = \lambda_s = 0$. \\
    Sind $y_1, \dotsc, y_s$ linear unabhängig modulo $U$, so sind sie
    linear unabhängig in $V$.
    Die Umkehrung gilt nicht.
\end{Def}

\begin{Satz}{Vereinigung von Zykeln ist linear unabhängig}
    Seien $f \in \End_K(V)$ und $\lambda \in K$ ein Eigenwert von $f$.
    Für $i = 1, \dotsc, s$ seien $\lambda$-Zyklen $Z_i$ von $f$ mit derselben
    Länge $t$ gegeben, wobei $y_i$ der Anfangsvektor von $Z_i$ ist. \\
    Ist die Menge der Anfangsvektoren $\{y_1, \dotsc, y_s\}$ linear
    unabhängig modulo $\ker(f - \ell_\lambda)^{t-1}$, so ist
    $Z = \bigcup_{i=1}^s Z_i$ ebenfalls linear unabhängig. \\
    Insbesondere ist daher die Summe der von den $Z_i$ aufgespannten
    Unterräume direkt.
\end{Satz}

\begin{Kor}
    Seien wie eben $y_1, \dotsc, y_s \in \ker(f - \ell_\lambda)^t$, deren
    Restklassen im Faktorraum \\
    $\ker(f - \ell_\lambda)^t / \ker(f - \ell_\lambda)^{t-1}$ linear
    unabhängig sind. \\
    Dann sind die von den $y_i$ erzeugten $\lambda$-Zykel paarweise
    disjunkt.
\end{Kor}

\begin{Lemma}{höhere Kerne bleiben gleich}
    Sei $\mathcal{N}_i = \ker(f - \ell_\lambda)^i$. \\
    Gilt $\mathcal{N}_r = \mathcal{N}_{r+1}$, so gilt
    $\mathcal{N}_r = \mathcal{N}_{r+i}$ für alle $i \in \natural$.
\end{Lemma}

\begin{Prozedur}{Bestimmung der Jordanbasis} \\
    Sei $f \in \End_K(V)$, sodass $\chi_f(t)$ in Linearfaktoren zerfällt. \\
    Dann kann folgendermaßen die Jordansche Normalform von
    $f$ bestimmt werden:
    \begin{enumerate}
        \item
        Sei $r \in \natural$ minimal mit $\mathcal{N}_r = \mathcal{N}_{r+1}$
        (Anzahl der Zeilen im $\lambda$-Diagramm).
        Man ergänzt eine Basis von $\mathcal{N}_{r-1}$ mit
        $y_1, \dotsc, y_{k_r}$  zu einer Basis von $\mathcal{N}_r$.

        \item
        Im $\lambda$-Diagramm ordnet man der $i$-ten Spalte von unten nach
        oben den Kreuzen die Elemente
        $y_i, (f - \ell_\lambda)(y_i), \dotsc, (f - \ell_\lambda)^{r-1}(y_i)$
        für $i = 1, \dotsc, k_r$ zu.
        Die Vektoren einer Spalte bilden dann einen $\lambda$-Zykel von $f$.
        Sei $U_1$ die Summe der von diesen $\lambda$-Zykeln aufgespannten
        Unterräumen, dann bilden die $(f - \ell_\lambda)^k y_i$
        mit $i = 1, \dotsc, k_r$ und $k = 1, \dotsc, r$ eine Basis von $U_1$.

        \item
        Die nächste, also die $k_r + 1$-te Spalte ist kürzer als die
        vorherigen.
        Sei sie von der Länge $t$ und $k_t$ die Anzahl der Spalten dieser
        Länge.
        Es gibt $k_t$ Basiselemente in einem Komplement von
        $(U_1 \cap \mathcal{N}_t) + \mathcal{N}_{t-1}$ in $\mathcal{N}_t$
        und nehmen wie eben die davon erzeugten $\lambda$-Zyklen von $f$.
        Diese erzeugen $U_2$ und sind eine Basis von $U_2$.

        \item
        Die nächste, also die $k_r + k_t + 1$-te Spalte ist kürzer als die
        vorherigen.
        Sei sie von der Länge $w$ und $k_w$ die Anzahl der Spalten dieser
        Länge.
        Es gibt $k_w$ Basiselemente in einem Komplement
        von $((U_1 + U_2) \cap \mathcal{N}_w) + \mathcal{N}_{w-1}$
        in $\mathcal{N}_w$
        und nehmen wie eben die davon erzeugten $\lambda$-Zyklen von $f$.
        Diese erzeugen $U_3$ und sind eine Basis von $U_3$.

        \item
        Man fährt so fort, bis man eine Basis von ganz
        $\mathcal{V}_\lambda(f)$ konstruiert hat.
        Jedem Kreuz im $\lambda$-Diagramm ist nun genau ein Basiselement
        zugeordnet.
        Diese werden nun spaltenweise (und von oben nach unten)
        durchnummeriert und bilden dann die Jordanbasis.
    \end{enumerate}
\end{Prozedur}

\begin{Def}{Fahne, angepasst}
    Sei $V$ ein $K$-Vektorraum.
    Eine \begriff{Fahne} der Länge $k$ in $V$ ist eine aufsteigende
    Kette $\mathcal{F}: (0) = U_0 \ur U_1 \ur \dotsb \ur U_k \ur V$ von
    Unterräumen $U_i$ von $V$. \\
    Eine Basis $\basis{B} = (v_1, \dotsc, v_n)$ von $V$ heißt
    \begriff{an $\mathcal{F}$ angepasst}, falls
    $(v_1, \dotsc, v_{m_i})$ eine Basis von $U_i$ ist, wobei
    $m_i = \dim_K U_i$ ist.
\end{Def}

Die Unterräume $\ker(f - \ell_\lambda)^i$ von $\mathcal{V}_\lambda(f)$
sind ein Beispiel von Fahnen, wobei die zugehörige Jordanbasis angepasst ist.

\begin{Lemma}{Eigenwerte und charakteristisches Polynom von
             nilpotenten/unipotenten Matrizen} \\
    Eine nilpotente Matrix $A \in M_n(K)$ kann nur $0$ als Eigenwert haben,
    d.\,h. $\chi_A(t) = t^n$. \\
    Ist $A$ unipotent, dann muss für jeden Eigenwert $\lambda^k = 1$ gelten,
    d.\,h. $\lambda$ ist eine $k$-te Einheitswurzel.
    Also ist $\chi_A(t) = \prod_{i=1}^n (t - \zeta_i)$ mit $\zeta_i^k = 1$.
\end{Lemma}

\begin{Lemma}{binomischer Lehrsatz im Ring}
    Seien $R$ ein Ring und $a, b \in R$ mit $ab = ba$. \\
    Dann gilt $(a + b)^n = \sum_{i=0}^n \binom{n}{i} a^i b^{n-i}$.
    Ist zusätzlich eines der beiden Ringelemente nilpotent, so lässt
    sich die Summe einfach auswerten.
\end{Lemma}

\begin{Lemma}{Jordanform ist Summe einer Diagonalmatrix und einer
              nilpotenten Matrix} \\
    Sei $A \in M_n(K)$ in Jordanform.
    Dann ist $A = D + N$ mit $DN = ND$, wobei $D$ eine Diagonalmatrix und
    $N$ eine nilpotente Matrix ist.
\end{Lemma}

\begin{Lemma}{ähnliche Matrizen zu nilpotenter Matrix sind nilpotent} \\
    Seien $A, N \in M_n(K)$ ähnlich, wobei $N$ nilpotent (unipotent) ist. \\
    Dann ist $A$ ebenfalls nilpotent (unipotent).
\end{Lemma}

\begin{Satz}{Jordanzerlegung}
    Sei $A \in M_n(K)$, sodass $\chi_A(t)$ in Linearfaktoren zerfällt. \\
    Dann ist $A = S + N$ mit $SN = NS$, wobei $S$ eine diagonalisierbare
    und $N$ eine nilpotente Matrix ist.
    Diese Zerlegung heißt \begriff{Jordanzerlegung} von $A$.
\end{Satz}

\subsection{%
    Das Minimalpolynom%
}

\begin{Def}{Ideal}
    Sei $R$ ein Ring (oder eine $K$-Algebra).
    Eine nicht-leere Teilmenge $I \subseteq R$ heißt \begriff{Rechtsideal},
    falls $a - b \in I$ und $ar \in I$ für alle $a, b \in I$, $r \in R$ ist.
    Gilt $a - b \in I$ und $ra \in I$ für alle $a, b \in I$, $r \in R$,
    so heißt I \begriff{Linksideal}. \\
    Ein \begriff{(zweiseitiges) Ideal} ist eine nicht-leere Teilmenge
    $I \subseteq R$, die zugleich Links- und Rechtsideal ist.
    In diesem Fall schreibt man $I \trianglelefteq R$.
\end{Def}

\begin{Bem}
    Es gilt $0 \cdot i = 0 \in I$ für jedes Ideal.
    Sind $a, b \in I$, so ist auch $a + b \in I$, da $0 - b = -b \in I$ ist.
    Jedes Ideal $I \trianglelefteq R$ ist auch ein Ring, indem man die
    Addition und Multiplikation von $R$ auf $I$ einschränkt.
    Ist $J \trianglelefteq R$ und $J \subseteq I$, so ist
    $J \trianglelefteq I$.
    Ist $R$ ein kommutativer Ring, so sind Ideale, Links- und Rechtsideale
    dasselbe.
\end{Bem}

\begin{Def}{Faktorring}
    Seien $R$ ein Ring und $I \trianglelefteq R$ ein Ideal. \\
    Dann wird durch $r \sim s \;\Leftrightarrow\; r - s \in I$
    für $r, s \in R$ auf $R$ eine Äquivalenzrelation definiert. \\
    Die Äquivalenzklasse von $r \in R$ heißt $r + I$ und
    die Menge der Äquivalenzklassen mit \\
    $R/I = \{r + I \;|\; r \in R\}$.
    $R/I$ wird zum Ring durch
    $(r + I) + (s + I) = (r + s) + I$ und
    $(r + I) \cdot (s + I) = (r \cdot s) + I$
    und heißt \begriff{Faktorring}. \\
    Die natürliche Projektion $\pi: R \rightarrow R/I$, $\pi(r) = r + I$
    ist ein Ringhomomorphismus.
\end{Def}

\begin{Lemma}{Kern von Ringhomomorphismen}
    Sei $f: R \rightarrow S$ ein Ringhomomorphismus. \\
    Dann ist $\ker f \trianglelefteq R$ und $f$ ist injektiv genau dann,
    wenn $\ker f = (0)$.
\end{Lemma}

\begin{Satz}{Isomorphiesätze für Ringe}
    \begin{enumerate}
        \item
        Seien $f: R \rightarrow S$ ein Ringhomomorphismus
        und $I \trianglelefteq R$ ein Ideal mit $I \subseteq \ker f$.
        Dann gibt es genau einen Ringhomomorphismus $\widetilde{f}$, sodass
        $f = \widetilde{f} \circ \pi$.
        Es gilt $\widetilde{f}: R/I \rightarrow S$,
        $\widetilde{f}(r + I) = f(r)$.
        Mit $I = \ker f$ gilt insbesondere, dass $R/\ker f$ isomorph
        zu $\im f$ ist.

        \item
        Seien $R$ ein Ring und $I, J \trianglelefteq R$ zwei Ideale.
        Dann sind $I \cap J$ und \\
        $I + J = \{i + j \;|\; i \in I\;, j \in J\}$
        ebenfalls Ideale von $R$ und es gilt
        $I/(I \cap J) \cong (I + J)/J$.

        \item
        Seien $R$ ein Ring und $I, J, K \trianglelefteq R$ drei Ideale
        mit $K \subseteq J \subseteq I$. \\
        Dann ist $I/J \cong (I/K)/(J/K)$.
    \end{enumerate}
\end{Satz}

\begin{Bem}
    Jeder Kern eines Ringhomomorphismus ist ein Ideal.
    Jedes Ideal $I \trianglelefteq R$ ist Kern eines Ringhomomorphismus,
    nämlich der von $\pi: R \rightarrow R/I$.
    Also sind Ideale genau die Kerne von Ringhomomorphismen.
\end{Bem}

\pagebreak

\begin{Def}{Verschwindungsideal}
    Sei $V$ ein endlich-dimensionaler Vektorraum und $f \in \End_K(V)$.
    Dann ist
    $\mathcal{I}_f = \{p(t) \in K[t] \;|\; p(f) \equiv 0\}$
    ein Ideal von $K[t]$ und wird \begriff{Verschwindungsideal} genannt.
\end{Def}

\begin{Satz}{Polynomdivision}
    Seien $h, g \in K[t]$ Polynome mit $\deg g \le \deg h$.
    Dann gibt es Polynome $q, r \in K[t]$ mit $\deg r < \deg g$, sodass
    $h = gq + r$ ist.
    Das Polynom $r$ ist der Rest bei der Polynomdivision.
\end{Satz}

\begin{Def}{normiert}
    Ein Polynom $g(t) \in K[t]$ heißt \begriff{normiert}, falls der
    \begriff{führende Koef"|fizient}
    (also der nicht-verschwindende Koef"|fizient
    bei der höchsten Potenz) gleich $1$ ist.
\end{Def}

\begin{Satz}{Ideale des Polynomrings}
    Seien $I \trianglelefteq K[t]$ ein Ideal mit $I \not= (0)$ und
    $p \in I$ ein nicht-triviales Polynom minimalen Grades in $I$.
    Dann ist $I = p K[t]$ und es gilt
    $I = r K[t] \;\Leftrightarrow\; r = \beta p$, wenn $r \in K[t]$ und
    $\beta \in K$ mit $\beta \not= 0$ ist.
\end{Satz}

\begin{Def}{Erzeuger, Hauptideal}
    Es gibt genau ein normiertes Polynom $q \in I$, sodass $I = q K[t]$ ist.
    $q$ heißt \begriff{normierter Erzeuger} von $I$.
    Ideale, die von einem Element erzeugt werden, heißen \begriff{Hauptideale}.
\end{Def}

\begin{Bem}
    Der Satz sagt also aus, dass alle Ideale von $K[t]$ Hauptideale sind.
\end{Bem}

\begin{Def}{Minimalpolynom}
    Sei $f \in \End_K(V)$.
    Das eindeutig bestimmte normierte Polynom kleinsten Grades
    in $\mathcal{I}_f$ heißt \begriff{Minimalpolynom} von $f$ und wird
    mit $\mu_f(t)$ bezeichnet. \\
    Analog ist das Minimalpolynom $\mu_A(t)$ einer Matrix $A \in M_n(K)$
    definiert.
\end{Def}

\begin{Kor}
    Sei $p \in K[t]$ ein Polynom mit $p(f) \equiv 0$.
    Dann gibt es $q \in K[t]$, sodass \\
    $p(t) = q(t) \cdot \mu_f(t)$ ist,
    d.\,h. das Minimalpolynom $\mu_f(t)$ teilt $p$.
    Insbesondere teilt das Minimalpolynom das charakteristische Polynom von
    $f$.
\end{Kor}

\begin{Satz}{Minimalpolynome ähnlicher Matrizen gleich} \\
    Die Minimalpolynome ähnlicher Matrizen stimmen überein. \\
    Analog: Konjugierte Endomorphismen $f, g \in \End_K(V)$
    (d.\,h. $f = h^{-1} g h$ für ein $h \in \Aut_K(V)$)
    haben dasselbe Minimalpolynom.
\end{Satz}

\begin{Satz}{$\chi_f(t)$ und $\mu_f(t)$ haben dieselben Nullstellen}
    Sei $f \in \End_K(V)$.
    Dann ist $\lambda \in K$ eine Nullstelle von $\mu_f(t)$ genau dann,
    wenn er Eigenwert von $f$ ist.
    Also haben $\chi_f(t)$ und $\mu_f(t)$ dieselben Nullstellen.
\end{Satz}

\begin{Satz}{Minimalpolynome teilen sich}
    Seien $f \in \End_K(V)$, wobei $\chi_f(t)$ in Linearfaktoren zerfalle,
    $V = V_1 \oplus \dotsb \oplus V_k$ eine Zerlegung in $f$-invariante
    Unterräume $V_i$ sowie $\mu_i$ das Minimalpolynom von der
    Einschränkung $f_i$ von $f$ auf $V_i$ für $i = 1, \dotsc, k$. \\
    Dann teilt $\mu_f(t)$ das Polynom $\prod_{i=1}^k \mu_i(t)$ und jedes
    $\mu_i(t)$ teilt $\mu_f(t)$. \\
    Insbesondere gilt $\mu_f(t) = \prod_{i=1}^k \mu_i(t)$, falls die
    $\mu_i(t)$ paarweise teilerfremd sind.
\end{Satz}

\begin{Kor} \\
    Sei $A = \diag\{J_1, \dotsc, J_k\}$ eine Blockdiagonalmatrix und
    $\chi_A(t)$ zerfalle in Linearfaktoren. \\
    Dann ist $\mu_A(t) = \prod_{i=1}^k \mu_{J_i}(t)$,
    falls die $\mu_{J_i}(t)$ paarweise teilerfremd sind.
\end{Kor}

\begin{Satz}{Minimalpolynom bestimmen}
    Sei $f \in \End_K(V)$ mit
    $\chi_f(t) = (t - \lambda_1)^{n_1} \dotsm (t - \lambda_k)^{n_k}$,
    wobei die $\lambda_i$ paarweise verschieden sind. \\
    Dann ist $\mu_f(t) = (t - \lambda_1)^{m_1} \dotsm (t - \lambda_k)^{m_k}$,
    wobei $m_i$ für $i = 1, \dotsc, k$
    die kleinste natürliche Zahl $s \in \natural$ mit
    $\ker(f - \ell_{\lambda_i})^s = \ker(f - \ell_{\lambda_i})^{s+1}$ ist
    (d.\,h. die Größe des größten Jordanblocks zum Eigenwert $\lambda_i$). \\
    Insbesondere ist $f$ diagonalisierbar genau dann, wenn
    $\mu_f(t) = (t - \lambda_1) \dotsm (t - \lambda_k)$ ist.
\end{Satz}

\pagebreak
