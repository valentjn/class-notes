\begin{landscape}
    \chapter{%
        Zusätzliches%
    }

    \section{%
        \emph{Zusatz}: Übersicht über die behandelten DGLs%
    }

    \footnotesize

    Alle Funktionen hängen von $(x, t) \in \Omega_T$ ab, soweit nicht anders erklärt.
    Es ist $\div := \div_x$ und $\Delta := \Delta_x$.

    \renewcommand*{\arraystretch}{1.2}
    \begin{tabular}{p{65mm}p{55mm}p{128mm}}
        \toprule

        \textbf{DGL} & \textbf{Name} & \textbf{Herleitung}\\

        \midrule

        $\partial_t u + \div F = G$&
        Transport-Reaktionsgleichung&
        Massenbilanz in Kontrollvolumen,
        $F(x,t) \in \real^d$ Fluss, $G(x,t) \in \real$ Konz.gewinn\\

        $\partial_t u = G(x, t, u(x,t))$&
        parametrisierte ODE&
        aus TRGL: $F :\equiv 0$, $G$ $u$-abhängig\\

        $\partial_t u + \div(vu) = 0$&
        Advektionsgleichung&
        aus TRGL: $F := vu$ mit $v \in \C^1(\Omega_T, \real^d)$ Geschw.feld, $G :\equiv 0$\\

        \midrule

        $\partial_t u + \div F(u) = 0$&
        nicht-lineare Konvektionsgl.&
        aus TRGL: $F := F(u)$, $G :\equiv 0$\\

        $\partial_t u + \partial_x (v(u) \cdot u) = 0$&
        Konvektionsgleichung&
        aus nicht-linearer Konvektionsgl.: $d := 1$, $F(u) := v(u) \cdot u$\\

        $\partial_t u + \partial_x(\frac{1}{2} u^2) = 0$&
        Burgersgleichung&
        aus nicht-linearer Konvektionsgl.: $d := 1$, $F(u) := \frac{1}{2} u^2$\\

        \midrule

        $\partial_t u - \div(a(x) \nabla u) = 0$&
        allg. Dif"|fusionsgleichung&
        aus TRGL: $F := -a(x) \nabla u$ (Ficksches Gesetz) mit $a \in \C^1(\Omega)$ Diff.koeff.,
        $G :\equiv 0$\\

        $\partial_t u - \Delta u = 0$&
        Dif"|fusionsgleichung/instat. WLG&
        aus allg. Dif"|fusionsgleichung: $a(x) :\equiv 1$\\

        $-\Delta u = 0$&
        Laplace-Gleichung&
        aus instat. Wärmeleitungsgleichung mit $t \to \infty$ und
        $u(\cdot, t) \to \overline{u}(\cdot) \in \C^2(\overline{\Omega})$ glm.\\

        $-\Delta u = f$&
        Poisson-Gleichung&
        aus Laplace-Gleichung mit $G := f(x)$\\

        \midrule

        $-\div_x(\nabla_p L(\nabla u, u, x)) + \partial_z L(\nabla u, u, x) = 0$&
        Euler-Lagrange-Gleichung&
        PDE für Lösung $u$ des Variationsproblems
        $I(u) \le I(w) := \int_\Omega L(\nabla w, w, x)\dx$\\

        $-a_{11}(x) u''(x) + c(x) u(x) = f(x)$&
        Sturm-Liouville-Problem&
        aus ELGL: $L(p, z, x) := \frac{1}{2} p^\tp A(x) p + \frac{1}{2} c(x) z^2 - z f(x)$,
        $d := 1$, $a_{11}(x) > 0$, $c(x) > 0$\\

        $\partial_t^2 u - c^2 \Delta u = 0$&
        Wellengleichung&
        aus ELGL: $L(p, z, x) := \frac{c^2}{2} \sum_{i=1}^d |p_i|^2 - \frac{1}{2} |p_{d+1}|^2$
        aus Hamilton-Prinzip\\

        $-\Delta u + W'(u) = 0$&
        stat. Allen-Cahn-Gleichung&
        aus ELGL: $L(p, z, x) := W(z) + \frac{1}{2} \norm{p}^2$ mit z.\,B. $W(z) := (z^2 - 1)^2$\\

        \bottomrule
    \end{tabular}
\end{landscape}

\begin{landscape}
    \section{%
        \emph{Zusatz}: Übersicht über die Aussagen über PDE-Klassen%
    }

    \footnotesize

    %\renewcommand*{\arraystretch}{1.2}
    \begin{tabular}{p{20mm}p{20mm}p{30mm}p{169mm}}
        \toprule

        \textbf{PDE}&
        \textbf{Problem}&
        \textbf{Definition/Satz}&
        \textbf{Voraussetzungen/Aussage}\\

        \midrule

        \multirow{11}{20mm}{Advektions-gleichung}&
        \multirow{7}{20mm}{konstante Adv.geschw.}&
        Definition&
        $\Omega := \real^d$, $T := \infty$, $b \in \real^d$, $u_0 \in \C^1(\Omega)$
        $\implies$
        $\partial_t u + \div(bu) = 0$ in $\Omega_T$, $u(\cdot, 0) = u_0$ in $\Omega$\\

        \mrowcell&\mrowcell&Translationsinv.&
        $\forall_{(x, t) \in \Omega_T} \forall_{s \in (-t, T-t)}\;
        \frac{\d}{\ds} u(x + bs, t + s) = 0$\\

        \mrowcell&\mrowcell&Ex. + Eind.&
        $u(x, t) := u_0(x - bt)$ eind. kl. Lsg.\\

        \mrowcell&\mrowcell&$L^\infty$-Stabilität&
        $u_0 \in \C^1(\Omega) \cap L^\infty(\Omega) \implies
        \forall_{t \in (0, T)}\; \norm{u(\cdot, t)}_{L^\infty} \le \norm{u_0}_{L^\infty}$\\

        \mrowcell&\mrowcell&Max.-/Min.prinzip&
        $u_0 \in \C^1(\Omega) \cap L^\infty(\Omega) \implies
        \forall_{(x, t) \in \Omega_T}\;
        \inf_{\overline{x} \in \Omega} u_0(\overline{x}) \le u(x, t) \le
        \sup_{\overline{x} \in \Omega} u_0(\overline{x})$\\

        \mrowcell&\mrowcell&st. Abh. von $u_0$&
        $u_0, u_0' \in \C^1(\Omega) \cap L^\infty(\Omega) \implies
        \forall_{t \in (0, T)}\;
        \norm{u(\cdot, t) - u'(\cdot, t)}_{L^\infty} \le \norm{u_0 - u_0'}_{L^\infty}$\\

        \mrowcell&\mrowcell&keine st. Abh. von $b$&
        $\lnot[\forall_{t \in (0, T)} \exists_{C(t) > 0}
        \forall_{u_0 \in \C^1(\Omega) \cap L^\infty(\Omega), \norm{u_0}_{L^\infty} \le 1}
        \forall_{b, b' \in \real}\;
        \norm{u(\cdot, t) - u'(\cdot, t)}_{L^\infty} \le C(t) \norm{b - b'}]$\\

        \cmidrule{2-4}

        \mrowcell&\multirow{2}{20mm}{Reaktions-/ Quellterm}&
        Definition&
        $q \in \C^0(\Omega_T)$
        $\implies$
        $\partial_t u + \div(bu) = q$ in $\Omega_T$, $u(\cdot, 0) = u_0$ in $\Omega$\\

        \mrowcell&\mrowcell&Ex. + Eind.&
        $u(x, t) := u_0(x - bt) + \int_0^t q(x + (s-t)b, s)\ds$ eind. kl. Lsg.\\

        \cmidrule{2-4}

        \mrowcell&\multirow{2}{20mm}{nicht-lineare Konvektion}&
        Definition&
        $\Omega := \real$, $T > 0$, $f \in \C^2(\real)$, $u_0 \in \C^1(\Omega)$
        $\implies$
        $\partial_t u + \partial_x(f(u)) = 0$ in $\Omega_T$, $u(\cdot, 0) = u_0$ in $\Omega$\\

        \mrowcell&\mrowcell&lokale Ex.&
        $\norm{f''}_\infty, \norm{u_0'}_\infty < \infty
        \implies \forall_{\overline{x} \in \real} \exists_{\varepsilon > 0} \exists_{T > 0}
        \exists_{u \in \C^1(B_\varepsilon(\overline{x}) \times (0, T))}\;
        [\text{$u$ kl. Lsg.}]$,
        $u(x, t) = u_0(x - tf'(u(x, t)))$\\

        \midrule

        \multirow{17}{20mm}{Poisson-Gleichung}&
        \multirow{8}{20mm}{Laplace-Gleichung}&
        Definition&
        $\Omega \subset \real^d$
        $\implies$
        $-\Delta u = 0$ in $\Omega$\\

        \mrowcell&\mrowcell&MW-Eigenschaft&
        $u \in C^2(\Omega)$ harm., $x \in \Omega$, $r > 0$, $\overline{B_r(x)} \subset \Omega$
        $\implies \fint_{B_r(y)} u(y) \dy = u(x) = \fint_{\partial B_r(x)} u(y) \dsigma(y)$\\

        \mrowcell&\mrowcell&Max.prinzip&
        $\Omega$ of"|fen, beschr., $u \in \C^2(\overline{\Omega})$ harm.
        $\implies \max_{x \in \overline{\Omega}} u(x) = \max_{x \in \partial\Omega} u(x)$\\

        \mrowcell&\mrowcell&verallg. Max.prinz.&
        $\Omega$ of"|fen, beschr., $u \in \C^2(\Omega) \cap \C^0(\overline{\Omega})$,
        $-\Delta u = f \le 0$
        $\implies$ $u$ nimmt Max. auf dem Rand an\\

        \mrowcell&\mrowcell&Vgl.prinzip&
        $\Omega$ of"|fen, beschr., $u, v \in \C^2(\Omega) \cap \C^0(\overline{\Omega})$,
        $-\Delta u \le -\Delta v$ in $\Omega$, $u \le v$ auf $\partial\Omega$
        $\implies$ $u \le v$ in $\Omega$\\

        \mrowcell&\mrowcell&Regularität&
        $\Omega := \real^d$, $u \in \C^2(\Omega)$ harm.
        $\implies u \in \C^\infty(\Omega)$\\

        \mrowcell&\mrowcell&Fundamentallsg.&
        $\Omega := \real^d \setminus \{0\}$, $d > 1$
        $\implies \Phi \in \C^\infty(\Omega)$,
        $\Phi(x) := -\frac{1}{2\pi} \cdot \ln(\norm{x})$ für $d = 2$,\quad
        $\Phi(x) := \frac{1}{(d-2)\omega_d} \cdot \frac{1}{\norm{x}^{d-2}}$ für $d \ge 3$\\

        \mrowcell&\mrowcell&Eigenschaften&
        $\int_{B_\varepsilon(0)} \Phi(x)\dx \to 0$,\quad
        $\Phi \in L^1_\loc(\real^d)$,\quad
        $\Phi(\varepsilon e_1) \varepsilon^{d-1} \to 0$,\quad
        $\forall_{\varepsilon>0}\;
        \int_{\partial B_\varepsilon(0)} \nabla\Phi(x) \cdot n\dsigma(x) = -1$\\

        \cmidrule{2-4}

        \mrowcell&\multirow{3}{20mm}{Poisson-Gleichung}&
        Definition&
        $\Omega \subset \real^d$, $f\colon \Omega \to \real$
        $\implies$
        $-\Delta u = f$ in $\Omega$\\

        \mrowcell&\mrowcell&Rotationsinv.&
        $u \in \C^2(\Omega)$ kl. Lsg.,
        $O \in \real^{d \times d}$ orth., $\Omega = O\Omega$, $f = f \circ O$
        $\implies v \in \C^2(\Omega)$ kl. Lsg., $v(x) := u(Ox)$\\

        \mrowcell&\mrowcell&Faltungslösung&
        $\Omega := \real^d$, $d \ge 2$, $f \in \C^2_0(\Omega)$
        $\implies u := \Phi \ast f$ kl. Lsg.\\

        \cmidrule{2-4}

        \mrowcell&\multirow{4}{20mm}{Poisson-RWP}&
        Definition&
        $\Omega \subset \real^d$ of"|fen, beschr., $f \in \C^0(\Omega)$,
        $g \in \C^0(\partial\Omega)$
        $\implies$
        $-\Delta u = f$ in $\Omega$, $u = g$ auf $\partial\Omega$\\

        \mrowcell&\mrowcell&Eind.&
        es gibt höchstens eine kl. Lsg. $u \in \C^2(\Omega) \cap \C^0(\overline{\Omega})$\\

        \mrowcell&\mrowcell&st. Abh. von $g$&
        $g, g' \in \C^0(\partial\Omega)
        \implies \norm{u - u'}_\infty \le \norm{g - g'}_\infty$\\

        \mrowcell&\mrowcell&st. Abh. von $f$&
        $f, f' \in C^0(\Omega)
        \implies \norm{u - u'}_\infty \le C \norm{f - f'}_\infty$, $C := \frac{R^2}{2}$,
        $R := \sup_{x \in \Omega} \norm{x}$\\

        \bottomrule
    \end{tabular}

    \pagebreak

    \begin{tabular}{p{20mm}p{20mm}p{30mm}p{169mm}}
        \toprule

        \textbf{PDE}&
        \textbf{Problem}&
        \textbf{Definition/Satz}&
        \textbf{Voraussetzungen/Aussage}\\

        \midrule

        \multirow{11}{20mm}{Dif"|fusions-gleichung}&
        \multirow{5}{20mm}{AWP}&
        Definition&
        $\Omega \subset \real^d$, $T > 0$, $u_0\colon \Omega \to \real$
        $\implies$
        $\partial_t u - \Delta u = 0$ in $\Omega_T$,
        $u(\cdot, 0) = u_0$ in $\Omega$\\

        \mrowcell&\mrowcell&Skal.inv.&
        $\Omega := \real^d$, $T := \infty$, $u \in \C^2(\Omega_T)$ kl. Lsg., $\lambda \in \real$
        $\implies u_\lambda$ kl. Lsg., $u_\lambda(x, t) := u(\lambda x, \lambda^2 t)$\\

        \mrowcell&\mrowcell&Fundamentallsg.&
        $\Omega := \real^d$, $T := \infty$
        $\implies \Phi \in \C^\infty(\Omega_T)$,
        $\Phi(x, t) := \frac{1}{(4\pi t)^{d/2}} e^{-\norm{x}^2/(4t)}$\\

        \mrowcell&\mrowcell&Faltungslösung&
        $\Omega := \real^d$, $T := \infty$, $u_0 \in L^\infty(\Omega)$
        $\implies u \in \C^\infty(\Omega_T)$,
        $u(\cdot, t) := \Phi(\cdot, t) \ast u_0$ kl. Lsg.,\newline
        für $u_0 \in \C^0(\real)$ gilt
        $\forall_{\overline{x} \in \Omega}\;
        \lim_{(x, t) \to (\overline{x}, 0)} u(x, t) = u_0(\overline{x})$,
        $\forall_{t > 0}\; \norm{u(\cdot, t)}_{L^\infty} \le \norm{u_0}_{L^\infty}$\\

        \cmidrule{2-4}

        \mrowcell&\multirow{2}{20mm}{ARWP}&
        Definition&
        $g\colon \partial\Omega \times (0, T) \to \real$
        $\implies$
        $\partial_t u - \Delta u = 0$ in $\Omega_T$,
        $u(\cdot, 0) = u_0$ in $\Omega$,
        $u = g$ auf $\partial\Omega \times (0, T)$\\

        \mrowcell&\mrowcell&Max.prinzip&
        $u$ nimmt Maximum auf parabolischem Rand
        $\Gamma := (\Omega \times \{0\}) \cup (\partial\Omega \times [0, T])$ an\\

        \cmidrule{2-4}

        \mrowcell&\multirow{3}{20mm}{inhom. ARWP}&
        Definition&
        $f\colon \Omega_T \to \real$
        $\implies$
        $\partial_t u - \Delta u = f$ in $\Omega_T$,
        $u(\cdot, 0) = u_0$ in $\Omega$,
        $u = g$ auf $\partial\Omega \times (0, T)$\\

        \mrowcell&\mrowcell&Eind.&
        $\Omega \subset \real^d$ Lipschitz
        $\implies$ es gibt höchstens eine kl. Lsg.\\

        \mrowcell&\mrowcell&Konv. gg. stat. Lsg.&
        $\Omega \subset \real^d$ Lipschitz,
        $f, g$ zeitunabh.,
        $-\Delta\overline{u} = f$ in $\Omega$, $\overline{u} = g$ auf $\partial\Omega$
        $\implies \norm{u(\cdot, t) - \overline{u}}_{L^2} \le
        e^{-t/c_p} \norm{u_0 - \overline{u}}_{L^2}$\\

        \midrule

        \multirow{8}{20mm}{Wellen-gleichung}&
        \multirow{5}{20mm}{AWP}&
        Definition&
        $\Omega := \real^d$, $T > 0$, $c > 0$, $u_0 \in \C^2(\Omega)$, $v_0 \in \C^1(\Omega)$
        $\implies$
        %$u \in \C^2(\Omega_T) \cap \C^1(\overline{\Omega_T})$,
        $\partial_t^2 u - c^2 \Delta u = 0$ in $\Omega_T$, $u(\cdot, 0) = u_0$,
        $\partial_t u(\cdot, 0) = v_0$ in $\Omega$\\

        \mrowcell&\mrowcell&Ex. + Eind.&
        $d := 1 \implies u(x, t) :=
        \frac{1}{2} (u_0(x+ct) + u_0(x-ct)) + \frac{1}{2c} \int_{x-ct}^{x+ct} v_0(s)\ds$
        eind. kl. Lsg.\\

        \mrowcell&\mrowcell&$L^\infty$-Stabilität&
        $d := 1$, $u_0 \in \C^2(\Omega) \cap L^\infty(\Omega)$,
        $v_0 \in \C^1(\Omega) \cap L^1(\Omega)$
        $\implies
        \forall_{t \ge 0}\;
        \norm{u(\cdot, t)}_{L^\infty} \le \norm{u_0}_{L^\infty} +
        \frac{1}{2c} \norm{v_0}_{L^1}$\\

        \mrowcell&\mrowcell&st. Abh. von $u_0, v_0$&
        $d := 1$, $u_0, \overline{u_0} \in \C^2 \cap L^\infty$,
        $v_0, \overline{v_0} \in \C^1 \cap L^1$
        $\implies
        \forall_{t \ge 0}\;
        \norm{u(\cdot, t) - \overline{u}(\cdot, t)}_{L^\infty}
        \le C \left(\norm{u_0 - \overline{u_0}}_{L^\infty} +
        \norm{v_0 - \overline{v_0}}_{L^1}\right)$\\

        \mrowcell&\mrowcell&Abh.kegel&
        $d := 1$, $(x_0, t_0) \in \Omega_T$,
        $\forall_{|x - x_0| \le ct_0}\; u_0(x) = v_0(x) = 0$
        $\implies$ $u(x,t) = 0$ für $t \in [0, t_0]$, $|x - x_0| \le c(t_0 - t)$\\

        \cmidrule{2-4}

        \mrowcell&\multirow{3}{20mm}{inhom. ARWP}&
        Definition&
        $\Omega \subset \real^d$,
        $f\colon \Omega_T \to \real$,
        $g\colon \partial\Omega \times (0, T) \to \real$,
        $u_0, v_0\colon \Omega \to \real$\newline
        $\implies$
        $\partial_t^2 u - c^2 \Delta u = f$ in $\Omega_T$,
        $u(\cdot, 0) = u_0$ in $\Omega$,
        $\partial_t u(\cdot, 0) = v_0$ in $\Omega$,
        $u = g$ auf $\partial\Omega \times (0, T)$\\

        \mrowcell&\mrowcell&Eind.&
        $\Omega \subset \real^d$ Lipschitz
        $\implies$ es gibt höchstens eine kl. Lsg.\\

        \bottomrule
    \end{tabular}

    \pagebreak
\end{landscape}

\pagebreak
