\section{%
    Determinanten%
}

\subsection{%
    Definition der Determinante%
}

\begin{Def}{Permutation}
    Eine \begriff{Permutation} ist eine bijektive Abbildung einer Menge in
    sich.
    Die Menge aller Permutationen der Zahlen $1, \ldots, n$ bezeichnet man
    als die \begriff{symmetrische Gruppe} $\perm_n$. \\
    $\perm_n$ wird mit der Komposition zu einer Gruppe mit $n!$ Elementen. \\
    Man schreibt $\pi =$ \matrixsize{$\begin{pmatrix}1 & \cdots & n \\
    \pi(1) & \cdots & \pi(n)\end{pmatrix}$} für eine Permuation
    $\pi \in \perm_n$. \\
    Eine \begriff{Transposition} ist eine Vertauschung von zwei Zahlen. \\
    Jede Permutation kann als Verkettung von Transpositionen dargestellt
    werden. \\
    Man kann zeigen, dass die Anzahl der Transpositionen modulo $2$ für eine
    Permutation eindeutig festgelegt ist.
    Für eine \begriff{gerade bzw. ungerade Permutation} sei $\sign(\pi) = 1$
    bzw. $\sign(\pi) = -1$ das \begriff{Signum (Vorzeichen)} der Permutation.
\end{Def}

\begin{Def}{Determinante}
    Sei $A = (\alpha_{ij}) \in M_n(K)$ eine quadratische
    $n \times n$-Matrix. \\
    Dann ist die \begriff{Determinante} $\det A = |A|$ definiert als
    $\det A = \sum_{\pi \in \perm_n} \sign(\pi) \alpha_{1 \pi(1)} \cdots
    \alpha_{n \pi(n)}$.
\end{Def}

\begin{Kor}
    In jedem Summanden $T_\pi$ von $\det A$ kommt aus jeder Zeile/Spalte
    genau ein Faktor vor.
    Für jedes Produkt von Elementen aus $A$, in dem aus jeder Zeile/Spalte
    genau ein Faktor vorkommt, gibt es einen Summanden, der bis aufs
    Vorzeichen gleich ist.
\end{Kor}

\begin{Satz}{Regel von \name{Sarrus}}
    Sei $A \in M_{3 \times 3}(K)$.
    Dann erhält man $\det A$, indem man die ersten zwei Spalten rechts neben
    die Matrix schreibt, die sechs aufsteigenden und absteigenden Diagonalen
    einzeichnet, die Produkte über diese Diagonalen bildet, Produkte
    aufsteigender Diagonalen mit negativem Vorzeichen versieht und aufsummiert.
\end{Satz}

\subsection{%
    Rechenregeln%
}

\begin{Lemma}{Nullzeile/-spalte}
    Enthält eine Zeile oder Spalte von $A$ nur Nullen, so ist $\det A = 0$.
\end{Lemma}

\begin{Def}{monomiale Matrix}
    Eine quadratische Matrix heißt \begriff{monomial}, falls in jeder Zeile und
    Spalte genau ein von Null verschiedener Eintrag vorkommt. \\
    Sind diese Einträge alle $1$, so spricht man von einer
    \begriff{Permutationsmatrix}.
\end{Def}

\begin{Kor}
    Die Determinante einer monomialen Matrix ist bis aufs Vorzeichen gleich
    dem Produkt ihrer nicht-verschwindenden Einträge.
\end{Kor}

\begin{Satz}{Rechenregeln}
    Für $A, B \in M_n(K)$ gilt
    $\det A^t = \det A$ und $\det(AB) = (\det A)(\det B)$.
    Ist $A$ invertierbar, dann ist $\det (A^{-1}) = (\det A)^{-1}$.
    \qquad Außerdem ist $\det(AB) = \det(BA)$.
\end{Satz}

\begin{Satz}{Elementaroperationen}
    Addiert man zu einer Zeile/Spalte ein Vielfaches einer anderen, so ändert
    sich $\det A$ nicht.
    Vertauscht man zwei Zeilen/Spalten, so ändert sich das Vorzeichen von
    $\det A$.
    Multipliziert man eine Zeile/Spalte mit einem $\lambda$, so wird $\det A$
    mit $\lambda$ multipliziert.
\end{Satz}

\begin{Satz}{Matrix invertierbar $\Leftrightarrow \det \not= 0$} \\
    Eine quadratische Matrix $A$ ist genau dann invertierbar, wenn
    $\det A \not= 0$.
\end{Satz}

\begin{Kor}
    Für eine quadratische Matrix $A$ mit zwei identischen Zeilen/Spalten gilt
    $\det A = 0$.
\end{Kor}

\begin{Def}{Kofaktor}
    Der \begriff{Kofaktor} $A_{ij}$ vom $(i, j)$-ten Eintrag $\alpha_{ij}$ von
    $A$ ($1 \le i, j \le n$) ist die $(n-1) \times (n-1)$-Matrix,
    die man erhält, wenn man aus $A$ die $i$-te Zeile und $j$-te Spalte
    streicht.
\end{Def}

\begin{Satz}{\name{Laplace}-Entwicklung}
    Sei $A = (\alpha_{ij}) \in M_n(K)$ mit
    $k \in \{1, \ldots, n\}$.
    Dann ist \\
    $\det A = \sum_{i=1}^n (-1)^{i+k} \alpha_{ik} \det(A_{ik})$
    (Entwicklung nach der $k$-ten Spalte) bzw. \\
    $\det A = \sum_{j=1}^n (-1)^{k+j} \alpha_{kj} \det(A_{kj})$
    (Entwicklung nach der $k$-ten Zeile).
\end{Satz}

\begin{Satz}{$\det$-Abbildung}
    Die Abbildung $\det: M_n(K) \rightarrow K$, $A \mapsto \det A$
    ist multiplikativ und surjektiv.
    Daher ist $\det: \GL_n(K) \rightarrow K^\ast$ ein Gruppenepimorphismus.
\end{Satz}

\begin{Def}{spezielle lineare Gruppe}
    Der Kern von $\det$ eingeschränkt auf $\GL_n(K)$ heißt
    \begriff{spezielle lineare Gruppe} $\SL_n(K)$, d.\,h. $\SL_n(K)$ ist die
    Menge aller $n \times n$-Matrizen mit Determinante $1$.
\end{Def}

\begin{xDef}{ähnliche Matrizen}{ahnlich}
    Zwei $n \times n$-Matrizen $A$ und $B$ heißen ähnlich, falls es eine
    invertierbare $n \times n$-Matrix $P$ gibt mit $B = P^{-1} A P$.
    Man schreibt dann $A \sim B$ und $\sim$ ist Äquivalenzrelation.
\end{xDef}

\begin{Satz}{Determinante ähnlicher Matrizen}
    Für zwei ähnliche Matrizen $A \sim B$ gilt $\det A = \det B$.
\end{Satz}

\begin{Def}{Determinante von Endomorphismen}
    Sei $f \in \End_K(V)$. \\
    Dann ist die Determinante $\det f$ von $f$ definiert als $\det f = \det A$,
    wobei $A = \hommatrix{f}{B}{B}$ für eine beliebige Basis $\basis{B}$ von
    $V$ ist (laut obigem Satz ist die Determinante bei jeder Basis gleich).
\end{Def}

\subsection{%
    Eine Anwendung%
}

\begin{Def}{Adjunkte}
    Sei $A \in M_n(K)$ eine quadratische Matrix. \\
    Dann ist die \begriff{Adjunkte} von $A$ die $n \times n$-Matrix
    $\adj A =$ \matrixsize{$\begin{pmatrix}
    (-1)^{1+1} |A_{11}| & \cdots & (-1)^{n+1} |A_{n1}| \\
    \vdots & & \vdots \\
    (-1)^{1+n} |A_{1n}| & \cdots & (-1)^{n+n} |A_{nn}|
    \end{pmatrix}$}.
\end{Def}

\begin{Satz}{Adjunkte und Determinante}
    Sei $A \in M_n(K)$ eine quadratische Matrix. \\
    Dann ist $A \cdot \adj(A) = \det(A) \cdot E_n$. \qquad
    Ist $A$ invertierbar, dann ist $A^{-1} = (\det A)^{-1} \cdot \adj(A)$.
\end{Satz}

\begin{Satz}{\name{Cramer}sche Regel}
    Sei $\lgs{G}: Ax = b$ mit $A = (\alpha_{ij}) \in M_n(K)$ und
    $b = $ \matrixsize{$\begin{pmatrix}\beta_1 \\ \vdots \\
    \beta_n\end{pmatrix}$} $ \in K^n$ ein LGS.
    Zusätzlich sei $\det A \not= 0$. \\
    Dann ist \lgs{G} eindeutig lösbar und die Lösung ist
    $x = A^{-1} b = (\det A)^{-1} \cdot \adj A \cdot b$.
\end{Satz}

\subsection{%
    \emph{Zusätzliches}: Nullstellen von Polynomen%
}

\begin{Def}{Nullstelle}
    $\alpha \in K$ ist eine \begriff{Nullstelle} des Polynoms $h(t) \in K[t]$,
    falls $h(\alpha) = 0$ ist.
\end{Def}

\begin{Satz}{Polynomdivision}
    Seien $h(t) \in K[t]$ ($h(t) \not= 0$) und $\alpha$ eine Nullstelle
    von $h$. \\
    Dann gibt es ein Polynom $g(t)$ mit $\deg g = \deg h - 1$, sodass
    $h(t) = g(t) (t - \alpha)$.
\end{Satz}

\begin{Satz}{Aufspaltung in Linearfaktoren durch Polynomdivision}
    Sind $\alpha_1, \ldots, \alpha_k$ genau die Nullstellen von $h$, dann
    gibt es $\nu_1, \ldots, \nu_k \in \natural$, sodass
    $h(t) = g_1(t) (t - \alpha_1)^{\nu_1} \cdots (t - \alpha_k)^{\nu_k}$,
    wobei $g_1(t)$ ein Polynom ohne Nullstellen in $K$ ist mit
    $\deg g_1 = \deg h - \nu_1 - \cdots - \nu_k$. \\
    $m_{\alpha_i}(h(t)) := \nu_i$ heißt \begriff{Vielfachheit} der Nullstelle
    $\alpha_i$ ($1 \le i \le k$).
\end{Satz}

\begin{Kor}
    Ein Polynom vom Grad $n$ hat höchstens $n$ verschiedene Nullstellen.
\end{Kor}

\begin{Def}{algebraisch abgeschlossen}
    Ein Körper $K$ heißt \begriff{algebraisch abgeschlossen}, falls jedes
    Polynom $p \in K[x]$ mit $\deg p \ge 1$ eine Nullstelle besitzt.
\end{Def}

\begin{Fakt}{Hauptsatz der Algebra}
    $\complex$ ist \begriff{algebraisch abgeschlossen}.
\end{Fakt}

\begin{Kor}
    Jedes Polynom $p \in K[x]$ mit $\deg p \ge 1$ über einem algebraisch
    abgeschlossenen Körper $K$ ist Produkt aus Linearfaktoren.
    Ein \begriff{lineares Polynom} ist ein Polynom vom Grad $1$.
\end{Kor}

\pagebreak
