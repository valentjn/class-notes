\section{%
    Netzwerkfluss-Probleme%
}

\subsection{%
    Maximaler Fluss (\emph{MaxFlow})%
}

\subsubsection{%
    Problem%
}

\textbf{Netzwerk}:
Ein \begriff{Netzwerk} $G = (V, E, \Cap, s, t)$ ist ein gerichteter Graph $(V, E)$ zusammen mit
zwei verschiedenen, ausgezeichneten Knoten
$s \in V$ und $t \in V$ (\begriff{Quelle} und \begriff{Senke}) und
einer \begriff{Kapazitätsfunktion} $\Cap\colon E \to \natural$.

OBdA kann man annehmen, dass $s$ nur aus- und $t$ nur eingehende Kanten besitzt.

\textbf{Fluss}:
Ein \begriff{Fluss} im Netzwerk $G$ ist eine Abbildung $f\colon E \to \natural_0$ mit
\begin{itemize}
    \item
    $\forall_{e \in E}\; f(e) \le \Cap(e)$
    (\begriff{Kapazitätsbedingung}) und
    
    \item
    $\forall_{v \in V \setminus \{s, t\}}\;
    \sum_{e = (\cdot, v)} f(e) = \sum_{e = (v, \cdot)} f(e)$
    (\begriff{Flusserhaltung}).
\end{itemize}
$f$ hat den \begriff{Wert} $\val(f) := \sum_{e = (s, \cdot)} f(e)$.

\textbf{MaxFlow-Problem}:
Das \begriff{MaxFlow-Problem} lautet wie folgt.
Gegeben ist $G = (V, E, \Cap, s, t)$.\\
Gesucht ist ein Fluss von $s$ nach $t$ mit größtmöglichem Wert
(\begriff{maximaler Fluss}).

\subsubsection{%
    \name{Ford}-\name{Fulkerson}-Algorithmus%
}

\textbf{Idee}:
Wähle einen beliebigen Pfad von $s$ nach $t$ und schicke einen größtmöglichen Fluss $f$ entlang
dieses Pfads.
$f$ ist i.\,A. nicht der maximale Fluss.
Allerdings kann man $f$ sozusagen vom Graphen "`abziehen"', d.\,h. man verändert den Graph,
sodass jeder Fluss im ursprünglichen Graph gleich einem Fluss im modifizierten Graph plus $f$ ist.
Dazu führt man umgekehrte Kanten ein, damit ein Teil des Flusses $f$ wieder "`rückgängig"' gemacht
werden kann.

\textbf{Residualnetzwerk}:
Seien $G = (V, E, \Cap, s, t)$ ein Netzwerk und $f$ ein Fluss in $G$.\\
Dann ist das \begriff{Residualnetzwerk} $G_f := (V, E_f, \Cap_f, s, t)$ ein Netzwerk
mit denselben Knoten, wobei die Kanten mit ihren Kapazitäten wie folgt definiert sind:
\begin{itemize}
    \item
    Für jede Kante $e \in E$ mit $f(e) < \Cap(e)$ sei
    $e \in E_f$ mit $\Cap_f(e) := \Cap(e) - f(e)$.
    
    \item
    Für jede Kante $e = (v, w) \in E$ mit $f(e) > 0$ sei
    $e' := (w, v) \in E_f$ mit $\Cap_f(e') := f(e)$.
\end{itemize}

\linie

\textbf{\name{Ford}-\name{Fulkerson}-Algorithmus}:\\
Der \begriff{\name{Ford}-\name{Fulkerson}-Alg.} bestimmt einen maximalen Fluss in einem Netzwerk
$G$ wie folgt.
\begin{enumerate}
    \item
    Starte mit $f$ als Nullfluss.
    
    \item
    Konstruiere das Residualnetzwerk $G_f = G$.
    
    \item
    Solange es einen \begriff{augmentierenden Pfad} $\pi$ in $G_f$ von $s$ nach $t$ gibt,
    wiederhole:
    \begin{enumerate}
        \item
        Sende den \begriff{Flaschenhals-Wert} von $\pi$ von $s$ nach $t$ über $\pi$,
        addiere den Fluss auf $f$.
        
        \item
        Berechne $G_f$ neu.
    \end{enumerate}
    
    \item
    Gebe $f$ zurück.
\end{enumerate}

\textbf{Lemma (Terminiertheit)}:
Der Algorithmus terminiert stets.

\begin{Beweis}
    In jeder Runde vergrößert sich der Wert von $f$ um mindestens $1$.
    Allerdings ist der Wert begrenzt durch $\sum_{e = (s, \cdot)} \Cap(e)$.
\end{Beweis}

\linie
\pagebreak

Die Korrektheit ist etwas schwieriger zu zeigen.

\textbf{gerichteter Schnitt}:
Seien $G = (V, E, \Cap)$ ein ger. Graph mit Kapazitätsfkt. $\Cap$ und $A \subset V$.\\
Dann heißt $\dcut(A, G) := \{(v, w) \in E \;|\; v \in A,\; w \in V \setminus A\}$
von $A$ induz. \begriff{gerichteter Schnitt}.

\textbf{Lemma}:
Sei $A \subset V$ mit $s \in A$ und $t \notin A$.
Dann ist der Wert jeden Flusses von $s$ nach $t$ in $G$ nach oben beschränkt durch
$\sum_{e \in \dcut(A, G)} \Cap(e)$.

\begin{Beweis}
    Der Fluss kann nur in den Kanten in $\dcut(A, G)$ von $A$ nach $V \setminus A$
    übergehen.
\end{Beweis}

\textbf{Lemma (Korrektheit)}:
Der Algorithmus arbeitet korrekt.

\begin{Beweis}
    Sei $f$ der Fluss, der durch den Algorithmus ausgegeben wird.
    Im Folgenden wird $A \subset V$ bestimmt mit $s \in A$, $t \notin A$ und
    $\sum_{e \in \dcut(A, G)} \Cap(e) = \val(f)$.
    Weil die linke Seite eine obere Schranke für den Wert eines optimalen Flusses ist,
    muss $f$ dann maximal sein.
    
    Sei dazu $A$ die Menge der Knoten, die im letzten Residualnetzwerk $G_f$ von $s$ aus
    erreichbar sind.
    Dann gilt $s \in A$ und $t \notin A$
    (sonst hätte der Algorithmus in dem Schritt nicht terminiert).
    Betrachte nun das Ausgangsnetzwerk $G = (V, E, \Cap)$.
    \begin{itemize}
        \item
        Für jede aus $A$ ausgehende Kante $e = (v, w)$ (also $v \in A$, $w \in V \setminus A$)
        gilt $f(e) = \Cap(e)$.\\
        Sonst würde aus $f(e) < \Cap(e)$ folgen, dass $e$ auch eine Kante in $G_f$ wäre.
        Dann wäre $w$ von $s$ aus in $G_f$ erreichbar, ein Widerspruch zu $w \notin A$.
        
        \item
        Für jede in $A$ eingehende Kante $e = (v, w)$ (also $v \in V \setminus A$, $w \in A$)
        gilt $f(e) = 0$.\\
        Sonst würde aus $f(e) > 0$ folgen, dass $e' = (w, v)$ eine Kante in $G_f$ wäre.
        Dann wäre $v$ von $s$ aus in $G_f$ erreichbar, ein Widerspruch zu $v \notin A$.
    \end{itemize}
    Damit muss der Wert $\val(f)$ von $f$ gleich $\sum_{e \in \dcut(A, G)} \Cap(e)$ sein.
\end{Beweis}

Der Ford-Fulkerson-Algorithmus liefert mit dem gerichteten Schnitt $A$ aus dem Beweis von eben
ein \begriff{Zertifikat der Optimalität}, weil leicht nachvollzogen werden kann, dass alle von $A$
ausgehenden Kanten in $G$ bis zu ihrer Kapazität ausgenutzt werden
(und die eingehenden gar nicht).

\linie

\textbf{Zeitbedarf}:
Wenn der augmentierende Pfad mithilfe Breiten- oder Tiefensuche bestimmt wird,
so benötigt der Algorithmus für jede Pfadbestimmung $\O(m + n)$ Zeit
(mit $n := |V|$ und $m := |E|$).
Auch die Konstruktion eines Residualnetzwerks kostet $\O(m + n)$ Zeit.
Weil aber momentan nur bekannt ist, dass der Wert des Flusses $f$ sich in jedem Schritt nur um
mindestens $1$ vergrößert, kann die Laufzeit des Ford-Fulkerson-Algorithmus nur durch\\
$\O((m + n) \cdot (\text{optimaler Flusswert})) \subset \O(C(m + n))$
mit $C := \sum_{e = (s, \cdot)} \Cap(e)$ beschränkt werden.
Das kann sehr groß sein, insbesondere ist der optimale Flusswert i.\,A. nicht polynomiell in der
Eingabelänge.

\pagebreak

\subsubsection{%
    \emph{Capacity Scaling}%
}

\textbf{Idee}:
Suche zunächst augmentierende Pfade mit einem hohen "`Flaschenhals-Wert"'.
Es wenn es solche Pfade nicht mehr gibt, erlaube auch Pfade mit einem kleineren Wert.

\textbf{\name{Ford}-\name{Fulkerson} mit Capacity Scaling}:\\
Der Ford-Fulkerson-Algorithmus mit \begriff{Capacity Scaling} sieht wie folgt aus.
\begin{enumerate}
    \item
    Starte mit $f$ als Nullfluss.
    
    \item
    Wähle $D$ als die größte Zweierpotenz kleiner als $C := \max_{e \in E} \Cap(e)$
    (d.\,h. $D := 2^{\lfloor \log_2 C \rfloor}$).
    
    \item
    Wiederhole Folgendes, solange $D \ge 1$ ist:
    \begin{enumerate}
        \item
        Konstruiere das Residualnetzwerk $G_f(D)$ von $G$ bzgl. $f$,
        eingeschränkt auf die Kanten mit Kapazität $\ge D$.
        
        \item
        Solange es einen Pfad $\pi$ in $G_f(D)$ von $s$ nach $t$ gibt, wiederhole:
        \begin{enumerate}
            \item
            Augmentiere $f$ um den maximalen Fluss entlang $\pi$.
            
            \item
            Berechne $G_f(D)$ neu.
        \end{enumerate}
        
        \item
        Halbiere $D$.
    \end{enumerate}
    
    \item
    Gebe $f$ zurück.
\end{enumerate}

\linie

Die äußere Schleife wird $(\log D)$-mal ausgeführt.
Für die innere Schleife muss etwas bewiesen werden.

\textbf{Lemma}:
Sei $f_D$ der Fluss aus dem Algorithmus, nachdem alle Augmentierungen mit Kapazitätsschranke $D$
durchgeführt wurden.\\
Dann ist der maximale Flusswert in $G$ beschränkt durch $\val(f_D) + mD$.

\begin{Beweis}
    Betrachte das Residualnetzwerk $G_{f_D}$
    nach der letzten Augmentierung mit Kapazitätsschranke $D$
    (mit allen Kanten, d.\,h. auch mit denen mit Kapazität $< D$).
    Dann induziert die Menge $A$ aller Knoten, die von $s$ über einen Pfad,
    der nur Kanten von Kapazität $\ge D$ enthält, erreichbar sind, einen gerichteten Schnitt
    $\dcut(A, G)$, der $s$ von $t$ trennt.
    Sei $c$ die Kapazität des gerichteten Schnitts.
    Zieht man von den Kapazitäten der Kanten aus $\dcut(A, G)$ den Fluss $f_D$ ab,
    so haben diese Kanten jeweils eine Kapazität $< D$,
    d.\,h. $c - \val(f_D) < mD$.
    Weil $c$ eine obere Schranke für den maximalen Flusswert ist, folgt die Behauptung.
%     Alle von $A$ ausgehenden Kanten haben Kapazität $< D$,
%     d.\,h. die Kapazität des gerichteten Schnitts ist $< mD$.
%     Der Wert eines Flusses von $s$ nach $t$ in $G$ ist beschränkt durch die Summe aus
%     dem Wert des aktuellen Flusses $f_D$ und der Kapazität des gerichteten Schnitts.
\end{Beweis}

\textbf{Lemma}:
Für festes $D$ wird die innere Schleife höchstens $2m$-mal durchlaufen.

\begin{Beweis}
    Beginnt man für festes $D$ den ersten Durchgang der inneren Schleife mit dem Fluss $f$,
    dann ist der maximale Flusswert von $G$ nach dem Lemma von eben $\le \val(f) + 2Dm$
    (wegen der vorherigen Runde der äußeren Schleife mit Kapazitätsschranke $2D$).
    In jeder Runde der inneren Schleife erhöht sich der Flusswert um mindestens $D$,
    d.\,h. es kann höchstens $2m$-viele Runden geben.
\end{Beweis}

\textbf{Zeitbedarf}:
FF mit Capacity Scaling benötigt die Laufzeit $\O((m + n)m \cdot \log C) = \O(m^2 \log C)$,
wobei $C := \max_{e \in E} \Cap(e)$.

\begin{Beweis}
    Die äußere Schleife wird $(\log D)$-mal durchlaufen, wobei $\O(\log C) = \O(\log D)$.
    Die innere Schleife wird $\O(m)$-mal durchlaufen und jede Pfadberechnung kostet
    $\O(m + n)$ Zeit.
\end{Beweis}

Die Laufzeitschranke $\O(m^2 \log C)$ ist zwar polynomiell in der Eingabelänge, allerdings hängt
sie immer noch von den Kapazitätszahlen ab, was man gerne vermeiden würde.

\pagebreak

\subsubsection{%
    \name{Edmonds}-\name{Karp}-Algorithmus%
}

\textbf{\name{Edmonds}-\name{Karp}-Algorithmus}:
Der \begriff{\name{Edmonds}-\name{Karp}-Algorithmus} verwendet
wieder den ursprünglichen FF-Algorithmus, nur mit der Maßgabe, dass
immer der kürzeste Pfad (im Sinne von Anzahl der verwendeten Kanten) als augmentierender Pfad
verwendet werden soll.
Ein solcher Pfad kann mit Breitensuche ebenfalls in $\O(m + n)$ Zeit gefunden werden.

\linie

\textbf{Lemma}:
Während des Verlaufs des Algorithmus verringert sich die Länge der augmentierenden Pfade nie.

\begin{Beweis}
    Seien $\ell(v)$ die Distanz von $s$ zu $v \in V$ im Residualnetzwerk und
    $G_\ell$ der Teilgraph des Residualnetzwerks, der die Kanten $(u, v)$ mit
    $\ell(v) = \ell(u) + 1$ enthält.
    Ein Pfad $\pi$ zwischen $s$ und einem Knoten $v$ im Residualnetzwerk ist am kürzesten
    genau dann, wenn $\pi$ auch ein Pfad in $G_\ell$ ist.
    Während der Augmentierung von $f$ entlang eines Pfades $\pi$ können prinzipiell zwei Ereignisse
    auftreten:
    \begin{itemize}
        \item
        Kanten im Residualnetzwerk können verschwinden (wegen voller Kapazität) oder
        
        \item
        Rückwärtskanten, die vorher noch nicht da waren, können erstellt werden.
    \end{itemize}
    In beiden Fällen verringert sich die Distanz von $s$ zu den Knoten nicht,
    was insb. für $t$ gilt.
\end{Beweis}

\textbf{Lemma}:
Nach höchstens $\O(m)$ Augmentierungen erhöht sich die Länge des augmentierenden Pfads um
mindestens $1$.

\begin{Beweis}
    Sei $E_k$ die Menge aller Kanten im Residualnetzwerk am Anfang einer "`Phase"', bei der
    die Distanz zwischen $s$ und $t$ genau $k$ beträgt.
    Sobald der kürzeste Pfad von $s$ nach $t$ eine Kante benutzt, die nicht in $E_k$ liegt,
    hat der Pfad eine Länge $> k$.
    Weil mit jeder Augmentierung mindestens eine Kante (die Flaschenhals-Kante) aus $E_k$
    eliminiert wird, muss sich die Länge des kürzesten Pfads von $s$ nach $t$ nach höchstens
    $|E_k| = \O(m)$ Schritten vergrößern.
\end{Beweis}

\textbf{Zeitbedarf}:
Der Edmonds-Karp-Algorithmus terminiert nach $\O(mn)$ Schritten.\\
Damit hat der Algorithmus einen Zeitbedarf von $\O((m+n)mn) = \O(m^2 n)$.

\begin{Beweis}
    Jeweils nach $\O(m)$ Runden erhöht sich nach dem letzten Lemma die Länge des augmentierenden
    Pfads um mindestens $1$.
    Weil jeder Pfad im Residualnetzwerk höchstens $\O(n)$ beteiligte Knoten haben kann,
    geht das höchstens $\O(n)$-mal.
\end{Beweis}

\linie

\textbf{nicht-ganzzahlige Kapazitäten}:
Bisher wurde immer angenommen, dass die Kapazitäten des Netzwerks ganzzahlig sind.
Wenn man allgemein nur reelle Zahlen voraussetzt, muss der FF-Algorithmus nicht terminieren.
Es gibt sogar einfache Beispiele, bei denen der FF-Algorithmus nicht einmal gegen einen
maximalen Fluss konvergiert.

\pagebreak

\subsection{%
    Fluss minimaler Kosten (\emph{MinCostFlow})%
}

\subsubsection{%
    Problem%
}

\textbf{MinCostFlow-Problem}:
Das \begriff{MinCostFlow-Problem} ist eine Verallgemeinerung des Max"-Flow-Problems.
Gegeben ist ein \begriff{erweitertes Netzwerk} $G = (V, E, b, c, \Cap)$ mit
\begin{itemize}
    \item
    \begriff{Überschussfunktion} $b\colon V \to \integer$,
    
    \item
    \begriff{Kostenfunktion} $c\colon E \to \integer$ und
    
    \item
    \begriff{Kapazitätsfunktion} $\Cap\colon E \to \natural$,
\end{itemize}
wobei $\sum_{v \in V} b(v) = 0$ gelten soll (\begriff{Gesamtüberschuss gleich Gesamtbedarf}).\\
Gesucht ist wie beim MaxFlow-Problem ein \begriff{zulässiger Fluss} $f\colon E \to \natural_0$,
d.\,h.
\begin{itemize}
    \item
    $\forall_{e \in E}\; f(e) \le \Cap(e)$
    (\begriff{Kapazitätsbedingung}) und
    
    \item
    $\forall_{v \in V}\;
    b(v) + \sum_{e = (\cdot, v)} f(e) = \sum_{e = (v, \cdot)} f(e)$
    (\begriff{Flusserhaltung}),
\end{itemize}
sodass $\sum_{e \in E} f(e)c(e)$ minimiert wird (\begriff{Fluss minimaler Kosten}).

\linie

\textbf{Berechnung eines zulässigen Flusses}:
Zur Berechnung eines zulässigen Flusses erstellt man das Netzwerk $G' = (V', E', \Cap', s, t)$
mit
\begin{itemize}
    \item
    $V' := V \cup \{s, t\}$
    (wobei $s \notin V$ die \begriff{Superquelle} und $t \notin V$ die \begriff{Supersenke} ist),
    
    \item
    $E' := E \cup \{(s, v) \;|\; v \in V,\; b(v) > 0\} \cup \{(w, t) \;|\; w \in V,\; b(w) < 0\}$
    sowie
    
    \item
    $\Cap'(e) := e$ für $e \in E$, $\Cap'((s, v)) := b(v)$ und $\Cap'((w, t)) := -b(w)$.
\end{itemize}
Dann gibt es in $G$ einen zulässigen Fluss genau dann, wenn der maximale Flusswert in $G'$
gleich $\sum_{v \in V,\; b(v) > 0} b(v)$ ist.
Durch Berechnung eines maximalen Flusses in $G'$ mit dem FF-Algorithmus erhält man so einen
zulässigen Fluss in $G$ (nach Einschränkung von $E$).

\linie

Zur Berechnung eines Flusses minimaler Kosten wird wieder das Residualnetzwerk definiert.

\textbf{Residualnetzwerk}:
Seien $G = (V, E, b, c, \Cap)$ ein Netzwerk und $f$ ein Fluss in $G$.\\
Dann ist das \begriff{Residualnetzwerk} $G_f := (V, E_f, b, c_f, \Cap_f)$ ein Netzwerk
mit denselben Knoten, wobei die Kanten mit ihren Kapazitäten wie folgt definiert sind:
\begin{itemize}
    \item
    Für jede Kante $e \in E$ mit $f(e) < \Cap(e)$ sei $e \in E_f$ mit\\
    $\Cap_f(e) := \Cap(e) - f(e)$ und $c_f(e) := c(e)$.
    
    \item
    Für jede Kante $e = (v, w) \in E$ mit $f(e) > 0$ sei $e' := (w, v) \in E_f$ mit\\
    $\Cap_f(e') := f(e)$ und $c_f(e) := -c(e)$.
\end{itemize}

\pagebreak

\subsubsection{%
    \emph{Cycle Canceling}%
}

\textbf{negativer Kreis}:
Ein \begriff{negativer Kreis} ist ein Pfad $(v_0, \dotsc, v_k)$ mit
$k \in \natural$,
$v_0 = v_k$, $v_i \not= v_j$ für $i, j \ge 1$ und $\sum_{i=1}^k c((v_{i-1}, v_i)) < 0$.

%\textbf{Idee}:
%Während der FF-Algorithmus nach augmentierenden $s$-$t$-Pfaden im Residualnetzwerk gesucht hat,
%wird nun nach \begriff{negativen Kreisen} gesucht, d.\,h. Pfaden $(v_0, \dotsc, v_k)$ mit
%$v_0 = v_k$, $v_i \not= v_j$ für $i, j \ge 1$ und $\sum_{i=1}^k c((v_{i-1}, v_i)) < 0$.
%Wird ein solcher negativer Kreis gefunden, dann wird 

\textbf{Cycle-Canceling-Algorithmus}:
Der \begriff{Cycle-Canceling-Algorithmus}
bestimmt einen Fluss minimaler Kosten in einem Netzwerk $G$ wie folgt.
\begin{enumerate}
    \item
    Berechne einen zulässigen Fluss $f$ wie oben.
    
    \item
    Konstruiere das Residualnetzwerk $G_f$.
    
    \item
    Solange es einen negativen Kreis $\pi$ in $G_f$ gibt, wiederhole:
    \begin{enumerate}
        \item
        Sende den Flaschenhals-Wert über $\pi$ und addiere den Fluss auf $f$.
        
        \item
        Berechne $G_f$ neu.
    \end{enumerate}
    
    \item
    Gebe $f$ zurück.
\end{enumerate}

\linie

\textbf{Lemma (Terminiertheit)}:
Der Algorithmus terminiert stets.

\begin{Beweis}
    Die Kosten jedes zulässigen Flusses $f$ sind nach unten durch
    $\sum_{e \in E,\; c(e) < 0} \Cap(e) c(e)$ beschränkt.
    Weil die Kosten von $f$ in jeder Runde um mindestens $1$ verringert werden,
    terminiert der Algorithmus nach endlich vielen Runden.
\end{Beweis}

\linie

\textbf{Satz (Korrektheit)}:
Sei $f$ ein zulässiger Fluss in $G$.\\
Dann ist $f$ ein Fluss minimaler Kosten genau dann, wenn $G_f$ keinen negativen Kreis enthält.

\begin{Beweis}
    "`$\implies$"':
    Angenommen, $f$ besitzt minimale Kosten, aber $G_f$ enthält einen negativen Kreis.
    Dann sende mindestens $1$ Fluss-Einheit entlang dieses Kreises, was die Kosten von $f$
    um mindestens $1$ reduziert, ein Widerspruch zur Minimalität von $f$.
    
    "`$\impliedby$"':
    Angenommen, $G_f$ enthält keinen negativen Kreis, aber $f$ besitzt nicht minimale Kosten.
    Dann gibt es einen zulässigen Fluss $f^\ast$ mit $c(f^\ast) < c(f)$.
    Betrachte nun die Flussdif"|ferenz $f' := f^\ast - f$.
    Dann gilt $c(f') < 0$ und an jedem Knoten $v \in V$ ist der eingehende Fluss
    genauso groß wie der ausgehende Fluss.
    Daher kann man $f'$ in eine Menge $\C$ von Zykeln zerlegen.
    Es gilt $\sum_{C \in \C} c(C) = c(f') < 0$,
    d.\,h. es gibt ein $C_0 \in \C$ mit $c(C_0) < 0$.
    Man kann sich überlegen, dass $C_0$ ebenfalls ein negativer Kreis in $G_f$ ist,
    was aber ein Widerspruch zur Annahme ist, dass $G_f$ keine negativen Kreise enthält.
\end{Beweis}

\linie

\textbf{Erkennung von negativen Kreisen}:
Gegeben sei ein Graph $G = (V, E, c)$ mit Kostenfunktion $c\colon E \to \integer$
($c(e) < 0$ ausdrücklich möglich) mit $n := |V|$ und $m := |E|$.
Um negative Kreise zu finden, kann man einen \begriff{geschichteten Graphen} mit $n$ Schichten
erstellen, wobei jede Schicht eine Kopie von $V$ enthält und es
eine Kante zwischen $v$ in Schicht $i$ und $w$ in Schicht $i+1$ gibt genau dann, wenn
$(v, w) \in E$ (wobei die Kantenkosten identisch seien).
Nun ist ein Knoten $v$ Teil eines negativen Kreises genau dann, wenn
es einen Pfad mit negativen Kosten von $v$ in der ersten Schicht zu $v$ in einer anderen Schicht
gibt.

\textbf{Komplexität}:
Der geschichtete Graph hat die Größe $\O(nm)$.
Der naive Weg, um negative Kreise zu finden (Distanzen mit kürzesten Pfaden von jedem Knoten in
der ersten Schicht zu sich selbst in einer anderen Schicht in $\O(nm)$), kostet Zeit $\O(n^2 m)$.
Besser ist der \begriff{\name{Bellman}-\name{Ford}\-Algorithmus}, der in $\O(nm)$ läuft.
Es gibt aber auch Algorithmen, die in Polynomialzeit laufen und einen negativen Kreis
$C$ zurückgeben, der $\frac{c(C)}{|C|}$ minimiert (mit $|C|$ der Pfadlänge).

\pagebreak

\subsubsection{%
    \emph{Successive Shortest Paths}%
}

\textbf{Idee}:
\begin{itemize}
    \item
    Der Cycle-Canceling-Algorithmus startet mit einem zulässigen Fluss
    (nicht kostenoptimal) und verkleinert dann die Kosten, währenddessen die Zulässigkeit erhalten
    bleibt.
    
    \item
    Der SSP-Algorithmus startet mit dem Nullfluss und erhöht schrittweise
    den Fluss, um Knoten-Überschuss/-Nachfrage zu decken,
    während die Kostenoptimalität erhalten bleibt.
\end{itemize}

\linie

\textbf{Successive-Shortest-Paths-Algorithmus}:
Der \begriff{Successive-Shortest-Paths-Alg.} bestimmt ei"-nen Fluss minimaler Kosten in einem
Netzwerk $G$ ohne negative Kreise wie folgt.
\begin{enumerate}
    \item
    Starte mit $f$ als Nullfluss und
    konstruiere das Residualnetzwerk $G_f = G$.
        
    \item
    Wiederhole Folgendes:
    \begin{enumerate}
        \item
        Berechne einen kürzesten Pfad $\pi$ (bzgl. Kosten) im Residualnetzwerk $G_f$
        zwischen Knoten $v, w \in V$ mit $b(v) > 0$ und $b(w) < 0$.\\
        Gibt es keinen solchen Pfad, dann ist $\forall_{v \in V}\; b(v) = 0$
        und gebe $f$ zurück.
        
        \item
        Sende so viel wie möglich von $v$ nach $w$
        (d.\,h. $\min\{\text{Flaschenhals}(\pi), b(v), -b(w)\}$).
        
        \item
        Aktualisiere $b(v), b(w), f$ und berechne $G_f$ neu.
    \end{enumerate}
\end{enumerate}
Die in jedem Schritt zu berechnenden Knoten $v, w$ können durch eine einzige
Kürzester-Pfad-Operation identifiziert werden, indem man eine Superquelle $s$ einführt,
sie mit Kanten der Kapazität $\infty$ und Kosten $0$ mit allen Überschussknoten
(d.\,h. $v \in V$ mit $b(v) > 0$) verbindet
und analog eine Supersenke $t$ einführt.

\linie

\textbf{\name{Johnson}-Lemma}:
Sei $G = (V, E, c)$ ein gerichteter Graph mit möglicherweise negativer Kostenfunktion
$c\colon E \to \integer$, der keine negativen Kreise besitzt.
Dann gibt es eine \begriff{Potentialfunktion} $\phi\colon V \to \integer$,
sodass für $c'\colon E \to \integer$, $c'(v, w) := c(v, w) + \phi(v) - \phi(w)$ gilt, dass
$\forall_{e \in E}\; c'(e) \ge 0$ und ein Pfad $\pi$ ist am kürzesten bzgl. $c$ $\iff$
$\pi$ ist am kürzesten bzgl. $c'$.

\begin{Beweis}
    OBdA gebe es einen Knoten $s \in V$, von dem aus alle anderen Knoten erreichbar sind.
    Sei $d_s(v) \in \integer$ für $v \in V$ die Distanz des kürzesten Pfads von $s$ nach $v$.
    $d_s(v)$ ist wohldefiniert, weil $G$ keine negativen Kreise enthält,
    und kann z.\,B. mit Bellman-Ford in $\O(mn)$ Zeit berechnet werden.
    Definiere $\phi(v) := d_s(v)$.
    Für $(v, w) \in E$ ist dann $d_s(w) \le d_s(v) + c(v, w)$, also\\
    $c'(v, w) := c(v, w) + \phi(v) - \phi(w) \ge 0$.
    Sei nun ein Pfad $\pi = v_0 \dotsb v_k$ in $G$ gegeben.
    Die Kosten von $\pi$ bzgl. $c'$ sind gleich
    $\sum_{i=0}^{k-1} c'(v_i, v_{i+1})
    = \sum_{i=0}^{k-1} (c(v_i, v_{i+1}) + \phi(v_i) - \phi(v_{i+1}))$\\
    $= \sum_{i=0}^{k-1} c(v_i, v_{i+1}) + \phi(v_0) - \phi(v_k)$,
    wobei der erste Summand die Kosten von $\pi$ bzgl. $c$ darstellt.
    Damit ist $\pi$ am kürzesten bzgl. $c$ ist genau dann,
    wenn $\pi$ am kürzesten bzgl. $c'$ ist.
\end{Beweis}

\linie

\textbf{Satz (Korrektheit)}:
Jedes Residualnetzwerk ist frei von negativen Kreisen.

\begin{Beweis}
    Betrachte die erste Augmentierung nach dem Nullfluss.
    Im entsprechenden Residualnetzwerk $G_f$ können sich Kanten mit negativen Kosten nur auf dem
    Augmentierungspfad $\pi$ befinden.
    Angenommen, $G_f$ besitzt einen negativen Kreis $C$,
    dann muss $C$ auch ein paar Rückwärtskanten von $\pi$ benutzen.
    $C$ lässt sich daher zerlegen in einen Pfad $C_1$,
    der mit einer Kante mit negativen Kosten beginnt und endet,
    und einen Pfad $C_2$, der nur Kanten mit nicht-negativen Kosten enthält.
    OBdA enthalte $C_1$ nur Rückwärtskanten von $v$ nach $w$ ($\ast$).
    Es gilt $c(C_1) + c(C_2) = c(C) < 0$ und $c(C_2) > 0$.
    $C_2$ liefert einen Pfad von $w$ nach $v$ mit Kosten $0 < c(C_2) < -c(C_1)$,
    was aber dem widerspricht, dass $\pi$ ein kürzester Pfad war.
    
    Für die späteren Augmentierungen können Kanten mit negativen Kosten überall in $G_f$ verteilt
    sein.
    Daher wendet man das Johnson-Lemma nach jeder Augmentierung an.
\end{Beweis}

\linie
\pagebreak

\textbf{Begründung für ($\ast$)}:
Angenommen, $C_2$ geht von $v_6$ nach $v_0$.
\begin{itemize}
    \item
    Wenn $C_1$ den augmentierenden Pfad in $v_1$ vor $v_0$ verlässt, aber in $v_5$ nach $v_6$
    wieder betritt und dann nach $v_6$ läuft, dann kann $C_1$ einfach durch
    $v_0 \rightsquigarrow v_1 \rightsquigarrow v_5 \rightsquigarrow v_6$
    ersetzt werden (Kosten kleiner als $c(C_2) < 0$),
    wobei "`$v_1 \rightsquigarrow v_5$"' den Teil auf dem augm. Pfad meint.
    
    \item
    Wenn $C_1$ in $v_1$ vor $v_0$ den augmentierenden Pfad verlässt, dann aber in $v_3$ vor $v_6$
    wieder betritt, in $v_4$ vor $v_3$ wieder verlässt, in $v_5$ nach $v_6$ wieder betritt und
    dann nach $v_6$ läuft, muss man etwas argumentieren.
    Es gilt $c(v_3 \rightsquigarrow v_4), c(v_5 \rightsquigarrow v_6),
    c(v_0 \rightsquigarrow v_1) < 0$ und
    $c(v_1 \rightsquigarrow v_3), c(v_4 \rightsquigarrow v_5), c(v_6 \rightsquigarrow v_0) > 0$.
    Dann folgt $|c(v_5 \rightsquigarrow v_6)| + |c(v_3 \rightsquigarrow v_4)| \le
    c(v_4 \rightsquigarrow v_5)$, weil
    $v_4 \rightsquigarrow v_3 \rightsquigarrow v_6 \rightsquigarrow v_5$ der kürzeste Pfad von
    $v_4$ nach $v_5$ war (Teilpfade des augmentierenden Pfads)
    und $v_5 \rightsquigarrow v_6$ und $v_3 \rightsquigarrow v_4$ Rückwärts-Teilpfade von
    $v_4 \rightsquigarrow v_3 \rightsquigarrow v_6 \rightsquigarrow v_5$ sind.
    Analog gilt $|c(v_0 \rightsquigarrow v_1)| \le c(v_6 \rightsquigarrow v_0)$,
    weil $v_6 \rightsquigarrow v_5 \rightsquigarrow v_1 \rightsquigarrow v_0$ der kürzeste Pfad von
    $v_6$ nach $v_0$ war.
    Daraus folgt $c(C) =
    (-|c(v_5 \rightsquigarrow v_6)| - |c(v_3 \rightsquigarrow v_4)| +
    c(v_4 \rightsquigarrow v_5))$\\
    $+\; (-|c(v_0 \rightsquigarrow v_1)| + c(v_6 \rightsquigarrow v_0)) + c(v_1 \rightsquigarrow v_3)
    \ge c(v_1 \rightsquigarrow v_3) \ge 0$,
    ein Widerspruch zu $C$ negativer Kreis.
    
    \item
    Die anderen Fälle gehen ähnlich.
\end{itemize}

\subsection{%
    Anwendungen der Netzwerkfluss-Berechnung%
}

\textbf{kürzester Pfad zwischen zwei Knoten}:
Gegeben sei ein Graph $G = (V, E, c)$ mit Kosten $c$ und
zwei Knoten $s, t \in V$ mit $s \not= t$.
Gesucht ist der bzgl. $c$ kürzeste Pfad von $s$ nach $t$.

\textbf{Lösung}:
Setze $b(s) := 1$, $b(t) := -1$, $\forall_{v \not= s, t}\; b(v) := 0$ und
$\forall_{e \in E}\; \Cap(e) := 1$.

\linie

\textbf{Transport-Problem}:
Gegeben seien $m$ Einrichtungen $f_1, \dotsc, f_m$, die jeweils $s_i$ Einheiten einer Ware
anbieten.
$n$ Kunden $u_1, \dotsc, u_n$ haben jeweils einen Bedarf an $d_j$ Einheiten der Ware,
wobei $\sum_{i=1}^m s_i = \sum_{j=1}^n d_j$ gelten soll.
Das Versenden einer Einheit der Ware von $f_i$ nach $u_j$ kostet $c_{i,j}$.
Gesucht ist eine Verteilung der Ware mit minimalen Kosten, sodass alle Wünsche der Kunden
erfüllt sind.

\textbf{Lösung}:
Erstelle einen vollständigen bipartiten Graph, d.\,h. Knotenmenge\\
$V := \{f_1, \dotsc, f_m\} \cup \{u_1, \dotsc, u_n\}$
und Kantenmenge $E := \{f_1, \dotsc, f_m\} \times \{u_1, \dotsc, u_n\}$,\\
wobei $b(f_i) := s_i$ und $b(u_j) := -d_j$ sowie
$\Cap(f_i, u_j) := \infty$ und $c(f_i, u_j) := c_{i,j}$.

\textbf{Spezialfall}:
Für $n = m$ und $s_i := d_j := 1$ für alle $i, j = 1, \dotsc, n$ erhält man das
\begriff{Zuweisungspro"-blem}.
Beispielsweise gibt es $n$ Arbeitsplätze und $n$ Arbeiter,
wobei Arbeiter $i$ an Arbeitsplatz $j$ zu den Kosten $c_{i,j}$ arbeiten kann.
Obige Lösung liefert dann eine Eins-zu-Eins-Zuweisung der Arbeiter auf die Arbeitsplätze
mit minimalen Kosten.

\linie

\textbf{Airplane-Hopping-Problem}:
Ein Flugzeug fliegt eine feste Route mit $n$ Zwischenhalten $v_1, \dotsc, v_n$
und kann höchstens $p$ Passagiere tragen.
Es gibt $t_{i,j}$ Passagiere, die von $v_i$ nach $v_j$ reisen wollen
und dafür $f_{i,j}$ ausgeben (wobei $i < j$).
Gesucht sind die Anteile an den $t_{i,j}$-vielen Passagieren, die die Fluglinie jeweils mitnehmen
soll, um ihren Gewinn zu maximieren, ohne jemals mehr als $p$ Passagiere mitfliegen zu lassen.

\textbf{Lösung}:
Führe Knoten $v_i$ und $w_{i,j}$ (für $i = 1, \dotsc, n$ und $i < j$),
wobei $w_{i,j}$ die Passagiere darstellt, die von $v_i$ nach $v_j$ reisen wollen.
Setze $b(v_j) :=-\sum_{i<j} t_{i,j}$ und $b(w_{i,j}) := t_{i,j}$.\\
Verbinde $v_i, v_{i+1}$ durch eine Kante mit Kapazität $p$ und Kosten $0$.
Erstelle zudem\\
Kanten $(w_{i,j}, v_i)$ mit $\Cap(w_{i,j}, v_i) := \infty$ und
$c(w_{i,j}, v_i) := -f_{i,j}$ sowie\\
Kanten $(w_{i,j}, v_j)$ mit $\Cap(w_{i,j}, v_j) := \infty$ und $c(w_{i,j}, v_j) := 0$.

\pagebreak
