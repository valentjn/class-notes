\section{%
    Nicht-LP-basierte Approximationen%
}

\subsection{%
    Lokale Suche für UFL%
}

\textbf{lokale Suche}:
\begin{enumerate}
    \item
    Starte mit einer zulässigen Lösung
    (bei UFL öffne ein Lager und verbinde alle Kunden mit dem Lager).
    
    \item
    Führe lokale Operationen solange aus, bis sich die Lösung nicht mehr verbessert.
    Bei UFL gibt es die folgenden Operationen:
    \begin{itemize}
        \item 
        ADD:
        Öffne ein neues Lager $i^\ast$ und
        verbinde alle Kunden mit $i^\ast$, für die $i^\ast$ am billigsten ist
        (nur, wenn die Ersparnisse die Öffnungskosten von $i^\ast$ überwiegen).
        
        \item
        SWAP:
        Öffne ein neues Lager $i^\ast$ und
        schließe alle Lager $i$ mit\\
        $f_i + \sum_{j \in D,\; \text{$j$ mit $i$ verbunden}} (c_{i,j} - c_{i^\ast,j}) > 0$\\
        (nur, wenn $\sum_{\text{$i$ zu schließen}}
        (f_i + \sum_{j \in D,\; \text{$j$ mit $i$ verbunden}} (c_{i,j} - c_{i^\ast,j})) >
        f_{i^\ast}$).
        
        \item
        DELETE:
        Lösche ein Lager und verbinde die verwaisten Kunden mit den nächstgelegenen Lagern.
    \end{itemize}
\end{enumerate}

\linie

Seien $F, F^\ast$ die Öffnungskosten der aktuellen/optimalen Lösung und
$C, C^\ast$ die Verbindungskosten der aktuellen/optimalen Lösung.

\textbf{Lemma 1}:
Wenn es keine verbessernde ADD-Operation mehr gibt, dann gilt
$C \le F^\ast + C^\ast$.

\textbf{Lemma 2}:
Wenn es keine verbessernde SWAP/DELETE-Operationen mehr gibt, dann gilt
$F \le F^\ast + C^\ast + C$.

\textbf{Satz ($3$-Approximation)}:
Es gilt $F + C \le 3(F^\ast + C^\ast)$.

\begin{Beweis}
    Aus den Lemmas folgt
    $F + C
    \le (F^\ast + C^\ast + C) + C
    = F^\ast + C^\ast + 2C
    \le 3(F^\ast + C^\ast)$.
\end{Beweis}

\linie

Der Algorithmus muss in nicht in Polynomialzeit laufen, weil die lokale
Verbesserung in jedem Schritt sehr klein sein kann.
Man kann diesen Algorithmus aber in einen Polynomialzeit-Algorithmus umwandeln,
wenn man nur solche Operationen durchführt,
die den Zielfunktionswert um mindestens einen Faktor von $1 + \alpha$ mit $\alpha > 0$
fest vergrößern.

Besitzt nämlich die Startlösung den Zielfunktionswert $S_0$ und die optimale Lösung den Wert
$S_\ast$,
so macht der Algorithmus $k$ Iterationen mit
$S_0 (1+\alpha)^k = S_\ast \iff k = \frac{\log(S_\ast/S_0)}{\log(1 + \alpha)}$,
was polynomiell ist.

Man kann zeigen, dass man mit diesem Algorithmus eine $3(1 + \alpha)$-Approximation erhält.

\pagebreak

\subsection{%
    \emph{Precedence Constraint Scheduling}%
}

\subsubsection{%
    Problem%
}

\textbf{Precedence Constraint Scheduling}:\\
Beim \begriff{PCS-Problem} ist
ein gerichteter azyklischer Graph (DAG) $G = (V, E)$ und $m \in \natural$ gegeben.\\
Gesucht ist eine Funktion $\phi\colon V \to \natural$ mit
$\forall_{i \in \natural}\; |\phi^{-1}(i)| \le m$ und
$\forall_{e = (v,w) \in E}\; \phi(v) < \phi(w)$, sodass $\max_{v \in V} \phi(v)$ minimal wird.
$\phi$ heißt \begriff{Schedule} und $\max_{v \in V} \phi(v)$ heißt \begriff{Länge} von $\phi$.\\
Das PCS-Problem ist NP-vollständig.

\textbf{Interpretation}:
Die Knoten des Graphen stellen Teilprojekte eines zu absolvierenden Projekts dar,
wobei die Kanten Abhängigkeiten zwischen den Teilprojekten angeben
($w$ muss nach $v$ begonnen werden, falls $(v, w) \in E$).
Jedes Projekt dauert $1$ Zeitschritt,
wobei $m$ Arbeiter/Maschinen zur Verfügung stehen,
d.\,h. niemals dürfen mehr als $m$ Teilprojekte gleichzeitig bearbeitet werden.
$\phi(v)$ gibt nun an, zu welchem Zeitschritt das Teilprojekt $v$ bearbeitet wird.

\subsubsection{%
    Algorithmus%
}

\textbf{Algorithmus}:
Angenommen, es gibt einen Knoten $s_0 \in V$ (\begriff{Startknoten})
ohne eingehende Kanten, aber mit ausgehenden Kanten zu jedem anderen Knoten.
\begin{enumerate}
    \item
    Berechne für jedes $v \in V$ ein \begriff{Distanz-Label} $d(v)$ als
    die Länge des längsten Pfads von $s_0$ nach $v$ (geht, da $G$ azyklisch).
    Es muss $\phi(v) \ge d(v)$ gelten.
    
    \item
    Sei $d_i := |\{v \in V \;|\; d(v) = i\}|$.
    Führe zunächst alle $v \in V$ mit $d(v) = 1$ aus
    (benötigt $\lceil\frac{d_1}{m}\rceil$ Zeitschritte),
    anschließend alle $v \in V$ mit $d(v) = 2$
    (benötigt $\lceil\frac{d_2}{m}\rceil$ Zeitschritte) usw.
\end{enumerate}

\textbf{Satz ($2$-Approximation)}:
Der Algorithmus produziert eine $2$-Approximation.

\begin{Beweis}
    Sei $t := \max_{v \in V} d(v)$ und $n := |V|$.
    Dann ist die Länge $L := \max_{v \in V} \phi(v)$ des
    vom Algorithmus erzeugten Schedules gegeben durch
    $L
    = \sum_{i=1}^t \lceil\frac{d_i}{m}\rceil
    < \sum_{i=1}^t (\frac{d_i}{m} + 1)
    = \frac{n}{m} + t
    \le 2L_\opt$
    mit $L_\opt$ der optimalen Schedulelänge
    ($\frac{n}{m} \le L_\opt$, weil $\frac{n}{m}$ die Länge wäre,
    wenn $m$ Arbeiter ununterbrochen arbeiten würden, und
    $t \le L_\opt$, weil $\forall_{v \in V}\; \phi_{\text{opt}}(v) \ge d(v)$).
    Somit produziert der Algorithmus eine $2$-Approximation.
\end{Beweis}

\pagebreak

\subsubsection{%
    Inapproximierbarkeit%
}

\textbf{$k$-CLIQUE}:
Gegeben seien ein unger. Graph $G = (V, E)$ und $k \in \natural$.
Das \begriff{$k$-CLIQUE-Problem} lautet nun:
Gibt es eine \begriff{$k$-Clique} in $G$, d.\,h. $C \subset V$ mit $|C| = k$ und
$\forall_{v, w \in C,\; v \not= w}\; \{v, w\} \in E$?\\
$k$-CLIQUE ist NP-vollständig.

\linie

\textbf{Konstruktion von $I$}:
Sei eine $k$-CLIQUE-Instanz $G = (V, E)$ mit $k \in \natural$ gegeben.
Im Folgenden wird daraus eine PCS-Instanz $I$
mit Graph $H = (W, D)$ und $m \in \natural$ konstruiert.
\begin{itemize}
    \item
    Setze $W := V \cup E \cup F_1 \cup F_2 \cup F_3$ für die Knoten,
    wobei $F_1, F_2, F_3$ paarweise disjunkte \begriff{Füllmengen} mit noch zu bestimmenden
    Kardinalitäten sind.
    
    \item
    Setze $D$ so, dass
    alle Jobs in $F_1$ vor denen in $F_2$ bearbeitet sein müssen,
    alle Jobs in $F_2$ vor denen in $F_3$ bearbeitet sein müssen und
    $(v, e), (w, e) \in D$ für $e = \{v, w\} \in E$
    (Knotenjobs vor zugehörigem Kantenjob).
    
    \item
    Wähle $m, |F_1|, |F_2|, |F_3|$, sodass
    $m = k + |F_1|$, $m = \frac{k(k-1)}{2} + (|V| - k) + |F_2|$ und\\
    $m = |E| - \frac{k(k-1)}{2} + |F_3|$
    (wähle z.\,B. $m := |V|^3$ und setze $|F_1|, |F_2|, |F_3|$ entsprechend).
\end{itemize}

\linie

\textbf{Lemma 1}:
$I$ kann immer in 4 Zeitschritten bearbeitet werden.

\begin{Beweis}
    1. Bearbeite $k$ beliebige Knotenjobs und $F_1$.
    2. Bearbeite die restlichen $|V| - k$ Knotenjobs und $F_2$
    (ein paar Arbeiter bleiben "`arbeitslos"').
    3. Bearbeite $|E| - \frac{k(k-1)}{2}$ beliebige Kantenjobs und $F_3$.
    4. Bearbeite die restlichen $\frac{k(k-1)}{2}$ Kantenjobs.
\end{Beweis}

\textbf{Lemma 2}:
$G = (V, E)$ enthält eine $k$-Clique $\iff$ $I$ kann in 3 Zeitschritten bearbeitet werden.

\begin{Beweis}
    "`$\implies$"':
    Angenommen, $G = (V, E)$ enthält eine $k$-Clique.
    1. Bearbeite die $k$ Knotenjobs der $k$-Clique und $F_1$.
    2. Bearbeite die restlichen $|V| - k$ Knotenjobs, $\frac{k(k-1)}{2}$ Kantenjobs der $k$-Clique
    und $F_2$.
    3. Bearbeite die restlichen $|E| - \frac{k(k-1)}{2}$ Kantenjobs und $F_3$.
    
    "`$\impliedby$"':
    Angenommen, $I$ kann in 3 Zeitschritten bearbeitet werden.
    Dann kann kein Arbeiter zu irgendeiner Zeit "`arbeitslos"' sein
    (die Jobs $F_i$ müssen im Schritt $i$ bearbeitet werden,
    dann bleiben für die 3 Zeitschritte insgesamt noch $|V| + |E|$ Arbeiter übrig).\\
    In Schritt 1 muss wegen der Abhängigkeiten
    $F_1$ und eine Teilmenge $V' \subset V$ von Knoten mit $|V'| = k$ bearbeitet werden.
    In Schritt 2 müssen mindestens $F_2$ und die restlichen Knoten $V \setminus V'$ bearbeitet.
    %werden (damit Kanten in Schritt 3 bearbeitet werden können).
    Damit müssen in Schritt 2 genau $\frac{k(k-1)}{2}$ der Kanten bearbeitet werden,
    was nur geht, wenn die bearbeiteten Knoten $V'$ aus Schritt 1 eine $k$-Clique von $G$
    darstellen.
\end{Beweis}

\linie

\textbf{Satz (Inapproximierbarkeit von PCS)}:
Es gibt keinen Polynomialzeit-Algorithmus,
der eine $\alpha$-Approximation für PCS mit $\alpha < \frac{4}{3}$ liefert,
wenn $\text{P} \not= \text{NP}$.

\begin{Beweis}
    Angenommen, ein Polynomialzeit-Algorithmus für $\alpha$-Approximationen von PCS existiert
    (mit $\alpha < \frac{4}{3}$).
    Dann könnte man $k$-CLIQUE wie folgt in Polynomialzeit entscheiden:
    Seien $G = (V, E)$ und $k \in \natural$ gegeben.
    Konstruiere nun $I$ für $G$ und $k$ und führe den PCS-Algorithmus für $I$ aus,
    um ein Schedule $\phi$ für $I$ zu erhalten.
    Dann gibt es zwei Möglichkeiten:
    \begin{itemize}
        \item
        \emph{$G$ enthält eine $k$-Clique}:
        In diesem Fall kann $I$ nach Lemma 2 in 3 Zeitschritten bearbeitet werden.
        Daher muss $\phi$ die Länge $3$ haben
        (ein Schedule der Länge $\ge 4$ würde der $\alpha$-Approximation mit $\alpha < \frac{4}{3}$
        widersprechen).
        
        \item
        \emph{$G$ enthält keine $k$-Clique}:
        In diesem Fall kann $I$ nach Lemma 1 in 4,
        nach Lemma 2 aber nicht in 3 Zeitschritten bearbeitet werden.
        Daher muss $\phi$ die Länge $\ge 4$ haben.
    \end{itemize}
    Man kann also aufgrund der Länge von $\phi$ in Polynomialzeit $k$-CLIQUE entscheiden,
    ein Widerspruch zur Annahme $\text{P} \not= \text{NP}$.
\end{Beweis}

\pagebreak

\subsection{%
    \emph{Vertex Cover}%
}

\textbf{Vertex-Cover-2-Approximation}:
Der folgende Algorithmus berechnet eine Knotenüberdeckung $C$, die höchstens doppelt so
groß ist wie eine kleinstmögliche Knotenüberdeckung $C_\opt$.
\begin{enumerate}
    \item
    Setze $C \leftarrow \emptyset$.
    
    \item
    Solange $E \not= \emptyset$, wiederhole:
    \begin{enumerate}
        \item
        Wähle eine beliebige Kante $e = \{v, w\} \in E$.
        
        \item
        Setze $C \leftarrow C \cup \{v, w\}$ und $M \leftarrow M \cup \{e\}$.
        
        \item
        Entferne alle Kanten aus $E$, die inzident zu $v$ oder $w$ sind.
    \end{enumerate}
    
    \item
    Gebe $C$ aus.
\end{enumerate}

\linie

\textbf{Satz ($2$-Approximation)}:
Der Algorithmus produziert eine $2$-Approximation.

\begin{Beweis}
    Weil im jeden Schritt nur Kanten entfernt werden, von denen ein Endpunkt sich bereits in der
    Knotenüberdeckung befindet, ist $C$ am Ende eine Knotenüberdeckung.
    Außerdem ist stets $|C| = 2|M|$, denn wenn eine neue Kante $e$ gewählt wird, dann sind beide
    Endpunkte noch nicht in $C$.
    Daher sind die Kanten von $M$ paarweise nicht-adjazent,
    d.\,h. $M$ ist am Ende ein Matching und es gilt
    $|C_\opt| \ge |M| = \frac{|C|}{2} \iff |C| \le 2|C_\opt|$.
    
    ($|M| \le |C_\opt|$ gilt, weil man eine injektive Abbildung
    $f\colon M \to C_\opt$ wie folgt konstruieren kann:
    Sei $e = \{v, w\} \in M$ beliebig.
    Dann ist $v \in C_\opt$ oder $w \in C_\opt$, d.\,h. setze z.\,B. $f(e) := v$.
    Es gilt $f(e) \not= f(e')$ für $e \not= e'$,
    weil sonst $f(e) = f(e')$ ein Endpunkt von $e$ und $e'$ wäre,
    ein Widerspruch zu $M$ Matching.)
\end{Beweis}

\linie

\textbf{Bemerkungen}:
\begin{itemize}
    \item
    Der Algorithmus läuft in Zeit $\O(m)$ mit $m := |E|$.
    
    \item
    Es gibt keinen Polynomialzeit-Algorithmus, der eine $1.1666$-Approximation produziert,
    wenn $\text{P} \not= \text{NP}$.
    
    \item
    Obwohl das VC-Problem NP-vollständig ist, ist die $\integer$-Version des dualen Problems
    in Polynomialzeit lösbar (bekannt als \begriff{maximum cardinality matching}).
    
    \item
    Es gibt VC-Instanzen, bei denen der Algorithmen tatsächlich Knotenüberdeckungen $C$
    mit $|C| = 2|C_\opt|$ ausgibt
    (z.\,B. vollständige bipartite Graphen),
    d.\,h. die Approximationsschranke ist scharf.
    
    \item
    Man kann keinen anderen Algorithmus entwerfen, der z.\,B. $C$ mit
    $|C| = \frac{3}{2} |M|$ zurückgibt (und daher besser ist):
    Sei dazu $K_n$ der vollständige Graph mit $n$ Knoten ($n$ ungerade).
    Dann hat jedes Matching $M$ die Größe $\le \frac{n-1}{2}$,
    aber jede Knotenüberdeckung $C$ hat die Größe $\ge n - 1$,
    d.\,h. $|C|/|M| \ge 2$.
\end{itemize}

\pagebreak
