\newcommand*{\vorlesung}{Mathematische Statistik}
\newcommand*{\dozent}{Priv-Doz.\ Dr.\ Jürgen \name{Dippon}}
\newcommand*{\semester}{Sommersemester 2012}

\hypersetup{pdftitle={Vorlesungsmitschrieb: \vorlesung{}}}

\automark{chapter}
\ihead{\vorlesung{}}
\ifoot{\vspace{-1.5mm}\headmark}

\thispagestyle{empty}
\vspace*{1em}

{%
  \usekomafont{disposition}\huge%
  Vorlesungsmitschrieb:
  \ifdef{\vorlesungInLectureHeading}{%
    \vorlesungInLectureHeading{}%
  }{%
    \vorlesung{}%
  }%
  \par%
}
\vspace*{1em}

\emph{Julian \name{Valentin}}

\vspace*{1em}

Dieser Vorlesungsmitschrieb entstand als Hörer in der Vorlesung \vorlesung{},
gelesen von \dozent{} an der Universität Stuttgart im \semester{}.
Sie dienten hauptsächlich als Lernhilfe für mich;
aus Zeitgründen fehlen viele Skizzen und mathematische Beweise.
Studentische Mitschriebe sind keine of"|fiziellen Skripte;
weder die Universität Stuttgart noch ihre Mitarbeiter sind für sie verantwortlich.
Fehler können auf \href{https://github.com/valentjn/class-notes}{GitHub} gemeldet werden.
Der Mitschrieb steht unter der
\href{https://creativecommons.org/licenses/by-sa/4.0/}{CC-BY-SA-4.0-Lizenz}.

{%
  % set chapter mark in footer and remove number 0
  \renewcommand*{\chaptermarkformat}{}
  \chaptermark{Inhaltsverzeichnis}

  \tableofcontents%
}

\pagebreak

% class-notes: begin-mathjax-commands
\newcommand*{\KK}{\mathbb{K}}
\renewcommand*{\A}{\mathcal{A}}
\renewcommand*{\C}{\mathcal{C}}
\newcommand*{\D}{\mathcal{D}}
\renewcommand*{\F}{\mathcal{F}}
\renewcommand*{\K}{\mathcal{K}}
\renewcommand*{\L}{\mathcal{L}}
\renewcommand*{\O}{\mathcal{O}}
\renewcommand*{\P}{\mathcal{P}}
\newcommand*{\R}{\mathcal{R}}
\newcommand*{\T}{\mathcal{T}}
\renewcommand*{\U}{\mathcal{U}}
\renewcommand*{\V}{\mathcal{V}}
\newcommand*{\kk}[1]{\overline{#1}}
\newcommand*{\orth}{\mathbin{\bot}}
\newcommand*{\iu}{\mathrm{i}}
\newcommand*{\Lip}{\text{Lip}}
\newcommand*{\unif}{\text{unif}}
\newcommand*{\supp}{\operatorname{supp}}
\newcommand*{\BUC}{\text{BUC}}
\newcommand*{\interior}[1]{\operatorname{int}(#1)}
\newcommand*{\Kern}{\text{Kern}}
\newcommand*{\Bild}{\text{Bild}}
\newcommand*{\pot}{\mathfrak{P}}
\newcommand*{\Lin}{\operatorname{Lin}}
\newcommand*{\id}{\text{id}}
\newcommand*{\dist}{\operatorname{dist}}
\let\oldorth\orth\renewcommand*{\orth}{{\oldorth}}
\renewcommand*{\Re}{\operatorname{Re}}
\renewcommand*{\Im}{\operatorname{Im}}
\renewcommand*{\graph}{\text{graph}}
\newcommand*{\codim}{\operatorname{codim}}
\renewcommand*{\div}{\operatorname{div}}
\newcommand*{\loc}{{\text{loc}}}
% class-notes: end-mathjax-commands

\optpart{%
    Grundlegende Räume und Abbildungen der Funktionalanalysis%
}

\section{%
    Skalarprodukte, Normen und Metriken%
}

\subsection{%
    Skalarprodukte%
}

\begin{Bem}
    Im Folgenden ist $\KK = \real$ oder $\KK = \complex$.
\end{Bem}

\begin{Def}{Skalarprodukt}
    Sei $V$ ein $\KK$-Vektorraum.\\
    Eine Abbildung $\sp{\cdot, \cdot}\colon V \times V \rightarrow \KK$ heißt
    \begriff{Skalarprodukt} (oder \begriff{inneres Produkt}) auf $V$, falls
    \begin{enumerate}
        \item
        $\forall_{\alpha \in \KK} \forall_{x, y, z \in V}\;
        \sp{\alpha x + y, z} = \alpha \sp{x, z} + \sp{y, z}$
        (\begriff{Linearität im ersten Argument}),

        \item
        $\forall_{x, y \in V}\; \sp{x, y} = \kk{\sp{y, x}}$
        (\begriff{Symmetrie} bzw. \begriff{\name{Hermite}sche Symmetrie}) und

        \item
        $\forall_{x \in V}\; \sp{x, x} \ge 0 \;\land\; \left[\sp{x, x} = 0 \iff x = 0\right]$
        (\begriff{positive Definitheit}).
    \end{enumerate}
    $V$ zusammen mit $\sp{\cdot, \cdot}$ heißt \begriff{Skalarproduktraum}
    (oder \begriff{Prä-\name{Hilbert}raum}).
\end{Def}

\begin{Bem}
    Aus \emph{(1)} und \emph{(2)} folgt
    $\forall_{\alpha \in \KK} \forall_{x, y, z \in V}\; \sp{x, \alpha y + z} =
    \kk{\alpha} \sp{x, y} + \sp{x, z}$.
    Ein Skalarprodukt ist also für $\KK = \real$ bzw. $\KK = \complex$
    eine positiv definite, symmetrische Bilinearform bzw.
    eine positiv definite, hermitesche Sesquilinearform.
\end{Bem}

\begin{Bsp}
    Folgende Vektorräume bilden mit den zugehörigen Abbildungen Skalarprodukträume.
    \begin{enumerate}[label=\emph{(\alph*)}]
        \item
        $V := \real^n$, $\sp{x, y} := \sum_{i=1}^n x_i y_i$

        \item
        $V := \complex^n$, $\sp{x, y} := \sum_{i=1}^n x_i \kk{y_i}$

        \item
        $V := \left\{\left.x \in \real^\natural \;\right|\; \sp{x, x} < \infty\right\}$,
        $\sp{x, y} := \sum_{i=1}^\infty x_i y_i$

        \item
        $V := \left\{\left.x \in \complex^\natural \;\right|\; \sp{x, x} < \infty\right\}$,
        $\sp{x, y} := \sum_{i=1}^\infty x_i \kk{y_i}$

        \item
        $V := \C([a,b], \real)$ mit $a < b$ reell, $\sp{x, y} := \int_a^b x(t)y(t)\dt$

        \item
        $V := \C([a,b], \complex)$ mit $a < b$ reell, $\sp{x, y} := \int_a^b x(t)\kk{y(t)}\dt$
    \end{enumerate}
\end{Bsp}

\begin{Satz}{\name{Cauchy}-\name{Schwarz}sche Ungleichung}
    Seien $X$ ein Skalarproduktraum und $x, y \in X$.\\
    Dann gilt $|\sp{x, y}| \le \sqrt{\sp{x, x}} \cdot \sqrt{\sp{y, y}}$.
    Gleichheit gilt genau dann, wenn $x$ und $y$ linear abhängig sind.
\end{Satz}

\subsection{%
    Normen%
}

\begin{Bem}
    Ein Skalarprodukt kann zur Abstandsmessung verwendet werden.
\end{Bem}

\begin{Def}{Norm}
    Sei $X$ ein $\KK$-Vektorraum.
    Eine Abbildung $\norm{\cdot}\colon X \rightarrow \real$ heißt \begriff{Norm}, falls
    \begin{enumerate}
        \item
        $\forall_{x \in X}\; \norm{x} \ge 0 \;\land\; \left[\norm{x} = 0 \iff x = 0\right]$
        (\begriff{Positivität} und \begriff{Definitheit}),

        \item
        $\forall_{\alpha \in \KK} \forall_{x \in X}\; \norm{\alpha x} = |\alpha| \cdot \norm{x}$
        (\begriff{Homogenität}) und

        \item
        $\forall_{x, y \in X}\; \norm{x + y} \le \norm{x} + \norm{y}$
        (\begriff{Dreiecksungleichung}).
    \end{enumerate}
    $V$ zusammen mit $\norm{\cdot}$ heißt \begriff{normierter Raum}.
\end{Def}

\linie
\pagebreak

\begin{Satz}{induzierte Norm}
    In jedem Skalarproduktraum $X$ lässt sich durch $\norm{x} := \sqrt{\sp{x, x}}$ eine Norm
    einführen.
    Man nennt sie die durch das Skalarprodukt \begriff{induzierte Norm}.
\end{Satz}

\begin{Satz}{Parallelogrammgleichung}
    Seien $(X, \sp{\cdot, \cdot})$ ein Skalarproduktraum und $\norm{\cdot}$ die durch
    $\sp{\cdot, \cdot}$ induzierte Norm.
    Dann gilt
    $\forall_{x, y \in X}\; \norm{x + y}^2 + \norm{x - y}^2 = 2(\norm{x}^2 + \norm{y}^2)$.
\end{Satz}

\begin{Bem}
    Nach dem Satz über die induzierte Norm ist jeder Skalarproduktraum auch ein normierter Raum.
    Allerdings wird nicht jede Norm von einem Skalarprodukt induziert:
    Sei $X := \real^2$ mit Norm $\norm{x} := \max_{k = 1, 2} |x_k|$ für $x \in X$.
    Für $x := (1, 2)^T$ und $y := (2, 0)^T$ gilt $\norm{x} = \norm{y} = 2$,
    $\norm{x + y} = 3$ und $\norm{x - y} = 2$, also
    $\norm{x + y}^2 + \norm{x - y}^2 = 13 \not= 16 = 2(\norm{x}^2 + \norm{y}^2)$.
    Die Parallelogrammgleichung ist nicht erfüllt, somit kann die Norm nicht von einem
    Skalarprodukt induziert werden.
\end{Bem}

\begin{Satz}{Bedingung für Induktion von Normen durch Skalarprodukte}
    Genau diejenigen normierten Räume $X$, in denen die Parallelogrammgleichung gilt,
    sind Skalarprodukträume, d.\,h. genau in diesen Räumen gibt es ein Skalarprodukt,
    welches die Norm induziert.\\
    In diesem Fall lässt sich für $\KK = \real$ durch
    $\sp{x, y} := \frac{1}{4} (\norm{x + y}^2 - \norm{x - y}^2)$
    und für $\KK = \complex$ durch
    $\sp{x, y} := \frac{1}{4} (\norm{x + y}^2 - \norm{x - y}^2 +
    \iu \cdot (\norm{x + \iu y}^2 - \norm{x - \iu y}^2))$
    (\begriff{Polarisationsformeln})
    ein Skalarprodukt auf $X$ erklären, das die Norm induziert.
\end{Satz}

\linie

\begin{Bem}
    Mithilfe von reellen Skalarprodukten kann man einen Winkelbegriff einführen, denn es gilt
    $\frac{|\sp{x, y}|}{\norm{x} \cdot \norm{y}} \le 1$ für $x, y \not= 0$ aufgrund der
    Cauchy-Schwarz-Ungleichung.
\end{Bem}

\begin{Def}{Winkel}
    Seien $X$ ein reeller Skalarproduktraum und $x, y \in X \setminus \{0\}$.\\
    Dann heißt $\alpha \in [0, \pi]$ mit $\cos(\alpha) = \frac{\sp{x, y}}{\norm{x} \cdot \norm{y}}$
    der \begriff{Winkel} zwischen $x$ und $y$.
\end{Def}

\begin{Def}{orthogonal}
    Sei $X$ ein Skalarproduktraum.
    \begin{enumerate}
        \item
        $x, y \in X$ heißen \begriff{orthogonal zueinander} ($x \orth y$), falls $\sp{x, y} = 0$.

        \item
        $X_1, X_2 \subset X$ mit $X_1, X_2 \not= \emptyset$ heißen \begriff{orthogonal zueinander}
        ($X_1 \orth X_2$), falls\\
        $\forall_{x \in X_1} \forall_{y \in X_2}\; x \orth y$.
    \end{enumerate}
\end{Def}

\begin{Satz}{\name{Pythagoras}}
    Seien $X$ ein Skalarproduktraum und $x, y \in X$ mit $x \orth y$.\\
    Dann gilt $\norm{x + y}^2 = \norm{x}^2 + \norm{y}^2$.
\end{Satz}

\subsection{%
    Beispiele für normierte Räume%
}

\begin{Bsp}
    \begriff{$\KK^n$ mit der $p$-Norm}
    \begin{enumerate}[label=\emph{(\alph*)}]
        \item
        $\norm{x}_p := \left(\sum_{k=1}^n |x_k|^p\right)^{1/p}$ für $p \in [1, \infty)$

        \item
        $\norm{x}_\infty := \max_{k=1,\dotsc,n} |x_k|$
    \end{enumerate}
\end{Bsp}

\begin{Bsp}
    \begriff{Folgenräume}
    \begin{enumerate}[label=\emph{(\alph*)}]
        \item
        $\ell^p := \{x \in \KK^\natural \;|\;
        \norm{x}_{\ell^p} < \infty\}$, $\norm{x}_{\ell^p} :=
        \left(\sum_{k=1}^\infty |x_k|^p\right)^{1/p}$ für $p \in [1, \infty)$

        \item
        $\ell^\infty := \{x \in \KK^\natural \;|\;
        \norm{x}_{\ell^\infty} < \infty\}$, $\norm{x}_{\ell^\infty} :=
        \sup_{k \in \natural} |x_k|$

        \item
        $c_0 := \{x \in \KK^\natural \;|\; \lim_{k \to \infty} x_k = 0\}$,
        $\norm{\cdot}_{\ell^\infty}$

        \item
        $c := \{x \in \KK^\natural \;|\; x \text{ konvergiert}\}$,
        $\norm{\cdot}_{\ell^\infty}$

        \item
        $c_\ast := \{x \in \KK^\natural \;|\; x_k = 0 \text{ für fast alle }
        k \in \natural\}$, $\norm{\cdot}_{\ell^p}$ für $p \in [1, \infty]$
    \end{enumerate}
\end{Bsp}

\linie
\pagebreak

\begin{Bsp}
    \begriff{Funktionenräume}\\
    Seien $M, K, \Omega \subset \real^n$ nicht-leer mit $K$ kompakt und $\Omega$ of"|fen.
    Die Räume sind auch definiert, falls $\KK$ weggelassen wird, in diesem Fall gilt
    $\KK = \real$.
    \begin{enumerate}[label=\emph{(\alph*)}]
        \item
        $B(M, \KK) := \{f\colon M \rightarrow \KK \;|\; \norm{f}_\infty < \infty\}$,
        $\norm{f}_\infty := \sup_{x \in M} |f(x)|$,\\
        \begriff{Raum der beschränkten Funktionen auf $M$}

        \item
        $\C^0(K, \KK) := \{f\colon K \rightarrow \KK \;|\; f \text{ stetig}\}$,
        $\norm{f}_{\C^0} := \sup_{x \in K} |f(x)|$,\\
        \begriff{Raum der stetigen Funktionen auf $K$}

        \item
        $\C^0_b(\Omega, \KK) := \{f\colon \Omega \rightarrow \KK \;|\; f \text{ stetig},\;
        \norm{f}_{\C_0} < \infty\}$,
        $\norm{f}_{\C^0} := \sup_{x \in \Omega} |f(x)|$,\\
        \begriff{Raum der stetigen, beschränkten Funktionen auf $\Omega$}

        \item
        $\C^0_c(\Omega, \KK) := \{f \in \C^0_b(\Omega, \KK) \;|\;
        \supp f \subset \Omega \text{ kompakt}\}$,
        $\norm{\cdot}_{\C^0}$,\\
        \begriff{Raum der stetigen, beschränkten Funktionen mit kompaktem Träger in $\Omega$}

        \item
        $\C^0_\unif(\Omega, \KK) := \BUC(\Omega, \KK) := \{f \in \C^0_b(\Omega, \KK) \;|\;
        f \text{ gleichmäßig stetig auf } \Omega\}$,
        $\norm{\cdot}_{\C^0}$,\\
        \begriff{Raum der gleichmäßig stetigen, beschränkten Funktionen auf $\Omega$}

        \item
        $\C^{0,\alpha}(\Omega, \KK) := \{f \in \C^0_b(\Omega, \KK) \;|\;
        \norm{f}_{\C^{0,\alpha}} < \infty\}$, $\alpha \in (0, 1]$,
        $\norm{f}_{\C^{0,\alpha}} := \norm{f}_{\C^0} + [f]_{\C^{0,\alpha}}$,\\
        $[f]_{\C^{0,\alpha}} := \sup_{x, y \in \Omega,\; x \not= y}
        \frac{|f(x) - f(y)|}{\norm{x - y}^\alpha}$,
        \begriff{Raum der \name{Hölder}-stetigen Funktionen auf $\Omega$},\\
        für $\alpha = 1$ ist
        $\C^{0,1}(\Omega, \KK) =: \Lip(\Omega, \KK)$
        der \begriff{Raum der \name{Lipschitz}-stetigen Funktionen auf $\Omega$}

        \item
        $\C^m(K, \KK) := \{f\colon K \rightarrow \KK \;|\;
        \partial_x^j f \text{ stetig auf } \overset{\circ}{K} = \interior K,\;
        \text{stetig fortsetzb. auf } K,\; |j| \le m\}$,\\
        $\norm{f}_{\C^m} := \sum_{|j| \le m} \norm{\partial_x^j f}_{\C^0}$,
        \begriff{Raum der $m$-fach stetig dif"|ferenzierbaren Funktionen auf $K$}\\
        (dabei ist $j = (j_1, \dotsc, j_n) \in \natural_0^n$ ein \begriff{Multiindex}
        mit $|j| := j_1 + \dotsb + j_n$
        sowie $x = (x_1, \dotsc, x_n)$ und
        $\partial_x^j = \partial_{x_1}^{j_1} \dotsb \partial_{x_n}^{j_n}$)

        \item
        $\C^m_b(\Omega, \KK) := \{f\colon \Omega \rightarrow \KK \;|\;
        \partial_x^j f \text{ stetig},\; \norm{\partial_x^j f}_{\C^0} < \infty,\; |j| \le m\}$,
        $\norm{\cdot}_{\C^m}$,\\
        \begriff{Raum der $m$-fach stetig diffb.,
        in allen Ableitungen beschränkten Funktionen auf $\Omega$}

        \item
        $\C^m_c(\Omega, \KK) := \{f \in \C^m_b(\Omega, \KK) \;|\;
        \partial_x^j f \text{ stetig},\; \supp f \subset \Omega \text{ kompakt}\}$,
        $\norm{\cdot}_{\C^m}$,\\
        \begriff{Raum der $m$-fach stetig diffb. Funktionen mit kompaktem Träger in $\Omega$}

        \item
        $\C^m_\unif(\Omega, \KK) := \{f \in \C^m_b(\Omega, \KK) \;|\;
        \partial_x^j f \in \C^0_\unif(\Omega, \KK),\; |j| \le m\}$,
        $\norm{\cdot}_{\C^m}$,\\
        \begriff{Raum der $m$-fach stetig diffb., in allen Ableitungen
        glm. stetigen Funktionen auf $\Omega$}

        \item
        $\C^{m,\alpha}(\Omega, \KK) := \{f \in \C^m_b(\Omega, \KK) \;|\;
        \partial_x^j f \in \C^{0,\alpha}(\Omega, \KK) \text{ für } |j| = m\}$,\\
        $\norm{f}_{\C^{m,\alpha}} := \norm{f}_{\C^{m-1}} +
        \sum_{|j|=m} \norm{\partial_x^j f}_{\C^{0,\alpha}}$ (für $m \ge 1$),\\
        \begriff{Raum der $m$-fach stetig diffb., in den $m$-ten Ableitungen
        \name{Hölder}-stetigen Fkt.en auf $\Omega$}
    \end{enumerate}
\end{Bsp}

\linie

\begin{Def}{Halbnorm}
    Sei $X$ ein $\KK$-Vektorraum.
    Eine Abbildung $[\cdot]\colon X \rightarrow \real$ heißt \begriff{Halbnorm},
    falls sie alle Norm-Eigenschaften außer
    die Definitheit ($[x] = 0 \iff x = 0$) erfüllt.\\
    $X$ zusammen mit $[\cdot]$ heißt \begriff{halbnormierter Raum}.
\end{Def}

\begin{Satz}{Faktorisierung von halbnormierten Räumen}
    Sei $(X, [\cdot])$ ein halbnormierter Raum.
    \begin{enumerate}
        \item
        $\Kern([\cdot]) := \{x \in X \;|\; [x] = 0\}$ ist ein Unterraum von $X$.

        \item
        $X/\Kern([\cdot])$ mit der kanonischen Quotientenvektorraum-Struktur und der Norm\\
        $\norm{x + \Kern([\cdot])} := [x]$ ist ein normierter Raum.
    \end{enumerate}
\end{Satz}

\begin{Bem}
    Dabei ist $X/\Kern([\cdot]) := \{\widehat{x} \;|\; x \in X\}$
    mit $\widehat{x} := x + \Kern([\cdot]) = \{y \in X \;|\; x \sim y\}$,
    wobei die Äquivalenzrelation $\sim$ durch
    $x \sim y \iff x - y \in \Kern([\cdot])$ definiert ist.
    Dadurch wird $X/\Kern([\cdot])$ mit den Operationen
    $\widehat{x} + \widehat{y} := \widehat{x + y}$ und
    $\alpha \widehat{x} := \widehat{\alpha x}$ zu einem Vektorraum mit
    Nullelement $\Kern([\cdot])$.
\end{Bem}

\linie
\pagebreak

\begin{Def}{$\L^p_\KK(\Omega)$-, $L^p_\KK(\Omega)$-, $\ell^p_\KK$-Räume}
    Sei $(\Omega, \Sigma, \lambda)$ ein Maßraum,
    also $\Sigma$ eine $\sigma$-Algebra über $\Omega$ und $\lambda$ ein Maß über
    $(\Omega, \Sigma)$.
    Definiere $\L^p_\KK(\Omega) := \{f\colon \Omega \rightarrow \KK \;|\;
    f \text{ ist } (\Sigma, \lambda)\text{-messbar},\; [f]_{L^p} < \infty\}$, wobei
    $[f]_{L^p} := \left(\int_\Omega |f|^p d\lambda\right)^{1/p}$ für $1 \le p < \infty$
    und $[f]_{L^\infty} := \inf_{B \in \Sigma,\; \lambda(B) = 0} \sup_{x \in \Omega \setminus B}
    |f(x)|$.\\
    Dadurch wird $(\L^p_\KK(\Omega), [\cdot]_{L^p})$ zum halbnormierten Raum.\\
    Gemäß obigem Satz ist $L^p_\KK(\Omega) := \L^p_\KK(\Omega)/\Kern([\cdot]_{L^p})$
    mit $\norm{f}_{L^p} := [f]_{L^p}$ ein normierter Raum,
    wobei $\Kern([\cdot]_{L^p}) = \{f \in \L^p_\KK(\Omega) \;|\; f = 0 \;\lambda\text{-f.ü.}\}$.\\
    Für $\Omega = \natural$, $\Sigma = \pot(\natural)$ und $\lambda$ gleich
    dem \begriff{Zählmaß} (oder \begriff{Diracmaß}), definiert durch $\lambda(B) := |B|$ für
    $B \subset \natural$, definiert man $\ell^p_\KK := L^p_\KK(\natural) \cong
    \L^p_\KK(\natural)$.\\
    Außerdem legt man fest, dass $\KK = \real$ ist, wenn $\KK$
    bei $\L^p_\KK(\Omega)$, $L^p_\KK(\Omega)$ oder $\ell^p_\KK$ weggelassen wird.
\end{Def}

%\begin{Bem}
%    Der schwierige Schritt beim Nachweis der Halbnorm-/Normeigenschaften von\\
%    $\L^p_\KK(\Omega)$/$L^p_\KK(\Omega)$ ist der Beweis der Dreiecksungleichung.
%    Dazu benötigt man ein paar Hilfssätze.
%\end{Bem}

\begin{Def}{konjugierte Zahl}
    Sei $p \in [1, \infty]$.\\\
    Dann heißt $p' \in [1, \infty]$ mit $\frac{1}{p} + \frac{1}{p'} = 1$ die zu $p$
    \begriff{konjugierte Zahl}
    (wobei $\frac{1}{\infty} := 0$).
\end{Def}

\begin{Lemma}{\name{Young}sche Ungleichung}
    Seien $a, b \ge 0$ und $p \in (1, \infty)$.
    Dann ist $ab \le \frac{1}{p} a^p + \frac{1}{p'} b^{p'}$.
\end{Lemma}

\begin{Satz}{\name{Hölder}sche Ungleichung}
    Seien $p \in [1, \infty]$, $f \in L^p(\Omega)$ und $g \in L^{p'}(\Omega)$.\\
    Dann ist $fg \in L^1(\Omega)$ und $\norm{fg}_{L^1} \le \norm{f}_{L^p} \norm{g}_{L^{p'}}$.
\end{Satz}

\begin{Satz}{\name{Minkowski}sche Ungleichung}
    Seien $p \in [1, \infty]$ und $f, g \in L^p(\Omega)$.\\
    Dann ist $f + g \in L^p(\Omega)$ und $\norm{f + g}_{L^p} \le \norm{f}_{L^p} + \norm{g}_{L^p}$.
\end{Satz}

\begin{Bem}
    Für $\lambda(\Omega) < \infty$ (d.\,h. $\lambda$ ist ein \begriff{endliches Maß})
    und $p, q \in [1, \infty]$ mit $p < q$ gilt
    $L^q(\Omega) \subset L^p(\Omega)$, genauer
    $\exists_{C > 0} \forall_{f \in L^q(\Omega)}\; \norm{f}_{L^p} \le C \norm{f}_{L^q}$.
\end{Bem}

\subsection{%
    Metriken%
}

\begin{Def}{Metrik}
    Sei $X \not= \emptyset$.
    Eine Abbildung $d\colon X \times X \rightarrow \real$ heißt \begriff{Metrik}, falls
    \begin{enumerate}
        \item
        $\forall_{x, y \in X}\; d(x, y) \ge 0 \;\land\; [d(x, y) = 0 \iff x = y]$
        (\begriff{Positivität} und \begriff{Definitheit}),

        \item
        $\forall_{x, y \in X}\; d(x, y) = d(y, x)$
        (\begriff{Symmetrie}) und

        \item
        $\forall_{x, y, z \in X}\; d(x, y) \le d(x, z) + d(z, y)$
        (\begriff{Dreiecksungleichung}).
    \end{enumerate}
    $X$ zusammen mit $d$ heißt \begriff{metrischer Raum}.
\end{Def}

\begin{Def}{Halbmetrik}
    Erfüllt $d$ alle Metrik-Eigenschaften außer die Definitheit\\
    ($d(x, y) = 0 \iff x = y$),
    so heißt $d$ \begriff{Halbmetrik}.\\
    $X$ zusammen mit $d$ heißt \begriff{halbmetrischer Raum}.
\end{Def}

\begin{Bem}
    Durch Verwendung von Quotientenräumen kann man wie bei halbnormierten Räumen
    halbmetrische Räume zu metrischen Räumen machen.
\end{Bem}

\begin{Satz}{induzierte Metrik}
    \begin{enumerate}
        \item
        Sei $(X, \norm{\cdot})$ ein normierter Raum.
        Dann ist durch $d(x, y) := \norm{x - y}$ eine Metrik
        (die sog. \begriff{induzierte Metrik}) definiert,
        die folgende zusätzliche Eigenschaften besitzt:
        \begin{enumerate}[start=4]
            \item
            $\forall_{x, y, z \in X}\; d(x + z, y + z) = d(x, y)$
            (\begriff{Translationsinvarianz}) und

            \item
            $\forall_{x, y \in X} \forall_{\alpha \in \KK}\;
            d(\alpha x, \alpha y) = |\alpha| \cdot d(x, y)$
            (\begriff{Homogenität}).
        \end{enumerate}

        \item
        Sei $(X, d)$ ein metrischer Raum.
        Außerdem sei $X$ ein $\KK$-Vektorraum, sodass
        $d$ translationsinvariant und homogen ist.
        Dann ist durch $\norm{x} := d(x, 0)$ eine Norm definiert,
        die die Metrik $d$ induziert.
    \end{enumerate}
\end{Satz}

\begin{Bsp}
    Für $X \not= \emptyset$ ist $d(x, y) := 0$ für $x = y$ und $d(x, y) := 1$ sonst
    eine Metrik, die \begriff{diskrete Metrik}.
    Falls $X$ ein $\KK$-Vektorraum ist, wird sie von keiner Norm induziert, wenn $|X| \ge 2$.
\end{Bsp}

\pagebreak

\chapter{%
    Topologie in Skalarprodukt-, normierten und metrischen Räumen%
}

\section{%
    Topologische Definitionen%
}

\begin{Bem}
    Im Folgenden ist $(X, d)$ ein metrischer Raum.
\end{Bem}

\begin{Def}{$\varepsilon$-Kugel}
    Für $x_0 \in X$ und $\varepsilon > 0$ heißt
    $B_\varepsilon(x_0) := \{x \in X \;|\; d(x, x_0) < \varepsilon\}$
    \begriff{$\varepsilon$-Kugel um $x_0$}.
\end{Def}

\begin{Def}{of"|fen}
    $O \subset X$ heißt \begriff{of"|fen}, falls
    $\forall_{x \in O} \exists_{\varepsilon > 0}\; B_\varepsilon(x) \subset O$.
\end{Def}

\begin{Def}{abgeschlossen}
    $A \subset X$ heißt \begriff{abgeschlossen}, falls $X \setminus A$ of"|fen ist.
\end{Def}

\vspace{-2mm}

\begin{Def}{Inneres}
    Für $M \subset X$ heißt $\overset{\circ}{M} = \interior{M} :=
    \{x \in M \;|\; \exists_{\varepsilon > 0}\; B_\varepsilon(x) \subset M\}$
    \begriff{Inneres von $M$}.
\end{Def}

\begin{Def}{Abschluss}
    Für $M \subset X$ heißt $\overline{M} := X \setminus \interior{X \setminus M}$
    \begriff{Abschluss von $M$}.
\end{Def}

\begin{Def}{Rand}
    Für $M \subset X$ heißt $\partial M := \overline{M} \setminus \interior{M}$
    \begriff{Rand von $M$}.
\end{Def}

\begin{Def}{dicht}
    $B \subset X$ \begriff{liegt dicht} in $A \subset X$, falls $\overline{B} = A$.
\end{Def}

\begin{Def}{beschränkt}
    $C \subset X$ heißt \begriff{beschränkt}, falls
    $\exists_{x \in X} \exists_{R > 0}\; C \subset B_R(x)$.
\end{Def}

\begin{Def}{zusammenhängend}
    $Z \subset X$ heißt \begriff{zusammenhängend}, falls
    es keine Zerlegung von $Z$ in zwei disjunkte, of"|fene und nicht-leere Mengen
    $Z_1, Z_2 \subset X$ gibt.
\end{Def}

\begin{Bem}
    Die Mengen $Z_1, Z_2 \subset X$ bei der Definition von Zusammenhang müssen of"|fen
    bzgl. der Teilraumtopologie auf $Z$ sein, d.\,h. Schnitte von of"|fenen Mengen in $X$
    mit $Z$.
\end{Bem}

\begin{Bsp}
    \begin{enumerate}[label=\emph{(\alph*)}]
        \item
        Sei $(X, d) = (\real^2, \norm{\cdot}_2)$.
        Dann ist $B_1(0) = \interior{B_1(0)}$ of"|fen und zusammenhängend und
        $\overline{B_1(0)} = \{x \in \real^2 \;|\; \norm{x}_2 \le 1\}$ ist  abgeschlossen und
        zusammenhängend.
        Außerdem ist $\partial B_1(0) = \{x \in \real^2 \;|\; \norm{x}_2 = 1\}$.

        \item
        Sei $(X, d) = (\real, |\cdot|)$.
        Dann ist $M = \bigcup_{n \in \natural} \left[\frac{1}{2n}, \frac{1}{2n-1}\right]$
        nicht zusammenhängend und weder of"|fen noch abgeschlossen.
        Es gilt $\partial M = \{\frac{1}{m} \;|\; m \in \natural\} \cup \{0\}$.
    \end{enumerate}
\end{Bsp}

\begin{Bem}
    Für normierte Räume $X$ gilt
    $\overline{B_\varepsilon(x_0)} = \{x \in X \;|\; \norm{x - x_0} \le \varepsilon\}$.
\end{Bem}

\section{%
    Konvergenz%
}

\begin{Def}{Konvergenz}
    Eine Folge $(x_n)_{n \in \natural}$ in einem metrischen Raum $(X, d)$
    heißt \begriff{konvergent} gegen den \begriff{Grenzwert} $x \in X$ für $n \to \infty$
    ($x_n \xrightarrow{n \to \infty} x$, $\lim_{n \to \infty} x_n = x$), falls
    $\lim_{n \to \infty} d(x_n, x) = 0$,
    also $\forall_{\varepsilon > 0} \exists_{n_\varepsilon \in \natural}
    \forall_{n \ge n_\varepsilon}\; d(x_n, x) < \varepsilon$.
\end{Def}

\begin{Bem}
    Der Grenzwert einer Folge $(x_n)_{n \in \natural}$ ist eindeutig bestimmt, wenn er existiert.
    Sind nämlich $x$ und $y$ Grenzwerte der Folge, dann gilt\\
    $0 \le d(x, y) \le d(x, x_n) + d(x_n, y) = d(x_n, x) + d(x_n, y) \xrightarrow{n \to \infty} 0$,
    also $d(x, y) = 0$ und $x = y$.
\end{Bem}

\begin{Satz}{Linearität des Grenzwerts}
    Seien $(X, \norm{\cdot})$ ein normierter Raum,
    $(x_n)_{n \in \natural}$ und $(y_n)_{n \in \natural}$ Folgen in $X$
    sowie $(\alpha_n)_{n \in \natural}$ eine Folge in $\KK$,
    wobei $x_n \xrightarrow{n \to \infty} x$, $y_n \xrightarrow{n \to \infty} y$ und
    $\alpha_n \xrightarrow{n \to \infty} \alpha$.\\
    Dann gilt $\alpha_n x_n + y_n \xrightarrow{n \to \infty} \alpha x + y$.
\end{Satz}

\begin{Satz}{Abschluss ist Menge aller Grenzwerte}
    Seien $(X, d)$ ein metrischer Raum und $M \subset X$.\\
    Dann gilt $\overline{M} = \{x \in X \;|\;
    \exists_{(x_n)_{n \in \natural} \text{ Folge in } M}\; x_n \xrightarrow{n \to \infty} x\}$.
\end{Satz}

\linie
\pagebreak

\begin{Bsp}
    \begin{enumerate}[label=\emph{(\alph*)}]
        \item
        Sei $(X, d) = (\real^m, \norm{\cdot}_2)$.
        Dann gilt $x_n \xrightarrow{n \to \infty} x$ genau dann, wenn\\
        $\sqrt{\sum_{i=1}^m ((x_n)_i - (x)_i)^2} \xrightarrow{n \to \infty} 0$.
        Dies ist äquivalent zu $\forall_{i=1,\dotsc,m}\; (x_n)_i \xrightarrow{n \to \infty} (x)_i$.

        \item
        Sei $(X, d) = (\C^0([0, 1]), d)$ mit $d(x, y) = \max_{t \in [0, 1]} |x(t) - y(t)|$.\\
        Dann gilt $x_n \xrightarrow{n \to \infty} x$ genau dann, wenn
        $\max_{t \in [0, 1]} |x_n(t) - x(t)| \xrightarrow{n \to \infty} 0$\\
        $\iff \forall_{\varepsilon > 0} \exists_{n_\varepsilon \in \natural}
        \forall_{n \ge n_\varepsilon}\; \max_{t \in [0, 1]} |x_n(t) - x(t)| < \varepsilon$\\
        $\iff \forall_{\varepsilon > 0} \exists_{n_\varepsilon \in \natural}
        \forall_{n \ge n_\varepsilon} \forall_{t \in [0, 1]}\; |x_n(t) - x(t)| < \varepsilon$
        ($x_n$ \begriff{konvergiert gleichmäßig} gegen $x$).

        \item
        Sei $(X, d) = (\C^0([0, 1]), d)$ mit
        $d(x, y) = \left(\int_0^1 |x(t) - y(t)|^p \dt\right)^{1/p}$ für $p \in [1, \infty)$.\\
        Dann gilt $x_n \xrightarrow{n \to \infty} x$ genau dann, wenn
        $\left(\int_0^1 |x_n(t) - x(t)|^p \dt\right)^{1/p} \xrightarrow{n \to \infty} 0$\\
        $\iff \forall_{\varepsilon > 0} \exists_{n_\varepsilon \in \natural}
        \forall_{n \ge n_\varepsilon}\; \int_0^1 |x_n(t) - x(t)|^p\dt < \varepsilon$
        ($x_n$ \begriff{konvergiert im $p$-ten Mittel} gegen $x$).
    \end{enumerate}
\end{Bsp}

\section{%
    Stetigkeit%
}

\begin{Bem}\\
    Im Folgenden sind $(X, d_X)$ und $(Y, d_Y)$ metrische Räume und
    $T\colon X \rightarrow Y$ eine Abbildung.
\end{Bem}

\begin{Def}{stetig in einem Punkt}
    $T$ heißt \begriff{stetig in $x_0 \in X$}, falls\\
    $\forall_{\varepsilon > 0} \exists_{\delta = \delta(x_0, \varepsilon) > 0}
    \forall_{x \in X,\; d_X(x, x_0) < \delta}\; d_Y(T(x), T(x_0)) < \varepsilon$.
\end{Def}

\begin{Def}{stetig}
    $T$ heißt \begriff{stetig (in $X$)}, falls $T$ in jedem Punkt $x_0 \in X$ stetig ist.
\end{Def}

\begin{Def}{Homöomorphismus}\\
    $T$ heißt \begriff{Homöomorphismus}, falls $T$ bijektiv ist sowie $T$ und $T^{-1}$ stetig sind.
\end{Def}

\begin{Def}{Isomorphismus}\\
    $T$ heißt \begriff{Isomorphismus}, falls $T$ bijektiv und linear ist sowie
    $T$ und $T^{-1}$ stetig sind.
\end{Def}

\begin{Def}{Isometrie}\\
    $T$ heißt \begriff{Isometrie}, falls $T$ bijektiv und stetig ist und
    $\forall_{x_1, x_2 \in X}\; d_Y(T(x_1), T(x_2)) = d_X(x_1, x_2)$.
\end{Def}

\begin{Bem}
    Isometrien werden oft ohne Voraussetzung der Bijektivität definiert.
    Bijektive Isometrien heißen in diesem Fall isometrische Isomorphismen.
\end{Bem}

\linie

\begin{Satz}{äquivalente Beschreibungen von Stetigkeit}
    Folgende Aussagen sind äquivalent:
    \begin{enumerate}
        \item
        $T$ ist stetig.

        \item
        $T$ ist \begriff{folgenstetig},
        d.\,h. $\forall_{x \in X}
        \forall_{(x_n)_{n \in \natural} \text{ Folge in } X,\; x_n \to x}\;
        T(x_n) \xrightarrow{n \to \infty} T(x)$.

        \item
        Für alle of"|fenen Teilmengen $O \subset Y$ ist $T^{-1}(O) \subset X$ of"|fen.

        \item
        Für alle abgeschlossenen Teilmengen $A \subset Y$ ist $T^{-1}(A) \subset X$ abgeschlossen.
    \end{enumerate}
\end{Satz}

\pagebreak

\section{%
    Vollständige Räume%
}

\begin{Def}{\name{Cauchy}-Folge}
    Eine Folge $(x_n)_{n \in \natural}$ in einem metrischen Raum $(X, d)$ heißt
    \begriff{\name{Cauchy}-Folge}, falls
    $\forall_{\varepsilon > 0} \exists_{n_\varepsilon \in \natural}
    \forall_{n, m \ge n_\varepsilon}\; d(x_n, x_m) < \varepsilon$.
\end{Def}

\begin{Lemma}{konvergente Folgen sind \name{Cauchy}-Folgen}\\
    Jede konvergente Folge in einem metrischen Raum ist eine Cauchy-Folge.
\end{Lemma}

\begin{Def}{vollständig}
    Ein metrischer Raum $(X, d)$ heißt \begriff{vollständig}, falls jede Cauchy-Folge
    $(x_n)_{n \in \natural}$ in $X$ gegen einen Punkt $x \in X$ konvergiert.
\end{Def}

\begin{Def}{\name{Fréchet}-, \name{Banach}-, \name{Hilbertraum}}
    Ein vollständiger metrischer Raum, normierter Raum oder Skalarproduktraum heißt
    \begriff{\name{Fréchet}-, \name{Banach}- bzw. \name{Hilbert}raum}.
\end{Def}

\begin{Bsp}
    \begin{enumerate}[label=\emph{(\alph*)}]
        \item
        $(\real, |\cdot|)$ und $(\complex, |\cdot|)$ sind Banachräume.

        \item
        $(\rational, d)$ mit $d(x, y) = |x - y|$ ist nicht vollständig.
        Wählt man z.\,B. die Folge $(x_n)_{n \in \natural}$ in $\rational$ mit
        $x_n$ gleich der Dezimaldarstellung von $\sqrt{2}$ bis zur $n$-ten Nachkommastelle,
        so konvergiert zwar $x_n \to \sqrt{2}$ in $\real$.
        Die Folge hat aber keinen Grenzwert in $\rational$ (obwohl sie eine Cauchy-Folge ist).
    \end{enumerate}
\end{Bsp}

\linie

\begin{Def}{äquivalent}
    Zwei Normen $\norm{\cdot}_a$ und $\norm{\cdot}_b$ auf $X$ heißen äquivalent, falls
    jede Folge, die bzgl. $\norm{\cdot}_a$ konvergiert, auch bzgl. $\norm{\cdot}_b$ konvergiert
    und umgekehrt.\\
    Äquivalent ist
    $\exists_{c_1, c_2 > 0} \forall_{x \in X}\; c_1 \norm{x}_b \le \norm{x}_a \le c_2 \norm{x}_b$.
\end{Def}

\begin{Satz}{äquivalente Normen in endlich-dimensionalen Räumen}\\
    In einem endlich-dimensionalen $\KK$-Vektorraum $X$ sind alle Normen äquivalent.
\end{Satz}

\begin{Kor}
    Jeder endlich-dimensionale normierte Raum ist ein Banachraum.
\end{Kor}

\begin{Bem}
    Jeder endlich-dimensionale Unterraum $U$ eines normierten Raums $X$ ist abgeschlossen.
    Ist nämlich $(x_n)_{n \in \natural}$ eine Folge in $U$ und $x \in X$ mit
    $x = \lim_{n \to \infty} x_n$, dann ist $(x_n)_{n \in \natural}$ eine Cauchy-Folge in $U$.
    Weil $U$ vollständig ist, existiert ein Grenzwert in $U$, d.\,h. auch in $X$.
    Wegen der Eindeutigkeit von Grenzwerten muss dieser mit $x$ übereinstimmen, also $x \in U$.
\end{Bem}

\linie

\begin{Satz}{vollständige Funktionenräume}
    Alle oben definierten, normierten Funktionenräume außer $C^m_c(\Omega, \KK)$ sind
    vollständig,
    also die Räume
    $B(M, \KK)$,
    $\C^m(K, \KK)$,
    $\C^m_b(\Omega, \KK)$,
    $\C^m_\unif(\Omega, \KK)$ und
    $\C^{0,\alpha}(\Omega, \KK)$
    für $M, K, \Omega \subset \real^n$ nicht-leer mit $K$ kompakt, $\Omega$ of"|fen und
    $m \in \natural_0$, $\alpha \in (0, 1]$.
\end{Satz}

\begin{Bem}
    Die $\C^m_c$-Räume sind nicht vollständig, da es Folgen gibt, bei denen der Träger immer
    breiter wird (die Grenzfunktion hätte keinen kompakten Träger mehr).
\end{Bem}

\begin{Satz}{$\ell^p_\KK$ vollständig}
    Die Räume $(\ell^p_\KK, \norm{\cdot}_p)$ mit $p \in [1, \infty]$
    sind vollständig, insbesondere handelt es sich bei $p = 2$ um einen Hilbertraum.
\end{Satz}

\linie
\pagebreak

\begin{Bem}
    $\C^0([0, 1])$ mit $\norm{f} := \left(\int_0^1 |f(x)|^p \dx\right)^{1/p}$ für $p \in [1, \infty)$
    ist nicht vollständig.\\
    Für $p = 2$ ist zum Beispiel $(f_n)_{n \in \natural}$ mit $f_n(x) := n^\alpha$ für
    $x \in [0, 1/n]$ und $f_n(x) := x^{-\alpha}$ für $x \in (1/n, 1]$ und $\alpha \in (0, 1/2)$
    eine nicht-konvergente Cauchy-Folge.
\end{Bem}

\begin{Satz}{$L^p$ vollständig}
    Die Räume $(L^p(\Omega), \norm{\cdot}_{L^p})$ mit $p \in [1, \infty]$
    sind vollständig, insbesondere handelt es sich bei $p = 2$ um einen Hilbertraum.
\end{Satz}

\begin{Satz}{Satz von \name{Beppo}-\name{Levi} zur monotonen Konvergenz}\\
    Seien $D$ messbar und $(f_n)_{n \in \natural}$ eine Folge messbarer Funktionen
    $f_n\colon D \rightarrow \real_0^+ \cup \{\infty\}$ mit $f_n \uparrow f$ für $n \to \infty$
    ($f_n$ \begriff{konvergiert monoton} gegen $f$, also
    $\forall_{x \in D}\; \lim_{n \to \infty} f_n(x) = f(x),\; f_n(x) \le f_{n+1}(x)$).\\
    Dann ist $f$ messbar und
    $\int_D f d\lambda = \lim_{n \to \infty} \left(\int_D f_n d\lambda\right)$.
\end{Satz}

\begin{Satz}{Satz von \name{Lebesgue} zur majorisierten Konvergenz}\\
    Seien $D$ messbar und $(f_n)_{n \in \natural}$ eine Folge messbarer Funktionen
    $f_n\colon D \rightarrow \real \cup \{\pm\infty\}$, sodass
    $\lim_{n \to \infty} f_n(x) =: f(x)$ $\lambda$-f.ü. existiert, sowie
    $g$ $\lambda$-integrierbar mit $\forall_{n \in \natural}\; |f_n| \le g$.\\
    Dann ist $f$ messbar und
    $\int_D f d\lambda = \lim_{n \to \infty} \left(\int_D f_n d\lambda\right)$ sowie
    $\lim_{n \to \infty} \left(\int_D |f - f_n| d\lambda\right) = 0$.
\end{Satz}

\begin{Lemma}{Äquivalenz für Banachraum}
    Sei $(X, \norm{\cdot})$ ein normierter Raum.\\
    Dann sind äquivalent:
    \begin{enumerate}
        \item
        $(X, \norm{\cdot})$ ist ein Banachraum.

        \item
        Jede \begriff{absolut konvergente} Reihe $\sum_{i=1}^\infty a_i$
        (d.\,h. $\sum_{i=1}^\infty \norm{a_i} < \infty$) ist konvergent.
    \end{enumerate}
\end{Lemma}

\linie

\begin{Bsp}
    $(C^\infty_b(\Omega), d)$ mit $d(f, g) := \sum_{n=1}^\infty 2^{-n} \cdot
    \frac{\norm{f^{(n)} - g^{(n)}}_{\C^0}}{1 + \norm{f^{(n)} - g^{(n)}}_{\C^0}}$
    ist ein Fréchetraum.
\end{Bsp}

\begin{Satz}{Vervollständigung}
    Jeder normierte Raum $(X, \norm{\cdot})$ ist \begriff{isometrisch isomorph} zu einem
    normierten Raum $(X_\ast, \norm{\cdot}_\ast)$
    (d.\,h. es gibt einen Isomorphismus $T\colon X \rightarrow X_\ast$, der gleichzeitig
    eine Isometrie ist),
    wobei $(X_\ast, \norm{\cdot}_\ast)$ ein dichter Unterraum eines Banachraums
    $(\widetilde{X}, \norm{\cdot}_{\widetilde{X}})$ und
    bis auf isometrische Isomorphie eindeutig bestimmt ist.
    $(\widetilde{X}, \norm{\cdot}_{\widetilde{X}})$
    heißt \begriff{Vervollständigung} von $(X, \norm{\cdot}_X)$.
\end{Satz}

\begin{Satz}{$\C^m_c$ dicht in $L^p$}
    Für $m \in \natural_0 \cup \{\infty\}$ und $p \in [1, \infty)$ ist
    $\C^m_c(\Omega)$ dicht in $(L^p(\Omega), \norm{\cdot}_{L^p})$.\\
    $(L^p(\Omega), \norm{\cdot}_{L^p})$ kann somit mit der Vervollständigung von $\C^m_c(\Omega)$
    bzgl. der $\norm{\cdot}_{L^p}$-Norm identifiziert werden.
\end{Satz}

\linie

\begin{Satz}{\name{Banach}scher Fixpunktsatz}
    Seien $(X, d)$ ein vollständiger metrischer Raum und\\
    $F\colon X \rightarrow X$ eine \begriff{Kontraktion}, d.\,h.
    $\exists_{\lambda \in (0, 1)} \forall_{x, y \in X}\;
    d(F(x), F(y)) \le \lambda \cdot d(x, y)$.\\
    Dann besitzt $F$ genau einen \begriff{Fixpunkt},
    d.\,h. $\exists!_{x^\ast \in X}\; F(x^\ast) = x^\ast$.
\end{Satz}

\section{%
    Kompaktheit%
}

\begin{Def}{kompakt}
    Seien $(X, d)$ ein metrischer Raum und $K \subset X$.\\
    Dann heißt $K$ \begriff{kompakt}, falls
    $\forall_{I \text{ Indexmenge}} \forall_{O_i \subset X \text{ of"|fen},\;
    K \subset \bigcup_{i \in I} O_i} \exists_{i_1, \dotsc, i_n \in I}\;
    K \subset \bigcup_{j=1}^n O_{i_j}$.
\end{Def}


\begin{Satz}{Äquivalenz zu Kompaktheit}
    Seien $(X, d)$ ein metrischer Raum und $K \subset X$.\\
    Dann sind äquivalent:
    \begin{enumerate}
        \item
        $K$ ist kompakt.

        \item
        $K$ ist \begriff{folgenkompakt}, d.\,h.
        $\forall_{(x_n)_{n \in \natural} \text{ Folge in} K}
        \exists_{(x_{n_k})_{k \in \natural} \text{ Teilfolge}} \exists_{x \in K}\;
        x = \lim_{k \to \infty} x_{n_k}$.

        \item
        $(K, d)$ ist vollständig und \begriff{präkompakt}, d.\,h.
        $\forall_{\varepsilon > 0} \exists_{H \subset X \text{ endlich}}\;
        K \subset \bigcup_{x \in H} B_\varepsilon(x)$.
    \end{enumerate}
\end{Satz}

\begin{Bem}
    $\overline{K} \subset X$ ist kompakt $\iff
    \forall_{(x_n)_{n \in \natural} \text{ Folge in} K}
    \exists_{(x_{n_k})_{k \in \natural} \text{ Teilfolge}} \exists_{x \in X}\;
    x = \lim_{k \to \infty} x_{n_k}$.
\end{Bem}

\linie
\pagebreak

\begin{Satz}{kompakt $\Rightarrow$ beschränkt und abgeschlossen}\\
    Jede kompakte Teilmenge eines metrischen Raumes ist beschränkt und abgeschlossen.
\end{Satz}

\begin{Satz}{Äquivalenz für Umkehrung}
    Sei $(X, \norm{\cdot})$ ein normierter Raum.\\
    Dann sind äquivalent:
    \begin{enumerate}
        \item
        Jede beschränkte und abgeschlossene Teilmenge ist kompakt.

        \item
        $X$ ist endlich-dimensional.

        \item
        $\overline{B_1(0)}$ ist kompakt.
    \end{enumerate}
\end{Satz}

%\begin{Bem}
%    Für den Beweis von \emph{(3)} nach \emph{(2)} wird folgendes Lemma benötigt.
%\end{Bem}

\begin{Lemma}{Lemma von \name{Riesz}}
    Seien $(X, \norm{\cdot})$ ein normierter Raum und
    $Y \subsetneqq X$ ein abgeschlossener Unterraum.
    Dann gilt $\forall_{r \in (0,1)} \exists_{x_r \in X \setminus Y}\;
    \norm{x_r} = 1,\; \dist(x_r, Y) := \inf_{y \in Y} \norm{x_r - y} \ge r$.
\end{Lemma}

\linie

\begin{Satz}{beste Approximation}\\
    Seien $(X, d)$ ein metrischer Raum und $K \subset X$ eine nicht-leere, kompakte Teilmenge.\\
    Dann gilt $\forall_{x_0 \in X} \exists_{y_0 \in K}\; d(x_0, y_0) = \dist(x_0, K)
    := \inf_{y \in K} d(x_0, y)$.\\
    %Dann gibt es zu jedem Punkt $y \in X$ einen Punkt $x_0 \in K$, der von $y$
    %den kleinsten Abstand hat.
    In diesem Fall heißt $y_0$ \begriff{beste Approximation} oder
    \begriff{bestapproximierendes Element} von $x_0$ in $K$.
\end{Satz}

\begin{Bem}
    In nicht-kompakten Mengen gibt es i.\,A. kein bestapproximierendes Element,
    z.\,B. geht dies nicht für $x_0 = -1$ und $M_1 = (0, 1]$ oder
    $x_0 = -1$ und $M_2 = \bigcup_{n \in \natural} \left[\frac{1}{2n}, \frac{1}{2n-1}\right]$.
\end{Bem}

\linie

\begin{Satz}{Satz von \name{Arzelà}-\name{Ascoli}}\\
    Seien $(K, d)$ ein kompakter metrischer Raum und $A \subset \C^0(K, \KK)$.
    Dann sind äquivalent:
    \begin{enumerate}
        \item
        $A$ ist \begriff{relativ kompakt} in $\C^0(K, \KK)$, d.\,h.
        $\overline{A}$ ist kompakt in $\C^0(K, \KK)$.

        \item
        $A$ ist beschränkt (d.\,h. $\sup_{f \in A} \norm{f}_{\C^0} < \infty$)
        und \begriff{gleichgradig stetig}, d.\,h.
        \\$\forall_{x \in K} \forall_{\varepsilon > 0}
        \exists_{\delta = \delta(x, \varepsilon) > 0}
        \forall_{y \in B_\delta(x)} \forall_{f \in A}\; |f(x) - f(y)| < \varepsilon$.
    \end{enumerate}
\end{Satz}

\begin{Bem}
    Da $K$ kompakt ist, gilt $\C^0(K, \KK) = \C^0_\unif(K, \KK)$,
    d.\,h. das $\delta(x)$ kann unabhängig von $x$ gewählt werden.
    Diesen als Satz von Heine-Cantor bekannten Sachverhalt kann man so beweisen:
    Sei $\varepsilon > 0$ beliebig.
    Zu $x \in K$ sei $\delta(x) := \delta(x, \varepsilon)$ wie in der Definition der Stetigkeit.
    Weil $K$ kompakt ist, gibt es $x_1, \dotsc, x_n \in K$ mit
    $K \subset \bigcup_{k=1}^n B_{\delta(x_k)/2}(x_k)$.
    Wähle $\delta := \min_{k=1,\dotsc,n} \frac{\delta(x_k)}{2}$.
    Seien $x \in K$ und $y \in B_\delta(x)$ beliebig.
    Dann gibt es ein $\ell \in \{1, \dotsc, n\}$, sodass
    $x \in B_{\delta(x_\ell)/2}(x_\ell)$.
    Aus $y \in B_\delta(x)$ folgt, dass $y \in B_{\delta(x_\ell)/2}(x)$.
    Insgesamt gilt also $y \in B_{\delta(x_\ell)}(x_\ell)$.
    Damit erhält man
    $|f(x) - f(y)| \le |f(x) - f(x_\ell)| + |f(x_\ell) - f(y)| < 2\varepsilon$,
    wobei man jeweils die Stetigkeit von $f$ in $x_\ell$ anwendet
    ($d(x, x_\ell) < \frac{\delta(x_\ell)}{2} < \delta(x_\ell)$ und
    $d(x_\ell, y) < \delta(x_\ell)$).
\end{Bem}

\begin{Bsp}
    Die Menge $A := B_1(0)$ in $(\C^1([-1, 1]), \norm{\cdot}_{\C^1})$
    ist beschränkt in $(\C^0([-1, 1]), \norm{\cdot}_{\C^0})$
    (da $\norm{f}_{\C^0} \le \norm{f}_{\C^1} < 1$ für alle $f \in A$)
    und gleichgradig stetig, da\\
    $\forall_{x \in [-1, 1]} \forall_{\varepsilon > 0}
    \exists_{\delta = \delta(x, \varepsilon) > 0} \forall_{y \in B_\delta(x)}
    \forall_{f \in A}\; |f(x) - f(y)| \le |x - y| \cdot
    \sup_{\xi \in [-1, 1]} |f'(\xi)| < \varepsilon$
    für $\delta(x, \varepsilon) := \varepsilon$,
    weil $\sup_{\xi \in [-1, 1]} |f'(\xi)| \le 1$ für alle $f \in A$.\\
    Nach dem Satz von Arzelà-Ascoli ist $A$ relativ kompakt in
    $(\C^0([-1, 1]), \norm{\cdot}_{\C^0})$.
\end{Bsp}

\begin{Satz}{Satz von \name{Fréchet}-\name{Kolmogorov}, \name{Riesz}}\\
    Für $p \in [1, \infty)$ ist $A \subset L^p(\real^m, \KK)$ relativ kompakt genau dann, wenn
    \begin{enumerate}
        \item
        $\sup_{f \in A} \norm{f}_{L^p} < \infty$,

        \item
        $\sup_{f \in A} \norm{f(\cdot + h) - f(\cdot)}_{L^p}
        \xrightarrow{h \in \real^m,\; \norm{h} \to 0} 0$ und

        \item
        $\sup_{f \in A} \norm{f}_{L^p(\real^m \setminus B_R(0))}
        \xrightarrow{R \to \infty} 0$.
    \end{enumerate}
\end{Satz}

\pagebreak

\section{%
    Lineare Abbildungen in normierten Räumen%
}

\subsection{%
    Stetigkeit und Beispiele%
}

\begin{Satz}{Äquivalenz für Stetigkeit bei linearen Operatoren}\\
    Seien $(E, \norm{\cdot}_E)$ und $(F, \norm{\cdot}_F)$ normierte Räume
    sowie $T\colon E \rightarrow F$ eine lineare Abbildung.\\
    Dann sind äquivalent:
    \begin{enumerate}
        \item
        $T$ ist stetig.

        \item
        $T$ ist stetig in $0$.

        \item
        Aus $(x_n)_{n \in \natural}$ Folge in $E$ mit $x_n \to 0$ folgt $Tx_n \to 0$.

        \item
        $\exists_{\alpha \ge 0}\; TB_E \subset \alpha B_F$, wobei
        $B_E := \{x \in E \;|\; \norm{x} \le 1\}$ und
        $\alpha B_F := \{y \in F \;|\; \norm{y} \le \alpha\}$.

        \item
        $T$ ist \begriff{beschränkt},
        d.\,h. $\exists_{\beta \ge 0} \forall_{x \in E}\; \norm{Tx}_F \le \beta \norm{x}_E$.
    \end{enumerate}
\end{Satz}

\linie

\begin{Def}{Dualraum}
    Sei $(E, \norm{\cdot}_E)$ ein normierter Raum.\\
    Dann heißt $E' := \{T\colon E \rightarrow \KK \;|\; T \text{ linear und stetig}\}$
    \begriff{Dualraum} von $E$.
\end{Def}

\begin{Bsp}
    \begin{enumerate}[label=\emph{(\alph*)}]
        \item
        Seien $E := (\real^n, \norm{\cdot}_2)$ und $F := (\real^m, \norm{\cdot}_2)$.
        Dann ist jede lineare Abbildung $T\colon E \rightarrow F$ stetig und kann
        durch eine Matrix dargestellt werden.
        Dasselbe gilt auch für alle anderen Normen (wegen der Normäquivalenz).

        \item
        Seien $E := (\C^0([a, b]), \norm{\cdot}_{\C^0})$ und $T\colon E \rightarrow \KK$
        mit $Tf := \int_a^b f(s)\ds$ (wobei $a, b \in \real$ mit $a \le b$).
        $T$ ist linear und stetig und damit $T \in E'$.
        Außerdem ist $V\colon E \rightarrow E$, $f \mapsto Vf$ mit
        $(Vf)(t) := \int_a^t f(s)\ds$ linear und stetig, denn
        $\norm{Vf}_{\C^0} \le (b - a) \norm{f}_{\C^0}$.
        $V$ ist auch stetig als Abbildung von
        $(\C^0([a, b]), \norm{\cdot}_{\C^0})$ nach
        $(\C^1([a, b]), \norm{\cdot}_{\C^1})$.
    \end{enumerate}
\end{Bsp}

\pagebreak

\subsection{%
    Lineare, stetige Abbildungen%
}

\begin{Def}{Raum der linearen, stetigen Abbildungen}
    Seien $(E, \norm{\cdot}_E)$ und $(F, \norm{\cdot}_F)$ normierte Räume.
    Dann heißt
    $\Lin(E, F) := \{T\colon E \rightarrow F \;|\; T \text{ linear und stetig}\}$
    der \begriff{Raum der linearen, stetigen Abbildungen} von $E$ nach $F$.
    Man schreibt $\Lin(E) := \Lin(E, E)$.
\end{Def}

\begin{Satz}{Operatornorm}
    Für $T \in \Lin(E, F)$ sei\\
    $\norm{T} := \sup_{x \in B_E} \norm{Tx}_F =
    \sup_{x \in \interior{B_E}} \norm{Tx}_F = \sup_{x \in \partial B_E} \norm{Tx}_F =
    \sup_{x \in E \setminus \{0\}} \frac{\norm{Tx}_F}{\norm{x}_E}$.\\
    Dann ist $\norm{\cdot}$ eine Norm auf $\Lin(E, F)$, die sog. \begriff{Operatornorm}.
    Ist $F$ vollständig, dann ist auch $(\Lin(E, F), \norm{\cdot})$ vollständig.
    Insbesondere ist der Dualraum $E'$ vollständig.
\end{Satz}

\begin{Bem}
    Das Supremum der Operatornorm muss auf dem Rand angenommen werden,
    denn würde es in $x \in E$ mit $\norm{x}_E < 1$ angenommen,
    dann wäre $\norm{Tx'}_F = \frac{\norm{Tx}_F}{\norm{x}_E} > \norm{Tx}_F$ mit
    $x' := \frac{x}{\norm{x}_E} \in \partial B_E$,
    d.\,h. wegen der Stetigkeit von $T$ gäbe es einen Punkt im Inneren von $B_E$,
    bei dem das Supremum überschritten wäre
    (zumindest, wenn $\norm{Tx}_F > 0$ -- falls das Supremum verschwindet, ist der
    Operator gleich dem Nulloperator).
\end{Bem}

\begin{Bsp}
    Sei $\psi \in \C^0([0, 1]^2)$.
    Dann ist $T\colon (\C^0([0, 1]), \norm{\cdot}_{\C^0}) \rightarrow
    (\C^0([0, 1]), \norm{\cdot}_{\C^0})$, $f \mapsto Tf$ mit\\
    $(Tf)(x) := \int_0^1 \psi(x, y) f(y)\dy$ linear und stetig und es gilt
    $\norm{T} = \sup_{x \in [0, 1]} \int_0^1 |\psi(x, y)| \dy$.
\end{Bsp}

\linie

\begin{Lemma}{Komposition von linearen, stetigen Abbildungen}\\
    Seien $(E, \norm{\cdot}_E)$, $(F, \norm{\cdot}_F)$ und $(G, \norm{\cdot}_G)$
    normierte Räume,
    $B \in \Lin(E, F)$ und $A \in \Lin(F, G)$.\\
    Dann gilt:
    \begin{enumerate}
        \item
        $A \circ B \in \Lin(E, G)$ und
        $\norm{A \circ B} \le \norm{A} \cdot \norm{B}$

        \item
        $M_r\colon \Lin(E, F) \rightarrow \Lin(E, G)$, $T \mapsto A \circ T$ und
        $M_\ell\colon \Lin(F, G) \rightarrow \Lin(E, G)$, $S \mapsto S \circ B$
        sind linear und stetig, wobei
        $\norm{M_r} \le \norm{A}$ und $\norm{M_\ell} \le \norm{B}$.
    \end{enumerate}
\end{Lemma}

\begin{Satz}{\name{Neumann}sche Reihe}
    Seien $(E, \norm{\cdot}_E)$ ein Banachraum und $T \in \Lin(E)$ mit\\
    $\limsup_{n \to \infty} \norm{T^n}^{1/n} < 1$
    (z.\,B. erfüllt, wenn $\norm{T} < 1$).\\
    Dann ist $\id - T$ bijektiv und es gilt
    $(\id - T)^{-1} = \sum_{n=0}^\infty T^n \in \Lin(E)$
    (die Reihe konvergiert bzgl. der Operatornorm).
    Die Reihe $\sum_{n=0}^\infty T^n$ heißt \begriff{\name{Neumann}sche Reihe}.
\end{Satz}

\subsection{%
    Operatornormen in \texorpdfstring{$\real^n$}{ℝⁿ}%
}

\begin{Satz}{Operatornormen in $\real^n$}
    \begin{enumerate}
        \item
        Seien $E := (\real^n, \norm{\cdot}_\infty)$ und $A \in \Lin(E)$
        beschrieben durch die $n \times n$-Matrix $(a_{ij})_{i,j=1,\dotsc,n}$.
        Dann kann die zugehörige Operatornorm berechnet werden durch\\
        $\norm{A} = \max_{i=1,\dotsc,n} \sum_{j=1}^n |a_{ij}|$,
        sie heißt \begriff{Zeilensummennorm} $\norm{A}_\infty$.

        \item
        Seien $E := (\real^n, \norm{\cdot}_1)$ und $A \in \Lin(E)$
        beschrieben durch die $n \times n$-Matrix $(a_{ij})_{i,j=1,\dotsc,n}$.
        Dann kann die zugehörige Operatornorm berechnet werden durch\\
        $\norm{A} = \max_{j=1,\dotsc,n} \sum_{i=1}^n |a_{ij}|$,
        sie heißt \begriff{Spaltensummennorm} $\norm{A}_1$.

        \item
        Seien $E := (\real^n, \norm{\cdot}_2)$ und $A \in \Lin(E)$
        beschrieben durch die $n \times n$-Matrix $(a_{ij})_{i,j=1,\dotsc,n}$.
        Dann ist die zugehörige Operatornorm gleich der Wurzel des größten Eigenwerts
        der symmetrischen, positiv definiten Matrix $A^T A$,
        sie heißt \begriff{Spektralnorm} $\norm{A}_2$.
    \end{enumerate}
\end{Satz}

\pagebreak

\section{%
    Dif"|ferentiation und Integration in Banachräumen%
}

\subsection{%
    \name{Gâteaux}- und \name{Fréchet}-Ableitung%
}

\begin{Def}{\name{Gâteaux}-Dif"|ferenzierbarkeit}
    Seien $(X, \norm{\cdot}_X)$ und $(Y, \norm{\cdot}_Y)$ Banachräume,
    $U \subset X$ of"|fen, $x \in U$ und $F\colon X \rightarrow Y$ eine Abbildung.
    Dann heißt $F$ \begriff{\name{Gâteaux}-dif"|ferenzierbar} in $x$,
    falls die \begriff{\name{Gâteaux}-Ableitung} $DF(x)[v]$
    an der Stelle $x$ in Richtung $v$ für alle $v \in X$ existiert, wobei\\
    $DF(x)[v] := \lim_{h \to 0} \frac{F(x + hv) - F(x)}{h}$ mit $h \in \real$.
\end{Def}

\begin{Def}{\name{Fréchet}-Dif"|ferenzierbarkeit}
    $F$ heißt \begriff{\name{Fréchet}-dif"|ferenzierbar} in $x$, falls die\\
    \begriff{\name{Fréchet}-Ableitung} $JF(x) \in \Lin(X, Y)$
    an der Stelle $x$ existiert, wobei\\
    $\lim_{h \to 0} \frac{\norm{F(x + h) - F(x) - JF(x)[h]}_Y}{\norm{h}_X} = 0$ mit $h \in X$.
\end{Def}

\begin{Bem}
    Gâteaux- und Fréchet-Ableitung verallgemeinern die Richtungsableitung bzw. totale Ableitung
    aus der reellen Dif"|ferentialrechnung.
    Für $X = \real$ gilt $JF(x) = DF(x)[1]$,
    d.\,h. $JF(x)[v] = v \cdot DF(x)[1]$ für alle $v \in \real$.
    Mithilfe von Gâteaux- und Fréchet-Ableitung lassen sich zentrale Sätze aus der
    reellen Dif"|ferentialrechnung
    (z.\,B. der Satz von Taylor, der Satz über implizite Funktionen und die Sätze über die
    Berechnung von Extremstellen ohne oder mit Nebenbedingungen)
    auf den Fall von Banachräumen verallgemeinern.
\end{Bem}

\subsection{%
    \name{Riemann}-Integrale in Banachräumen
}

\begin{Def}{\name{Riemann}-Summe}
    Seien $(X, \norm{\cdot}_X)$ ein Banachraum, $a < b$ und
    $f\colon [a, b] \rightarrow X$ eine Abbildung.
    Seien außerdem $P = \{x_0, \dotsc, x_n\}$
    mit $a = x_0 < \dotsb < x_n = b$ eine \begriff{Partition} des Intervalls $[a, b]$
    und $\xi = (\xi_1, \dotsc, \xi_n)$ \begriff{Stützstellen}
    mit $\xi_k \in [x_{k-1}, x_k]$ für alle $k = 1, \dotsc, n$.
    Dann heißt $S(f, P, \xi) := \sum_{k=1}^n (x_k - x_{k-1}) f(\xi_k)$
    \begriff{\name{Riemann}-Summe} von $f$ zur Partition $P$ mit Stützstellen $\xi$.
\end{Def}

\begin{Def}{\name{Riemann}-integrierbar}
    $f$ heißt \begriff{\name{Riemann}-integrierbar}, falls der Grenzwert\\
    $\lim_{n \to \infty} S(f, P(n), \xi(n))$ für alle Folgen
    $(P(n), \xi(n))_{n \in \natural}$ von Partitionen $P(n)$ und Stützstellen $\xi(n)$,
    die $\lim_{n \to \infty} |P(n)| = 0$ erfüllen, existiert
    und unabhängig von den Folgen ist
    (dabei ist $|P| := \max_{k=1,\dotsc,n} (x_k - x_{k-1})$ die \begriff{Feinheit} der Partition
    $P$).
    In diesem Fall nennt man $\int_a^b f(x)\dx := \lim_{|P| \to 0} S(f, P, \xi)$
    \begriff{\name{Riemann}-Integral} von $f$ von $a$ bis $b$.
\end{Def}

\begin{Bem}
    Mithilfe dieses Integralbegrif"|fs lassen sich zentrale Sätze aus der reellen
    Integralrechnung auf den Fall von Banachräumen verallgemeinern, z.\,B. gilt:
    Jede stetige Funktion $f\colon [a, b] \rightarrow X$ ist Riemann-integrierbar.
    Außerdem kann man den lokalen Existenz- und Eindeutigkeitssatz von Picard-Lindelöf auf
    den Fall von gewöhnlichen Dif"|ferentialgleichungen mit Werten in Banachräumen verallgemeinern.
\end{Bem}

\pagebreak

\optpart{%
    \name{Hilbert}raumtheorie%
}

\chapter{%
    Orthogonale Projektionen%
}

\section{%
    Der Projektionssatz%
}

\begin{Satz}{Existenz und Eindeutigkeit des bestappr. Elements}\\
    Seien $H$ ein Hilbertraum und
    $A \subset H$ eine nicht-leere, abgeschlossene und \begriff{konvexe} Teilmenge, d.\,h.
    $\forall_{x, y \in A} \forall_{\lambda \in [0, 1]}\; \lambda x + (1 - \lambda) y \in A$.
    Dann gilt
    $\forall_{x_0 \in H} \exists!_{y_0 \in A}\;
    \norm{x_0 - y_0} = \dist(x_0, A)$.\\
    $y_0$ heißt \begriff{bestapproximierendes Element} an $x_0$ in $A$.
\end{Satz}

\begin{Satz}{Charakterisierung des bestappr. Elements als orthogonale Projektion}\\
    Seien $H$ ein Hilbertraum und $M \subset H$ ein Unterraum.
    Dann ist $y_0 \in M$ bestapproximierend an $x_0 \in H$ in $M$ genau dann, wenn
    $\forall_{y \in M}\; \sp{x_0 - y_0, y} = 0$
    (also $x_0 - y_0 \in M^\orth$).\\
    $y_0$ heißt in diesem Fall die \begriff{orthogonale Projektion} von $x_0$ auf $M$.
\end{Satz}

\begin{Satz}{Projektionssatz}
    Seien $H$ ein Hilbertraum und $M \subset H$ ein abgeschlossener Unterraum.\\
    Dann gilt $\forall_{x_0 \in H} \exists!_{y_0 \in M} \exists!_{y_1 \in M^\orth}\;
    x_0 = y_0 + y_1$,
    also $H = M \oplus M^\orth$ (direkte Summe).\\
    Dabei ist $M^\orth := \{y \in H \;|\; \forall_{x \in M}\; \sp{x, y} = 0\}$
    das \begriff{orthogonale Komplement} von $M$ in $H$.
\end{Satz}

\begin{Kor}
    Zu jedem abgeschlossenen, echten Unterraum $M$ eines Hilbertraums $H$ ($M \not= H$)
    gibt es ein $z_0 \in M^\orth$ mit $z_0 \not= 0$ ($M^\orth \not= \{0\}$).
\end{Kor}

\begin{Bem}
    Für jeden Unterraum $M \subset H$ gilt stets $M \cap M^\orth = \{0\}$.\\
    Außerdem ist
    $M^\orth = \bigcap_{x \in M} \{y \in H \;|\; \sp{x, y} = 0\}
    = \bigcap_{x \in M} \sp{x, \cdot}^{-1}(0)$
    abgeschlossen.
\end{Bem}

\pagebreak

\section{%
    Orthonormalsysteme%
}

\begin{Def}{Orthonormalsystem}
    Seien $(E, \sp{\cdot, \cdot})$ ein Skalarproduktraum und $e_i \in E$ für $i \in I$
    ($I \not= \emptyset$ Indexmenge).
    Die Familie $(e_i)_{i \in I}$ heißt \begriff{Orthonormalsystem (ONS)}, falls
    $\forall_{i, j \in I}\; \sp{e_i, e_j} = \delta_{ij}$.
\end{Def}

\begin{Lemma}{orthogonale Projektion durch endliche ONS}
    Sei $(e_i)_{i \in I}$ ein endliches ONS in $E$.\\
    Dann liefert die Zuordnung $P_I\colon E \rightarrow E_I$,
    $P_I(x) := \sum_{i \in I} \sp{x, e_i} e_i$ die orthogonale Projektion von $x$ auf
    $E_I := [\{e_i \;|\; i \in I\}]$
    und es gilt $\forall_{x \in E}\; \norm{x}^2 =
    \sum_{i \in I} |\sp{x, e_i}|^2 + \norm{x - P_I(x)}^2$.\\
    Außerdem sind die $(e_i)_{i \in I}$ linear unabhängig.
\end{Lemma}

\begin{Lemma}{\name{Bessel}sche Ungleichung}
    Sei $(e_i)_{i \in I}$ ein beliebiges ONS in $E$.\\
    Dann gilt $\forall_{x \in E}\; \sum_{i \in I} |\sp{x, e_i}|^2 \le \norm{x}^2$.
\end{Lemma}

\begin{Satz}{Äquivalenzen für abzählbare ONS}
    Für jedes höchstens abzählbare ONS $(e_i)_{i \in I}$ ($I \subset \natural$)
    in einem Skalarproduktraum $(E, \sp{\cdot, \cdot})$ sind äquivalent:
    \begin{enumerate}
        \item
        $[\{e_i \;|\; i \in I\}]$ ist dicht in $E$.

        \item
        $\forall_{x \in E}\; x = \sum_{i \in I} \sp{x, e_i} e_i$

        \item
        $\forall_{x \in E}\; \norm{x}^2 = \sum_{i \in I} |\sp{x, e_i}|^2$
        (\begriff{\name{Parseval}sche Gleichung})
    \end{enumerate}
    Ist $E$ ein Hilbertraum, dann ist zusätzlich jede dieser Aussagen äquivalent zu
    \begin{enumerate}[start=4]
        \item
        $(e_i)_{i \in I}$ \begriff{maximal}, d.\,h. es gibt kein $y \in E \setminus \{0\}$ mit
        $\forall_{i \in I}\; \sp{y, e_i} = 0$.
    \end{enumerate}
\end{Satz}

\begin{Bem}
    Wenn die Parsevalsche Gleichung oder eine der äquivalenten Aussagen gilt,
    so spricht man auch oft von einer
    \begriff{Orthonormalbasis (ONB)} $(e_i)_{i \in I}$
    (i.\,A. aber keine Vektorraum-Basis)
    oder einem \begriff{vollständigen ONS}.
    In diesem Fall gilt
    $\norm{\sum_{i \in I} \alpha_i e_i}^2 = \sum_{i \in I} |\alpha_i|^2$
    für jede Folge $(\alpha_i)_{i \in I}$ in $\KK$, wie man sich leicht herleiten kann
    (Verallgemeinerung des Satzes von Pythagoras).
\end{Bem}

\linie

\begin{Def}{separabel}
    Sei $(M, d)$ ein metrischer Raum.
    Eine Teilmenge $T \subset M$ heißt \begriff{separabel}, falls es eine höchstens abzählbare
    Teilmenge $A \subset M$ gibt, die dicht in $T$ ist.
\end{Def}

\begin{Satz}{Äquivalenz für separable Hilberträume}
    Sei $H$ ein Hilbertraum. Dann sind äquivalent:
    \begin{enumerate}
        \item
        $H$ ist separabel.

        \item
        $H$ besitzt ein maximales, höchstens abzählbares ONS.
    \end{enumerate}
\end{Satz}

\begin{Bsp}
    \begin{enumerate}[label=\emph{(\alph*)}]
        \item
        Sei $H := L^2([0, 2\pi], \real)$.
        Dann ist $\{\frac{1}{\sqrt{2\pi}}, g_1, h_1, g_2, h_2, \dotsc\}$
        mit $g_n(x) := \frac{1}{\sqrt{\pi}} \cos(nx)$,\\
        $h_n(x) := \frac{1}{\sqrt{\pi}} \sin(nx)$
        eine abzählbare ONB.
        Es gilt für alle $f \in H$, dass\\
        $f(x) = \frac{1}{2\pi} \int_0^{2\pi} f(t)\dt +
        \frac{1}{\pi} \sum_{n=1}^\infty \left(\int_0^{2\pi} f(t)\cos(nt)\dt\right) \cos(nx) \;+$\\
        $+\; \frac{1}{\pi} \sum_{n=1}^\infty \left(\int_0^{2\pi} f(t)\sin(nt)\dt\right) \sin(nx)$,
        wobei diese Reihen bzgl. der $L^2$-Norm konvergieren.

        \item
        Sei $H := L^2([0, 2\pi], \complex)$.
        Dann ist $(f_n)_{n \in \integer}$ mit $f_n(x) := \frac{1}{\sqrt{2\pi}} e^{\iu nx}$
        eine abzählbare ONB.
        Es gilt für alle $f \in H$, dass
        $f(x) = \frac{1}{2\pi} \sum_{n=-\infty}^{+\infty}
        \left(\int_0^{2\pi} f(t) e^{-\iu nt} \dt\right) e^{\iu nx}$,
        wobei diese Reihe bzgl. der $L^2$-Norm konvergiert.
    \end{enumerate}
\end{Bsp}

\pagebreak

\section{%
    Der \name{Riesz}sche Darstellungssatz%
}

\begin{Bem}
    Jede lineare Abbildung $\ell\colon \real^n \rightarrow \real$ lässt sich durch eine
    Matrix $L = \smallpmatrix{L_1 & \cdots & L_n}$ mit $L \in \real^{1 \times n}$ darstellen,
    d.\,h. es gilt $\ell(x) = Lx = \sp{\smallpmatrix{L_1 \\ \vdots \\ L_n},
    \smallpmatrix{x_1 \\ \vdots \\ x_n}}_2 = \sp{L^T, x}$ mit $L^T \in \real^n$.
    Es ist überraschend, dass sich das auf Hilberträume verallgemeinern lässt.
\end{Bem}

\begin{Satz}{\name{Riesz}scher Darstellungssatz}
    Seien $H$ ein Hilbertraum und $\ell \in H'$.\\
    Dann gibt es genau ein $y \in H$ mit $\forall_{x \in H}\; \ell(x) = \sp{x, y}$.
    Es gilt $\norm{\ell} = \norm{y}$.
\end{Satz}

\begin{Kor}\\
    Seien $H$ ein Hilbertraum und $\R\colon H \rightarrow H'$, $y \mapsto \R y$ mit
    $(\R y)(x) := \sp{x, y}$ für $x \in H$.\\
    Dann ist $\R$ für $\KK = \real$ ein isometrischer Isomorphismus und\\
    für $\KK = \complex$ ein \begriff{isometrischer, konjugiert linearer Isomorphismus}
    (d.\,h. $\R$ ist eine Isometrie,\\
    $\forall_{y_1, y_2 \in H} \forall_{\alpha \in \complex}\;
    \R(y_1 + \alpha y_2) = \R y_1 + \overline{\alpha} \R y_2$,
    $\R$ ist bijektiv und $\R, \R^{-1}$ sind stetig).
\end{Kor}

\linie

\begin{Satz}{Charakterisierung des darstellenden Elements}\\
    Seien $H$ ein Hilbertraum, $y \in H$ und $\ell \in H'$.\\
    Dann gilt $\forall_{x \in H}\; \ell(x) = \sp{x, y}$
    genau dann, wenn\\
    $\frac{1}{2} \sp{y, y} - \Re(\ell(y)) =
    \min_{x \in H} \left(\frac{1}{2} \sp{x, x} - \Re(\ell(x))\right)$.
\end{Satz}

\linie

\begin{Satz}{Satz von \upshape\,\!\name{Lax}-\name{Milgram}}
    Seien $H$ ein Hilbertraum und $a\colon H \times H \rightarrow \KK$ \begriff{sesquilinear}\\
    (d.\,h. linear im ersten und konjugiert linear im zweiten Argument).\\
    Außerdem gebe es Konstanten $c_0, C_0 \in \real$ mit $0 < c_0 < C_0 < \infty$, sodass
    \begin{enumerate}
        \item
        $\forall_{x, y \in H}\; |a(x, y)| \le C_0 \norm{x} \norm{y}$
        (Stetigkeit von $a$) und

        \item
        $\forall_{x \in H}\; \Re(a(x, x)) \ge c_0 \norm{x}^2$
        (\begriff{Koerzitivität} von $a$).
    \end{enumerate}
    Dann gibt es zu jedem $\ell \in H'$ genau ein $z \in H$ mit
    $\forall_{y \in H}\; \ell(y) = a(y, z)$.
    Es gilt $\norm{z} \le \frac{1}{c_0} \norm{\ell}$.\\
    Außerdem existiert genau eine Abbildung $A\colon H \rightarrow H$ mit
    $\forall_{x, y \in H}\; a(y, x) = \sp{y, Ax}$.\\
    $A$ ist ein Isomorphismus mit $\norm{A} \le C_0$ und
    $\norm{A^{-1}} \le \frac{1}{c_0}$.
\end{Satz}

\pagebreak

\section{%
    Anwendungen bei elliptischen RWP und \name{Sobolev}räume%
}

\subsection{%
    \name{Poisson}-Gleichung mit \name{Dirichlet}-Randbedingungen%
}

\begin{Bem}
    Sei $\Omega \subset \real^n$ ein beschränktes \begriff{Normalgebiet},
    d.\,h. eine beschränkte, of"|fene, nicht-leere und zusammenhängende Teilmenge von $\real^n$,
    sodass der Gaußsche Integralsatz anwendbar ist.
    Für $f \in \C^0(\overline{\Omega})$ sei außerdem
    $E(w) := \int_\Omega (\frac{1}{2} |\nabla w|^2 - fw) \dx$.\\
    Zusätzlich sei $\A_g := \C^1_g(\overline{\Omega}) \cap \C^2(\Omega)$,
    wobei $g \in \C^0(\partial\Omega)$ und
    $\C^1_g(\overline{\Omega}) := \{w \in \C^1(\overline{\Omega}) \;|\;
    w|_{\partial \Omega} = g\}$.
    
    Das \begriff{Minimumproblem} lautet nun:
    Nimmt $E$ auf $\A_g$ ein Minimum an?
    
    Beispiele aus der Physik beinhalten
    eingespannte Membranen im Schwerefeld der Erde,
    elektrische Potentiale oder
    stationäre Temperaturverteilungen.
\end{Bem}

\linie

\begin{Bem}
    Die Lösung (falls existent) lässt sich wie folgt charakterisieren.
\end{Bem}

\begin{Satz}{Charakterisierung der Lösung des Minimumproblems}
    Sei $u \in \A_g$.
    Dann sind äquivalent:
    \begin{enumerate}
        \item
        $E(u) = \min_{w \in \A_g} E(w)$
        
        \item
        $\forall_{\varphi \in \C^\infty_c(\Omega)}\;
        \int_\Omega (\nabla u \nabla \varphi - f \varphi)\dx = 0$
        
        \item
        $-\Delta u = f$ in $\Omega$, $u = g$ auf $\partial \Omega$
    \end{enumerate}
\end{Satz}

\begin{Lemma}{Fundamentallemma der Variationsrechnung}
    Sei $f \in \C^0(\Omega)$.\\
    Dann gilt $\forall_{\varphi \in \C^\infty_c(\Omega)}\; \int_\Omega f\varphi \dx = 0$
    genau dann, wenn $f \equiv 0$.
\end{Lemma}

\begin{Lemma}{\name{Green}sche Formel}
    Für alle $u, w \in \C^2(\overline{\Omega})$ gilt\\
    $\int_\Omega \nabla u \nabla w \dx = -\int_\Omega (\Delta u) w \dx
    + \int_{\partial \Omega} \frac{\partial u}{\partial \nu} w do$,
    wobei $\frac{\partial u}{\partial \nu}$ die Ableitung von $u$ in Richtung des äußeren
    Einheitsnormalenvektors ist.
\end{Lemma}

\linie

\begin{Bem}
    "`\emph{(1)} $\Rightarrow$ \emph{(2)}"' kann man wie folgt beweisen:
    Für $\varphi \in \C^\infty_c(\Omega)$ und $h > 0$ gilt\\
    $E(u) \le E(u \pm h\varphi)
    = \int_\Omega (\frac{1}{2} |\nabla (u \pm h\varphi)|^2 - f(u \pm h\varphi)) \dx$\\
    $= \int_\Omega (\frac{1}{2} |\nabla u|^2 + \frac{h^2}{2} |\nabla \varphi|^2 \pm
    h \nabla u \nabla \varphi - fu \mp h f\varphi) \dx
    = E(u) \pm h \int_\Omega (\nabla u \nabla \varphi - f\varphi) \dx +
    \frac{h^2}{2} \int_\Omega |\nabla \varphi|^2 \dx$,
    also $0 \le \pm\int_\Omega (\nabla u \nabla \varphi - f\varphi) \dx +
    \frac{h}{2} \int_\Omega |\nabla \varphi|^2 \dx$.
    Für $h \to 0$ fällt der zweite Summand weg und man erhält
    $0 \le \pm\int_\Omega (\nabla u \nabla \varphi - f\varphi) \dx$.
    
    "`\emph{(2)} $\iff$ \emph{(3)}"' sieht man wie folgt:
    Mit der Greenschen Formel ist (2) äquivalent zu\\
    $\forall_{\varphi \in \C^\infty_c(\Omega)}\;
    0 = \int_\Omega (\nabla u \nabla \varphi - f\varphi) \dx
    = \int_\Omega (-\Delta u - f) \varphi \dx$, weil das Integral über $\partial \Omega$
    wegfällt (da $\varphi = 0$ auf $\partial \Omega$).
    Nach dem Fundamentallemma der Variationsrechnung ist dies äquivalent zu $-\Delta u = f$
    in $\Omega$.
    $u = g$ auf $\partial G$ gilt immer, da $u \in \A_g$ nach Voraussetzung.
    
    "`\emph{(3)} $\Rightarrow$ \emph{(1)}"' zeigt man folgendermaßen:
    Für $w \in \A_g$ beliebig gilt nach der Greenschen Formel
    $\int_\Omega (\nabla u \nabla (u - w) - f(u - w))\dx
    = \int_\Omega ((-\Delta u)(u - w) - f(u - w)) \dx +
    \int_{\partial\Omega} \frac{\partial u}{\partial \nu} (u - w) do = 0$,
    weil $-\Delta u = f$ in $\Omega$ und $u|_{\partial\Omega} = w|_{\partial\Omega} = g$.
    Daraus folgt\\
    $\int_\Omega (|\nabla u|^2 - fu) \dx
    = \int_\Omega (\nabla u \nabla w - fw) \dx
    \le \frac{1}{2} \int_\Omega |\nabla u|^2 \dx + \frac{1}{2} \int_\Omega |\nabla w|^2 \dx -
    \int_\Omega fw \dx$ wegen der Ungleichung
    $0 \le |\nabla u - \nabla w|^2 = |\nabla u|^2 + |\nabla w|^2 - 2 \nabla u \nabla w$.\\
    Damit gilt
    $E(u) = \int_\Omega (\frac{1}{2} |\nabla u|^2 - fu) \dx \le
    \int_\Omega (\frac{1}{2} |\nabla w|^2 - fw) \dx = E(w)$.
\end{Bem}

\linie

\begin{Bem}
    Notwendige Bedingung für die Existenz einer Lösung von \emph{(3)}
    (\begriff{\name{Poisson}-Gleichung mit inhomogenen \name{Dirichlet}-Randbedingungen})
    ist die Existenz einer Funktion $u_g \in \A_g$
    (d.\,h. $\A_g = \C^1_g(\overline{\Omega}) \cap \C^2(\Omega) \not= \emptyset$).
    Existiert eine solche Funktion,
    dann ist \emph{(3)} äquivalent zu
    $-\Delta \widetilde{u} = \widetilde{f}$ in $\Omega$, $\widetilde{u} = 0$ auf $\partial \Omega$
    mit $\widetilde{u} := u - u_g$, $\widetilde{f} := f + \Delta u_g$.
    Daher genügt es, wenn im Folgenden nur homogene Dirichlet-Randbedingungen (also $g \equiv 0$)
    betrachtet werden.
    (Achtung: $\C^1_g = \C^1_0$ darf nicht mit $\C^1_c$ verwechselt werden!)
\end{Bem}

\linie
\pagebreak

\begin{Bem}
    Nun zeigt man, dass das Minimum überhaupt existiert.
\end{Bem}

\begin{Satz}{\name{Poincaré}-Ungleichung}
    Sei $\Omega \subset \real^n$ ein Gebiet, das zwischen zwei parallelen Hyperebenen
    mit Abstand $C$ liegt.
    Dann gilt $\forall_{u \in \C^1_0(\overline{\Omega})}\;
    \norm{u}_{L^2} \le \frac{C}{\sqrt{2}} \norm{\nabla u}_{L^2}$.
\end{Satz}

\begin{Bem}
    Dabei gilt $\norm{\nabla u}_{L^2}^2
    = \sum_{i=1}^n \int_\Omega |\partial_{x_i} u|^2 \dx
    = \sum_{i=1}^n \norm{\partial_{x_i} u}_{L^2}^2$.
\end{Bem}

\begin{Lemma}{$\varepsilon$-Ungleichung}
    Für $a, b \in \real$ und $\varepsilon > 0$ gilt
    $ab \le \varepsilon a^2 + \frac{b^2}{4\varepsilon}$.
\end{Lemma}

\begin{Satz}{Beschränktheit nach unten}
    $E$ ist auf $\A_0$ nach unten beschränkt.
\end{Satz}

\linie

\begin{Bem}
    Da $E$ auf $\A_0$ nach unten beschränkt ist, existiert eine Minimalfolge
    $(u_n)_{n \in \natural}$ in $\A_0$.
    Weil $\A_0$ konvex ist, kann man wie im Beweis des Projektionssatzes mithilfe der
    Parallelogrammgleichung zeigen, dass $(\partial_{x_i} u_n)_{n \in \natural}$ für alle
    $i = 1, \dotsc, n$ eine Cauchy-Folge bzgl. $\norm{\cdot}_{L^2}$ ist.
    Aufgrund der Poincaré-Ungleichung folgt, dass auch $(u_n)_{n \in \natural}$ eine
    Cauchy-Folge bzgl. $\norm{\cdot}_{L^2}$ ist.
    
    ($(u_n)_{n \in \natural}$ ist auch eine Cauchy-Folge bzgl. der Norm $\norm{\cdot}_{H^1}$
    mit $\norm{f}_{H^1} := \norm{f}_{L^2} + \norm{\nabla f}_{L^2}$
    sowie bzgl. der (in diesem Fall zur $H_1$-Norm äquivalenten) Norm $\norm{\cdot}_{H^1_0}$
    mit $\norm{f}_{H^1_0} := \norm{\nabla f}_{L^2}$.
    Allerdings ist $\A_0$ bzgl. dieser Normen nicht vollständig.)
    
    $L^2$ ist vollständig,
    daher existieren $u \in L^2$ mit $u_n \xrightarrow{\norm{\cdot}_{L^2}} u$ und
    "`$\partial_{x_i} u$"' mit $\partial_{x_i} u_n \xrightarrow{\norm{\cdot}_{L^2}}
    \partial_{x_i} u$.
    "`$\partial_{x_i} u$"' ist aber nur eine Schreibweise, i.\,A. besitzt $u$ keine partiellen
    Ableitungen.
    Zwischen $u$ und den Funktionen "`$\partial_{x_i} u$"' besteht folgende Beziehung:
    $\forall_{\varphi \in \C^\infty_c(\Omega)}\;
    \int_\Omega (\partial_{x_i} u) \varphi \dx = -\int_\Omega u \partial_{x_i} \varphi \dx$
    (weil $\int_\Omega (\partial_{x_i} u) \varphi \dx
    = \lim_{n \to \infty} \int_\Omega (\partial_{x_i} u_n) \varphi \dx
    = -\lim_{n \to \infty} \int_\Omega u_n (\partial_{x_i} \varphi) \dx
    = -\int_\Omega u \partial_{x_i} \varphi \dx$).
    Dies motiviert die Definition der Sobolevräume.
\end{Bem}

\subsection{%
    \name{Sobolev}räume und schwache Ableitungen%
}

\begin{Def}{\name{Sobolev}raum}
    Seien $\Omega \subset \real^n$ of"|fen, $m \in \natural$ und $p \in [1, \infty]$.\\
    Dann heißt der Vektorraum
    $W^{m,p}(\Omega) := \{f \in L^p(\Omega) \;|\; \forall_{s \in \natural_0^n,\, |s| \le m}
    \exists_{f^{(s)} \in L^p(\Omega)}\; f^{(0)} = f,$\\
    $\forall_{\varphi \in \C^\infty_c(\Omega)}\;
    \int_\Omega (\partial_x^s \varphi) f \dx = (-1)^{|s|} \int_\Omega \varphi f^{(s)} \dx\}$
    \begriff{\name{Sobolev}raum} der Ordnung $m$ mit Exponent $p$.\\
    $W^{m,p}(\Omega)$ wird mit der Norm
    $\norm{f}_{W^{m,p}(\Omega)} := \sum_{|s| \le m} \norm{f^{(s)}}_{L^p(\Omega)}$ versehen.
    Für $p = 2$ schreibt man auch $H^m(\Omega) := W^{m,2}(\Omega)$ bzw.
    $\norm{\cdot}_{H^m(\Omega)} := \norm{\cdot}_{W^{m,2}(\Omega)}$.
\end{Def}

\begin{Def}{schwache Ableitung}
    Die Funktionen $f^{(s)}$ für $|s| \ge 1$ heißen \begriff{schwache Ableitungen} von $f$
    und werden mit $\partial_x^s f := f^{(s)}$ bezeichnet.
\end{Def}

\begin{Bem}
    Eine alternative Definition der Norm lautet
    $\norm{f}_{W^{m,p}(\Omega)}' :=
    \left(\sum_{|s| \le m} \norm{\partial_x^s f}_{L^p(\Omega)}^p\right)^{1/p}$
    (bzw. für $p = \infty$ das Maximum
    $\norm{f}_{W^{m,\infty}(\Omega)}' :=
    \max_{|s| \le m} \norm{\partial_x^s f}_{L^\infty(\Omega)}$).
    Allerdings kann man zeigen, dass
    $\norm{\cdot}_{W^{m,p}(\Omega)}$ und $\norm{\cdot}_{W^{m,p}(\Omega)}'$ äquivalent sind.
    %Der Sobolevraum $W^{m,p}(\Omega)$ ist gerade die Vervollständigung bzgl.
    %$\norm{\cdot}_{W^{m,p}(\Omega)}$ derer Funktionen in $\C^\infty(\Omega)$,
    %deren partielle Ableitungen bis zur Ordnung $m$ in $L^p(\Omega)$ sind.
\end{Bem}

\begin{Def}{\name{Sobolev}raum mit Nullrandwerten}
    Der Raum $W_0^{m,p}(\Omega) := \overline{\C^\infty_c(\Omega)}^{\norm{\cdot}_{W^{m,p}(\Omega)}}$
    für\\
    $p \in [1, \infty)$ heißt
    \begriff{\name{Sobolev}raum mit (verallgemeinerten) Nullrandwerten}
    der Ordnung $m$ mit Exponent $p$.
    Für $p = 2$ schreibt man auch $H_0^m(\Omega) := W_0^{m,2}(\Omega)$.
\end{Def}

\begin{Bem}
    Für $m = 1$ gilt $W_0^{1,p}(\Omega) =
    \{f \in W^{1,p}(\Omega) \;|\; f|_{\partial\Omega} = 0\}$.\\
    Für $p = 2$ ist $\sp{f, g}_{H^m(\Omega)}
    := \sum_{|s| \le m} \sp{\partial_x^s f, \partial_x^s g}_{L^2(\Omega)}
    = \sum_{|s| \le m} \int_\Omega (\partial_x^s f) (\partial_x^s g) \dx$
    ein Skalarprodukt auf $H^m(\Omega)$.
    Für $m = 1$ und $p = 2$ ist
    $\sp{f, g}_{H^1_0(\Omega)}
    := \sp{\nabla f, \nabla g}_{L^2(\Omega)}
    := \sum_{i=1}^n \int_\Omega (\partial_x^{e_i} f) (\partial_x^{e_i} g) \dx$
    mit $e_i := (0, \dotsc, 0, 1, 0, \dotsc, 0) \in \natural_0^n$
    ein Skalarprodukt auf $H^1_0(\Omega)$.\\
    Es gilt
    $\sp{f, g}_{H^1(\Omega)}
    = \sp{f, g}_{L^2(\Omega)} + \sp{\nabla f, \nabla g}_{L^2(\Omega)}$.
\end{Bem}

\linie
\pagebreak

\begin{Satz}{schwache Ableitungen}
    \begin{enumerate}
        \item
        Alle schwachen Ableitungen sind eindeutig bestimmt (wenn sie existieren).
        
        \item
        Besitzt $f \in W^{m,p}(\Omega)$ eine partielle Ableitung $\partial_x^s f$ mit $|s| \le m$,
        dann stimmt $\partial_x^s f$ fast überall mit der schwachen Ableitung $f^{(s)}$ überein.
    \end{enumerate}
\end{Satz}

\begin{Lemma}{verallgemeinertes Fundamentallemma der Variationsrechnung}\\
    Seien $\Omega \subset \real^n$ of"|fen und $f \in L^1(\Omega)$.\\
    Dann gilt $\forall_{\varphi \in \C^\infty_c(\Omega)}\; \int_\Omega f\varphi \dx = 0$
    genau dann, wenn $f = 0$ f.ü.
\end{Lemma}

\begin{Satz}{Eigenschaften der \name{Sobolev}räume}
    \begin{enumerate}
        \item
        $(W^{m,p}(\Omega), \norm{\cdot}_{W^{m,p}(\Omega)})$ ist ein Banachraum.
        $(H^m(\Omega), \norm{\cdot}_{H^m(\Omega)})$ ist ein Hilbertraum.
        
        \item
        Für $p \in [1, \infty)$ ist $W^{m,p}(\Omega)$ separabel.
        
        \item
        $(W^{m,p}(\Omega), \norm{\cdot}_{W^{m,p}(\Omega)})$ ist
        (bis auf isometrische Isomorphie) die Vervollständigung der Räume
        $W^{m,p}(\Omega) \cap \C^\infty(\Omega) =
        \{f \in \C^\infty(\Omega) \;|\; \norm{f}_{W^{m,p}(\Omega)} < \infty\}$.
        
        \item
        Für $p \in [1, \infty)$ und alle $f \in W^{m,p}(\Omega)$ gibt es eine Folge
        $(f_n)_{n \in \natural}$ in $W^{m,p}(\Omega) \cap \C^\infty(\Omega)$ mit
        $f_n \xrightarrow{\norm{\cdot}_{W^{m,p}(\Omega)}} f$,
        es gilt also $W^{m,p}(\Omega) =
        \overline{W^{m,p}(\Omega) \cap \C^\infty(\Omega)}^{\norm{\cdot}_{W^{m,p}(\Omega)}}$
        für $p \in [1, \infty)$.
    \end{enumerate}
\end{Satz}

\subsection{%
    Schwache Lösung der \name{Poisson}-Gleichung mit \name{Dirichlet}-RB%
}

\begin{Satz}{verallgemeinerte \name{Poincaré}-Ungleichung}
    Sei $\Omega \subset \real^n$ ein Gebiet, das zwischen zwei parallelen Hyperebenen
    mit Abstand $C$ liegt.
    Dann gilt $\forall_{u \in H^1_0(\Omega)}\;
    \norm{u}_{L^2} \le \frac{C}{\sqrt{2}} \norm{\nabla u}_{L^2}$,\\
    wobei $\norm{\nabla u}_{L^2} :=
    \left(\sum_{i=1}^n \norm{\partial_{x_i} u}_{L^2}^2\right)^{1/2}$.
\end{Satz}

\begin{Kor}
    Die Normen $\norm{\cdot}_{H^1(\Omega)}$ und $\norm{\cdot}_{H_0^1(\Omega)}$ auf $H_0^1(\Omega)$
    sind äquivalent, wenn $\Omega$ ein Gebiet wie im vorherigen Satz ist.
\end{Kor}

\linie

\begin{Satz}{schwache Lösung}\\
    Seien $\Omega \subset \real^n$ ein beschränktes Normalgebiet, $f \in L^2(\Omega)$ und
    $E(w) := \int_\Omega (\frac{1}{2} |\nabla w|^2 - fw)\dx$.\\
    Dann besitzt $E$ auf $H_0^1(\Omega)$ eine eindeutige Minimalstelle $u$ und $u$ ist die
    eindeutige schwache Lösung des Dirichlet-Problems für die Poisson-Gleichung
    $-\Delta u = f$ in $\Omega$, $u = 0$ auf $\partial\Omega$, d.\,h. es gilt
    $\forall_{\varphi \in H_0^1(\Omega)}\; \int_\Omega (\nabla u \nabla \varphi - f\varphi)\dx = 0$.
\end{Satz}

\begin{Bem}
    Es gibt eine nur von $\Omega$ abhängige Konstante $C > 0$ mit
    $\norm{u}_{H^1} \le C \norm{f}_{L^2}$.
\end{Bem}

\pagebreak

\subsection{%
    \emph{Zusatz}: \name{Poisson}-Gleichung mit \name{Neumann}-Randbedingungen%
}

\begin{Bem}
    Sei $\Omega \subset \real^n$ ein beschränktes Normalgebiet.\\
    Für $f \in \C^0(\overline{\Omega})$ und $g \in \C^0(\partial\Omega)$ sei außerdem
    $E_g(w) := \int_\Omega (\frac{1}{2} |\nabla w|^2 - fw) \dx - \int_{\partial\Omega} gw do$.\\
    Zusätzlich sei $\A := \C^1(\overline{\Omega}) \cap \C^2(\Omega)$.
    
    Das \begriff{Minimumproblem} lautet nun:
    Nimmt $E_g$ auf $\A$ ein Minimum an?
\end{Bem}

\linie

\begin{Satz}{Charakterisierung der Lösung des Minimumproblems}
    Sei $u \in \A$.
    Dann sind äquivalent:
    \begin{enumerate}
        \item
        $E_g(u) = \min_{w \in \A} E_g(w)$
        
        \item
        $\forall_{\varphi \in \C^\infty(\Omega)}\;
        \int_\Omega (\nabla u \nabla \varphi - f \varphi)\dx -
        \int_{\partial\Omega} g\varphi do = 0$
        
        \item
        $-\Delta u = f$ in $\Omega$, $\frac{\partial u}{\partial\nu} = g$ auf $\partial\Omega$
    \end{enumerate}
    In diesem Fall gilt notwendigerweise $\int_\Omega f\dx + \int_{\partial\Omega} g do = 0$.
\end{Satz}

\linie

\begin{Satz}{\name{Poincaré}-Ungleichung mit Mittelwert}
    Sei $\Omega \subset \real^n$ ein beschränktes und konvexes Gebiet mit Durchmesser $h$.
    Dann gibt es ein $C > 0$ mit $\forall_{u \in \C^1(\overline{\Omega})}\;
    \norm{u - Mu}_{L^2} \le Ch \norm{\nabla u}_{L^2}$,
    wobei $Mu := \frac{\int_\Omega u\dx}{\int_\Omega 1\dx}$
    der \begriff{Mittelwert} von $u$ auf $\Omega$ ist.
\end{Satz}

\begin{Satz}{Beschränktheit nach unten}
    Seien $\Omega \subset \real^n$ ein beschränktes und konvexes Normalgebiet
    und $\int_\Omega f\dx + \int_{\partial\Omega} g do = 0$.
    Dann ist $E_g$ auf $\A$ nach unten beschränkt.
\end{Satz}

\linie

\begin{Satz}{schwache Lösung}
    Seien $\Omega \subset \real^n$ ein beschränktes und konvexes Normalgebiet,\\
    $f \in L^2(\Omega)$ mit $\int_\Omega f\dx = 0$ und
    $E_0(w) := \int_\Omega (\frac{1}{2} |\nabla w|^2 - fw)\dx$.\\
    Dann besitzt $E_0$ auf $H^1(\Omega)$ eine eindeutige Minimalstelle $u$ und $u$ ist die
    eindeutige schwache Lösung des Neumann-Problems für die Poisson-Gleichung
    $-\Delta u = f$ in $\Omega$,
    $\frac{\partial f}{\partial\nu} = 0$ auf $\partial\Omega$, d.\,h. es gilt
    $\forall_{\varphi \in H^1(\Omega)}\; \int_\Omega (\nabla u \nabla \varphi - f\varphi)\dx = 0$.
\end{Satz}

\subsection{%
    Verallgemeinerung auf elliptische Randwertprobleme%
}

\begin{Def}{elliptische DGL}
    Sei $\Omega \subset \real^n$ ein beschränktes Normalgebiet.\\
    Gesucht sind Funktionen $u \in \C^2(\Omega)$, die die \begriff{elliptische DGL}
    $-\div(A \nabla u + h) + bu + f = 0$\\
    (d.\,h. $-\sum_{i=1}^n \partial_{x_i}
    \!\left(\sum_{j=1}^n a_{ij} \partial_{x_j} u + h_i\right) + bu + f = 0$)
    erfüllen.\\
    Dabei ist $a_{ij}, h_i \in \C^1(\Omega)$ für $i, j = 1, \dotsc, n$,
    $f, b \in \C^0(\Omega)$ und $(a_{ij}(x))_{i,j=1,\dotsc,n}$ sei
    \begriff{gleichmäßig elliptisch in $x$}, d.\,h.
    $\exists_{c_0 > 0} \forall_{x \in \Omega} \forall_{\xi \in \real^n}\;
    \xi^T A(x) \xi = \sum_{i,j=1}^n a_{ij}(x) \xi_i \xi_j \ge c_0 |\xi|^2$.
    (Für jedes $c > 0$ und $x \in \Omega$ beschreibt die Menge
    $\left\{\xi \in \real^n \;\left|\; \xi^T A(x) \xi = c\right.\right\}$
    eine Ellipse.)\\
    Die Matrix $(a_{ij}(x))_{i,j=1,\dotsc,n}$ kann auch unsymmetrisch sein.
\end{Def}

\begin{Bem}
    Ohne zusätzliche Bedingungen sind elliptische DGL nicht eindeutig lösbar.
    Meist bekommt man die eindeutige Lösbarkeit durch Einführung von Randbedingungen.
    Es folgen die beiden Randbedingungen, die in der mathematischen Physik am häufigsten
    vorkommen.
\end{Bem}

\begin{Def}{\name{Dirichlet}-Randbedingungen}\\
    $u$ löst die elliptische DGL in $\Omega$ und erfüllt
    $u = g$ auf $\partial\Omega$ mit $g \in \C^0(\partial\Omega)$.
\end{Def}

\begin{Def}{\name{Neumann}-Randbedingungen}\\
    $u$ löst die elliptische DGL in $\Omega$ und erfüllt
    $-\nu (A \nabla u + h) =
    -\sum_{i=1}^n \nu_i \left(\sum_{j=1}^n a_{ij} \partial_{x_j} u + h_i\right) = g$ auf
    $\partial\Omega$ mit $g \in \C^0(\partial\Omega)$, wobei $\nu$ der äußere
    Einheitsnormalenvektor an $\partial\Omega$ ist.
\end{Def}

\linie
\pagebreak

\begin{Bem}
    Wie bei der Poisson-Gleichung führt man den Begriff einer schwachen Lösung ein.
    Seien dafür nun $a_{ij} \in L^\infty(\Omega)$ und $(a_{ij}(x))_{i,j=1,\dotsc,n}$
    erfülle die Bedingung der gleichmäßigen Elliptizität fast überall auf $\Omega$,
    $b \in L^\infty(\Omega)$ und $h_i, f \in L^2(\Omega)$.
    Aus denselben Gründen wie bei der Poisson-Gleichung genügt es, wenn man nur den Fall $g = 0$
    betrachtet.
\end{Bem}

\begin{Def}{schwache Lösung des \name{Dirichlet}-Problems}
    $u \in H_0^1(\Omega)$ heißt \begriff{schwache Lösung des \name{Dirichlet}-Problems},
    falls
    $\forall_{\varphi \in H_0^1(\Omega)}\;
    \int_\Omega (\nabla \varphi (A \nabla u + h) + \varphi(bu + f))\dx = 0$.
\end{Def}

\begin{Def}{schwache Lösung des \name{Neumann}-Problems}
    $u \in H^1(\Omega)$ heißt \begriff{schwache Lösung des \name{Neumann}-Problems},
    falls
    $\forall_{\varphi \in H^1(\Omega)}\;
    \int_\Omega (\nabla \varphi (A \nabla u + h) + \varphi(bu + f))\dx = 0$.
\end{Def}

\begin{Bem}
    Zusätzlich sei vorausgesetzt, dass $b \ge 0$ für das Dirichlet-Problem und
    $b \ge b_0 > 0$ für das Neumann-Problem gilt.
    Dann gilt folgender Satz.
\end{Bem}

\begin{Satz}{eindeutige Lösung von elliptischen DGL}
    Unter obigen Voraussetzungen existiert genau eine schwache Lösung des Dirichlet-
    bzw. des Neumann-Problems.
\end{Satz}

\begin{Bem}
    Unter zusätzlichen Regularitätsannahmen an die Daten $a_{ij}$, $h_i$, $b$, $f$ und
    $\partial\Omega$ kann man zeigen, dass die schwache Lösung so regulär ist,
    dass sie auch eine klassische Lösung ist.
    Beispielsweise folgt aus $a_{ij} \in \C^{m,1}(\Omega)$, $h_i \in H^{m+1}(\Omega)$,
    $f \in H^m(\Omega)$ und $\partial\Omega$ lokal als Graph von $\C^{m+1,1}$-Funktionen
    darstellbar, dass $u \in H^{m+2}(\Omega)$,
    und damit für hinreichend großes $m = m(n)$, dass $u \in \C^2(\Omega)$.
    Details siehe elliptische Regularitätstheorie ($L^2$-, $L^p$- und $\C^{0,\alpha}$-Theorie)
    mithilfe der Sobolevschen Einbettungssätze (siehe Funktionalanalysis 2).
\end{Bem}

\subsection{%
    \name{Ritz}-\name{Galerkin}-Approximation für elliptische RWP%
}

\begin{Satz}{\name{Ritz}-\name{Galerkin}-Approximation}
    Sei $u \in H^1_0(\Omega)$ bzw. $u \in H^1(\Omega)$ die schwache Lösung des
    Dirichlet- bzw. Neumann-Problems.
    Für $N \in \natural$ sei $X_N$ ein $N$-dimensionaler
    Unterraum von $H_0^1(\Omega)$ bzw. von $H^1(\Omega)$
    mit der Basis $\{\varphi_k^{(N)} \;|\; k = 1, \dotsc, N\}$.\\
    Dann existiert genau ein $u_N \in X_N$
    (\begriff{\name{Ritz}-\name{Galerkin}-Approximation}), sodass\\
    $\forall_{\varphi \in X_n}\;
    \int_\Omega (\nabla \varphi (A \nabla u + h) + \varphi(bu_N + f))\dx = 0$.\\
    Es gilt $u_N = \sum_{k=1}^N u_{N,k} \varphi_k^{(N)}$,
    wobei sich die Koef"|fizienten $u_{N,k} \in \real$ als eindeutige Lösung des LGS
    $\sum_{\ell=1}^N a_{k\ell}^{(N)} u_{N,\ell} + c_k^{(N)} = 0$, $k = 1, \dotsc, N$ mit
    $c_k^{(N)} := \int_\Omega (\nabla \varphi_k^{(N)} h + \varphi_k^{(N)} f)\dx$ und
    $a_{k\ell}^{(N)} := \int_\Omega (A \nabla \varphi_k^{(N)} \nabla \varphi_\ell^{(N)} + b\varphi_k^{(N)} \varphi_\ell^{(N)}) \dx$
    bestimmen lassen.
\end{Satz}

\begin{Bem}
    Die Nachweis der Struktur des LGS erfolgt durch direktes Nachrechnen.
    Der Beweis der eindeutigen Existenz von $u_N$ kann man mit Lax-Milgram
    (angewendet im Hilbertraum $X_N$) durchführen
    oder man zeigt, dass die Voraussetzungen an $a_{ij}$,
    insbesondere die gleichmäßige Elliptizitätsbedingung,
    die Invertierbarkeit der Matrix des LGS implizieren.
\end{Bem}

\linie

\begin{Lemma}{\name{Céa}-Lemma}
    Es gilt $\norm{u - u_N}_{H^1} \le C \cdot \inf_{v \in X_N} \norm{u - v}_{H^1}$,
    wobei die Konstante $C > 0$ nur von den Konstanten im Satz von Lax-Milgram abhängt.
\end{Lemma}

\begin{Bem}
    Das Céa-Lemma ist die zentrale Fehlerabschätzung für Ritz-Galerkin-Approxi"-mationen.
    Es besagt, dass die Ritz-Galerkin-Approximation bis auf eine multiplikative Konstante
    die beste Approximation ist.
    Weil $H^1$ separabel ist, können die $X_N$ so gewählt werden, dass
    $\inf_{v \in X_N} \norm{u - v}_{H^1} \xrightarrow{N \to \infty} 0$.\\
    Für weitere Fehlerabschätzungen bzgl. numerischer Verfahren, die bei der numerischen Berechnung
    der Ritz-Galerkin-Approximation eingesetzt werden
    (Interpolation, numerische Integra\-tion, iterative LGS-Löser) siehe Numerik-Veranstaltungen.
\end{Bem}

\pagebreak

\section{%
    Der Spektralsatz für kompakte, selbstadjungierte Operatoren%
}

\subsection{%
    \name{Hilbert}raum-Adjungierte%
}

\begin{Bem}
    Seien $H_1, H_2$ Hilberträume und $T \in \Lin(H_1, H_2)$.
    Für $y \in H_2$ ist die Abbildung $x \mapsto \sp{Tx, y}_{H_2}$ ein Element des Dualraums
    von $H_1$.
    Nach dem Rieszschen Darstellungssatz gibt es daher genau ein $T^\ast y \in H_1$ mit
    $\forall_{x \in H_1}\; \sp{Tx, y}_{H_2} = \sp{x, T^\ast y}_{H_1}$.
    Somit existiert die Hilbertraum-Adjungierte $T^\ast$ und ist eindeutig.
\end{Bem}

\begin{Def}{\name{Hilbert}raum-Adjungierte}
    Seien $H_1, H_2$ Hilberträume und $T \in \Lin(H_1, H_2)$.
    Dann heißt die Abbildung $T^\ast\colon H_2 \rightarrow H_1$ mit
    $\forall_{x \in H_1,\, y \in H_2}\; \sp{Tx, y}_{H_2} = \sp{x, T^\ast y}_{H_1}$
    \begriff{\name{Hilbert}raum-Adjungierte} von $T$.
\end{Def}

\begin{Lemma}{Eigenschaften der \name{Hilbert}raum-Adjungierten}
    \begin{enumerate}
        \item
        $T^\ast \in \Lin(H_2, H_1)$ mit $\norm{T^\ast} = \norm{T}$

        \item
        $(T + S)^\ast = T^\ast + S^\ast$, $(\alpha T)^\ast = \overline{\alpha} T^\ast$,
        $(T \circ S)^\ast = S^\ast \circ T^\ast$

        \item
        $T^{\ast\ast} = T$
    \end{enumerate}
\end{Lemma}

\linie

\begin{Bsp}
    \begin{enumerate}[label=\emph{(\alph*)}]
        \item
        Für $H_1 = H_2 = \real^n$ (mit eukl. Skalarprodukt) und
        $T = (a_{ij})_{i,j=1,\dotsc,n}$ ist $T^\ast = (a_{ji})_{i,j=1,\dotsc,n}$.

        \item
        Für $H_1 = H_2 = \complex^n$ ist $T^\ast = (\overline{a_{ji}})_{i,j=1,\dotsc,n}$.

        \item
        Für $H_1 = H_2 = \ell^2_\real$ und
        $T((x_n)_{n \in \natural}) := (a_n x_n)_{n \in \natural}$ für eine Folge
        $(a_n)_{n \in \natural}$, $\sup_{n \in \natural} |a_n| < \infty$, ist
        $T^\ast((x_n)_{n \in \natural}) = (a_n x_n)_{n \in \natural} = T((x_n)_{n \in \natural})$.

        \item
        Für $H_1 = H_2 = \ell^2_\complex$ ist
        $T^\ast((x_n)_{n \in \natural}) = (\overline{a_n} x_n)_{n \in \natural}$.
    \end{enumerate}
\end{Bsp}

\linie

\begin{Bsp}
    Seien $H_1 = L^2(\Omega_1, \complex)$ und $H_2 = L^2(\Omega_2, \complex)$,
    wobei $\Omega_1 \subset \real^n$ und $\Omega_2 \subset \real^m$ messbar seien.
    Außerdem sei $K\colon \Omega_1 \times \Omega_2 \rightarrow \complex$ messbar
    mit $\norm{K} := \left(\int_{\Omega_1} \int_{\Omega_2}
    |K(x, y)|^2 \dy \dx\right)^{1/2} < \infty$.
    Sei für $f \in L^2(\Omega_1, \complex)$ die Abbildung $Tf\colon \Omega_2 \rightarrow \complex$
    definiert durch $(Tf)(y) := \int_{\Omega_1} K(x, y)f(x)\dx$.\\
    Dann ist $T \in \Lin(H_1, H_2)$ und $\norm{T} \le \norm{K}$.
    Außerdem gilt $(T^\ast g)(x) = \int_{\Omega_2} \overline{K(x, y)} g(y) \dy$,
    wenn $n = m$ und $\Omega_1 = \Omega_2$.

    Dies sieht man wie folgt:
    Es gilt $\norm{Tf}_{H_2}^2 = \int_{\Omega_2} |(Tf)(y)|^2 \dy
    = \int_{\Omega_2} \left|\int_{\Omega_1} K(x, y)f(x)\dx\right|^2 \dy\\
    %$\le \int_{\Omega_2} \left|\int_{\Omega_1} |K(x, y)| |f(x)| \dx\right|^2 \dy
    = \int_{\Omega_2} \left|\sp{K(\cdot, y), f}_{H_1}\right|^2 \dy
    \le \int_{\Omega_2} \norm{K(\cdot, y)}_{H_1}^2 \norm{f}_{H_1}^2 \dy
    = \norm{K}^2 \norm{f}_{H_1}^2$,
    also $Tf \in H_2$,\\
    $T \in \Lin(H_1, H_2)$ und $\norm{T} \le \norm{K}$.
    Die Adjungierte $T^\ast$ erhält man durch direktes Nachrechnen
    (wobei man die konjugierte Linearität im zweiten Argument beachten muss).
    Ersetzt man $\complex$ durch $\real$, so ist $T^\ast = T$.
\end{Bsp}

\linie

\begin{Def}{selbstadjungiert}
    Sei $H$ ein Hilbertraum.\\
    Dann heißt $T \in \Lin(H)$ \begriff{selbstadjungiert}, falls $T^\ast = T$
    (d.\,h. $\forall_{x, y \in H}\; \sp{Tx, y} = \sp{x, Ty}$).
\end{Def}

\begin{Bem}
    Ist $T \in \Lin(H)$ selbstadjungiert, so gilt für $x = y$, dass
    $\sp{Tx, x} = \sp{x, Tx} = \overline{\sp{Tx, x}}$,
    also $\sp{Tx, x} = \sp{x, Tx} \in \real$ für alle $x \in H$.
    Manchmal ist Selbstadjungiertheit eine zu starke Eigenschaft,
    in diesem Fall verwendet man die Verallgemeinerung von normalen Abbildungen.
\end{Bem}

\begin{Def}{normal}
    Sei $H$ ein Hilbertraum.
    Dann heißt $T \in \Lin(H)$ \begriff{normal}, falls $T^\ast T = TT^\ast$.
\end{Def}

\begin{Lemma}{Charakterisierung}
    $T$ ist normal genau dann, wenn
    $\forall_{x \in H}\; \norm{Tx} = \norm{T^\ast x}$.
\end{Lemma}

\pagebreak

\subsection{%
    Kompakte Operatoren%
}

\begin{Def}{kompakter Operator}
    Seien $E, F$ Banachräume.\\
    Dann heißt $T \in \Lin(E, F)$ \begriff{kompakt}, falls
    $\overline{T B_E} = \overline{\{Tx \;|\; \norm{x}_E \le 1\}}$ kompakt in $F$ ist.\\
    Äquivalent dazu sind:
    \begin{enumerate}
        \item
        Für jede Folge $(x_n)_{n \in \natural}$ in $B_E$ besitzt $(Tx_n)_{n \in \natural}$
        eine konvergente Teilfolge in $F$.

        \item
        Für alle $\varepsilon > 0$ gibt es eine endliche Menge $M \subset F$ mit
        $TB_E \subset M + \varepsilon B_F$.
    \end{enumerate}
    Die \begriff{Menge aller kompakten Operatoren} von $E$ nach $F$ bezeichnet man mit
    $\K(E, F)$ und man schreibt $\K(E) := \K(E, E)$.
\end{Def}

\begin{Bem}
    Ist $X$ ein Banachraum, dann gilt $\id \in \K(X) \iff X \;\text{endl.-dim.}$
\end{Bem}

\begin{Lemma}{$\K(E, F)$ abg. UVR}
    $\K(E, F)$ ist ein abgeschlossener Unterraum von $\Lin(E, F)$.
\end{Lemma}

\linie

\begin{Def}{Operator mit endlichem Rang}
    Seien $E, F$ Banachräume.
    Die \begriff{Menge aller Operatoren mit endlichem Rang} ist definiert durch
    $\F(E, F) := \{T \in \Lin(E, F) \;|\; \dim TE < \infty\}$.
\end{Def}

\begin{Bsp}
    \begin{enumerate}[label=\emph{(\alph*)}]
        \item
        Für $T \in \F(E, F)$ gilt $T \in \K(E, F)$, denn
        $TB_E$ ist beschränkt in $TE$ (für alle $x \in B_E$ gilt
        $\norm{Tx}_F \le \norm{T} \norm{x}_E \le \norm{T}$) und
        somit ist $\overline{TB_E}$ beschränkt und abgeschlossen.
        Damit ist $\overline{TB_E}$ kompakt in $TE$ (wegen $\dim TE < \infty$)
        und insbesondere kompakt in $F$.

        \item
        Für $\dim E < \infty$ ist $\dim TE < \infty$ für alle $T \in \Lin(E, F)$,
        also gilt\\
        $\Lin(E, F) \subset \F(E, F) \subset \K(E, F) \subset \Lin(E, F)$,
        d.\,h. jeder lineare, stetige Operator ist kompakt, wenn $E$ endlich-dimensional ist.

        \item
        Es gilt $\overline{\F(E, F)} \subset \K(E, F)$, weil $\K(E, F)$ abgeschlossen ist.
    \end{enumerate}
\end{Bsp}

\begin{Bem}
    Lange war ungeklärt, ob die Umkehrung auch gilt,
    d.\,h. ob $\overline{\F(E, F)} = \K(E, F)$.
    Die Frage war also, ob jeder kompakte Operator durch Operatoren von endlichem Rang
    approximiert werden kann.
    Per Enflo konnte als Erster ein Gegenbeispiel liefern (1973).
    Allerdings stimmt die Aussage, wenn $F$ ein Hilbertraum ist.
\end{Bem}

\begin{Lemma}{kpkt.e Operatoren als GW von Operatoren mit endl. Rang}\\
    Seien $E$ ein Banachraum und $F$ ein Hilbertraum.
    Dann gilt $\overline{\F(E, F)} = \K(E, F)$.
\end{Lemma}

\linie
\pagebreak

\begin{Bsp}
    Obiger Integraloperator $T \in \Lin(H_1, H_2)$ ist kompakt.
    Wählt man ein vollständiges ONS $(e_k)_{k \in \natural}$ von $H_1$, dann gilt nach
    Parseval
    $\norm{K}^2 = \int_{\Omega_2} \norm{\overline{K(\cdot, y)}}_{H_1}^2 \dy$\\
    $= \int_{\Omega_2} \sum_{k \in \natural}
    \left|\sp{\overline{K(\cdot, y)}, e_k}_{H_1}\right|^2 \dy
    = \int_{\Omega_2} \sum_{k \in \natural} |(Te_k)(y)|^2 \dy
    = \sum_{k \in \natural} \norm{Te_k}_{H_2}^2$.\\
    Sei $P_n$ die orthogonale Projektion von $H_1$ auf $[e_1, \dotsc, e_n]$, d.\,h.
    $P_n f := \sum_{k=1}^n \sp{f, e_k}_{H_1} e_k$.\\
    Dann gilt
    $\norm{(T - TP_n) f}_{H_2}^2
    = \norm{T (f - P_n f)}_{H_2}^2
    = \norm{T\!\left(\sum_{k > n} \sp{f, e_k}_{H_1} e_k\right)}_{H_2}^2$\\
    $= \norm{\sum_{k > n} \sp{f, e_k}_{H_1} Te_k}_{H_2}^2
    \le \left(\sum_{k > n} \left|\sp{f, e_k}_{H_1}\right| \norm{Te_k}_{H_2}\right)^2$\\
    $\le \sum_{k > n} \left|\sp{f, e_k}_{H_1}\right|^2 \cdot \sum_{k > n} \norm{Te_k}_{H_2}^2$
    wegen der Cauchy-Schwarz-Ungleichung für $\ell^2$.
    Der erste Faktor ist mit Parseval durch $\norm{f}_{H_1}^2$ nach oben beschränkt, während
    der zweite für $n \to \infty$ gegen Null geht
    (weil $\sum_{k \in \natural} \norm{Te_k}_{H_2}^2 = \norm{K}^2 < \infty$).
    Damit gilt\\
    $\norm{(T - TP_n) f}_{H_2}^2 \le \sum_{k > n} \norm{Te_k}_{H_2}^2
    \norm{f}_{H_1}^2$
    und somit $\norm{T - TP_n}^2 \le \sum_{k > n} \norm{Te_k}_{H_2}^2 \to 0$
    für $n \to \infty$.
    Wegen $\Bild(TP_n) = T(\Bild(P_n))$ endlich-dimensional für alle $n \in \natural$ ist
    $T$ kompakt.
\end{Bsp}

\begin{Bem}
    Man kann bei Vorhandensein entsprechender Integrabilität von $K$ auch Integraloperatoren
    $T^{p,q} \in \Lin(L^p(\Omega_1, \KK), L^q(\Omega_2, \KK))$ für $\frac{1}{p} + \frac{1}{q} = 1$
    bekommen.
    Auch sie sind stetig (Nachweis mit Hölder statt Cauchy-Schwarz, ähnlich wie für $p = q = 2$)
    und kompakt (Nachweis mithilfe von Fréchet-Kolmogorov, Riesz).
    Man nennt diese Operatoren\\
    \begriff{\name{Hilbert}-\name{Schmidt}-Integraloperatoren}.
\end{Bem}

\linie

\begin{Lemma}{Komposition kpkt.}
    Seien $X, Y, Z$ Banachräume, $T_1 \in \Lin(X, Y)$ und $T_2 \in \Lin(Y, Z)$.\\
    Dann folgt aus $T_1$ kompakt oder $T_2$ kompakt, dass $T_2 T_1$ kompakt ist.
\end{Lemma}

\begin{Bem}
    Algebraisch lässt sich das für $X = Y = Z$ wie folgt ausdrücken:
    Mit der Verkettung $\circ$ als Multiplikation ist der Vektorraum
    $(\Lin(X), +, \circ)$ eine nicht-kommutative Algebra,
    d.\,h. $\circ$ ist assoziativ (aber i.\,A. nicht-kommutativ),
    $+$ und $\circ$ sind distributiv und für alle $\alpha \in \KK$ gilt
    $\alpha (f \circ g) = (\alpha f) \circ g = f \circ (\alpha g)$.
    Für $S, T \in \Lin(X)$ gilt außerdem $\norm{S \circ T} \le \norm{S} \cdot \norm{T}$.
    Ein Banachraum, der eine Algebra ist und dessen Multiplikation diese Beziehung erfüllt,
    heißt \begriff{\name{Banach}algebra}.
    $(\Lin(X), +, \circ)$ ist also eine Banachalgebra und obiges Lemma besagt nun,
    dass $\K(X)$ ein Ideal in $\Lin(X)$ ist.
\end{Bem}

\linie

\begin{Satz}{Eigenwerte kompakter Operatoren}\\
    Seien $X$ ein Banachraum, $T \in \K(X)$ und $\lambda \in \KK \setminus \{0\}$.
    Dann gilt:
    \begin{enumerate}
        \item
        $\dim \Kern(\lambda\id - T) < \infty$

        \item
        $\Bild(\lambda\id - T) \subset X$ abgeschlossen

        \item
        $\lambda\id - T$ injektiv $\iff$ $\lambda\id - T$ surjektiv
    \end{enumerate}
\end{Satz}

\pagebreak

\subsection{%
    Das Spektrum linearer Abbildungen über Banachräumen%
}

\begin{Bem}
    Im Folgenden seien $X$ ein $\complex$-Banachraum und $T \in \Lin(X)$.
\end{Bem}

\begin{Def}{Resolventenmenge}\\
    Die Menge $\varrho(T) := \{\lambda \in \complex \;|\; \lambda\id - T \text{ bijektiv}\}$
    heißt \begriff{Resolventenmenge} von $T$.
\end{Def}

\begin{Def}{Spektrum}
    Die Menge $\sigma(T) := \complex \setminus \varrho(T)$
    heißt \begriff{Spektrum} von $T$.
    Es kann zerlegt werden in
    \begin{itemize}
        \item
        das \begriff{Punktspektrum}\\
        $\sigma_p(T) := \{\lambda \in \complex \;|\; \lambda\id - T \text{ nicht injektiv}\}$,

        \item
        das \begriff{kontinuierliche Spektrum}\\
        $\sigma_c(T) := \{\lambda \in \complex \;|\; \lambda\id - T
        \text{ injektiv, aber nicht surjektiv und } \overline{\Bild(\lambda\id - T)} = X\}$ und

        \item
        das \begriff{Residualspektrum}\\
        $\sigma_r(T) := \{\lambda \in \complex \;|\; \lambda\id - T
        \text{ injektiv und } \overline{\Bild(\lambda\id - T)} \not= X\}$.
    \end{itemize}
\end{Def}

\begin{Def}{Eigenvektor, Eigenwert, Eigenraum}\\
    Für $\lambda \in \complex$ gilt $\lambda \in \sigma_p(T)$ genau dann, wenn
    $\exists_{x \in X \setminus \{0\}}\; Tx = \lambda x$.
    In diesem Fall heißt $x$ \begriff{Eigenvektor} von $T$ zum \begriff{Eigenwert} $\lambda$.
    Ist $X$ ein Funktionenraum, so heißt $x$ auch \begriff{Eigenfunktion}.
    Der Unterraum $\Kern(\lambda\id - T)$ von $X$ heißt \begriff{Eigenraum} von $T$ zum
    Eigenwert $\lambda$.
    Seine Dimension heißt \begriff{Vielfachheit} des Eigenwerts $\lambda$.
    Der Eigenraum ist ein $T$-invarianter Unterraum, d.\,h.
    $T(\Kern(\lambda\id - T)) \subset \Kern(\lambda\id - T)$.
\end{Def}

\begin{Bem}
    Für $\dim(X) < \infty$ gilt $\sigma(T) = \sigma_p(T)$ für alle $T \in \Lin(X)$
    (da in diesem Fall $\lambda\id - T$ injektiv $\iff$ $\lambda\id - T$ surjektiv für alle
    $\lambda \in \complex$).
\end{Bem}

\linie

\begin{Bem}
    Im weiteren Verlauf wird der folgende (nicht-triviale) Satz aus der Banachraum-Theorie
    benötigt, der später bewiesen wird.
\end{Bem}

\begin{Satz}{Umkehrabbildung stetig}
    Seien $E$ und $F$ Banachräume und $L \in \Lin(E, F)$ bijektiv.\\
    Dann gilt $L^{-1} \in \Lin(F, E)$.
\end{Satz}

\begin{Def}{Resolvente}
    Sei $\lambda \in \varrho(T)$.
    Dann heißt $R(\lambda, T) := (\lambda\id - T)^{-1} \in \Lin(X)$
    \begriff{Resolvente} von $T$ in $\lambda$ und
    $R(\cdot, T)\colon \varrho(T) \rightarrow \Lin(X)$,
    $\lambda \mapsto R(\lambda, T)$ heißt \begriff{Resolventenfunktion}.
\end{Def}


\begin{Satz}{Resolventenfunktion holomorph}
    $\varrho(T) \subset \complex$ ist of"|fen und
    $R(\cdot, T)\colon \varrho(T) \rightarrow \Lin(X)$ ist holomorph, d.\,h.
    $\lim_{h \to 0} \frac{R(\lambda + h, T) - R(\lambda, T)}{h}$ existiert in $\Lin(X)$.\\
    Außerdem gilt $\forall_{\lambda \in \varrho(T)}\;
    \norm{R(\lambda, T)}^{-1} \le \dist(\lambda, \sigma(T))$.
\end{Satz}

\linie

\begin{Def}{Spektralradius}
    $\sup_{\lambda \in \sigma(T)} |\lambda|$ heißt \begriff{Spektralradius} von $T$.
\end{Def}

\begin{Satz}{Spektrum kompakt}
    $\sigma(T)$ ist kompakt und für $X \not= \{0\}$ auch nicht-leer mit\\
    $\sup_{\lambda \in \sigma(T)} |\lambda| = \lim_{m \to \infty} \norm{T^m}^{1/m} \le \norm{T}$.
\end{Satz}

\begin{Satz}{Spektralradius normaler Operatoren über Hilberträume}\\
    Sei $X \not= \{0\}$ ein $\complex$-Hilbertraum und
    $T \in \Lin(X)$ normal.
    Dann gilt $\sup_{\lambda \in \sigma(T)} |\lambda| = \norm{T}$.
\end{Satz}

\pagebreak

\subsection{%
    Das Spektrum kompakter Operatoren und der Spektralsatz%
}

\begin{Satz}{Spektrum kompakter Operatoren}
    Sei $T \in \K(X)$.\\
    Dann stimmt $\sigma(T)$ mit den Eigenwerten $\sigma_p(T)$ bis auf $0$ überein,
    d.\,h. $\sigma(T) \setminus \{0\} = \sigma_p(T) \setminus \{0\}$.\\
    Außerdem besteht $\sigma(T) \setminus \{0\}$
    \begin{enumerate}
        \item
        aus endlich vielen Eigenwerten oder

        \item
        aus abzählbar unendlich vielen Eigenwerten mit $0$ als einzigem Häufungspunkt.
    \end{enumerate}
    Die Vielfachheit jeden von $0$ verschiedenen Eigenwerts
    $\lambda \in \sigma(T) \setminus \{0\}$ ist endlich.\\
    Für $\dim X = \infty$ ist $0 \in \sigma(T)$.
\end{Satz}

\linie

\begin{Def}{positiv semidefinit}
    Seien $H$ ein $\complex$-Hilbertraum und $T \in \Lin(H)$ selbstadjungiert.\\
    Dann heißt $T$ \begriff{positiv semidefinit}, falls
    $\forall_{x \in H}\; \sp{x, Tx} \ge 0$.
\end{Def}

\begin{Satz}{Spektralsatz für kompakte, selbstadjungierte Operatoren}\\
    Seien $H$ ein $\complex$-Hilbertraum und $T \in \Lin(H) \setminus \{0\}$
    kompakt und selbstadjungiert.
    Dann gilt:
    \begin{enumerate}
        \item
        $\sigma_p(T) \setminus \{0\} = \{\lambda_k \;|\; k \in N\}$
        mit $N = \{1, \dotsc, n\}$ oder $N = \natural$ und
        $\lambda_k$ paarweise verschieden.\\
        Für alle $k \in N$ gilt $\dim(\Kern(\lambda_k \id - T)) < \infty$ und gibt es Eigenvektoren
        $e_{k,j_k}$,\\
        $j_k = 1, \dotsc, \dim(\Kern(\lambda_k \id - T))$,
        von $T$ zu $\lambda_k$, sodass
        $(e_{k,j_k})_{k,j_k}$ ein ONS in $H$ ist.\\
        Für $N = \natural$ gilt $\lim_{k \to \infty} \lambda_k = 0$.

        \item
        $H = \Kern(T) \oplus \overline{[\{e_{k,j_k} \;|\; k, j_k\}]}$
        mit $\Kern(T) \,\orth\, \overline{[\{e_{k,j_k} \;|\; k, j_k\}]}$

        \item
        $\forall_{x \in H}\; Tx = \sum_k \sum_{j_k} \lambda_k \sp{x, e_{k,j_k}} e_{k,j_k}$

        \item
        $\sigma_p(T) \subset [-\norm{T}, \norm{T}] \subset \real$

        \item
        $\norm{T} \in \sigma_p(T)$ oder $-\norm{T} \in \sigma_p(T)$

        \item
        Ist $T$ positiv semidefinit, dann gilt $\sigma_p(T) \subset [0, \norm{T}]$.
    \end{enumerate}
\end{Satz}

\begin{Bem}
    Dieser Satz ist eine unendlich-dimensionale Verallgemeinerung des Theorems
    aus der linearen Algebra, dass jede
    symmetrische Matrix mithilfe von ONBen aus Eigenvektoren reell diagonalisierbar ist.
\end{Bem}

\begin{Satz}{Spektralsatz für kompakte, normale Operatoren}\\
    Seien $H$ ein $\complex$-Hilbertraum und $T \in \Lin(H) \setminus \{0\}$
    kompakt und normal.\\
    Dann gelten die Aussagen \emph{(1)}, \emph{(2)} und \emph{(3)} aus obigem Satz.
\end{Satz}

\begin{Bem}
    Anhand des Beweises erkennt man, dass die Aussagen \emph{(4)} und \emph{(6)} gelten,
    wenn $T$ nur selbstadjungiert (und stetig) ist, aber nicht kompakt.
\end{Bem}

\linie

\begin{Bem}
    Ist $X$ ein $\real$-Banachraum, so kann man $X$ \begriff{komplexifizieren},
    d.\,h. $\widetilde{X} := X \times X$ mit
    $\alpha \cdot (x_1, x_2) := (a x_1 - b x_2, a x_2 + b x_1)$ und
    $\overline{(x_1, x_2)} := (x_1, -x_2)$ für $(x_1, x_2) \in \widetilde{X}$ und
    $\alpha := a + \iu b \in \complex$ mit $a, b \in \real$.
    Damit wird $\widetilde{X}$ ein $\complex$-Vektorraum.\\
    Mit $\norm{x}_{\widetilde{X}} := \sup_{\theta \in \real}
    \left(\norm{\cos(\theta)x_1 - \sin(\theta)x_2}_X^2 +
    \norm{\sin(\theta)x_1 + \cos(\theta)x_2}_X^2\right)^{1/2}$ gilt dann\\
    $\forall_{x \in \widetilde{X}} \forall_{\theta \in \real}\;
    \norm{e^{\iu\theta}x}_{\widetilde{X}} = \norm{x}_{\widetilde{X}}$
    und $\widetilde{X}$ ist ein $\complex$-Banachraum.\\
    Falls $X$ ein $\real$-Hilbertraum ist, so ist $\widetilde{X}$ ein $\complex$-Hilbertraum
    sowie $\norm{x}_{\widetilde{X}} = \left(\norm{x_1}_X^2 + \norm{x_2}_X^2\right)^{1/2}$.\\
    Für $T \in \Lin(X)$ ist $\widetilde{T} \in \Lin(\widetilde{X})$ mit
    $\widetilde{T}x := (Tx_1, Tx_2)$.
    Zusätzliche Eigenschaften wie Kompaktheit oder Selbstadjungiertheit von $T$ übertragen sich
    auf $\widetilde{T}$.
    Somit kann man mit dieser Komplexifizierung Spektralsätze wie oben auch auf reelle
    Hilberträume übertragen (analog auch von komplexen auf reelle Banachräume).
\end{Bem}

\linie
\pagebreak

\begin{Def}{\name{Rayleigh}-Quotient}
    Seien $H$ ein $\complex$-Hilbertraum und $T \in \Lin(H)$ selbstadjungiert.\\
    Dann heißt $R_T(u) := \frac{\sp{Tu, u}}{\sp{u, u}}$
    der \begriff{\name{Rayleigh}-Quotient} von $u \in H \setminus \{0\}$.
\end{Def}

\begin{Bem}
    Der Rayleigh-Quotient von Eigenvektoren ist
    gleich dem jeweiligen Eigenwert.
\end{Bem}

\begin{Satz}{Eigenwerte kompakter, selbstadjungierter Operatoren}\\
    Seien $H$ ein $\complex$-Hilbertraum und $T \in \Lin(H) \setminus \{0\}$
    kompakt und selbstadjungiert.
    Dann gilt:
    \begin{enumerate}
        \item
        Wenn $\lambda \not= 0$ mit
        $\lambda := \sup_{u \in H \setminus \{0\}} R_T(u) = \sup_{u \in H,\;\norm{u}=1} \sp{Tu, u}$
        gilt, dann ist\\
        $\lambda = \max(\sigma_p(T) \setminus \{0\})$.
        Das Supremum wird in diesem Fall von allen Eigenvektoren zum Eigenwert $\lambda$
        angenommen.

        \item
        Wenn $\mu \not= 0$ mit
        $\mu := \inf_{u \in H \setminus \{0\}} R_T(u) = \inf_{u \in H,\;\norm{u}=1} \sp{Tu, u}$
        gilt, dann ist\\
        $\mu = \min(\sigma_p(T) \setminus \{0\})$.
        Das Infimum wird in diesem Fall von allen Eigenvektoren zum Eigenwert $\lambda$
        angenommen.

        \item
        Für $\sup_{u \in \Kern(\lambda\id - T)^\orth \setminus \{0\}} R_T(u) \not= 0$
        ist dies der zweitgrößte von $0$ verschiedene Eigenwert usw.
    \end{enumerate}
\end{Satz}

\begin{Bem}
    Für alle von $0$ verschiedenen Eigenwerte sind die Lösungen der jeweiligen
    Eigen"-wert-Gleichungen die Lösungen von Variationsproblemen mit Nebenbedingungen,
    wobei die Eigenwerte als Lagrange-Parameter auftreten.
\end{Bem}

\subsection{%
    Der Spektralsatz für den \name{Laplace}-Operator%
}

\begin{Def}{inverser \name{Laplace}-Operator}
    Sei $\Omega \subset \real^n$ ein beschränktes und stückweise $\C^1$-berandetes Gebiet.
    Dann ist der \begriff{(schwache) inverse \name{Laplace}-Operator
    (mit homogenen \name{Dirichlet}-RB)}
    $\Delta^{-1}\colon L^2(\Omega) \rightarrow H_0^1(\Omega)$ definiert durch die für
    $f \in L^2(\Omega)$ eindeutige Lösung $-\Delta^{-1} f \in H_0^1(\Omega)$ von
    $\forall_{\varphi \in H_0^1(\Omega)}\;
    \int_\Omega (\nabla (-\Delta^{-1} f) \nabla \varphi - f \varphi) \dx = 0$
    (schwache Lösung des Dirichlet-Problems für die Poisson-Gleichung mit
    homogenen Randbedingungen).
\end{Def}

\begin{Satz}{Eigenschaften von $-\Delta^{-1}$}
    $-\Delta^{-1}\colon L^2(\Omega) \rightarrow L^2(\Omega)$ ist
    linear, stetig, injektiv, kompakt, selbstadjungiert und positiv semidefinit.
\end{Satz}

\begin{Satz}{Satz von \scshape\,\!\name{Rellich}}
    Sei $\Omega \subset \real^n$ ein beschränktes und stückweise $\C^1$-berandetes Gebiet.
    Dann ist die Einbettung $\id\colon H^1(\Omega) \hookrightarrow L^2(\Omega)$ ein kompakter
    Operator, d.\,h. jede in $H^1(\Omega)$ beschränkte Folge enthält eine in
    $L^2(\Omega)$ konvergente Teilfolge.
\end{Satz}

\linie

\begin{Def}{schwacher \name{Laplace}-Operator}\\
    $\Delta := (\Delta^{-1})^{-1}\colon \Delta^{-1}(L^2(\Omega)) \rightarrow L^2(\Omega)$
    heißt \begriff{schwacher \name{Laplace}-Operator}.
\end{Def}

\begin{Satz}{Spektralsatz für den \name{Laplace}-Operator}\\
    Sei $\Omega \subset \real^n$ ein beschränktes und stückweise $\C^1$-berandetes Gebiet.
    Dann gilt:
    \begin{enumerate}
        \item
        $\sigma_p(-\Delta) = \{\lambda_k \;|\; k \in \natural\}$ mit
        $0 < \lambda_1 \le \lambda_2 \le \dotsb$,
        $\dim(\Kern(\lambda_k\id + \Delta)) < \infty$ und\\
        $\lim_{k \to \infty} \lambda_k = \infty$

        \item
        Es gibt eine Folge $(e_k)_{k \in \natural}$ in $H_0^1(\Omega)$, sodass
        $(e_k)_{k \in \natural}$ ein vollständiges ONS in $L^2(\Omega)$ aus Eigenvektoren von
        $-\Delta$ ist, d.\,h.
        $\forall_{\varphi \in H_0^1(\Omega)}\;
        \sp{e_k, \varphi}_{H_0^1} = \lambda_k \sp{e_k, \varphi}_{L^2}$ und\\
        $\forall_{u \in L^2(\Omega)}\;
        u \overset{L^2}{=} \sum_{k=1}^\infty \sp{u, e_k}_{L^2} e_k,\;
        \norm{u}_{L^2}^2 = \sum_{k=1}^\infty |\sp{u, e_k}_{L^2}|^2$.

        \item
        Für $k \in \natural$ gilt
        $\lambda_k = \min\left\{\left.\frac{\norm{u}_{H_0^1}^2}{\norm{u}_{L^2}^2} \;\right|\;
        u \in H_0^1 \setminus \{0\},\; u \;\orth\; [e_1, \dotsc, e_{k-1}]\right\}$.
    \end{enumerate}
\end{Satz}

\pagebreak

\optpart{%
    \name{Banach}raumtheorie%
}

\chapter{%
    Der Satz von \name{Hahn}-\name{Banach} und die Hauptsätze der \name{Banach}raumtheorie%
}

\section{%
    Der Satz von \name{Hahn}-\name{Banach}, Projektions- und Trennungssatz
}

\begin{Bem}
    Sämtliche Aussagen in diesem Abschnitt basieren auf dem Satz von Hahn-Banach, für dessen
    Beweis man das Auswahlaxiom benötigt.
\end{Bem}

\begin{Satz}{Satz von \name{Hahn}-\name{Banach}}
    Sei $X$ ein $\real$-Vektorraum und $Y \subset X$ ein Unterraum.\\
    Außerdem seien
    \begin{enumerate}
        \item
        $p\colon X \rightarrow \real$ \begriff{sublinear},
        d.\,h. $\forall_{x, y \in X}\; p(x + y) \le p(x) + p(y)$ und
        $\forall_{x \in X} \forall_{\alpha \ge 0}\; p(\alpha x) = \alpha p(x)$,

        \item
        $f\colon Y \rightarrow \real$ linear und

        \item
        $f \le p$ auf $Y$.
    \end{enumerate}
    Dann gibt es eine lineare Abbildung $F\colon X \rightarrow \real$ mit
    $F|_Y = f$ und $F \le p$ auf $X$.
\end{Satz}

\begin{Satz}{Satz von \name{Hahn}-\name{Banach} für lineare Funktionale}\\
    Seien $X$ ein normierter Raum und $Y \subset X$ ein Unterraum (mit der Norm von $X$).
    Dann gilt\\
    $\forall_{y' \in Y'} \exists_{x' \in X'}\; [x'|_Y = y',\; \norm{x'}_{X'} = \norm{y'}_{Y'}]$.
\end{Satz}

\linie

\begin{Satz}{Projektionssatz für norm. Räume}\\
    Seien $X$ ein normierter Raum, $Y \subset X$ ein abgeschlossener Unterraum und
    $x_0 \in X \setminus Y$.\\
    Dann gilt
    $\exists_{x' \in X'}\; [x'|_Y = 0,\; \norm{x'}_{X'} = 1,\; x'(x_0) = \dist(x_0, Y)]$.
\end{Satz}

\begin{Bem}
    $x'$ ist also eine Art lineare Näherung der Abstandsabbildung $\dist(\cdot, Y)$.\\
    Der Satz kann als Verallgemeinerung des Projektionssatzes für Hilberträume aufgefasst werden:
    Ist $X$ sogar ein Hilbertraum,
    dann erfüllt $x' \in X'$ mit $x'(x) := \sp{x, \frac{(\id - P)x_0}{\norm{(\id - P)x_0}}}$ mit
    $P$ der orthogonalen Projektion auf $Y$ die Eigenschaften des obigen Satzes.
    Es gilt $(\id - P)x_0 \in Y^\orth$,
    weil $\id - P$ die orthogonale Projektion auf $Y^\orth$ ist
    (daraus folgt $x'|_Y = 0$).
    Außerdem gilt mit $x'(Px_0) = 0$ (wegen $Px_0 \in Y$), dass
    $x'(x_0)
    = x'((\id - P)x_0)
    = \norm{(\id - P)x_0}
    = \dist(x_0, Y)$,
    insbesondere gilt also $\norm{x'}_{X'} \ge 1$.
    $x' \in X'$ gilt wegen $|x'(x)| \le \norm{x}_X$, also $\norm{x'}_{X'} \le 1$.
\end{Bem}

\linie

\begin{Kor}
    Seien $X$ ein normierter Raum und $x_0 \in X$.
    Dann gilt:
    \begin{enumerate}
        \item
        Ist $x_0 \not= 0$, so gibt es ein $x_0' \in X'$ mit $\norm{x_0'}_{X'} = 1$ und
        $x_0'(x_0) = \norm{x_0}_X$.

        \item
        Wenn $\forall_{x' \in X'}\; x'(x_0) = 0$ gilt, dann ist $x_0 = 0$.

        \item
        Sei $J_{x_0}\colon X' \rightarrow \KK$, $J_{x_0} x' := x'(x_0)$.
        Dann ist $J_{x_0} \in X''$ mit $\norm{J_{x_0}}_{X''} = \norm{x_0}_X$.
    \end{enumerate}
\end{Kor}

\begin{Bem}
    $X''$ heißt \begriff{Bidualraum} von $X$.
\end{Bem}

\linie

\begin{Satz}{Trennungssatz}
    Seien $X$ ein normierter Raum, $M \subset X$ eine nicht-leere, abgeschlossene und
    konvexe Teilmenge und $x_0 \in X \setminus M$.\\
    Dann gilt $\exists_{x' \in X'} \exists_{\alpha \in \real} \forall_{x \in M}\;
    \Re(x'(x)) \le \alpha,\; \Re(x'(x_0)) > \alpha$.\\
    Insbesondere ist $x' \not= 0$ und $\{x \in X \;|\; \Re(x'(x)) = \alpha\}$ ist
    eine Hyperebene in $X$.
\end{Satz}

\begin{Bem}
    Man kann sich den Satz so vorstellen, dass die Hyperebene $\Re(x'(x)) = \alpha$
    den Raum $X$ in $\Re(x'(x)) \le \alpha$ und $\Re(x'(x)) > \alpha$ aufteilt, wobei diese beiden
    Mengen $M$ bzw. $x_0$ enthalten.
    Für nicht-konvexe Mengen gilt die Aussage i.\,A. nicht.
\end{Bem}

\pagebreak

\section{%
    \name{Baire}scher Kategoriensatz und der Satz von \name{Banach}-\name{Steinhaus}%
}

\begin{Bem}
    Der folgende Bairesche Kategoriensatz gilt nur in vollständigen metrischen Räumen.
    Ein Gegenbeispiel für nicht-vollständige metrische Räume
    ist $\rational = \bigcup_{q \in \rational} \{q\}$.
\end{Bem}

\begin{Satz}{\name{Baire}scher Kategoriensatz}
    Seien $X$ ein nicht-leerer, vollständiger metrischer Raum und $A_k \subset X$ abgeschlossen
    mit $X = \bigcup_{k \in \natural} A_k$.\\
    Dann gibt es ein $k_0 \in \natural$ mit $\interior{A_{k_0}} \not= \emptyset$.
\end{Satz}

\begin{Satz}{Prinzip der gleichmäßigen Beschränktheit}
    Seien $X$ ein nicht-leerer, vollständiger metrischer Raum, $Y$ ein normierter Raum und
    $\F \subset \C^0(X, Y)$ mit $\forall_{x \in X} \sup_{f \in \F} \norm{f(x)}_Y < \infty$.\\
    Dann gilt $\exists_{x_0 \in X} \exists_{\varepsilon_0 > 0}\;
    \sup_{x \in \overline{B_{\varepsilon_0}(x_0)}} \sup_{f \in \F} \norm{f(x)}_Y < \infty$.
\end{Satz}

\begin{Satz}{Satz von \name{Banach}-\name{Steinhaus}}\\
    Seien $X$ ein Banachraum, $Y$ ein normierter Raum und
    $\T \subset \Lin(X, Y)$ mit\\
    $\forall_{x \in X}\; \sup_{T \in \T} \norm{Tx}_Y < \infty$.\\
    Dann ist $\T$ beschränkt, d.\,h. $\sup_{T \in \T} \norm{T}_{\Lin(X, Y)} < \infty$.
\end{Satz}

\begin{Satz}{Satz von \name{Banach}-\name{Steinhaus} für lineare, stetige Funktionale}\\
    Seien $X$ ein Banachraum, $Y$ ein normierter Raum und
    $\T \subset \Lin(X, Y)$ mit\\
    $\forall_{x \in X} \forall_{y' \in Y'}\; \sup_{T \in \T} |y'(Tx)| < \infty$.\\
    Dann ist $\T$ beschränkt, d.\,h. $\sup_{T \in \T} \norm{T}_{\Lin(X, Y)} < \infty$.
\end{Satz}

\linie

\begin{Def}{of"|fene Abbildung}
    Seien $X, Y$ metrische Räume.\\
    Dann heißt eine Abbildung $f\colon X \rightarrow Y$ of"|fen, falls
    $\forall_{U \subset X \text{ of"|fen}}\; f(U) \subset Y \text{ of"|fen}$.
\end{Def}

\begin{Bem}
    Ist $f$ bijektiv, dann ist $f$ of"|fen genau dann, wenn $f^{-1}$ stetig ist.
\end{Bem}

\begin{Bem}
    Sind $X, Y$ normierte Räume und $T\colon X \rightarrow Y$ linear,
    dann ist $T$ of"|fen genau dann, wenn $\exists_{\delta > 0}\; B_\delta(0) \subset TB_1(0)$
    (d.\,h. $0 \in \interior{TB_1(0)}$).\\
    Wenn $T$ nämlich of"|fen ist, dann ist $TB_1(0)$ of"|fen in $Y$
    (als Bild einer of"|fenen Menge in $X$) und weil $0 \in TB_1(0)$, gibt es eine
    $\delta$-Kugel um $0$ in $TB_1(0)$.\\
    Sei umgekehrt $B_\delta(0) \subset TB_1(0)$ für ein $\delta > 0$.
    Ist $U \subset X$ of"|fen und $Tx \in TU$, dann gibt es ein $\varepsilon > 0$ mit
    $B_\varepsilon(x) \subset U$.
    Sei $y \in B_{\varepsilon\delta}(Tx)$, also $\norm{y - Tx}_Y < \varepsilon\delta$,
    dann gilt $\frac{1}{\varepsilon} (y - Tx) \in B_\delta(0)$,
    d.\,h. $\frac{1}{\varepsilon} (y - Tx) \in TB_1(0)$.
    Daher gibt es ein $z \in B_1(0)$ mit $\frac{1}{\varepsilon} (y - Tx) = Tz$,
    also $y = T(\varepsilon z + x)$.
    Es gilt $\varepsilon z + x \in B_\varepsilon(x) \subset U$,
    d.\,h. $y \in TU$ und $B_{\varepsilon\delta}(Tx) \subset TU$.
    Damit ist $TU$ of"|fen.
\end{Bem}

\begin{Satz}{Satz von der of{}fenen Abbildung}
    Seien $X, Y$ Banachräume und $T \in \Lin(X, Y)$.\\
    Dann ist $T$ surjektiv genau dann, wenn $T$ of"|fen ist.
\end{Satz}

\begin{Satz}{Satz von der inversen Abbildung}\\
    Seien $X, Y$ Banachräume und $T \in \Lin(X, Y)$ bijektiv.
    Dann ist $T^{-1} \in \Lin(Y, X)$.
\end{Satz}

\linie

\begin{Def}{Graph}
    Seien $X, Y$ Banachräume und $T\colon X \rightarrow Y$ eine Abbildung.\\
    Dann heißt
    $\graph(T) := \{(x, Tx) \;|\; x \in X\} \subset X \times Y$ der \begriff{Graph} von $T$.
\end{Def}

\begin{Satz}{Satz vom abgeschlossenen Graphen}
    Seien $X, Y$ Banachräume und $T\colon X \rightarrow Y$ linear.\\
    Dann ist $\graph(T) \subset X \times Y$ abgeschlossen genau dann, wenn $T \in \Lin(X, Y)$.
\end{Satz}

\begin{Bem}
    $X \times Y$ wird dabei mit der Norm $\norm{(x, y)}_{X \times Y} := \norm{x}_X + \norm{y}_Y$
    für $(x, y) \in X \times Y$ versehen.
    Äquivalent dazu ist die Norm $\norm{(x, y)}_{X \times Y}' := \max(\norm{x}_X, \norm{y}_Y)$
    (oder allgemeiner $\norm{(x, y)}_{X \times Y}'' := (\norm{x}_X^p + \norm{y}_Y^p)^{1/p}$
    für $p \in [1, \infty]$).
\end{Bem}

\pagebreak

\section{%
    Projektionen in Banachräumen%
}

\begin{Def}{Projektion}
    Seien $X$ ein $\KK$-Vektorraum, $Y \subset X$ ein Unterraum und
    $P\colon X \rightarrow X$ linear.\\
    Dann heißt $P$ \begriff{Projektion} auf $Y$,
    falls $P^2 = P$ und $\Bild(P) = Y$.
\end{Def}

\begin{Lemma}{Eigenschaften von Projektionen}
    \begin{enumerate}
        \item
        $P$ ist eine Projektion auf $Y$ genau dann, wenn $P\colon X \rightarrow Y$ und
        $P|_Y = \id$.

        \item
        Wenn $P$ eine Projektion ist, dann ist $X = \Kern(P) \oplus \Bild(P)$.

        \item
        Wenn $P$ eine Projektion ist, dann ist $\id - P$ auch eine Projektion mit\\
        $\Kern(\id - P) = \Bild(P)$ und $\Bild(\id - P) = \Kern(P)$.

        \item
        Zu jedem Unterraum $Y \subset X$ existiert eine Projektion auf $Y$.
    \end{enumerate}
\end{Lemma}

\begin{Bem}
    Für den Beweis der vierten Eigenschaft benötigt man das Auswahlaxiom.
\end{Bem}

\linie

\begin{Def}{Menge der stetigen Projektionen}
    Sei $X$ ein normierter Raum.\\
    Dann heißt $\P(X) := \{P \in \Lin(X) \;|\; P^2 = P\}$ die
    \begriff{Menge der stetigen Projektionen}.
\end{Def}

\begin{Lemma}{Eigenschaften von stetigen Projektionen}
    Sei $P \in \P(X)$.
    Dann gilt:
    \begin{enumerate}
        \item
        $\Kern(P)$ und $\Bild(P)$ sind abgeschlossen in $X$.

        \item
        $\norm{P} \ge 1$ oder $P = 0$
    \end{enumerate}
\end{Lemma}

\linie

\begin{Satz}{Satz vom abgeschlossenen Komplement}
    Seien $X$ ein Banachraum, $Y \subset X$ ein abgeschlossener Unterraum und
    $Z \subset X$ ein Unterraum mit $Y \oplus Z = X$.
    Dann sind äquivalent:
    \begin{enumerate}
        \item
        Es gibt eine stetige Projektion $P$ auf $Y$ mit $Z = \Kern(P)$.

        \item
        $Z$ ist abgeschlossen.
    \end{enumerate}
\end{Satz}

\begin{Bem}
    Ist $H$ ein Hilbertraum und $Y \subset H$ ein abgeschlossener Unterraum,
    dann ist nach dem Projektionssatz die orthogonale Projektion $P$ auf $Y$ eine stetige
    Projektion auf $Y$ im Sinne der obigen Definition und $H = Y \oplus Y^\orth$ mit $Y^\orth$
    abgeschlossen.
    Wegen der Besselschen Ungleichung ist $\norm{P} \le 1$, d.\,h. $\norm{P} = 1$ oder $P = 0$.
\end{Bem}

\begin{Satz}{Projektionen auf endl.-dim. Unterräume}\\
    Seien $X$ ein normierter Raum, $E \subset X$ ein endlich-dimensionaler Unterraum mit Basis\\
    $\{e_1, \dotsc, e_n\}$ und $Y \subset X$ ein abgeschlossener Unterraum mit
    $Y \cap E = \{0\}$.
    Dann gilt:
    \begin{enumerate}
        \item
        $\exists_{e_1', \dotsc, e_n' \in X'} \forall_{i,j=1,\dotsc,n}\;
        e_j'|_Y = 0,\; e_j'(e_i) = \delta_{ij}$

        \item
        Es gibt eine stetige Projektion $P$ auf $E$ mit $Y \subset \Kern(P)$.
    \end{enumerate}
\end{Satz}

\pagebreak

\section{%
    Kompakte Operatoren und adjungierte Operatoren auf Banachräumen%
}

\subsection{%
    \name{Jordan}sche Normalform für kompakte Operatoren%
}

\begin{Satz}{\name{Jordan}sche Normalform für kompakte Operatoren}\\
    Seien $X$ ein Banachraum und $T \in \K(X)$.
    Dann gilt:
    \begin{enumerate}
        \item
        Die Aussagen aus dem Satz über das Spektrum kompakter Operatoren gelten, d.\,h.\\
        $\sigma(T) \setminus \{0\} = \sigma_p(T) \setminus \{0\}$
        besteht aus höchstens abzählbar vielen Eigenwerten
        mit $0$ als einzigem Häufungspunkt (falls $|\sigma(T)| = \infty$),
        die Vielfachheit von $\lambda \in \sigma(T) \setminus \{0\}$ ist endlich und
        für $\dim X = \infty$ ist $0 \in \sigma(T)$.

        \item
        Für $\lambda \in \sigma(T) \setminus \{0\}$ und
        $n_\lambda := \max\{n \in \natural \;|\; \Kern((\lambda\id - T)^{n-1}) \not=
        \Kern((\lambda\id - T)^n)\}$ der \begriff{Ordnung} von $\lambda$ gilt
        $1 \le n_\lambda < \infty$.

        \item
        Für $\lambda \in \sigma(T) \setminus \{0\}$ gilt
        $X = \Kern((\lambda\id - T)^{n_\lambda}) \oplus \Bild((\lambda\id - T)^{n_\lambda})$
        (\begriff{\name{Riesz}-Zerlegung}).
        Beide Unterräume sind abgeschlossen und $T$-invariant.
        Der \begriff{charakteristische Unterraum} $\Kern((\lambda\id - T)^{n_\lambda})$
        von $T$ zum Eigenwert $\lambda$ ist endlich-dimensional.

        \item
        Für $\lambda \in \sigma(T) \setminus \{0\}$ gilt:
        Für $n = 1, \dotsc, n_\lambda$ gibt es Unterräume
        $E_n \subset \Kern((\lambda\id - T)^n)$ mit
        $E_n \cap \Kern((\lambda\id - T)^{n-1}) = \{0\}$, sodass
        $\Kern((\lambda\id - T)^{n_\lambda}) = \bigoplus_{k=1}^{n_\lambda} N_k$ mit\\
        $N_k := \bigoplus_{\ell=0}^{k-1} (\lambda\id - T)^\ell (E_k)$.

        \item
        $N_k$, $k = 1, \dotsc, n_\lambda$, ist $T$-invariant und die Dimensionen
        $d_k := \dim((\lambda\id - T)^{\ell} (E_k))$ sind unabhängig von
        $\ell \in \{0, \dotsc, k - 1\}$.

        \item
        Ist $\{e_{k,j} \;|\; j = 1, \dotsc, d_k\}$ eine Basis von $E_k$ für
        $k = 1, \dotsc, n_\lambda$, dann ist\\
        $\{(\lambda\id - T)^\ell e_k \;|\; 0 \le \ell < k \le n_\lambda,\; 1 \le j \le d_k\}$
        eine Basis von $\Kern((\lambda\id - T)^{n_\lambda})$.\\
        Mit $x = \sum_{k,j,\ell} \alpha_{k,j,\ell} (\lambda\id - T)^\ell e_{k,j}$
        und $y = \sum_{k,j,\ell} \beta_{k,j,\ell} (\lambda\id - T)^\ell e_{k,j}$ gilt\\
        $Tx = y \iff \smallpmatrix{\lambda & -1 & & \\ & \ddots & \ddots & \\
        & & \lambda & -1 \\ & & & \lambda}
        \smallpmatrix{\alpha_{k,j,0} \\ \vdots \\ \alpha_{k,j,k-1}} =
        \smallpmatrix{\beta_{k,j,0} \\ \vdots \\ \beta_{k,j,k-1}}$.
    \end{enumerate}
\end{Satz}

\vspace{3mm}
\linie

\begin{Kor}
    Seien $X$ ein Banachraum und $T \in \K(X)$.
    Dann gilt:
    \begin{enumerate}
        \item
        Für $\lambda \in \sigma(T) \setminus \{0\}$ gilt
        $\sigma(T|_{\Bild((\lambda\id - T)^{n_\lambda})}) = \sigma(T) \setminus \{\lambda\}$.

        \item
        Ist $P_\lambda$ für $\lambda \in \sigma(T) \setminus \{0\}$ die stetige Projektion
        auf $\Kern((\lambda\id - T)^{n_\lambda})$ gemäß der Riesz-Zerlegung,
        dann gilt $\forall_{\lambda, \mu \in \sigma(T) \setminus \{0\}}\;
        P_\lambda P_\mu = \delta_{\lambda\mu} P_\lambda$.
    \end{enumerate}
\end{Kor}

\linie

\begin{Kor}
    Seien $X$ ein Banachraum, $T \in \K(X)$ und $\lambda \in \sigma(T) \setminus \{0\}$.
    Dann hat die Resolventenfunktion $R(\cdot, T)$ in $\lambda$ einen isolierten Pol
    der Ordnung $n_\lambda$, d.\,h.
    $\mu \mapsto (\mu - \lambda)^{n_\lambda} R(\mu, T)$ kann in $\lambda$ holomorph fortgesetzt
    werden und der fortgesetzte Wert in $\lambda$ ist ungleich Null.
\end{Kor}

\pagebreak

\subsection{%
    Adjungierter Operator%
}

\begin{Def}{Adjungierte}
    Seien $X, Y$ normierte Räume und $T \in \Lin(X, Y)$.\\
    Dann heißt der Operator $T' \in \Lin(Y', X')$ definiert durch
    $(T'y')(x) := y'(Tx)$ für $y' \in Y'$ und $x \in X$ der zu $T$ \begriff{adjungierte Operator}.
\end{Def}

\begin{Satz}{Eigenschaften der Adjungierten}
    \begin{enumerate}
        \item
        $T \mapsto T'$ ist eine lineare, isometrische Einbettung
        von $\Lin(X, Y)$ nach $\Lin(Y', X')$.

        \item
        Seien $X, Y, Z$ normierte Räume, $T_1 \in \Lin(X, Y)$ und $T_2 \in \Lin(Y, Z)$.\\
        Dann ist $(T_2 T_1)' = T_1' T_2'$.

        \item
        Seien $J_X\colon X \rightarrow X''$, $x_0 \mapsto J_{x_0}$ mit
        $J_{x_0}(x') := x'(x_0)$ für $x' \in X'$ und analog $J_Y\colon Y \rightarrow Y''$.\\
        Dann gilt $T'' J_X = J_Y T$.
    \end{enumerate}
\end{Satz}

\begin{Bsp}
    \begin{enumerate}[label=\emph{(\alph*)}]
        \item
        Für $X = Y = \real^n$ mit der euklidischen Norm und
        $T = (a_{ij})_{i,j=1,\dotsc,n}$ ist\\
        $T' = (a_{ji})_{i,j=1,\dotsc,n} = T^\ast$,
        wobei $T^\ast$ die Hilbertraum-Adjungierte ist.

        \item
        Für $X = Y = \complex^n$ mit der euklidischen Norm und
        $T = (a_{ij})_{i,j=1,\dotsc,n}$ ist\\
        $T' = (a_{ji})_{i,j=1,\dotsc,n} \not= (\overline{a_{ji}})_{i,j=1,\dotsc,n} = T^\ast$.

        \item
        Für $X = Y = L^2([0,1], \complex)$ und
        $T\colon X \rightarrow X$,
        $(Tf)(y) := \int_0^1 K(x, y)f(x)\dx$ ist\\
        $(T'g)(x) := \int_0^1 K(x, y)g(y)\dy$
        (nicht gleich
        $(T^\ast g)(x) = \int_0^1 \overline{K(x, y)} g(y) \dy$).

        \item
        Sind $X, Y$ Hilberträume und $\R_X\colon X \rightarrow X'$ und
        $\R_Y\colon Y \rightarrow Y'$ die Isometrien aus dem Rieszschen Darstellungssatz
        (z.\,B. $(\R_X x_1)(x_2) := \sp{x_2, x_1}_X$),
        dann gilt $T^\ast = \R_X^{-1} T' \R_Y$.\\
        Für $x \in X$ und $y \in Y$ gilt nämlich
        $((T' \R_Y)(y))(x)
        = (T'(\R_Y y))(x)
        = (\R_Y y)(Tx)$\\
        $= \sp{Tx, y}_Y
        = \sp{x, T^\ast y}_X
        = (\R_X (T^\ast y))(x)
        = ((\R_X T^\ast)(y))(x)$.
    \end{enumerate}
\end{Bsp}

\pagebreak

\subsection{%
    \name{Fredholm}sche Alternative%
}

\begin{Satz}{Satz von \scshape\,\!\name{Schauder}}
    Seien $X, Y$ Banachräume und $T \in \Lin(X, Y)$.\\
    Dann gilt $T \in \K(X, Y)$ genau dann, wenn $T' \in \K(Y', X')$.
\end{Satz}

\linie

\begin{Def}{Annihilator}
    Seien $X$ ein Banachraum und $Z \subset X$ ein Unterraum.\\
    Dann heißt $Z^\circ := \{x' \in X' \;|\; x'|_Z = 0\}$
    \begriff{Annihilator} von $Z$.
\end{Def}

\begin{Def}{Kodimension}
    Seien $X$ ein $\KK$-Vektorraum und $Z \subset X$ ein Unterraum.\\
    Dann ist $\codim Z := \dim X/Z$ die \begriff{Kodimension} von $Z$ in $X$.
\end{Def}

\begin{Bem}
    Ist $Y$ ein Komplement von $Z$ in $X$ (d.\,h. $X = Y \oplus Z$), dann gilt
    $\codim Z = \dim Y$.
    %Komplemente existieren immer, allerdings sind sie i.\,A. nicht eindeutig.
\end{Bem}

\begin{Satz}{Eigenschaften des Annihilators}
    Seien $X, Y$ Banachräume und $Z \subset X$ ein Unterraum.
    \begin{enumerate}
        \item
        Ist $X$ ein Hilbertraum, dann ist $Z^\circ = \R_X(Z^\orth)$.

        \item
        Für $T \in \Lin(X, Y)$ gilt $\Kern(T') = \Bild(T)^\circ$.

        \item
        Ist $Z$ abgeschlossen und $\codim Z < \infty$, dann ist $\dim Z^\circ = \codim Z$.
    \end{enumerate}
\end{Satz}

\linie

\begin{Satz}{Inverse der Adjungierten}
    Seien $X, Y$ Banachräume und $T \in \Lin(X, Y)$.\\
    Dann existiert $T^{-1} \in \Lin(Y, X)$ genau dann,
    wenn $(T')^{-1} \in \Lin(X', Y')$ existiert.\\
    In diesem Fall gilt $(T^{-1})' = (T')^{-1}$.
\end{Satz}

\linie

\begin{Satz}{\name{Fredholm}sche Alternative}
    Seien $X$ ein Banachraum, $T \in \K(X)$ und
    $\lambda \in \KK \setminus \{0\}$.\\
    Dann gilt:
    Zu $y \in X$ besitzt die Gleichung $Tx - \lambda x = y$ eine Lösung $x \in X$ genau dann,
    wenn $x'(y) = 0$ für alle Lösungen $x' \in X'$ der
    \begriff{homogenen adjungierten Gleichung} $T'x' - \lambda x' = 0$ gilt.
    Die dadurch gegebene endliche Anzahl der Nebenbedingungen an $y$ ist gleich der Anzahl
    linear unabhängiger Lösungen $z$ der \begriff{homogenen Gleichung} $Tz - \lambda z = 0$.
\end{Satz}

\begin{Bem}
    Der Satz lässt sich auch wie folgt formulieren:
    Entweder
    \begin{itemize}
        \item
        $Tz - \lambda z = 0$ besitzt nur die triviale Lösung,

        \item
        $T'x' - \lambda x' = 0$ besitzt nur die triviale Lösung und

        \item
        $Tx - \lambda x = y$ ist für alle $y \in Y$ eindeutig lösbar
    \end{itemize}
    oder
    \begin{itemize}
        \item
        $Tz - \lambda z = 0$ besitzt $n := \dim(\Kern(\lambda\id - T))$
        ($1 \le n < \infty$) linear unabhängige Lösungen,

        \item
        $T'x' - \lambda x' = 0$ besitzt $n$ linear unabhängige Lösungen und

        \item
        $Tx - \lambda x = y$ ist für $y \in Y$ genau dann lösbar, wenn
        $x'(y) = 0$ für alle $x' \in \Kern(\lambda\id' - T')$.
    \end{itemize}
\end{Bem}

\pagebreak

\section{%
    Lokalkonvexe und schwache Topologien%
}

\subsection{%
    Grundbegrif"|fe aus der Topologie%
}

\begin{Def}{topologischer Raum}
    Seien $X$ eine Menge und $\T \subset \P(X)$.\\
    Dann heißt $(X, \T)$ \begriff{topologischer Raum}, falls
    \begin{enumerate}
        \item
        $\emptyset \in \T$, $X \in \T$,

        \item
        $\forall_{\T' \subset \T}\; \bigcup_{U \in \T'} U \in \T$ und

        \item
        $\forall_{U_1, U_2 \in \T}\; U_1 \cap U_2 \in \T$.
    \end{enumerate}
    In diesem Fall heißt $\T$ \begriff{Topologie} auf $X$ und die Elemente von $\T$ heißen
    \begriff{of"|fen}.
\end{Def}

\begin{Bem}
    Im Folgenden ist $(X, \T)$ ein topologischer Raum und $M \subset X$.
\end{Bem}

\begin{Def}{abgeschlossen}
    $M \subset X$ heißt \begriff{abgeschlossen}, falls $X \setminus M$ of"|fen ist.
\end{Def}

\begin{Def}{Inneres}
    $\interior{M} := \{x \in M \;|\; \exists_{O \in \T}\; O \subset M,\; x \in O\}$
    heißt das \begriff{Innere} von $M$.
\end{Def}

\begin{Def}{Abschluss}
    $\overline{M} := X \setminus \interior{X \setminus M}$ heißt \begriff{Abschluss} von $M$.
\end{Def}

\begin{Def}{Rand}
    $\partial M := \overline{M} \setminus \interior{M}$ heißt \begriff{Rand} von $M$.
\end{Def}

\begin{Def}{dicht}
    $M$ heißt \begriff{dicht} in $X$, falls $\overline{M} = X$.
\end{Def}

\begin{Satz}{abgeschlossene Mengen}
    $\emptyset$ und $X$ sind abgeschlossen.
    Schnitte beliebig vieler und Vereinigungen endlicher vieler abgeschlossener Mengen sind
    abgeschlossen.
\end{Satz}

\linie

\begin{Def}{Umgebung}
    Seien $(X, \T)$ ein topologischer Raum und $x \in X$.\\
    Dann heißt $U \subset X$ \begriff{Umgebung} von $x$, falls
    $\exists_{O \in \T}\; O \subset U,\; x \in O$
    (d.\,h. $x \in \interior{U}$).
\end{Def}

\begin{Def}{Umgebungsfilter}
    $\U(x) := \{U \subset X \;|\; U \text{ Umgebung von } x\}$ heißt
    \begriff{Umgebungsfilter} von $x$.
\end{Def}

\begin{Def}{Umgebungsbasis}\\
    Eine Teilfamilie $\V(x) \subset \U(x)$ heißt \begriff{Umgebungsbasis} von $x$, falls
    $\forall_{U \in \U(x)} \exists_{V \in \V(x)}\; V \subset U$.
\end{Def}

\begin{Satz}{Eigenschaften des Umgebungsfilters}
    Seien $(X, \T)$ ein topologischer Raum und $x \in X$.\\
    Dann gilt:
    \begin{enumerate}
        \item
        $\forall_{U \in \U(x)}\; x \in U$

        \item
        $\forall_{U \in \U(x)} \exists_{V \in \U(x)} \forall_{y \in V}\; U \in \U(y)$

        \item
        $\forall_{U \in \U(x)} \forall_{V \supset U}\; V \in \U(x)$

        \item
        $\forall_{U, V \in \U(x)}\; U \cap V \in \U(x)$
    \end{enumerate}
\end{Satz}

\begin{Satz}{Umgebungsfilter induziert Topologie}
    Sei $X$ eine Menge und $\U(x) \subset \P(X)$ für jedes $x \in X$,
    sodass \emph{(1)} bis \emph{(4)} von eben erfüllt sind.
    Dann gibt es genau eine Topologie $\T$ auf $X$, sodass $\U(x)$ für $x \in X$ der
    Umgebungsfilter von $x$ ist.
    Es gilt $\T = \bigcup_{x \in X} \O(x) \cup \{\emptyset\}$,
    wobei $\O(x) := \{\interior{U} \;|\; U \in \U(x)\}$ und
    $\interior{U} := \{y \in X \;|\; U \in \U(y)\}$.
\end{Satz}

\linie
\pagebreak

\begin{Satz}{Metrik induziert Topologie}
    Jeder metrische Raum induziert einen topologischen Raum.
    In diesem Fall besitzt jeder Punkt $x$ des topologischen Raums eine abzählbare
    Umgebungsbasis $\V(x)$.
    Allerdings ist nicht jeder topologische Raum \begriff{metrisierbar}
    (d.\,h. die Topologie wird nicht von einer Metrik induziert).
\end{Satz}

\begin{Def}{feiner/gröber}
    Seien $\T_1, \T_2$ Topologien auf $X$.\\
    Dann heißt $\T_2$ \begriff{stärker/feiner} als $\T_1$ bzw.
    $\T_1$ \begriff{schwächer/gröber} als $\T_2$, falls $\T_1 \subsetneqq \T_2$.
\end{Def}

\begin{Def}{\name{Hausdorff}-Raum}
    Ein topologischer Raum $(X, \T)$ heißt \begriff{\name{Hausdorff}-Raum}, falls\\
    $\forall_{x, y \in X,\; x \not= y} \exists_{U \in \U(x)} \exists_{V \in \U(y)}\;
    U \cap V = \emptyset$.
\end{Def}

\begin{Def}{Konvergenz}
    Eine Folge $(x_n)_{n \in \natural}$ in $X$ konvergiert gegen $x \in X$
    ($x_n \xrightarrow{n \to \infty} x$), falls\\
    $\forall_{U \in \U(x)} \exists_{n_U \in \natural} \forall_{n \ge n_U}\; x_n \in V$.
\end{Def}

\begin{Satz}{GWe in \name{Hausdorff}-Räumen eindeutig}\\
    Grenzwerte von Folgen in Hausdorff-Räumen sind eindeutig.
\end{Satz}

\begin{Def}{folgenabgeschlossen}\\
    $A \subset X$ heißt \begriff{folgenabgeschlossen}, falls
    $\forall_{x \in X} \forall_{(x_n)_{n \in \natural} \text{ Folge in } A,\; x_n \to x}\;
    x \in A$.
\end{Def}

\begin{Satz}{abg. $\Rightarrow$ folgenabg.}
    Wenn $A \subset X$ abgeschlossen ist,\\
    dann ist $A$ auch folgenabgeschlossen.
    Die Umkehrung gilt i.\,A. nicht.
\end{Satz}

\linie

\begin{Def}{stetig}
    Seien $(X, \T_X), (Y, \T_Y)$ topologische Räume.\\
    Eine Abbildung $T\colon X \rightarrow Y$ heißt \begriff{stetig}, falls
    $\forall_{x \in X} \forall_{V \in \U(Tx)} \exists_{U \in \U(x)}\; T(U) \subset V$.
\end{Def}

\begin{Satz}{äquivalente Beschreibungen von Stetigkeit}
    Folgende Aussagen sind äquivalent:
    \begin{enumerate}
        \item
        $T$ ist stetig.

        \item
        Für alle of"|fenen Teilmengen $O \subset Y$ ist $T^{-1}(O) \subset X$ of"|fen.

        \item
        Für alle abgeschlossenen Teilmengen $A \subset Y$ ist $T^{-1}(A) \subset X$ abgeschlossen.
    \end{enumerate}
\end{Satz}

\begin{Satz}{stetig $\Rightarrow$ folgenstetig}
    Wenn $T$ stetig ist, dann ist $T$ auch \begriff{folgenstetig}, d.\,h.\\
    $\forall_{x \in X}
    \forall_{(x_n)_{n \in \natural} \text{ Folge in } X,\; x_n \to x}\;
    T(x_n) \xrightarrow{n \to \infty} T(x)$.
    Die Umkehrung gilt i.\,A. nicht.
\end{Satz}

\linie

\begin{Def}{kompakt}
    Sei $(X, \T)$ ein topologischer Raum.
    $K \subset X$ heißt \begriff{kompakt}, falls\\
    $\forall_{I \text{ Indexmenge}} \forall_{O_i \subset X \text{ of"|fen},\;
    K \subset \bigcup_{i \in I} O_i} \exists_{i_1, \dotsc, i_n \in I}\;
    K \subset \bigcup_{j=1}^n O_{i_j}$.
\end{Def}

\begin{Def}{folgenkompakt}
    $K$ heißt \begriff{folgenkompakt}, falls\\
    $\forall_{(x_n)_{n \in \natural} \text{ Folge in } K}
    \exists_{(x_{n_k})_{k \in \natural} \text{ Teilfolge}} \exists_{x \in K}\;
    x = \lim_{k \to \infty} x_{n_k}$.
\end{Def}

\begin{Bem}
    Kompaktheit und Folgenkompaktheit sind i.\,A. nicht äquivalent.
\end{Bem}

\linie

\begin{Def}{separabel}
    Sei $(X, \T)$ ein topologischer Raum.\\
    Dann heißt $(X, \T)$ separabel, falls $X$ eine abzählbare, dichte Teilmenge enthält.
\end{Def}

\linie

\begin{Satz}{Relativtopologie}
    Seien $(X, \T)$ ein topologischer Raum und $A \subset X$.\\
    Dann ist $(A, \T_A)$ ein topologischer Raum mit der \begriff{Relativtopologie}
    $\T_A := \{U \cap A \;|\; U \in \T\}$.
\end{Satz}

\begin{Satz}{Produkttopologie}
    Seien $I$ eine Indexmenge, $(X_i, \T_i)_{i \in I}$ eine Familie topologischer Räume und
    $X := \prod_{i \in I} X_i$.
    Dann ist $(X, \T)$ ein topologischer Raum mit der \begriff{Produkttopologie} $\T$ mit Basis
    $\{\prod_{i \in I} O_i \;|\; \forall_{i \in I}\; O_i \in \T_i,\;
    \text{fast alle } O_i = X_i\}$
    (beliebige Vereinigungen hinzunehmen).
\end{Satz}

\begin{Satz}{Satz von \scshape\,\!\name{Tychonov}}
    Seien $I$ eine Indexmenge, $(X_i, \T_i)_{i \in I}$ eine Familie topologischer Räume,
    $X := \prod_{i \in I} X_i$ und $\T$ die Produkttopologie auf $X$.\\
    Dann ist $X$ kompakt genau dann, wenn $X_i$ für alle $i \in I$ kompakt ist.
\end{Satz}

\begin{Bem}
    Dieser Satz ist äquivalent zum Auswahlaxiom.
\end{Bem}

\pagebreak

\subsection{%
    Lokalkonvexe Topologie%
}

\begin{Def}{lokalkonvexe Topologie}
    Seien $X$ ein $\KK$-Vektorraum und $(p_\alpha)_{\alpha \in I}$ eine Familie von
    Halbnormen auf $X$ ($I$ Indexmenge).
    Für $x \in X$ definiert man
    \begin{itemize}
        \item
        $U_{\varepsilon,H}(x) := \{y \in X \;|\; \forall_{\alpha \in H}\;
        p_\alpha(x - y) < \varepsilon\}$
        für $\varepsilon > 0$ und $H \subset I$ endlich,

        \item
        $\V(x) := \{U_{\varepsilon,H}(x) \;|\; \varepsilon > 0,\; H \subset I \text{ endlich}\}$,

        \item
        $\U(x) := \{U \subset X \;|\; \exists_{V \in \V(x)}\; V \subset U\}$ und

        \item
        $\T := \{O \subset X \;|\; \forall_{x \in O} \exists_{V \in \V(x)}\; V \subset O\}$.
    \end{itemize}
    Man kann zeigen, dass $(X, \T)$ ein topologischer Raum ist,
    wobei $\U(x)$ der Umgebungsfilter und $\V(x)$ eine Umgebungsbasis von $x \in X$ ist.
    $\T$ heißt die von $(p_\alpha)_{\alpha \in I}$ induzierte \begriff{lokalkonvexe Topologie}
    auf $X$ und $(X, \T)$ heißt \begriff{lokalkonvexer (topologischer) Raum}.
\end{Def}

\begin{Bem}
    Die Topologie heißt deshalb lokalkonvex, weil es für jeden Punkt $x \in X$ eine
    Umgebungsbasis aus konvexen Mengen $U_{\varepsilon,H}(x)$ gibt.
\end{Bem}

\begin{Bem}
    $(X, \T)$ ist bereits eindeutig durch die Nullumgebungsbasis $\V(0)$ festgelegt,
    da $\V(x) = x + \V(0)$ und $\U(x) = x + \U(0)$
    (weil $U_{\varepsilon,H}(x) = x + U_{\varepsilon,H}(0)$).
\end{Bem}

\linie

\begin{Lemma}{Charakterisierung der Konvergenz}
    Seien $(X, \T)$ ein lokalkonvexer Raum, der durch $(p_\alpha)_{\alpha \in I}$ induziert wird,
    $(x_n)_{n \in \natural}$ eine Folge in $X$ und $x \in X$.
    Dann sind äquivalent:
    \begin{enumerate}
        \item
        $x_n \xrightarrow{n \to \infty} x$

        \item
        $x_n - x \xrightarrow{n \to \infty} 0$

        \item
        $\forall_{\alpha \in I}\; p_\alpha(x_n - x) \xrightarrow{n \to \infty} 0$
    \end{enumerate}
\end{Lemma}

\linie

\begin{Lemma}{Charakterisierung von \name{hausdorff}sch}\\
    Sei $(X, \T)$ ein lokalkonvexer Raum, der durch $(p_\alpha)_{\alpha \in I}$ induziert wird.
    Dann sind äquivalent:
    \begin{enumerate}
        \item
        $(X, \T)$ ist hausdorffsch.

        \item
        $\forall_{x \in X \setminus \{0\}} \exists_{\alpha \in I}\; p_\alpha(x) \not= 0$
    \end{enumerate}
\end{Lemma}

\linie

\begin{Lemma}{Charakterisierung von Stetigkeit}
    Seien $(X, \T_X), (Y, \T_Y)$ von den Halbnormfamilien $(p_\alpha)_{\alpha \in I_X}$ bzw.
    $(q_\beta)_{\beta \in I_Y}$ induzierte lokalkonvexe Räume und $T\colon X \rightarrow Y$
    linear.\\
    Dann sind äquivalent:
    \begin{enumerate}
        \item
        $T$ ist stetig.

        \item
        $T$ ist stetig in $0$.

        \item
        $\forall_{\beta \in I_Y} \exists_{H \subset I_X \text{ endlich}} \exists_{M \ge 0}
        \forall_{x \in X}\; q_\beta(Tx) \le M \cdot \max_{\alpha \in H} p_\alpha(x)$
    \end{enumerate}
\end{Lemma}

\begin{Kor}
    Seien $(X, \T)$ ein lokalkonvexer Raum und $T\colon X \rightarrow \KK$ linear.\\
    Dann ist $T$ stetig genau dann, wenn
    $\exists_{H \subset I \text{ endlich}} \exists_{M \ge 0}
    \forall_{x \in X}\; |Tx| \le M \cdot \max_{\alpha \in H} p_{\alpha_i}(x)$.
\end{Kor}

\begin{Def}{Dualraum}
    Sei $(X, \T)$ ein lokalkonvexer Raum.\\
    Dann heißt $X' := \{T\colon X \rightarrow \KK \;|\; T \text{ linear und stetig}\}$
    \begriff{Dualraum} von $X$.
\end{Def}

\begin{Bem}
    Es gibt Verallgemeinerungen des Satzes von Hahn-Banach und der Trennungssätze
    für lokalkonvexe Räume.
\end{Bem}

\pagebreak

\subsection{%
    Schwache Konvergenz und \texorpdfstring{Schwach$\ast$}{Schwach*}-Konvergenz%
}

\begin{Def}{schwache Topologie}
    Seien $X$ ein normierter Raum und $X'$ der Dualraum von $X$.\\
    $(p_{x'})_{x' \in X'}$ mit $p_{x'}(x) := |x'(x)|$ für $x \in X$ ist eine Familie von
    Halbnormen auf $X$.\\
    Die induzierte lokalkonvexe Topologie heißt \begriff{schwache Topologie}
    $\sigma(X, X')$ auf $X$.
\end{Def}

\begin{Def}{Schwach$\ast$-Topologie}
    Seien $X$ ein normierter Raum und $X'$ der Dualraum von $X$.\\
    $(p_x)_{x \in X}$ mit $p_x(x') := |x'(x)|$ für $x' \in X'$ ist eine Familie von
    Halbnormen auf $X'$.\\
    Die induzierte lokalkonvexe Topologie heißt \begriff{Schwach$\ast$-Topologie}
    $\sigma(X', X)$ auf $X'$.
\end{Def}

\begin{Def}{schwache Konvergenz}
    Seien $X$ ein normierter Raum, $(x_n)_{n \in \natural}$ eine Folge in $X$ und $x \in X$.\\
    Dann \begriff{konvergiert} $(x_n)_{n \in \natural}$ \begriff{schwach} gegen $x$
    ($x_n \rightharpoonup x$), falls
    $x_n \xrightarrow{n \to \infty} x$ bzgl. $\sigma(X, X')$.
\end{Def}

\begin{Def}{Schwach$\ast$-Konvergenz}
    Seien $X$ ein normierter Raum, $(x_n')_{n \in \natural}$ eine Folge in $X'$ und $x \in X'$.\\
    Dann \begriff{konvergiert} $(x_n')_{n \in \natural}$ \begriff{schwach$\ast$} gegen $x'$
    ($x_n' \xrightharpoonup{\ast} x'$), falls
    $x_n' \xrightarrow{n \to \infty} x'$ bzgl. $\sigma(X', X)$.
\end{Def}

\begin{Bem}
    Schwache Konvergenz ist äquivalent
    zu $\forall_{x' \in X'}\; x'(x_n) \xrightarrow{n \to \infty} x'(x)$.\\
    Analog ist Schwach$\ast$-Konvergenz äquivalent
    zu $\forall_{x \in X}\; x_n'(x) \xrightarrow{n \to \infty} x'(x)$.
\end{Bem}

\subsection{%
    Distributionen%
}

\begin{Def}{Distributionen}
    Seien $\Omega \subset \real^d$ of"|fen und $\D(\Omega) := \C^\infty_c(\Omega)$.
    Zunächst definiert man Halbnormen $(p_m)_{m \in \natural_0}$
    durch $p_m(\varphi) := \sup_{|\beta| \le m} \norm{\partial_x^\beta \varphi}_{\C^0(\Omega)}$.
    Anschließend definiert man $(p_\alpha)_{\alpha \in I}$ als Familie aller Halbnormen
    $p_\alpha$, sodass
    $\forall_{K \subset \Omega \text{ kpkt.}} \exists_{C \ge 0} \exists_{m \in \natural_0}
    \forall_{\varphi \in \C^\infty_c(K)}\; p_\alpha(\varphi) \le C \cdot p_m(\varphi)$.\\
    Dann heißt der Dualraum $\D'(\Omega)$ von $(\D(\Omega), \T_\D)$
    \begriff{Raum der Distributionen} auf $\Omega$,
    wobei $\T_\D$ die von $(p_\alpha)_{\alpha \in I}$ induzierte lokalkonvexe Topologie auf
    $\D(\Omega)$ ist.
\end{Def}

\begin{Bem}
    Sei $(\varphi_n)_{n \in \natural}$ eine Folge in $\D(\Omega)$ und $\varphi \in \D(\Omega)$.\\
    Dann gilt $\varphi_n \xrightarrow{n \to \infty} \varphi$ (bzgl. $\T_\D$) genau dann, wenn
    es $K \subset \Omega$ kompakt gibt mit\\
    $\supp(\varphi) \subset K$,
    $\forall_{n \in \natural}\; \supp(\varphi_n) \subset K$ und
    $\forall_{\beta \in \natural_0^d}\; \partial_x^\beta \varphi_n
    \xrightarrow{\norm{\cdot}_{\C^0(K)}} \partial_x^\beta \varphi$.
\end{Bem}

\linie

\begin{Lemma}{Eigenschaften des Testfunktionenraums}
    \begin{enumerate}
        \item
        Seien $K \subset \Omega$ kompakt und $\D_K(\Omega) := \C^\infty_c(K)$.
        Dann ist die von $(p_m)_{m \in \natural_0}$ erzeugte lokalkonvexe Topologie
        auf $\D_K(\Omega)$ gleich der
        Relativtopologie von $(D(\Omega), \T_\D)$ auf $\D_K(\Omega)$.

        \item
        $(\D(\Omega), \T_\D)$ ist hausdorffsch.
    \end{enumerate}
\end{Lemma}

\begin{Lemma}{Charakterisierung der Stetigkeit}
    Sei $T\colon \D(\Omega) \rightarrow \KK$ linear.
    Dann sind äquivalent:
    \begin{enumerate}
        \item
        $T \in \D'(\Omega)$

        \item
        $\forall_{K \subset \Omega \text{ kpkt.}}\; T|_{\D_K(\Omega)} \in (\D_K(\Omega))'$

        \item
        $\forall_{K \subset \Omega \text{ kpkt.}} \exists_{m \in \natural_0}
        \exists_{C \ge 0} \forall_{\varphi \in \D_K(\Omega)}\; |T\varphi| \le C p_m(\varphi)$

        \item
        $T$ ist \begriff{folgenstetig}, d.\,h.
        aus $\varphi_n \to \varphi$ in $(\D(\Omega), \T_D)$ folgt $T\varphi_n \to T\varphi$.

        \item
        $T$ ist folgenstetig in $0$, d.\,h.
        aus $\varphi_n \to 0$ in $(\D(\Omega), \T_D)$ folgt $T\varphi_n \to 0$.
    \end{enumerate}
\end{Lemma}

\linie

\begin{Def}{Schwach$\ast$-Topologie für Distributionen}
    Sei die Familie $(p_\varphi)_{\varphi \in \D(\Omega)}$ von Halbnormen auf $\D'(\Omega)$
    definiert durch $p_\varphi(T) := |T\varphi|$ für alle $T \in \D'(\Omega)$.\\
    Dann heißt die von $(p_\varphi)_{\varphi \in \D(\Omega)}$ induzierte lokalkonvexe Topologie
    \begriff{Schwach$\ast$-Topologie}\\
    $\sigma(\D'(\Omega), \D(\Omega))$ auf $\D'(\Omega)$.
\end{Def}

\begin{Bem}
    Sei $(T_n)_{n \in \natural}$ eine Folge in $\D'(\Omega)$ und $T \in \D'(\Omega)$.\\
    Dann gilt $T_n \xrightarrow{n \to \infty} T$ (bzgl. $\sigma(\D'(\Omega), \D(\Omega))$)
    genau dann, wenn $\forall_{\varphi \in \D(\Omega)}\;
    T_n \varphi \xrightarrow{n \to \infty} T \varphi$.
\end{Bem}

\pagebreak

\subsection{%
    Beispiele für Distributionen und distributionelle Ableitung%
}

\begin{Def}{induzierte reguläre Distribution}\\
    Sei $f \in L^1_\loc(\Omega)$
    mit $L^1_\loc(\Omega) := \{f\colon \Omega \rightarrow \real \text{ messbar} \;|\;
    \forall_{K \subset \Omega \text{ kpkt.}}\; f \in L^1(K)\}$.\\
    Dann heißt $T_f \in \D'(\Omega)$ mit $T_f(\varphi) := \int_\Omega f(y)\varphi(y)\dy$
    für $\varphi \in \D(\Omega)$
    die durch $f$ \begriff{induzierte reguläre Distribution}.
\end{Def}

\begin{Bem}
    Die Abbildung $L^1_\loc(\Omega) \rightarrow \D'(\Omega)$, $f \mapsto T_f$ ist injektiv
    (sie ist linear und aus $T_f = 0$ folgt $T_f(\varphi) = 0$ für alle $\varphi \in \D(\Omega)$,
    also $f = 0$ f.ü. nach dem Fundamentallemma der Variationsrechnung).
    Daher kann man die Funktionen $f \in L^1_\loc(\Omega)$ mit den
    induzierten regulären Distributionen $T_f \in \D'(\Omega)$ identifizieren.\\
    Eine Distribution $T \in \D'(\Omega)$ heißt \begriff{regulär},
    falls $\exists_{f \in L^1_\loc(\Omega)}\; T = T_f$,
    d.\,h. falls sie im Bild dieser Abbildung ist.
    Nicht jede Distribution ist regulär, wie die Delta-Distribution zeigt.
\end{Bem}

\begin{Def}{\name{Dirac}-/Delta-Distribution}
    Sei $x \in \Omega$.
    Dann heißt $\delta_x \in \D'(\Omega)$ mit $\delta_x(\varphi) := \varphi(x)$ für
    $\varphi \in \D(\Omega)$ \begriff{\name{Dirac}- oder Delta-Distribution} zum Punkt $x$.
    Man schreibt $\delta := \delta_0$.
\end{Def}

\begin{Bem}
    $\delta_x$ ist nicht regulär.
    Angenommen, es gilt $\delta_x = T_f$ für ein $f \in L^1_\loc(\Omega)$.
    Definiere für $\varphi \in \D(\Omega)$ die Testfunktion
    $\psi_\varphi \in \D(\Omega)$ durch $\psi_\varphi(y) := |y - x|^2 \varphi(y)$.
    Dann gilt für alle $\varphi \in \D(\Omega)$, dass
    $0 = \psi_\varphi(x) = \delta_x(\psi_\varphi) = T_f(\psi_\varphi) =
    \int_\Omega f(y) |y - x|^2 \varphi(y) \dy$.
    Nach dem Fundamentallemma der Variationsrechnung folgt,
    dass $f(y) |y - x|^2 = 0$ für fast alle $y \in \Omega$, d.\,h. $f = 0$ f.ü.
    Damit wäre aber $\delta_x = T_f = 0$, ein Widerspruch
    (es gibt $\varphi \in \D(\Omega)$ mit $\delta_x(\varphi) = \varphi(x) \not= 0$).\\
    Trotzdem schreibt man formal häufig
    $\int_\Omega \delta_x(y) \varphi(y) \dy := \varphi(x)$.
\end{Bem}

\begin{Satz}{\name{Dirac}-Folge}
    Sei $f_n \in L^1_\loc(\Omega)$ mit
    $f_n(x) := \left(\frac{n}{4\pi}\right)^{d/2} \exp\!\left(-\frac{n|x|^2}{4}\right)$
    für $n \in \natural$.\\
    Dann gilt $T_{f_n} \to \delta_0$ bzgl. $\sigma(\D'(\Omega), \D(\Omega))$.
\end{Satz}

\linie

\begin{Bem}
    Seien $f \in L^1_\loc(\Omega)$ und $\beta \in \natural_0^d$, sodass
    die partielle Ableitung $\partial^\beta_x f$ der Ordnung $\beta$ existiert.
    Dann gilt $T_{\partial^\beta_x f}(\varphi)
    = \int_{\real^d} (\partial^\beta_x f)(x) \varphi(x) \dx
    = (-1)^{|\beta|} \int_{\real^d} f(x) (\partial^\beta_x \varphi)(x) \dx$\\
    $= (-1)^{|\beta|} T_f(\partial^\beta_x \varphi)$ wegen partieller Integration.
    Die folgende Definition erklärt $T_{\partial^\beta_x f}$ zur "`Ableitung"' von $T_f$
    und verallgemeinert dies für nicht-reguläre Distributionen.
\end{Bem}

\begin{Def}{distributionelle Ableitung}
    Seien $T \in \D'(\Omega)$ und $\beta \in \natural_0^d$.\\
    Dann heißt $\partial^\beta_x T \in \D'(\Omega)$ mit
    $(\partial^\beta_x T)(\varphi) := (-1)^{|\beta|} T(\partial^\beta_x \varphi)$
    \begriff{distributionelle Ableitung} von $T$ der Ordnung $\beta$.
\end{Def}

\begin{Bsp}
    Seien $\Omega := (-1, 1)$ und $f \in L^1_\loc((-1, 1))$ mit $f(x) := |x|$.\\
    Dann gilt $T_f(\partial_x \varphi) = \int_{-1}^0 (-x)(\partial_x \varphi)(x) \dx +
    \int_0^1 x (\partial_x \varphi)(x) \dx = \int_{-1}^0 \varphi(x) \dx - \int_0^1 \varphi(x)\dx
    = -T_g(\varphi)$ für alle $\varphi \in \D((-1, 1))$ mit
    $g(x) := -1$ für $x < 0$, $g(x) := 0$ für $x = 0$ und $g(x) := 1$ für $x > 0$
    (Vorzeichenfunktion).
    Somit ist $T_g$ die distributionelle Ableitung von $T_f$
    (man identifiziert $f$ und $g$ mit $T_f$ bzw. $T_g$ und spricht oft davon,
    dass $g$ die distributionelle Ableitung von $f$ ist).
    Wegen $g \in L^2((-1, 1))$ ist $g$ auch die schwache Ableitung von $f$.

    Außerdem gilt
    $T_f(\partial_x^2 \varphi) =
    \int_{-1}^0 (\partial_x \varphi)(x)\dx - \int_0^1 (\partial_x \varphi)(x) \dx =
    2\varphi(0) = 2\delta(\varphi)$.
    Daher ist $2\delta$ die zweite distributionelle Ableitung von $T_f$ (bzw. von $f$),
    allerdings besitzt $g$ keine schwache Ableitung ($\delta$ ist keine Funktion).
\end{Bsp}

\begin{Satz}{distr. Ableitungsoperator stetig}\\
    Die Abbildung $\partial^\beta_x\colon \D'(\Omega) \rightarrow \D'(\Omega)$,
    $T \mapsto \partial^\beta_x T$ ist stetig bzgl. $\sigma(\D'(\Omega), \D(\Omega))$.
\end{Satz}

\pagebreak

\subsection{%
    Eigenschaften der schwachen Konvergenz und der Satz von \name{Alaoglu}%
}

\begin{Lemma}{schwache Konvergenz und Schwach$\ast$-Konvergenz}
    Seien $X$ ein normierter Raum,
    $(x_n)_{n \in \natural}$ eine Folge in $X$,
    $(x_n')_{n \in \natural}$ eine Folge in $X'$,
    $x \in X$ und
    $x' \in X'$.
    Dann gilt:
    \begin{enumerate}
        \item
        $x_n \rightharpoonup x \iff J_X x_n \xrightharpoonup{\ast} J_X x$ mit
        $J_X\colon X \rightarrow X''$, $(J_X x)(x') := x'(x)$ für $x \in X$, $x' \in X'$

        \item
        Aus $x_n' \rightharpoonup x'$ folgt $x_n' \xrightharpoonup{\ast} x'$.

        \item
        Der schwache Grenzwert von $(x_n)_{n \in \natural}$ (falls existent) und
        der Schwach$\ast$-Grenzwert von $(x_n')_{n \in \natural}$ (falls existent) sind eindeutig.

        \item
        Aus $x_n \to x$ (bzgl. $\norm{\cdot}_X$) folgt $x_n \rightharpoonup x$ und
        aus $x_n' \to x'$ (bzgl. $\norm{\cdot}_{X'}$) folgt $x_n' \xrightharpoonup{\ast} x'$.

        \item
        Aus $x_n' \xrightharpoonup{\ast} x'$ folgt
        $\norm{x'}_{X'} \le \liminf_{n \to \infty} \norm{x_n'}_{X'}$.

        \item
        Aus $x_n \rightharpoonup x$ folgt
        $\norm{x}_X \le \liminf_{n \to \infty} \norm{x_n}_X$\\
        (\begriff{Unterhalbstetigkeit} der Norm bzgl. der schwachen Konvergenz von Folgen).

        \item
        Konvergiert $(x_n)_{n \in \natural}$ schwach,
        so ist $(x_n)_{n \in \natural}$ beschränkt.\\
        Konvergiert $(x_n')_{n \in \natural}$ schwach$\ast$,
        so ist $(x_n')_{n \in \natural}$ beschränkt.

        \item
        Aus $x_n \to x$ und $x_n' \xrightharpoonup{\ast} x'$ folgt
        $x_n'(x_n) \to x'(x)$ in $\KK$.\\
        Aus $x_n \rightharpoonup x$ und $x_n' \to x'$ folgt
        $x_n'(x_n) \to x'(x)$ in $\KK$.
    \end{enumerate}
\end{Lemma}

\linie

\begin{Bsp}
    \begin{enumerate}[label=\emph{(\alph*)}]
        \item
        Sei $(\Omega, \Sigma, \mu)$ ein Maßraum, $p \in [1, \infty)$ und
        $p'$ mit $\frac{1}{p} + \frac{1}{p'} = 1$.
        Im Fall $p = 1$ sei $\mu$ zusätzlich $\sigma$-endlich.
        Dann ist $J_{p'}\colon L^{p'}(\Omega) \rightarrow (L^p(\Omega))'$ mit
        $(J_{p'} f)(g) := \int_\Omega g\overline{f} d\mu$ für $g \in L^p(\mu)$
        ein konjugiert linearer, isometrischer Isomorphismus.
        Für $p = 2$ ist $J_2 = \R_{L^2(\Omega)}$ gleich dem
        konjugiert linearen Isomorphismus aus dem Rieszschen Darstellungssatz.\\
        Seien $(f_k)_{k \in \natural}$ eine Folge in $L^p(\Omega)$ und $f \in L^p(\Omega)$.\\
        Dann gilt $f_k \rightharpoonup f$ in $L^p(\Omega)$ genau dann,
        wenn $\forall_{g \in L^{p'}(\Omega)}\; \int_\Omega f_k \overline{g} d\mu
        \xrightarrow{k \to \infty} \int_\Omega f\overline{g} d\mu$.

        \item
        Seien $K \subset \real^n$ kompakt und
        $\text{rca}(K)$ der Raum der signierten Borelmaße auf $K$.\\
        Dann ist $J\colon \text{rca}(K) \rightarrow (\C^0(K))'$ mit
        $(J\nu)(f) := \int_K fd\nu$ ein isometrischer Isomorphismus.\\
        Seien $(f_k)_{k \in \natural}$ eine Folge in $\C^0(K)$ und $f \in \C^0(K)$.\\
        Dann gilt $f_k \rightharpoonup f$ in $\C^0(K)$ genau dann,
        wenn $\forall_{\nu \in \text{rca}(K)}\; \int_K f_k d\nu
        \xrightarrow{k \to \infty} \int_K f d\nu$.

        \item
        Seien $\Omega \subset \real^n$ of"|fen, $m \in \natural$ und $p \in [1, \infty]$,\\
        außerdem $(u_k)_{k \in \natural}$ eine
        Folge in $W^{m,p}(\Omega)$ und $u \in W^{m,p}(\Omega)$.\\
        Dann gilt $u_k \rightharpoonup u$ in $W^{m,p}(\Omega)$ genau dann,
        wenn $\forall_{|s| \le m}\; \partial_x^s u_k \rightharpoonup \partial_x^s u$
        in $L^p(\Omega)$.\\
        Die gleiche Aussage gilt für $W^{m,p}_0(\Omega)$.
    \end{enumerate}
\end{Bsp}

\linie

\begin{Satz}{beschr. Folge in $X'$ besitzt schwach$\ast$ konv. TF für $X$ separabel}\\
    Sei $X$ ein separabler normierter Raum.\\
    Dann ist $\overline{B_1(0)} \subset X'$ schwach$\ast$ folgenkompakt.
\end{Satz}

\begin{Bem}
    In diesem Fall gilt diese Aussage auch für jede andere abgeschlossene Kugel
    $\overline{B_R(0)}$.
    Insbesondere besitzt jede beschränkte Folge in $X'$ eine schwach$\ast$ konvergente Teilfolge.\\
    Die Aussage gilt i.\,A. nicht, wenn $X$ nicht separabel ist.
\end{Bem}

\begin{Satz}{Satz von \scshape\,\!\name{Alaoglu}}
    Sei $X$ ein Banachraum.\\
    Dann ist $\overline{B_1(0)} \subset X'$ kompakt bzgl. der Schwach$\ast$-Topologie auf $X'$.
\end{Satz}

\pagebreak

\subsection{%
    Beste Approximationen in reflexiven Räumen%
}

\begin{Def}{reflexiv}
    Sei $X$ ein Banachraum.\\
    Dann heißt $X$ \begriff{reflexiv}, falls $J_X\colon X \rightarrow X''$
    (mit $(J_X x)x' = x'(x)$) surjektiv, also bijektiv ist.
\end{Def}

\begin{Satz}{(Gegen-)Beispiele für reflexive Räume}
    \begin{enumerate}
        \item
        Jeder Hilbertraum ist reflexiv.

        \item
        $L^p(\Omega)$ ist für $p \in (1, \infty)$ reflexiv.

        \item
        $W^{m,p}(\Omega)$ ist für $p \in (1, \infty)$ reflexiv.

        \item
        $\C^0(K)$ ist für $K$ kompakt und unendlich nicht reflexiv.
    \end{enumerate}
\end{Satz}

\begin{Lemma}{Eigenschaften reflexiver Räume}
    Sei $X$ ein Banachraum.
    \begin{enumerate}
        \item
        Ist $X$ reflexiv, dann stimmen schwache Konvergenz in $X'$ und
        Schwach$\ast$-Konvergenz in $X'$ überein.

        \item
        Ist $X$ reflexiv, dann ist auch jeder abgeschlossene Unterraum von $X$ reflexiv.

        \item
        Sei $Y$ ein zu $X$ isomorpher Banachraum.\\
        Dann ist $X$ reflexiv genau dann, wenn $Y$ reflexiv ist.

        \item
        $X$ ist reflexiv genau dann, wenn $X'$ reflexiv ist.
    \end{enumerate}
\end{Lemma}

\linie

\begin{Satz}{beschr. Folge in $X$ besitzt schwach konv. TF für $X$ reflexiv}\\
    Sei $X$ ein reflexiver Banachraum.\\
    Dann ist $\overline{B_1(0)} \subset X$ schwach folgenkompakt.
\end{Satz}

\begin{Bem}
    In diesem Fall gilt diese Aussage auch für jede andere abgeschlossene Kugel
    $\overline{B_R(0)}$.
    Insbesondere besitzt jede beschränkte Folge in $X$ eine schwach konvergente Teilfolge.
\end{Bem}

\begin{Lemma}{$X'$ separabel $\Rightarrow$ $X$ separabel}
    Sei $X$ ein Banachraum mit $X'$ separabel.\\
    Dann ist auch $X$ separabel.
\end{Lemma}

\linie

\begin{Def}{Vollstetigkeit}
    Seien $X, Y$ Banachräume und $T\colon X \rightarrow Y$ linear.
    Dann heißt $T$ \begriff{vollstetig}, falls
    für alle Folgen $(x_n)_{n \in \natural}$ in $X$ und $x \in X$ mit $x_n \rightharpoonup x$
    gilt, dass $Tx_n \to Tx$.
\end{Def}

\begin{Satz}{Vollstetigkeit}
    Seien $X, Y$ Banachräume und $T\colon X \rightarrow Y$ linear.
    \begin{enumerate}
        \item
        Ist $T$ kompakt, dann ist $T$ vollstetig.

        \item
        Ist $X$ reflexiv und $T$ vollstetig, dann ist $T$ kompakt.
    \end{enumerate}
\end{Satz}

\linie

\begin{Satz}{konvexe abg. Menge schwach folgenabg.}\\
    Seien $X$ ein normierter Raum und $M \subset X$
    nicht-leer, konvex und abgeschlossen.\\
    Dann ist $M$ schwach folgenabgeschlossen.
\end{Satz}

\linie

\begin{Satz}{bestapproximierendes Element für reflexive Räume}\\
    Seien $X$ ein reflexiver Banachraum und $M \subset X$
    nicht-leer, konvex und abgeschlossen.\\
    Dann gilt $\forall_{x_0 \in X} \exists_{y_0 \in M}\;
    \norm{x_0 - y_0} = \dist(x_0, M)$.
\end{Satz}

\pagebreak


