\chapter{%
    Geophysik des Erdkörpers%
}

\section{%
    Erdbeben%
}

\textbf{Erdbeben}:
Bruchfestigkeit der Gesteine wird überschritten (spröd nahe der Oberfläche, duktil in der Tiefe),
Spannung in der Kruste durch Plattentektonik aufgrund inhomogenen Wärmestroms
führt zum Bruch von Gestein am Hypozentrum,
Epizentrum liegt direkt über Hypozentrum auf der Oberfläche,
kann durch Laufzeituntersuchung ermittelt werden

\textbf{Richterskala}:
eine Stufe mehr bedeutet 30 Mal mehr freigesetzte Energie,
maximale Stärke von \num{9.4} wegen Maximum der Bruchfestigkeit der Gesteine in der Kruste

\textbf{konvergente Plattenränder}:
in Subduktionszonen (auch Wadati-Benioff-Zonen), wo eine Platte sich unter eine andere schiebt,
entstehen tief"|liegende Erdbeben

\textbf{Raumwellen}:
gehen von der Quelle des Erdbeben in der Masse aus,
gehen bei Übertritt zur Oberfläche in eine Oberflächenwelle über,
Geschwindigkeitsbereich in \si{\kilo\meter/\second},
zwei Arten von Raumwellen:
Longitudialwellen (Primärwellen, Ausbreitung in Längsrichtung durch alles,
abwechselnde Kompression und Dehnung der Partikel) und
Scherwellen (Sekundärwellen, senkrecht zur Ausbreitungsrichtung,
nicht durch Flüssigkeiten und Gase, langsamer als Primärwellen),
Wellen bewegen sich schneller durch dichtere Materie

\textbf{Ausbreitungsbereich von Raumwellen}:
direkte P-Wellen, die nicht durch den Kern laufen, haben einen Ausbreitungsbereich von
\SIrange{0}{103}{\degree},
gebeugte P-Wellen, die durch den Kern laufen: \SIrange{143}{180}{\degree},
Schattenzone der P-Wellen: \SIrange{103}{143}{\degree},
direkte S-Wellen: \SIrange{0}{103}{\degree},
Schattenzone der S-Wellen: \SIrange{103}{180}{\degree}
(S-Wellen laufen nicht durch den Kern und hören an der Kern-Mantel-Grenze auf)

\textbf{Oberflächenwellen}:
zwei Arten von Oberflächenwellen:
Love-Wellen (entstehen aus horizontal polarisierten S-Wellen,
Schwingung parallel zur Oberfläche, treten nur in geschichteten Gesteinen auf) und
Rayleigh-Wellen (entstehen aus der Interferenz von P- und S-Wellen,
Schwingung senkrecht zur Oberfläche,
Rollbewegung gegenläufig zur Ausbreitungsrichtung der Welle,
treten in homogenen und geschichteten Gesteinen auf)

\textbf{Dichte der Erde}:
Durchschnitt \SI{5.5}{\gram/\centi\meter\cubed},
ozeanische Kruste \SI{3.0}{\gram/\centi\meter\cubed},
kontinentale Kruste \SIrange{2.6}{2.7}{\gram/\centi\meter\cubed}

\textbf{Aufbau der Erdkruste und des oberen Mantels}:
Lithosphäre \SIrange{0}{100}{\kilo\meter} und spröd,
Asthenosphäre \SIrange{100}{300}{\kilo\meter} und duktil,
Mesosphäre $> \SI{300}{\kilo\meter}$ und spröd

\begin{wichtig}
    \item
    \textbf{um welchen Betrag steigt die Freisetzung der Energie bei einer Stufe mehr
    auf der Richterskala}:
    \emph{um 30 Mal}

    \item
    \textbf{Arten von Raumwellen}:
    \emph{P- und S-Wellen}

    \item
    \textbf{Arten von Oberflächenwellen}:
    \emph{Love- und Rayleigh-Wellen}

    \item
    \textbf{Wellen sind schneller, wenn das Medium}:
    \emph{dichter ist}

    \item
    \textbf{Geschwindigkeitsbereich einer Welle}:
    \emph{\si[math-rm=\mathit,text-rm=\itshape]{\kilo\meter/\second}}

    \item
    \textbf{S-Wellen durchlaufen nicht}:
    \emph{Flüssigkeiten und Gase}

    \item
    \textbf{Durchschnittsdichte von kontinentaler Kruste}:
    \emph{ca. \SI[math-rm=\mathit,text-rm=\itshape]{2.7}{\gram/\centi\meter^3}}

    \item
    \textbf{Durchschnittsdichte von ozeanischer Kruste}:
    \emph{ca. \SI[math-rm=\mathit,text-rm=\itshape]{3.0}{\gram/\centi\meter^3}}
\end{wichtig}

\pagebreak

\section{%
    Wärmefluss%
}

\textbf{Wärmefluss}:
überhitzter Eisenkern,
Konvektion ist ef"|fektiver Wärmetransport,
aber Abstrahlung des Kerns,
inkompatible Elemente (z.\,B. zu groß) gehen in die Schmelze und wandern mit nach oben,
so entsteht die Wärmeproduktion in der Kruste

\begin{wichtig}
    \item
    \textbf{Ursachen des Wärmeflusses im Inneren der Erde}:\\
    \emph{radioaktiver Zerfall von \ce{K}, \ce{Th}, \ce{U},\\
    Aufprallenergie aus der Kollision mit Himmelskörpern in der Frühzeit,\\
    Rotationsenergieverlust des Systems Erde -- Mond}
\end{wichtig}

\section{%
    Gravimetrie%
}

\textbf{Schwimmgleichgewicht von Krusten}:
die kontinentale Kruste (Dicke von \SI{35}{\kilo\meter}) schwimmt aufgrund der geringeren
Dichte von \SI{2.8}{\gram/\centi\meter\cubed} über der ozeanischen Kruste und dem
darüberliegenden Ozean (Dicke von \SI{4.7}{\kilo\meter} bzw. \SI{8}{\kilo\meter})
mit einer Krustendichte von \SI{3.0}{\gram/\centi\meter\cubed}

\textbf{Geoid}:
Abbild des Schwerefelds der Erde

\textbf{Ursache von Schwereanomalien}:
unterschiedliche Gesteinszusammensetzung in der Kruste

\textbf{Bestimmung von Schwereanomalien}:
Schweremessung an der Oberfläche,
Freiluft-Korrek"-tur für die topografische Höhe über dem Referenz-Geoid,
Bouguer-Korrektur für die Massenanziehung des Gebirges über dem Referenz-Geoid,
verbleibt nach Berücksichtigung beider Korrekturen eine Abweichung vom globalen Mittelwert
(Schwereanomalie), so muss die Ursache dafür in der Gesteinszusammensetzung der Kruste liegen

\textbf{positive/negative Schwereanomalien}:
Mantelgesteine und ozeanische Kruste bzw. Sedimentgesteine,
auch auf der Ozeanoberfläche gibt es Berge und Täler

\begin{wichtig}
    \item
    \textbf{positive Schwereanomalien werden verursacht durch}:
    \emph{ozeanische Kruste}

    \item
    \textbf{auch auf der Ozeanoberfläche gibt es}:
    \emph{Berge und Täler}
\end{wichtig}

\pagebreak

\section{%
    Magnetismus%
}

\textbf{Magnetit}:
Bildung von \ce{Fe3O4} durch Vulkanismus,
ungeordnete Ausrichtung von Magnetkristallen (Elementarmagneten) in Magnetitschmelzen bei
Temperaturen über \SI{500}{\celsius},
bei Abkühlung ohne Magnetfeld Bildung von gleichorientierten Bereichen, die aber keine
gemeinsame Orientierung haben,
mit Magnetfeld (fast) alle in dieselbe Richtung ausgerichtet

\textbf{Bildung des Erdmagnetfelds im Erdkern}:
Entstehung von ringförmigen, torodialen Magnetfeldern durch meridionale Ringströme im
äußeren Erdkern,
Verdrillung der torodialen Magnetfelder durch die Coriolis-Kraft (Taylor-Säulen)

\textbf{Ursache der hohen Feldstärke}:
flüssiger äußerer Kern und fester innerer Kern,
rasche Rotation der Erde

\textbf{Verteilung des Magnetfelds}:
Bereiche höherer und niedrigerer Intensität,
ändert sich ständig,
Wanderung des magnetischen Nordpols durch Kanada

\textbf{Deklination/Inklination}:
horizontale/vertikale Komponente des Magnetfelds

\textbf{Polaritätswechsel}:
Erdmagnetfeld wechselt Polarität öfters,
durch paläomagnetische Untersuchungen von ozeanischen Rücken ist daher eine
Altersbestimmung der Kruste möglich
(Unterteilung in Gilbert-, Gauß-, Matuyama- und Brunhes-Epoche),
aus paläömagnetischen Daten geht hervor, dass die heutigen Ozeanböden nicht älter sind als
ca. \SI{180}{\mega\year}
(dann sind sie so kalt und dicht, dass sie im Mantel recycelt werden)

\textbf{Superkontiente}:
Rodinia vor \SI{800}{\mega\year},
Gondwana vor \SI{450}{\mega\year},
Pangäa vor \SI{220}{\mega\year}

\begin{wichtig}
    \item
    \textbf{Ursache der hohen Feldstärke}:\\
    \emph{flüssiger äußerer Kern und fester innerer Kern,\\
    rasche Rotation der Erde}

    \item
    \textbf{aus palömagnetischen Daten geht hervor, dass die heutigen Ozeanböden nicht
    älter sind als}:
    \emph{ca. \SI[math-rm=\mathit,text-rm=\itshape]{180}{\mega\year}}
\end{wichtig}

\pagebreak

\section{%
    Rotation und Gezeiten%
}

\textbf{Milankovitch-Parameter}:
Exzentrizität,
Ekliptikschiefe,
Präzession

\textbf{Exzentrizität}:
Abweichung der Erdbahn von einem Kreis,
Periode von \SI{100}{\kilo\year},
starke Klimawirksamkeit

\textbf{Ekliptikschiefe}:
Neigung der Rotationsachse gegenüber der Erdbahn
(heute \SI{23.5}{\degree}),
Periode von \SI{40}{\kilo\year},
mittlere Klimawirksamkeit

\textbf{Präzession}:
Kreisbewegung der Erdachse
(Ursache: Äquatorwulst),
Periode von \SI{20}{\kilo\year},
schwache Klimawirksamkeit

\textbf{Ursache des bidiurnalen Gezeitenregimes}:
zentripetal-gravitativer Flutberg (höher) auf der dem Mond zugewandten Seite,
zentripetaler Flutberg (niedriger) auf der dem Mond abgewandten Seite

\textbf{Abweichungen von der Tageslänge in den letzten \SI{300}{\year}}:
Massenverlagerung wegen Verlagerung des Wassers

\begin{wichtig}
    \item
    \textbf{Veränderung wichtiger planetarer Kenngrößen im Lauf der Erdgeschichte}:\\
    \emph{die Rotation der Erde wird langsamer,\\
    die Entfernung zum Mond wird größer,\\
    die Gezeitenkräfte werden schwächer,\\
    die Tage werden länger,\\
    die Wärmeproduktion wird geringer,\\
    die ozeanische Kruste wird dichter,\\
    die Ozeane daher tiefer und Gebirge höher}
\end{wichtig}

\pagebreak

\section{%
    Atmosphäre%
}

\textbf{Licht- und Radiowellenfenster der Erdatmosphäre}:
Durchlässigkeit für sichtbares Licht (\SIrange{400}{700}{\nano\meter} Wellenlänge)
und Radiowellen (\SI{5}{\centi\meter} bis \SI{10}{\meter} Wellenlänge),
der Rest wird relativ gut abgeschirmt

\textbf{Aufbau der Atmosphäre}:
Troposphäre,
Tropopause (\SI{10}{\kilo\meter}),
Stratosphäre,\\
Stratopause (\SI{50}{\kilo\meter}),
Mesosphäre,
Mesopause (\SI{80}{\kilo\meter}),
Thermosphäre,
Exosphäre (ab \SI{500}{\kilo\meter}),
Ozonschicht in Höhe von \SIrange{25}{30}{\kilo\meter},
Meso- und Thermosphäre gehören zur Ionosphäre

\textbf{globale atmosphärische Konvektionszellen}:
am Äquator steht die Sonne am höchsten, daher wird Luft stark erwärmt,
kann viel Feuchtigkeit aufnehmen,
in großer Höhe kühlt sie sich ab und es regnet,
dann wandert die trockene Luft nach Norden und Süden und sinkt in den Ross-Breiten zu Boden,
diese Luft strömt als Nordost- bzw. Südostpasst wieder zum Äquator (Hadley-Zelle),
analog Ferrel-Zellen zwischen Breiten von \SIrange{30}{60}{\degree},
Westwinde in mittleren Breiten,
polare Ostwinde

\textbf{Coriolis-Kraft}:
Scheinkraft aufgrund der Erdrotation,
lässt Winde Richtung Äquator als Ostwinde wehen (Passate)

\begin{wichtig}
    \item
    \textbf{\ce{O2}-Gehalt heute}:
    \emph{\SI[math-rm=\mathit,text-rm=\itshape]{21}{\percent}}

    \item
    \textbf{\ce{O2}-Gehalt ursprünglich}:
    \emph{\SI[math-rm=\mathit,text-rm=\itshape]{0}{\percent}}

    \item
    \textbf{\ce{CO2}-Gehalt heute}:
    \SI[math-rm=\mathit,text-rm=\itshape]{0.0390}{\percent}

    \item
    \textbf{\ce{CO2}-Gehalt ursprünglich}:
    \emph{höher}

    \item
    \textbf{Passatwinde wehen auf der Nord-/Südhalbkugel aus}:
    \emph{Nord-/Südost}

    \item
    \textbf{der meiste Niederschlag fällt}:
    \emph{in den Tropen}
\end{wichtig}

\pagebreak

\section{%
    Ozeane und Meeresströmungen%
}

\textbf{Zusammensetzung von Meerwasser}:
\SI{96.5}{\percent} Wasser und
\SI{3.5}{\percent} gelöste Salze,
davon \SI{55}{\percent} Chlorid,
\SI{31}{\percent} Natrium und
\SI{8}{\percent} Sulfat

\textbf{Meeresströmungen}:
Antrieb durch Passatwinde (rechtsdrehend auf der Nordhalbkugel)

\textbf{Golfstrom}:
oberflächlicher Warmwasserstrom mit wenig Salzgehalt,
gibt es seit \SI{3.4}{\mega\year}
(Bildung der Landbrücke von Mittelamerika),
warmes Wasser kommt aus dem Pazifik,
wandert zwischen Australien und Asien an der Südspitze von Afrika vorbei,
wird in der Karibik nochmals erwärmt und gelangt in die Nordsee,
gibt dort Wärme ab und sinkt ab,
wandert als kalter, salzhaltiger Tiefenstrom wieder zurück
(diesmal südlich von Australien vorbei)

\textbf{Versiegen des Golfstroms}:
Golfstrom bewirkt wärmeres Klima,
wenn das Klima noch weiter erwärmt wird,
dann ergibt sich ein größerer Süßwasserzustrom im arktischen Ozean (Flüsse in Russland),
was zum Versiegen des Golfstroms führen könnte

\textbf{Vertikalprofil der ozeanischen Wassersäule}:
der \ce{O2}-Gehalt erreicht in \SI{1}{\kilo\meter} Tiefe sein Minimum
(Plankton-Regen wird oxidiert und verbraucht Sauerstoff),
steigt in tieferem Wasser wieder an
(wegen arktischen, \ce{O2}-gesättigten Tiefenströmungen)

\textbf{Ekman-Strömung an den Westküsten der Südkontinente}:
Strömung weg von der Küste bei vorherrschender Windrichtung aus Süden
(kaltes Tiefenwasser erzeugt Küstenwüsten),
Strömung hin zu Küste bei vorherrschender Windrichtung aus Norden
(warmes Oberflächenwasser)

\textbf{El Niño}:
Ursache ist autozyklisch

\textbf{Meeresspiegel}:
Temperaturschwankungen haben dazu geführt, dass der Meeresspiegel während der letzten Eiszeit
um \SI{130}{\meter} tiefer war

\textbf{Vereisung}:
Südpol ist kontinuierlich vereist seit \SI{34}{\mega\year}
(Vereisungsschwelle \SI{750}{\ppm} \ce{CO2}),
Nordpol ist kontinuierlich vereist seit \SI{3.4}{\mega\year}
(Vereisungsschwelle \SI{280}{\ppm} \ce{CO2})

\begin{wichtig}
    \item
    \textbf{Salzgehalt}:
    \emph{\SI[math-rm=\mathit,text-rm=\itshape]{3.5}{\percent}}

    \item
    \textbf{Golfstrom gibt es seit}:
    \emph{\SI[math-rm=\mathit,text-rm=\itshape]{3.4}{\mega\year} (Landbrücke Mittelamerika)}

    \item
    \textbf{Auftriebsgebiete von kaltem Tiefenwasser erzeugen}:
    \emph{Küstenwüsten}

    \item
    \textbf{Ursache für El-Niño-Phänomen}:
    \emph{autozyklisch}

    \item
    \textbf{Südpol ist kontinuierlich vereist}:
    \emph{seit \SI[math-rm=\mathit,text-rm=\itshape]{34}{\mega\year}}

    \item
    \textbf{Nordpol ist kontinuierlich vereist}:
    \emph{seit \SI[math-rm=\mathit,text-rm=\itshape]{3.4}{\mega\year}}

    \item
    \textbf{Milankovitch-Parameter mit dem stärksten Klimaeinfluss}:
    \emph{Exzentrizität}
\end{wichtig}

\pagebreak
