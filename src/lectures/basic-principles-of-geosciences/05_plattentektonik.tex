\chapter{%
    Plattentektonik%
}

\section{%
    Allgemeines%
}

\textbf{Plattentektonik}:
älterer Ozeanboden liegt tiefer (dichter, da mehr abgekühlt),
drei Arten von Gebirgen
(Platten schieben sich aufeinander wie im Himalaya,
Graben sackt ab wie im Schwarzwald und Ostafrika und
Subduktionszone wie in den Anden),
Schelf ist der Meeresboden von \SIrange{0}{180}{\meter} Tiefe,
Ozeankrusten subduzieren steiler, je älter,
Umwandlung von Basalt in Eklogit erfolgt ab \SI{35}{\kilo\meter} Subduktionstiefe

\begin{wichtig}
    \item
    \textbf{Plattentektonik ist das Ergebnis von}:
    \emph{dem inhomogenen Wärmestrom}

    \item
    \textbf{Plattentektonik ist möglich, solange}:\\
    \emph{die Asthenosphäre existiert (die man von beiden Seiten durchdringen kann)}

    \item
    \textbf{maximale Driftraten heute}:
    \emph{\SI[math-rm=\mathit,text-rm=\itshape]{17}{\centi\meter/\year}}

    \item
    \textbf{Spreizungsraten maximal}:
    \emph{am Rotationsäquator}

    \item
    \textbf{Eklogitisierung}:
    \emph{ab \SI[math-rm=\mathit,text-rm=\itshape]{35}{\kilo\meter} Tiefe}

    \item
    \textbf{Magmenbildung}:
    \emph{\SI[math-rm=\mathit,text-rm=\itshape]{34}{\kilo\meter^3/\year}}

    \item
    \textbf{breite Schelfe gibt es nicht}
    \emph{an Subduktionszonen}

    \item
    \textbf{korreliert die Wassertiefe mit dem Alter des Ozeanbodens}:
    \emph{ja}
\end{wichtig}

\section{%
    Kontinentale Divergenzränder%
}

\textbf{Bildung von Grabensystemen}:
Antrieb ist eine Mantelaufwölbung im Scheitelbereich einer Mantel-Konvektionszelle

\textbf{zwei Arten von Scherungen}:
reine Scherung (Auseinanderziehen wie ein Kaugummi, benötigt viel Temperatur aus dem Mantel,
damit die Kruste duktil ist) und
einfache Scherung (Seitwärtsverschiebung, tiefer Bruch in der Kruste)

\textbf{Spreizungsäste}:
haben oft einen \SI{120}{\degree}-Winkel

\textbf{Rift-Gebiete}:
zum Beispiel Ostafrika, Jordan-Graben und Oberrhein,
typisch sind Vulkane (Kilimandscharo), viele Seen (Totes Meer)

\pagebreak

\section{%
    Ozeanische Divergenzränder%
}

\textbf{Magmenbildung}:
partielle Aufschmelzung von Mantelgestein,
zwei Ursachen (Druckentlastung und Injektion von Wasser)

\textbf{slab pull/ridge push}:
Antrieb bei ozeanischen Divergenzrändern einmal durch
Gefälle durch mittelozeanischen Rücken (ridge push) und andererseits durch
Subduktion am anderen Rand (slab pull)

\textbf{Rotes Meer}:
Spreizungsraten von \SIrange{1.0}{1.5}{\centi\meter/\year}

\textbf{Aufbau der ozeanischen Kruste}:
kein explosiver Vulkanismus, da der Wasserdruck zu hoch ist
(\SI{10}{\meter} Wassertiefe entsprechen \SI{1}{\bar} Druck),
daher Kissenbasalt mit Abschreckungskruste
und darunter Gabbro (in der Tiefe erstarrtes basaltisches Magma)

\textbf{Schwarze Raucher}:
Wasser dringt in Spalten ein,
die Erhitzung führt zu überkritischem Wasser (Temperatur größer als \SI{600}{\celsius}),
was sehr aggressiv ist und viele Mineralien löst,
aufgrund der geringen Dichte steigt das überkritische Wasser wieder auf und
Sulfide fallen bei Kontakt mit Meerwasser aus

\section{%
    Passive Kontinentalränder%
}

\textbf{Sedimentkeile}:
Ablagerungen an Küsten durch Flüsse,
Entstehung von Canyons (submarine Täler, in denen Sedimente unter Wasser weitertransportiert
werden),
Rutschungen können große Tsunamis verursachen

\section{%
    Intraplatten-Magmatismus%
}

\textbf{Entstehung von "`Heißen Punkten"' (Plumes)}:
kaltes Subduktionsmaterial sinkt durch die Grenze von oberem und unterem Erdmantel
(Diskontinuität in der Tiefe von \SI{670}{\kilo\meter}) bis an die Kern-Mantel-Grenze
(Tiefe von \SI{3000}{\kilo\meter}), wird dort erwärmt und steigt wieder auf,
durch Druckentlastung kommt es zur Bildung von Magmen,
z.\,B. Hawaii, Island

\textbf{Zusammenhang zwischen der Hotspot-Aktivität und dem Paläoklima}:
durch mehr \ce{CO2}-Entgasung steigen die Temperatur und der Meeresspiegel,
es gibt auch einen Zusammenhang mit dem Erdmagnetfeld

\textbf{Spur von Hotspots}:
Platten bewegen sich über dem Hotspot hinweg,
dieser hinterlässt so eine Spur,
z.\,B. Hawaii und Midway-Inseln, Island

\textbf{Seamounts}:
alte Vulkane, die aufgrund eines Hotspots entstanden sind, sich aber wegen der Plattentektonik
wegbewegt haben,
Berg wird abgetragen (Darwins Prinzip der Entstehung von Atollen),
auch an Land möglich (z.\,B. Yellowstone-Hotspot)

\pagebreak

\section{%
    Konvergenzränder%
}

\textbf{Ozean-Ozean-Konvergenz}:
ältere, kältere Kruste sinkt, steiler Abstieg

\textbf{Ozean-Kontinent-Konvergenz}:
Konvergenz unter Kontinentsplitter (Japan) oder Konvergenz unter kontinentaler Kruste (Anden),
gestufte Subduktionszonen, breitet sich an Diskontinuität aus

\textbf{Konvergenzzonen-Magmatismus}:
Ursache ist injiziertes Wasser, das sich in der heißen Asthenosphäre löst,
dadurch erhält man höher dif"|ferenzierte Magmen, von denen es aber nur \SI{10}{\percent}
an die Oberfläche schaf"|fen,
daraus ist hauptsächlich die kontinentale Kruste entstanden,
notwendig ist dafür steile Subduktion
(Low-Stress-Subduktion), damit das Wasser die Asthenosphäre erreicht
(bei flacher Subduktion (High-Stress-Subduktion) verschwindet das Wasser schon vorher),
bei der Austreibung des Wassers wird das Basalt der ozeanischen Kruste in Eklogit umgewandelt
(Eklogitisierung)

\textbf{Kontinent in Konvergenzzonen}:
der konvergente Plattenrand wird bei der Bildung von Magmen duktiler,
dadurch kann der Kontinent zusammengepresst werden und hohe Gebirge können entstehen
(z.\,B. Anden), dazwischen Hochplateaus

\textbf{Anwachskeile}:
in Tiefseerinnen wird dort viel Sediment eingetragen, wo ein Gebirge und feuchtes Klima ist,
dadurch entstehen Anwachskeile oder auch Akkretionskeile
(akkretionäre Subduktionszonen im Gegensatz zu erosiven Subduktionszonen)

\textbf{Seamounts}:
werden in Subduktionszonen zerlegt und erzeugen Buchten in der Küste

\begin{wichtig}
    \item
    \textbf{Ursache der Magmenbildung an konvergenten Plattenrändern}:\\
    \emph{Injektion von Wasser}

    \item
    \textbf{Magmen von Konvergenzrändern sind}:
    \emph{wasserhaltig}

    \item
    \textbf{Konvergenzzonen-Magmatismus ist die Folge von}:
    \emph{steiler Subduktion}

    \item
    \textbf{Ursachen der Kontinente}:
    \emph{nur Konvergenzzonen-Magmatismus}

    \item
    \textbf{Akkretionskeile entstehen}:
    \emph{im humiden Klima}
\end{wichtig}

\section{%
    Seitenverschiebungsränder%
}

\textbf{Seitenverschiebungsränder}:
z.\,B. San-Andreas-Verwerfung, mittelozeanische Rücken und Totes Meer,
entstehen wegen Kugelkalotten,
tiefe Löcher können entstehen bei nicht-geraden Plattengrenzen
(z.\,B. Totes Meer, Himalaya)

\section{%
    Terrancollage%
}

\textbf{Terrancollage}:
z.\,B. Alaska,
Akkretion von verschiedenen Kontinentalsplittern durch Seitenverschiebung

\pagebreak

\section{%
    Kollisionszonen%
}

\textbf{Entstehung von Kontinentaldrifts}:
Back-Arc-Spreizung,
Antrieb ist ein Winkelstrom

\textbf{Wilson-Zyklus}:
Periode von \SIrange{100}{200}{\mega\year},
Spreizung eines Kontinents,
Entstehung eines Ozeans,
irgendwann konvergieren die Platten wieder und der Ozean verschwindet
(z.\,B. vor dem Atlantik gab es einen Vorläufer-Atlantik),
Suturzone ist die Nahtstelle, wo früher der Ozean war,
Ophiolithe sind Späne von ozeanischer Kruste in Gebirgen

\textbf{Kollision}:
bei der Kollision kommt es zu Krustenstapelung, z.\,B. im Himalaya
(Stapelung von Krustenspänen, i.\,A. keine Faltung),
das Tibet-Plateau hat zu einer Temperaturminderung seit \SI{20}{\mega\year} geführt
(Gebirge in Äquatornähe sind \ce{CO2}-Verbraucher)

\textbf{Lithosphärenkeile}:
Entstehung durch Subduktion,
die erhöhte Krustendicke von \SI{70}{\kilo\meter} wird innerhalb von
\SI{150}{\mega\year} wieder auf die Durchschnittsdicke von \SI{40}{\kilo\meter} abgetragen

\begin{wichtig}
    \item
    \textbf{in Gebirgen wird die Lage des verschwundenen Ozeans markiert durch}:\\
    \emph{Ophiolithe}

    \item
    \textbf{Krustendicke unter Gebirgen}:
    \emph{\SI[math-rm=\mathit,text-rm=\itshape]{70}{\kilo\meter}}

    \item
    \textbf{typische Krustenstruktur in Gebirgen}:
    \emph{Lithosphärenkeil}
\end{wichtig}

\section{%
    Magmatische Tiefengesteine%
}

\textbf{Aufbau der ozeanischen Kruste}:
ganz unten ist der lithosphärische Mantel,
darüber Gab"-bro-Stockwerk (in der Tiefe erstarrtes, basaltisches Magma),
darüber Gang-in-Gang-Stockwerk,
darüber Kissenbasalt-Stockwerk und
ganz oben Sedimente

\textbf{Rhyolith, Granit}:
Tiefengesteine (auch Plutonite),
im Gegensatz zu Basalt und Gabbro höher chemisch und magamatisch dif"|ferenziert,
Entstehung von Granit in Konvergenzzonen oder in Kollisionszonen
(bei der Stapelung von wasserhaltigen Krusten kommt es zur Entwässerung
und zur Bildung von Kollisionszonen-Granit, z.\,B. im Schwarzwald)

\textbf{Ganggesteine}:
auch Pegmatite,
Seltene Erden wie Lithium etc. reichern sich in Gängen an, die durch
Spreizung aufgrund von aufsteigender Magma entstehen

\textbf{hydrothermale Gänge}:
wandparallel zonierte Kristallisation in of"|fenen Klüften aus
überhitztem (unter \SI{407}{\celsius}) oder überkritischem (unter \SI{650}{\celsius}) Wasser,
meistens Quarzgänge,
Erzlagerstätten enthalten Metalle wie Eisen und Kupfer

\begin{wichtig}
    \item
    \textbf{häufigstes magmatisches Tiefengestein der kontinentalen Kruste}:
    \emph{Granit}

    \item
    \textbf{häufigstes magmatisches Tiefengestein der ozeanischen Kruste}:
    \emph{Gabbro}

    \item
    \textbf{typisches Intrusionsniveau von Graniten}:
    \emph{\SIrange[math-rm=\mathit,text-rm=\itshape]{10}{15}{\kilo\meter}}

    \item
    \textbf{granitische Tiefengesteine enthalten}:
    \emph{keine Erzlagerstatten}

    \item
    \textbf{wo konzentriert sich die Vererzung}:\\
    \emph{im Dach des Plutons (pegmatische und hydrothermale Gänge)}
\end{wichtig}

\pagebreak
