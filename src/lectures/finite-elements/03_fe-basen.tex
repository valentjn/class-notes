\section{%
    Finite-Elemente-Basen%
}

\subsection{%
    Multivariate B-Splines%
}

\textbf{Tensorprodukt-B-Splines}:
Der $m$-variate B-Spline $b_{k,h}^n$ mit Grad $n_\nu$ in der $\nu$-ten Variable,
Index $k = (k_1, \dotsc, k_m)$ und Gitterweite $h$ ist definiert durch
\begin{align*}
    b_{k,h}^n(x) := \prod_{\nu=1}^m b_{k_\nu,h}^{n_\nu}(x_\nu)
\end{align*}
mit der Konvention, dass $n_1 = \dotsb = n_m$, wenn nicht anders angegeben.
Für diese Standardwahl ist der hochgestellte Index $n$ eine Zahl (und kein Vektor).

\linie

\textbf{Eigenschaften von multivariaten B-Splines}:\\
Multivariate B-Splines erfüllen die folgenden Eigenschaften:
\begin{itemize}
    \item
    \emph{Positivität und lokaler Träger}:
    $b_{k,h}^n$ ist positiv auf $kh + (0, n + 1)^m h$ und verschwindet außerhalb dieses
    $m$-dimensionalen Hyperwürfels.

    \item
    \emph{Glattheit}:
    $b_{k,h}^n$ ist in jeder Variablen $(n - 1)$-fach stetig dif"|ferenzierbar.

    \item
    \emph{Struktur als stückweises Polynom}:
    $b_{k,h}^n$ ist auf jeder Gitterzelle $\ell h + [0, 1)^m h$,\\
    $\ell = (\ell_1, \dotsc, \ell_m) \in \{0, \dotsc, n\}^m$,
    ein Polynom vom Grad $n$ in jeder Variable, d.\,h. der B-Spline ist gleich
    $\sum_{\alpha_\nu \le n} c_\alpha x^\alpha$ mit $c_{(n, \dotsc, n)} \not= 0$
    und $x^\alpha = x_1^{\alpha_1} \dotsm x_m^{\alpha_m}$.
\end{itemize}

\textbf{partielle Ableitungen}:
Die partiellen Ableitungen erster Ordnung eines multivariaten B-Splines $b_{k,h}^n$ vom
Grad $n = (n_1, \dotsc, n_m)$ sind Dif"|ferenzen von B-Splines niedrigerem Grad:
\begin{align*}
    \partial^\alpha b_{k,h}^n = h^{-1} (b_{k,h}^{n-\alpha} - b_{k+\alpha,h}^{n-\alpha})
\end{align*}
für die Einheitsvektoren $\alpha = (1, 0, \dotsc, 0), (0, 1, \dotsc, 0), \dotsc,
(0, 0, \dotsc, 1)$.

\subsection{%
    Splines auf beschränkten Gebieten%
}

\textbf{Spline}:
Die \begriff{Splines} $\BB_h^n(D)$ auf einem beschränkten Gebiet $D \subset \real^m$ bestehen aus
allen Linearkombinationen
\begin{align*}
    \sum_{k \in K} c_k b_{k,h}^n
\end{align*}
von relevanten B-Splines ($K$ ist die Menge der relevanten Indizes $k \sim D$, die alle
Indizes $k$ mit
$b_{k,h}^n(x) \not= 0$ für ein $x \in D$ enthält).

\textbf{Darstellung von Polynomen}:
Jedes multivariate Polynom $p(x) = \sum_{\alpha_\nu \le n} c_\alpha x^\alpha$ kann auf dem Gebiet
$D$ als Linearkombination
\begin{align*}
    \sum_{k \in K} q(k) b_k(x),\quad
    x \in D,
\end{align*}
geschrieben werden, wobei $q$ ein multivariates Polynom vom Grad $\le n$ in jeder Variable $k_\nu$
ist.

\textbf{lokale lineare Unabhängigkeit}:
Für jede of"|fene Teilmenge $D' \subset D$ sind die B-Splines
\begin{align*}
    b_k,\quad
    D' \cap \supp b_k \not= \emptyset,
\end{align*}
linear unabhängig.

\pagebreak

\subsection{%
    Gewichtsfunktionen%
}

\textbf{Gewichtsfunktion}:
Eine \begriff{Gewichtsfunktion} $w$
der Ordnung $\gamma \in \natural_0$ ist stetig auf $\overline{D}$ und erfüllt
\begin{align*}
    w(x) \asymp \dist(x, \Gamma)^\gamma,\quad
    x \in D,
\end{align*}
für eine Teilmenge $\Gamma \subset \partial D$.
Es wird angenommen, dass $\Gamma$ positives $(m - 1)$-dimensionales Maß besitzt und genügend
regulär ist, sodass der Gradient der Abstandsfunktion beschränkt ist.
Falls $w$ glatt und auf dem ganzen Rand linear verschwindet (also $\gamma = 1$),
dann heißt $w$ \begriff{Standard-Gewichtsfunktion}.

Um eine Gewichtsfunktion für einen Dreiviertelskreis zu erhalten (Pacman), kann man nicht einfach
$1 - x^2 - y^2$ mit $x$ und $y$ multiplizieren, denn dann erhält man im Inneren des Gebiets
Nullstellen der Gewichtsfunktion, was einen Genauigkeitsverlust bedeuten wurde.
Jede Approximation als Linearkombination der gewichteten Basis wäre ja in diesen Punkten null.
Wenn die tatsächliche Lösung nicht auch zufällig diese Eigenschaft hat, könnte in diesen Punkten
ein sehr großer Fehler entstehen.
Es gibt aber eine systematische Methode zur Definition von Gewichtsfunktionen mit booleschen
Operationen.

\linie

\textbf{Methode der R-Funktionen}:
Eine \begriff{vorzeichenbehaftete Gewichtsfunktion} ist eine global definierte Funktion,
die in $D$ positiv und im Komplement von $\overline{D}$ negativ ist.
Solche Gewichtsfunktionen können mit den zu booleschen Mengenoperationen (Komplement, Schnitt usw.)
gehörigen \begriff{R-Funktionen} $r$ konstruiert werden.
Wenn $w_1$ und $w_2$ vorzeichenbehaftete Gewichtsfunktionen für $D_1$ und $D_2$ sind, dann sind
\begin{align*}
    w = r_c(w_1),\quad
    w = r_\cap(w_1, w_2),\quad
    w = r_\cup(w_1, w_2)
\end{align*}
vorzeichenbehaftete Gewichtsfunktionen für $D_1^c, D_1 \cap D_2, D_1 \cup D_2$.\\
Die R-Funktionen für die Booleschen Operationen lauten:
\begin{align*}
    r_c(w) &:= -w\\
    r_\cap(w_1, w_2) &:= w_1 + w_2 - \sqrt{w_1^2 + w_2^2}\\
    r_\cup(w_1, w_2) &:= w_1 + w_2 + \sqrt{w_1^2 + w_2^2}
\end{align*}

\linie

\textbf{numerische Gewichtsfunktionen}:
Für allgemeine Gebiete, die durch Freihandkurven begrenzt sind,
müssen Gewichtsfunktionen numerisch konstruiert werden.
Dabei gibt es eine einfache Prozedur, die auf glatte Ränder angewendet werden kann.
Durch die Formel
\begin{align*}
    w(x) = 1 - \max(0, 1 - \dist(x, \partial D)/\delta)^\gamma
\end{align*}
wird die Abstandfunktion in einem kleinen Streifen $D \setminus D_\delta$ der Breite $\delta$
verwendet, wo sie frei von Singularitäten ist.
Im Inneren des Gebiets wird sie mit einem Plateau der Höhe $1$ übergeblendet.
$\gamma$ beeinflusst die Glattheit (Ordnung der Nullstelle beim Abstand $\delta$).
Für zweidimensionale Gebiete muss $\delta$ kleiner als der minimale Krümmungsradius $1/\kappa$
und kleiner als die Hälfte der Breite von kleinen Kanälen gewählt werden.
Allerdings sollte der Streifen $D \setminus D_\delta$ nicht zu schmal sein,
damit die Ableitungen der Gewichtsfunktionen klein bleiben.

\pagebreak

\subsection{%
    WEB-Splines%
}

Die Räume $\BB$ und $w\BB$ bieten zwar optimale Approximationsordnung,
aber die B-Spline-Basis ist nicht gleichmäßig stabil in Bezug zur Gitterweite $h$.
Diese Instabilität, die aufgrund von B-Splines auftritt, die nur einen kleinen Teil des Trägers
im Gebiet $D$ haben, verursacht für $h \to 0$ massive numerische Probleme.
Beispielsweise werden die Ritz-Galerkin-Systeme sehr schlecht konditioniert,
was die Konvergenz von iterativen Schemata und die Genauigkeit von numerischen Lösungen
beeinflusst.
Dieses Problem wird gelöst, indem B-Splines mit kleinem Träger zu einer stabilen Teilmenge der
Basis von $\BB$ zusammengefügt werden.

\textbf{innere und äußere B-Splines}:\\
Die Gitterzellen $Q = \ell h + [0, 1]^m h$ werden in folgende Typen unterteilt:
\begin{itemize}
    \item
    \begriff{innere Gitterzelle}, falls $Q \subset \overline{D}$,

    \item
    \begriff{äußere Gitterzelle}, falls $Q \cap D = \emptyset$,

    \item
    \begriff{Rand-Gitterzelle} sonst
    (falls das Innere von $Q$ den Rand $\partial D$ schneidet).
\end{itemize}
Die relevanten B-Splines $b_k$, $k \in K$, werden ebenfalls unterteilt:
\begin{itemize}
    \item
    \begriff{innerer B-Spline} $b_i$, $i \in I$,
    falls der Träger mindestens eine innere Zelle $Q_i$ enthält,

    \item
    \begriff{äußerer B-Spline} $b_j$, $j \in J = K \setminus I$,
    falls der Träger nur aus äußeren Gitterzellen und Rand-Gitterzellen besteht.
\end{itemize}

\linie

\textbf{gewichtete erweiterte B-Splines (WEB-Splines)}:
Für einen äußeren Index $j \in J$ sei $I(j) := \ell + \{0, \dotsc, n\}^m \subset I$ ein
$m$-dimensionales Array von inneren Indizes, die $j$ am nächsten liegen,
wobei angenommen wird, dass $h$ so klein ist, dass solch ein Array existiert.\\
Außerdem seien
\begin{align*}
    e_{i,j} := \prod_{\nu=1}^m \prod_{\genfrac{}{}{0pt}{}{\mu=0,}{\ell_\nu+\mu\not=i_\nu}}^n
    \frac{j_\nu - \ell_\nu - \mu}{i_\nu-\ell_\nu-\mu}
\end{align*}
die Werte der mit $I(j)$ assoziierten Lagrange-Polynome
(Polynom, das gleich null ist in $\ell_\nu + \mu$ und gleich eins in $i_\nu$,
ausgewertet in $j_\nu$) und
\begin{align*}
    J(i) := \{j \;|\; i \in I(j)\}.
\end{align*}
Dann formen die \begriff{gewichteten erweiterten B-Splines (WEB-Splines)}
\begin{align*}
    B_i := \frac{w}{w(x_i)} \left[b_i + \sum_{j \in J(i)} e_{i,j} b_j\right],\quad i \in I,
\end{align*}
eine Basis des \begriff{WEB-Raums} $w^\e\BB_h^n(D)$
(wobei $x_i$ der Mittelpunkt der inneren Gitterzelle\\
$Q_i \subset \overline{D} \cap \supp b_i$ ist).

\linie
\pagebreak

\textbf{Eigenschaften von WEB-Splines}:
WEB-Splines erben (außer Positivität) alle wesentlichen Eigenschaften von B-Splines.
Die folgenden Eigenschaften sind für FE-Approximationen besonders wichtig:
\begin{itemize}
    \item
    \begriff{Erweiterungskoef"|fizienten}:
    Es gilt $|e_{i,j}| \preceq 1$ und
    \begin{align*}
        e_{i,j} = 0,\quad
        \norm{i - j} \succeq 1.
    \end{align*}
    Außerdem müssen nur $\preceq h^{1-m}$ B-Splines am Rand verändert werden.
    Für die überwiegende
    Mehrheit der inneren Indizes $i$ gilt $B_i = w/w(x_i) b_i$, wenn $h$ klein wird.

    \item
    \begriff{Stabilität}:
    Die WEB-Splines sind linear unabhängig und
    \begin{align*}
        \norm{\sum_i c_i B_i}_0 \asymp h^{m/2} \norm{C}.
    \end{align*}
    Insbesondere gilt $\norm{B_i}_0 \asymp h^{m/2}$.

    \item
    \begriff{Approximationsordnung}:
    Der WEB-Raum $w^\e\BB_h^n$ enthält gewichtete Polynome vom Koordinatengrad $\le n$.
    Außerdem gilt
    \begin{align*}
        \inf_{u_h \in w^\e\BB_h^n} \norm{u - u_h}_0 \preceq h^{n+1},
    \end{align*}
    wenn $w$ und $u/w$ glatt sind.
\end{itemize}
In den Abschätzungen hängen die Konstanten vom Grad $n$, vom Gebiet $D$ und von der
Gewichtsfunktion $w$ ab (in der letzten Abschätzung auch von der approximierten
Funktion $u$).

\subsection{%
    Hierarchische Basen%
}

\textbf{hierarchische B-Splines}:
Der \begriff{hierarchische Spline-Raum} $\BB_h^n(\DD)$,
der zu einer verschachtelten Folge von Gebieten
\begin{align*}
    \DD\colon D = D_0 \supset D_1 \supset D_2 \supset \dotsb \supset D_\ell = \emptyset
\end{align*}
gehört, wird von den B-Splines
\begin{align*}
    b_{k,h_\nu},\quad
    k \in K_\nu, h_\nu = 2^{-\nu} h,\quad
    0 \le \nu < \ell,
\end{align*}
aufgespannt, wobei $K_\nu := \{k \;|\; \overline{D_\nu} \supset
\overline{D} \cap \supp b_{k,h_\nu} \not\subset \overline{D_{\nu+1}}\}$.
Durch Multiplikation mit einer Gewichtsfunktion $w$ erhält man den
\begriff{gewichteten hierarchischen Spline-Raum} $w\BB_h^n(\DD)$.
Zusätzlich ist es möglich, äußere B-Splines mit der Erweiterungsmethode zu eliminieren.

\textbf{adaptive Konstruktion}:
Zunächst wählt man eine Teilmenge der relevanten B-Splines für $D$ mit Gitterweite $h$
(die B-Splines mit $\overline{D} \cap \supp b_{k,h_0} \subset \overline{D_1}$, also
$k \in K \setminus K_0$)
und ersetzt diese mittels Subdivision durch B-Splines der Gitterweite $h/2$.
Von den relevanten B-Splines $b_{k,h_1}$ auf dem feineren Gitter mit
$\overline{D} \cap \supp b_{k,h_1} \subset \overline{D_1}$
(da sind insbesondere die B-Splines aus der Subdivision dabei) wird wieder eine Teilmenge
gewählt und verfeinert.
Dieses Verfahren wird entsprechend der Folge von Gebieten $D_\nu$ wiederholt.

\textbf{lineare Unabhängigkeit und lokale Struktur}:\\
Die B-Splines, die den Raum $\BB_h(\DD)$ aufspannen, sind linear unabhängig und es gilt
\begin{align*}
    \overline{D} \cap \supp b_{k,h_\nu} \subset \overline{D_\nu}
    \;\Rightarrow\; b_{k,h_\nu} \in \BB_h(\DD).
\end{align*}

\pagebreak
