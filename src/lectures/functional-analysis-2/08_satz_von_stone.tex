\chapter{%
    Der Satz von \name{Stone}%
}

\begin{Bem}
    Im Folgenden seien $H$ ein Hilbertraum und $A\colon D(A) \to H$ ein linearer Operator.
\end{Bem}

\section{%
    Adjungierter Operator%
}

\begin{Def}{symmetrisch}
    $A$ heißt symmetrisch, falls $\forall_{x, y \in D(A)}\; \sp{Ax, y} = \sp{x, Ay}$.
\end{Def}

\begin{Def}{Adjungierte}
    Sei $A$ dicht definiert.\\
    Dann heißt der Operator $(A^\ast, D(A^\ast))$ mit
    $D(A^\ast) := \{y \in H \;|\; \exists_{y^\ast \in H} \forall_{x \in D(A)}\;
    \sp{Ax, y} = \sp{x, y^\ast}\}$ und $A^\ast y := y^\ast$ für $y \in D(A^\ast)$
    der \begriff{zu $A$ adjungierte Operator}.
\end{Def}

\begin{Bem}
    Für $y \in D(A^\ast)$ ist $y^\ast \in H$ mit
    $\forall_{x \in D(A)}\; \sp{Ax, y} = \sp{x, y^\ast}$
    wegen $D(A^\ast)$ dicht in $H$ eindeutig bestimmt.
    $(A^\ast, D(A^\ast))$ ist ein linearer Operator auf $H$.
\end{Bem}

\linie

\begin{Lemma}{Eigenschaften von $A^\ast$}
    Sei $A$ dicht definiert.
    Dann gilt:
    \begin{enumerate}
        \item
        $A^\ast$ ist abgeschlossen.

        \item
        Ist $A$ symmetrisch, dann gilt $(A, D(A)) \subset (A^\ast, D(A^\ast))$.
    \end{enumerate}
\end{Lemma}

\section{%
    Selbstadjungierte Operatoren%
}

\begin{Def}{Abschließung}
    $A$ heißt \begriff{abschließbar}, falls es eine abgeschlossene Erweiterung von $A$ gibt.\\
    In diesem Fall heißt die kleinste abgeschlossene Erweiterung $(\overline{A}, D(\overline{A}))$
    \begriff{Abschließung} von $A$.
\end{Def}

\begin{Def}{selbstadjungiert}
    Sei $A$ dicht definiert.\\
    Dann heißt $A$ \begriff{selbstadjungiert}, falls $(A, D(A)) = (A^\ast, D(A^\ast))$.
\end{Def}

\begin{Def}{wesentlich selbstadjungiert}
    Sei $A$ symmetrisch und dicht definiert.\\
    Dann heißt $A$ \begriff{wesentlich selbstadjungiert}, falls
    $(\overline{A}, D(\overline{A}))$ selbstadjungiert ist.
\end{Def}

\begin{Bem}
    Nach dem Lemma von eben ist jeder symmetrische, dicht definierte Operator $A$ abschließbar,
    wobei $(\overline{A}, D(\overline{A})) \subset (A^\ast, D(A^\ast))$.
    Jeder selbstadjungierte Operator ist symmetrisch
    (wegen $\forall_{x, y \in D(A)}\; \sp{Ax, y} = \sp{x, A^\ast y} = \sp{x, Ay}$)
    und abgeschlossen.
\end{Bem}

\linie

\begin{Lemma}{Bild von $(A - \lambda)$}
    Sei $A$ dicht definiert.
    Dann gilt $\forall_{\lambda \in \complex}\;
    (\Bild(A - \lambda))^{\orth} = \Kern(A^\ast - \overline{\lambda})$.
\end{Lemma}

\begin{Satz}{Spektrum von selbstadj. Operatoren reell}
    Sei $A$ selbstadjungiert.
    Dann ist $\sigma(A) \subset \real$.
\end{Satz}

\begin{Satz}{Charakterisierung von Selbstadjungiertheit}
    Sei $A$ symmetrisch und dicht definiert.\\
    Dann sind äquivalent:
    \begin{enumerate}
        \item
        $A$ ist selbstadjungiert.

        \item
        Es gilt
        $\exists_{\lambda \in \complex}\; \Bild(A - \lambda) = H = \Bild(A - \overline{\lambda})$.
    \end{enumerate}
    In diesem Fall gilt $\forall_{\lambda \in \complex \setminus \real}\; \Bild(A - \lambda) = H$.
\end{Satz}

\linie

\begin{Bsp}
    Im Folgenden wird gezeigt, dass $(\Delta, H^2(\real^n))$ auf $L^2(\real^n)$
    selbstadjungiert ist.\\
    Wegen partieller Integration gilt
    $\sp{\Delta u, v}_{L^2} = \int_{\real^n} \Delta u \overline{v} \dx
    = \int_{\real^n} u \overline{\Delta v} \dx = \sp{u, \Delta v}_{L^2}$,
    d.\,h. $\Delta$ ist symmetrisch.
    Seien nun $\lambda \in \complex \setminus \real$ und $f \in L^2(\real^n)$ und
    betrachte $\Delta u - \lambda u = f$.
    Mit Fouriertransformation gilt
    $u(x) = (2\pi)^{-n/2} \int_{\real^n}
    \left(-\frac{\widehat{f}(k)}{\lambda + k^2}\right) e^{\iu\sp{k, x}} dk$
    (der Nenner verschwindet nicht, da $\Im(\lambda) \not= 0$),
    daraus folgt, dass es eine Lösung $u \in H^2(\real^n)$ gibt.
    Mit dem Satz von eben folgt, dass $(\Delta, H^2(\real^n))$ selbstadjungiert ist.
\end{Bsp}

\pagebreak

\section{%
    Satz von \name{Stone}%
}

\begin{Def}{unitär}
    Sei $U \in \Lin(H)$.
    Dann heißt $U$ unitär, falls $U$ bijektiv ist und $U^\ast = U^{-1}$.
\end{Def}

\begin{Lemma}{Charakterisierung von Unitärität}\\
    $U \in \Lin(H)$ ist unitär genau dann,
    wenn $U$ eine surjektive Isometrie ist.
\end{Lemma}

\begin{Satz}{Satz von \name{Stone}}\\
    Sei $A$ ein dicht definierter, linearer Operator auf einem Hilbertraum $H$.\\
    $A$ ist Erzeuger einer $\C_0$-Gruppe $(U(t))_{t \in \real}$ von unitären Operatoren
    auf $H$ genau dann, wenn $\iu A$ selbstadjungiert ist.
\end{Satz}

\begin{Lemma}{Fall $A, A^\ast$ dissipativ und abg.}\\
    Sei $(A, D(A))$ ein dicht definierter, abgeschlossener, linearer Operator auf $H$.\\
    Sind sowohl $A$ als auch $A^\ast$ dissipativ, dann ist $A$ der Erzeuger einer
    Kontraktions-HG auf $H$.
\end{Lemma}

\linie

\begin{Bsp}
    Wegen dem Satz von Stone und obigem Beispiel erzeugt $A := \iu\Delta$ eine $\C_0$-Gruppe
    von unitären Operatoren.
    Insbesondere ist die sogenannte \begriff{lineare \name{Schrödinger}-Gleichung}
    $\partial_t u = \iu\Delta u$ und $u(t = 0) = u_0$ lösbar in $H^2$
    und die $L^2$-Norm der Lösung bleibt erhalten
    (kann man mit der Fouriertransformation auch direkt nachrechnen).
\end{Bsp}

\pagebreak
