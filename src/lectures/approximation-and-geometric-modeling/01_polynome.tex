\section{%
    Polynome%
}

\subsection{%
    Monomdarstellung%
}

\textbf{Polynom in Monomdarstellung}:
Ein (reelles) \begriff{Polynom} $p$ vom Grad $n$ ist eine Linearkombination
\begin{align*}
    p(x) = c_0 + c_1 x + \dotsb + c_n x^n
\end{align*}
der Monome $x \mapsto x^k$ mit \begriff{Koef"|fizienten} $c_k \in \real$ und $c_n \not= 0$
(\begriff{Monomdarstellung}).

Die Koef"|fizienten hängen durch $k! c_k = p^{(k)}(0)$ für $k = 0, \dotsc, n$ mit den Ableitungen
an $x = 0$ zusammmen.
Die Polynome vom Grad $\le n$ formen einen Vektorraum $\P^n$ der Dimension $n + 1$.
Man schreibt $\P^n(D)$, falls die Variable $x$ auf ein bestimmtes Intervall $D$ beschränkt ist.

\linie

\textbf{geschachtelte Multiplikation (\name{Horner}-Schema)}:
Beim Auswerten eines Polynoms\\
$p(x) = c_0 + c_1 x + \dotsb + c_n x^n$ kann man zu häufiges Potenzieren von $x$
vermeiden, indem man $p$ in \begriff{geschachtelter Form (\name{Horner}-Schema)} schreibt, d.\,h.
\begin{align*}
    p(x) = c_0 + (c_1 + \dotsb (c_{n-2} + (c_{n-1} + c_n x)x) \dotsb)x.
\end{align*}
Mit $p_n := c_n$ ergibt die Rekursion $p_k := c_k + p_{k+1} x$, $k = n - 1, \dotsc, 0$,
den Wert $p(x) = p_0$ in $n$ Schritten mit $2n$ Operationen.

Die Rekursion kann auch zur Auswertung der Ableitung $p$ benutzt werden.
Dazu leitet man die Rekursion nach $x$ ab und erhält mit der Produktregel
$p_n' := 0$ und $p_k' := p_{k+1}' x + p_{k+1}$, $k = n - 1, \dotsc, 0$,
mit $p'(x) = p_0'$.

\subsection{%
    \name{Taylor}-Approximation%
}

\textbf{\name{Taylor}-Polynom}:
Das \begriff{\name{Taylor}-Polynom} $p_n$ vom Grad $\le n$ einer Funktion $f$ im Punkt $x_0$
stimmt in den Ableitungen bis zum Grad $n$ mit $f$ in $x_0$ überein:
\begin{align*}
    p_n(x) := \sum_{k=0}^n \frac{f^{(k)}(x_0)}{k!} (x - x_0)^k.
\end{align*}
Der Approximationsfehler bzw. das Restglied kann in der Form
\begin{align*}
    f(x) - p_n(x) = \frac{f^{(n+1)}(\xi)}{(n + 1)!} (x - x_0)^{n+1}
\end{align*}
dargestellt werden, wobei $\xi$ ein Punkt zwischen $x$ und $x_0$ ist.
Als Folgerung approximieren Taylor-Polynome vom Grad $\le n$ glatte Funktionen auf einem
Intervall $[x_0 - h, x_0 + h]$ mit der Ordnung $\O(h^{n+1})$.

\linie

\textbf{Beispiel (Auswertung einer Funktion mit tabulierten Werten)}:
Als Anwendung des Taylor-Polynoms kann man eine Tabelle mit eingespeicherten
Funktionswerten einer bestimmten Funktion betrachten (z.\,B. Sinusfunktion).
Um den Wert der Funktion an einer Zwischenstelle zu errechnen, kann man das Taylor-Polynom
errechnen und in der Zwischenstelle auswerten.
Die Fehlerformel kann benutzt werden, um auszurechnen, wie nah die gegebenen Funktionswerte
beieinander sein müssen, damit eine gewisse Genauigkeit erreicht wird.

\pagebreak

\subsection{%
    Interpolation%
}

\textbf{Interpolation}:
Funktionswerte $f_0, \dotsc, f_n$ an $n + 1$ paarweise verschiedenen Stellen $x_0, \dotsc, x_n$
können eindeutig durch ein Polynom $p$ vom Grad $\le n$ interpoliert werden,
d.\,h. $p(x_k) = f_k$ für $k = 0, \dotsc, n$.
Das Polynom kann explizit in der \begriff{\name{Lagrange}-Form} angegeben werden:
\begin{align*}
    p(x) := \sum_{k=0}^n f_k q_k(x),\quad
    q_k(x) := \prod_{\ell\not=k} \frac{x - x_\ell}{x_k - x_\ell}.
\end{align*}
Dabei sind die \begriff{\name{Lagrange}-Polynome} $q_k$ in $x_k$ gleich 1 und verschwinden an
allen anderen Interpolationspunkten.

\linie

\textbf{4-Punkt-Formel}:
Polynomiale Interpolation mit niedrigen Polynomgraden wird oft benutzt, um Funktionswerte
an Zwischenstellen zu generieren.
Ein Beispiel ist die \begriff{4-Punkt-Formel}
\begin{align*}
    f_{k+1/2} \approx (-f_{k-1} + 9f_k + 9f_{k+1} - f_{k+2})/16,
\end{align*}
die $f(kh + h/2)$ durch benachbarte Funktionswerte $f(\ell h)$ an äquidistanten Stellen
$x_\ell = \ell h$ schätzt.
Die Formel basiert dabei auf kubischer Interpolation,
d.\,h. die Gewichte $-1/16$, $9/16$, $9/16$ und $-1/16$ sind die Lagrange-Polynome
der vier Punkte $x_{k-1}$, $x_k$, $x_{k+1}$ und $x_{k+2}$ ausgewertet in $(k + 1/2)h$.

\linie

\textbf{\name{Aitken}-\name{Neville}-Schema}:
Wenn ein Polynom $p_k^{m-1}$ eine Funktion $f$ an paarweise verschiedenen Punkten
$x_k, \dotsc, x_{k+m-1}$ interpoliert, dann interpoliert
\begin{align*}
    p_k^m := (1 - w_k^m) p_k^{m-1} + w_k^m p_{k+1}^{m-1},\quad
    w_k^m(x) := \frac{x - x_k}{x_{k+m} - x_k}
\end{align*}
an $x_k, \dotsc, x_{k+m}$.

Der Wert $p(x)$ des Interpolationspolynoms
von $n + 1$ Datenpunkten $(x_k, f(x_k))$, $k = 0, \dotsc, n$, mit $x_0 < \dotsb < x_n$
an einer Stelle $x \in \real$ lässt sich somit mithilfe eines Dreiecksschemas berechnen,
startend mit $p_k^0 := f(x_k)$:
\begin{align*}
    \begin{array}{ccccccc}
        f(x_0) = p_0^0 & \rightarrow & p_0^1 & \cdots & p_0^{n-1} & \rightarrow & p_0^n = p(x)\\
        & \nearrow & & & & \nearrow\\
        f(x_1) = p_1^0 & & p_1^1 & & p_1^{n-1}\\
        \vdots & & \vdots &\\
        f(x_{n-1}) = p_{n-1}^0 & \rightarrow & p_{n-1}^1\\
        & \nearrow\\
        f(x_n) = p_n^0
    \end{array}
\end{align*}
Die Pfeile $\rightarrow$ bzw. $\nearrow$, die zu $p_k^m$ zeigen,
deuten Multiplikation mit $(1 - w_k^m)$ bzw. $w_k^m$ an.
Das letzte Polynom $p_0^n$ hat Grad $\le n$ und interpoliert $x_0, \dotsc, x_n$.

Der Vorteil dieses Dreiecksschemas ist, dass zur Verbesserung der Genauigkeit
weitere Datenpunkte sehr einfach als neue Zeile am unteren Rand hinzugefügt
werden können, ohne alle Werte neu zu berechnen
(anders als z.\,B. mit Lagrange-Polynomen).

\linie

\textbf{\name{Runge}s Phänomen}:
Äquidistante polynomiale Interpolation der rationalen Funktion $f(x) = 1/(1 + x^2)$
(\begriff{\name{Runge}-Funktion}) im Intervall $[-5, 5]$ führt
im Grenzwert zu keiner gleichmäßigen Konvergenz.
Der Fehler der Interpolation, der absolut bei Grad 10 schon bei circa 2 liegt, kommt durch die
Singularitäten von $f$ an $\pm \i$ nahe der reellen Achse zustande.
Der Konvergenzradius der Taylor-Reihe beträgt daher nur 1.

\subsection{%
    \name{Bernstein}-Polynome%
}

\textbf{\name{Bernstein}-Polynome}:
Die \begriff{\name{Bernstein}-Polynome} vom Grad $n$ sind definiert durch
\begin{align*}
    b_k^n(x) := \binom{n}{k} (1 - x)^{n-k} x^k,\quad
    k = 0, \dotsc, n.
\end{align*}
Sie bilden eine Basis des Raums $\P^n$ der Polynome vom Grad $\le n$.
Die Basis ist symmetrisch bzgl. des Standardintervalls $[0, 1]$.
Genauer gilt für $j = 0, \dotsc, n$ bzw. für $k = 0, \dotsc, n$
\begin{align*}
    x^j = \sum_{k=j}^n \left.\binom{k}{j}\right/\binom{n}{j} b_k^n(x),\quad
    b_k^n(x) = \sum_{j=0}^{n-k} (-1)^j \binom{n}{k} \binom{n-k}{j} x^{j+k}.
\end{align*}
Die beiden Gleichungen beschreiben die Umrechnung zwischen der Monom- und der
Bernstein-Darstellung eines Polynoms.
In Gleichungen und Rekursionen mit Bernstein-Polynomen ist es oft üblich, allgemeine
Indizes $k \in \integer$ zu benutzen.
Für $k \notin \{0, \dotsc, n\}$ sei dabei $b_k^n(x) :\equiv 0$.

\subsection{%
    Eigenschaften der \name{Bernstein}-Polynome%
}

\textbf{Eigenschaften der \name{Bernstein}-Polynome}:
Die Bernstein-Polynome vom Grad $n$ sind auf dem Standardintervall $[0, 1]$ nicht-negativ und
summieren zu Eins:
\begin{align*}
    \sum_{k=0}^n b_k^n(x) = 1.
\end{align*}
Außerdem hat $b_k^n$ auf $[0, 1]$ ein eindeutiges Maximum bei
\begin{align*}
    x = \frac{k}{n}.
\end{align*}
In den Intervallendpunkten $0$ und $1$ ist nur das erste und das letzte Bernstein-Polynom
nicht Null, genauer:
\begin{align*}
    &b_0^n(0) = 1,\quad
    b_1^n(0) = \dotsb = b_n^n(0) = 0,\\
    &b_0^n(1) = \dotsb = b_{n-1}^n(1) = 0,\quad
    b_n^n(1) = 1.
\end{align*}
Als Folgerung ist ein Polynom $p = \sum_{k=0}^n c_k b_k^n$ in Bernstein-Darstellung
gleich $c_0$ in $x = 0$ und gleich $c_n$ in $x = 1$.
Man nennt diese Eigenschaft \begriff{Endpunktinterpolation}.

Außerdem kann man ein Polynom in Bernstein-Darstellung durch das betragsmäßige Maximum der
Koef"|fizienten abschätzen:
$|p(x)| \le \sum_{k=0}^n |c_k| \cdot b_k^n(x) \le \max_{k=0,\dotsc,n} |c_k|$.

\linie

\textbf{\name{Bernstein}-Approximation}:
Man kann eine Funktion $f$ durch Verwendung ihrer Werte an den Stellen, in denen die
Bernstein-Polynome $b_k^n$ maximal werden, als Bernstein-Koef"|fizienten approximieren:
\begin{align*}
    f \approx p = \sum_{k=0}^n f(k/n) b_k^n.
\end{align*}
Die sogenannte \begriff{\name{Bernstein}-Approximation} modelliert
die Form des Graphen von $f$ auf $[0, 1]$ recht genau.
Aufgrund der Eigenschaften der Bernstein-Polynome interpoliert $p$ die Funktion $f$ in $0$ und $1$.
Außerdem ist $p$ nicht-negativ, wenn $f$ nicht-negativ ist.
Die Approximation ist für lineare Polynome exakt.

\linie
\pagebreak

\textbf{Identitäten für \name{Bernstein}-Polynome}:\\
Die Bernstein-Polynome $b_k^n$, $k = 0, \dots, n$ erfüllen folgende Identitäten:
\begin{align*}
    \begin{array}{cl}
        b_k^n(1 - x) = b_{n-k}^n(x) & \text{(Symmetrie)}\\[2mm]
        b_k^n(x) = x b_{k-1}^{n-1}(x) + (1 - x) b_k^{n-1}(x) & \text{(Rekursion)}\\[2mm]
        (b_k^n)' = n (b_{k-1}^{n-1} - b_k^{n-1}) & \text{(Dif"|ferentiation)}\\[2mm]
        \int_0^1 b_k^n = \frac{1}{n+1} & \text{(Integration)}
    \end{array}
\end{align*}
Dabei ist $b_{-1}^{n-1} \equiv b_n^{n-1} \equiv 0$ in der zweiten und dritten Identität
aufgrund der Konvention.

\linie

\textbf{Auswertung von Polynomen in \name{Bernstein}-Darstellung mit vorgenerierten Werten}:
Wenn viele Polynome in Bernstein-Darstellung an denselben Punkten $x_\ell$ ausgewertet
werden müssen, sollten die Werte $a_{\ell,k}^n := b_k^n(x_\ell)$ vorher berechnet werden.
Wegen der Identitäten für Bernstein-Polynome können die Matrizen $A^n$ durch die Rekursion
$a_{\ell,k}^{n+1} := x_\ell a_{\ell,k-1}^n + (1 - x_\ell) a_{\ell,k}^n$
mit $a_{\ell,-1}^n := a_{\ell,n+1}^n := 0$ bestimmt werden.
Die Werte $p(x_\ell) = \sum_{k=0}^n c_k b_k^n(x_\ell)$ können dann durch eine einzige
Matrixmultiplikation $A^n c$ errechnet werden.

\subsection{%
    \name{Hermite}-Interpolation%
}

\textbf{\name{Hermite}-Interpolation}:
Werte $f_0, f_1$ und Ableitungen $d_0, d_1$ an zwei Punkten $x_0 < x_1$ können durch ein
kubisches Polynom $p$ interpoliert werden.
Dieser \begriff{\name{Hermite}-Interpolant} kann als Linearkombination der auf das Intervall
$[x_0, x_1]$ transformierten Bernstein-Polynome dargestellt werden:
\begin{align*}
    p(x) := f_0 b_0^3(y) + (f_0 + d_0 h/3) b_1^3(y) + (f_1 - d_1 h/3) b_2^3(y) + f_1 b_3^3(y)
\end{align*}
mit $h := x_1 - x_0$ und $y := (x - x_0)/h$.

Die Bernstein-Koef"|fizienten des Interpolanten $p$ können auch zeichnerisch ermittelt werden:
Teilt man das Intervall $[x_0, x_1]$ äquidistant in Drittel auf, so ist der zweite
Bernstein-Koef"|fizient gleich der Ordinate des Schnittpunkts von der Tangente an $p$ in $x_0$
mit der Geraden $x = x_0 + h/3$.
Der dritte Koef"|fizient ist zum zweiten symmetrisch.

\linie

\textbf{\name{Hermite}-Spline}:
Falls Hermite-Daten an mehr als zwei Punkten gegeben sind, formen die kubischen Interpolanten
einen sogenannten \begriff{\name{Hermite}-Spline} $q$.
Nach Konstruktion ist $q$ durch seine Werte und Ableitungen an den
Interpolationspunkten eindeutig bestimmt und an diesen stetig dif"|ferenzierbar.

\subsection{%
    Approximation von stetigen Funktionen%
}

\textbf{Approximationssatz von \name{Weierstraß}}:
Jede stetige Funktion $f$ kann auf einem beschränkten, abgeschlossenen Intervall $[a, b]$
durch Polynome mit beliebiger Genauigkeit approximiert werden.
Genauer existiert für jedes $\varepsilon > 0$ ein Polynom $p$ mit
\begin{align*}
    \max_{x \in [a, b]} |f(x) - p(x)| < \varepsilon.
\end{align*}
Als Beweis kann man z.\,B. zeigen,
dass die Bernstein-Approximationen\\
$f \approx p_n = \sum_{k=0}^n f(k/n) b_k^n$ für $n \to \infty$ gegen $f$ gehen.

\pagebreak
