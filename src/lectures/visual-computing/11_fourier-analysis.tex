\chapter{%
    \name{Fourier}-Analysis%
}

\section{%
    Trigonometrische Approximation und \name{Fourier}-Reihen%
}

\textbf{trigonometrische Approximation}:
Gegeben sei eine $2\pi$-periodische, stückweise stetige Funktion $f\colon \real \to \real$, d.\,h.
$\forall_{x \in \real}\; f(x + 2\pi) = f(x)$, sodass für alle $x_0 \in \real$ die Grenzwerte\\
$\lim_{h \to 0+0} f(x_0 - h) = y_0^-$ und $\lim_{h \to 0+0} f(x_0 + h) = y_0^+$ existieren.\\
Gesucht ist $g_n(x) := \frac{1}{2} a_0 + \sum_{k=1}^n (a_k\cos(kx) + b_k\sin(kx))$, sodass\\
$F := \norm{g_n - f}_{L^2}^2 = \int_{-\pi}^\pi (g_n - f)^2 \dx \to \min$.\\
Durch $\partial_{a_0} F = \partial_{a_j} F = \partial_{b_j} F \overset{!}{=} 0$ für
$j = 1, \dotsc, n$ erhält man folgende Formeln.

\linie

\textbf{reelle \name{Fourier}-Reihe}:\\
Seien $a_k := \frac{1}{\pi} \int_{-\pi}^\pi f(x)\cos(kx) \dx$ für $k \in \natural_0$
und $b_k := \frac{1}{\pi} \int_{-\pi}^\pi f(x)\sin(kx) \dx$ für $k \in \natural$.\\
Das \begriff{reelle \name{Fourier}-Polynom} ist gegeben durch
$g_n(x) := \frac{1}{2} a_0 + \sum_{k=1}^n (a_k\cos(kx) + b_k\sin(kx))$.\\
Die \begriff{reelle \name{Fourier}-Reihe} ist gegeben durch
$g(x) := \frac{1}{2} a_0 + \sum_{k=1}^\infty (a_k\cos(kx) + b_k\sin(kx))$.

\textbf{komplexe \name{Fourier}-Reihe}:
Durch Einsetzen von $\cos(kx) = \frac{e^{\iu kx} + e^{-\iu kx}}{2}$ und
$\sin(kx) = \frac{e^{\iu kx} - e^{-\iu kx}}{2\iu}$ in die reelle Fourier-Reihe erhält man
die \begriff{komplexe \name{Fourier}-Reihe} $g(x) = \sum_{k\in\integer} c_k e^{\iu kx}$
mit $c_k := \frac{2}{\pi} \int_{-\pi}^\pi f(x) e^{-\iu kx} \dx$ bzw.
$c_0 := \frac{a_0}{2}$ für $k = 0$,
$c_k := \frac{a_k - \iu b_k}{2}$ und $c_{-k} := \frac{a_k + \iu b_k}{2}$ für $k \in \natural$.

\linie

\textbf{Satz (Konvergenz der \name{Fourier}-Reihe)}:
Gegeben sei eine $2\pi$-periodische, stückweise stetige Funktion $f\colon \real \to \real$
mit stückweise stetiger Ableitung.\\
Dann konvergiert die Fourier-Reihe
$g(x) := \frac{1}{2} a_0 + \sum_{k=1}^\infty (a_k\cos(kx) + b_k\sin(kx))$ in $x_0 \in \real$
\begin{itemize}
    \item
    gegen $f(x_0)$, wenn $f$ stetig in $x_0$ ist, und

    \item
    gegen $\frac{y_0^- + y_0^+}{2}$, wenn $x_0$ eine Sprungstelle von $f$ ist.
\end{itemize}

\textbf{\name{Gibbs}-Phänomen}:
Anhand der Rechteck-Funktion $f := \chi_{[-\pi/2,\pi/2]}$ erkennt man,
dass die Fourier-Reihe zwar punktweise f.\,ü. konvergiert,
die $L^\infty$-Norm der Dif"|ferenz aber nicht konvergiert,
weil $g_n(x_0) \to 1.08949$ für $n \to \infty$ und $x_0 = x_0(n)$ der Maximalstelle von $g_n$.

\pagebreak

\section{%
    \name{Fourier}-Transformation%
}

\textbf{\name{Fourier}-Transformation}:
Sei $f \in L^1(\real) := L^1(\real, \complex)$.\\
Dann ist $(\F(f))(k) = F(k) := \int_\real f(x)e^{-\iu kx} \dx$
die \begriff{\name{Fourier}-Transformation}.

\textbf{inverse \name{Fourier}-Transformation}:
Sei $f \in L^1(\real)$ mit $F = \F(f) \in L^1(\real)$.\\
Dann ist $f(x) = \frac{1}{2\pi} \int_\real F(k) e^{\iu kx}\d k$ die
\begriff{inverse \name{Fourier}-Transformation}.

\textbf{Eigenschaften der \name{Fourier}-Transformation}:

{\small\begin{tabular}{lll}
    \toprule
    \emph{Eigenschaft} & \emph{Funktion} & \emph{\name{Fourier}-Transformierte}\\
    \midrule
    Fourier-Transformation & $f(x)$ & $F(k)$\\
    inverse Fourier-Transformation & $F(x)$ & $2\pi f(-k)$\\
    \midrule
    Faltung & $(f_1 \ast f_2)(x)$ & $F_1(k) F_2(k)$\\
    Multiplikation & $f_1(x) f_2(x)$ & $\frac{1}{2\pi} (F_1 \ast F_2)(k)$\\
    \midrule
    Translation & $f(x - a)$ & $e^{-\iu ak} F(k)$\\
    Modulation & $e^{\iu ax} f(x)$ & $F(k - a)$\\
    \midrule
    Skalierung & $f(x/a)$ & $|a| F(ak)$\\
    \midrule
    Ableitung & $f^{(p)}(x)$ & $(\iu k)^p F(k)$\\
    Frequenzableitung & $(-\iu x)^p f(x)$ & $F^{(p)}(k)$\\
    \midrule
    komplexes Konjugat & $\overline{f(x)}$ & $\overline{F(-k)}$\\
    hermitesche Symmetrie & $f(x) \in \real$ & $F(k) = \overline{F(-k)}$\\
    \bottomrule
\end{tabular}}

\section{%
    \name{Dirac}sche Delta-Distribution%
}

\textbf{\name{Dirac}-Delta}:
Die \begriff{\name{dirac}sche Delta-Distribution} $\delta$
ist eine Distribution, die eine stetige Funktion $\phi$ im Punkt $t = 0$ auswertet,
d.\,h. $\int_\real \phi(t) \delta(t) \dt := \phi(0)$
(keine Funktion, da $\int_\real \delta(t) \dt = 1$).
Für die Auswertung in $u \in \real$ schreibt man
$(\phi \ast \delta)(u) = \int_\real \phi(t) \delta(u - t) \dt := \phi(u)$.
Die Ableitung ist gegeben durch
$\int_\real \phi(t) \delta^{(n)}(t) \dt := (-1)^n \phi^{(n)}(0)$.

\textbf{Delta-Folgen}:
\begriff{Delta-Folgen} approximieren $\delta$,
z.\,B. die \begriff{Glockenkurve}
$\delta_\varepsilon(x) := \frac{1}{\sqrt{2\pi\varepsilon}} e^{-x^2/(2\varepsilon)}$ und
die \begriff{\name{Lorentz}-Kurve}
$\delta_\varepsilon(x) := \frac{1}{\pi} \cdot \frac{\varepsilon}{x^2 + \varepsilon^2}$
(erfüllen jeweils $\int_\real \delta_\varepsilon(x) \dx = 1$).

\textbf{\name{Fourier}-Transformation mit dem \name{Dirac}-Delta}:

\begin{tabular}{p{60mm}p{60mm}}
    \toprule
    \emph{Funktion} $f(x)$ & \emph{\name{Fourier}-Transformierte} $F(k)$\\
    \midrule
    $\delta(x - a)$ & $e^{-\iu ka}$\\
    $c$ & $c \delta(k)$\\
    \midrule
    $\cos(k_0 x)$ & $\frac{1}{2} (\delta(k - k_0) + \delta(k + k_0))$\\
    $\sin(k_0 x)$ & $\frac{1}{2\iu} (\delta(k - k_0) - \delta(k + k_0))$\\
    %\midrule
    %$\sum_{n\in\integer} c_n e^{\iu nx/\Delta x}$&
    %$\sum_{n\in\integer} c_n \delta(k - \frac{n}{\Delta x})$\\
    %$\sum_{n\in\integer} \delta(x - n\Delta x)
    %= \frac{1}{\Delta x} \sum_{n \in \integer} e^{\iu n}$&
    %$\frac{1}{\Delta x} \sum_{n\in\integer} \delta(k - n/\Delta x)$\\
    \bottomrule
\end{tabular}

\pagebreak

\section{%
    Sampling-Theorem%
}

In diesem Abschnitt wird eine andere Definition für die Fourier-Transformation benutzt,
nämlich $(\F(f))(k) = F(k) := \int_\real f(x) e^{-2\pi\iu kx} \dx$
(bzw. $f(x) = \int_\real F(k) e^{2\pi\iu kx} \d k$).

\textbf{Delta-Kamm}:
Die Transformation eines kontinuierlichen Signals $f$ in ein diskretes Signal $\widehat{f}$
kann man mit dem \begriff{Delta-Kamm} $c(x) := \sum_{n \in \integer} \delta(x - n\Delta x)$
darstellen durch\\
$\widehat{f}(x) := f(x) c(x) = \sum_{n \in \integer} f(n\Delta x) \delta(x - n\Delta x)$,
wobei $\Delta x$ die \begriff{Sampling-Gitterweite} ist.\\
Dabei besitzt $\widehat{f}(x)$ die Fourier-Transformierte
$\widehat{F}(k) = \frac{1}{\Delta x} \sum_{n \in \integer} F(k - \frac{n}{\Delta x})$.

\linie

\textbf{Rekonstruktion des originalen Signals}:\\
Um $f(x)$ aus $\widehat{F}(k)$ zu rekonstruieren, verfährt man wie folgt.
\begin{enumerate}
    \item
    Multipliziere $\widehat{F}(k)$ mit der Rechtecksfunktion
    $R(k) := \Delta x \cdot \chi_{[-k_c,k_c]}(k)$\\
    (es gilt $F(k) = \widehat{F}(k) R(k)$ genau dann, wenn $f$ bandbegrenzt durch $k_c$ ist).

    \item
    Wende die inverse FT auf $\widehat{F}(k) R(k)$ an, d.\,h.\\
    $\F^{-1}(\widehat{F}(k) R(k))
    = \sum_{n \in \integer} f(n\Delta x) 2k_c \Delta x \sinc(2k_c (x - n\Delta x))$
    mit $\sinc(x) := \frac{\sin(\pi x)}{\pi x}$.
\end{enumerate}

\textbf{Sampling-Theorem}:
Sei $f(\cdot)$ eine Funktion.
Gibt es eine \begriff{Abschneidefrequenz} $k_c > 0$ mit\\
$\forall_{|k| \ge k_c}\; F(k) = 0$, dann kann $f$ aus der
gesampelten Funktion $\widehat{f}$ exakt rekonstruiert werden,
wenn $k_c \le \frac{1}{2 \Delta x} = \frac{k_s}{2}$
mit der \begriff{Samplerate} $k_s := \frac{1}{\Delta x}$ und
der \begriff{\name{Nyquist}-Frequenz} $\frac{k_s}{2}$.\\
In diesem Fall gilt
$f(x) = \sum_{n \in \integer} f(n\Delta x) 2k_c \Delta x \sinc(2k_c (x - n\Delta x))$.

\textbf{Aliasing}:
Wenn $f(x)$ unterabgetastet wird, d.\,h. $k_s \not\ge 2k_c$,
dann besitzt die rekonstruierte Funktion Artefakte, sog. \begriff{Aliasing-Ef"|fekte}.

\section{%
    Diskrete \name{Fourier}-Transformation%
}

\textbf{diskrete 1D-\name{Fourier}-Transformation}:\\
Seien $\vec{g} := (g_n)_{n=0}^{N-1} \in \complex^N$
und $\omega_N := e^{2\pi\iu/N}$ die \begriff{$N$-te Einheitswurzel}.\\
Dann heißt
$G_v := \sum_{n=0}^{N-1} \omega_N^{-vn} g_n$ für $v = 0, \dotsc, N - 1$
\begriff{diskrete \name{Fourier}-Transf. (DFT)}.\\
$g_n = \frac{1}{N} \sum_{v=0}^{N-1} \omega_N^{vn} G_v$ für $n = 0, \dotsc, N - 1$
heißt \begriff{inverse DFT}.\\
Es gilt $G_v = \sqrt{N} \sp{\vec*{b}{v}, \vec{g}}$
und $g_n = \frac{1}{\sqrt{N}} \sp{\vec*{b}{-n}, \vec{G}}$ mit den Basisvektoren\\
$\vec*{b}{v} := \frac{1}{\sqrt{N}}
(\omega_N^0, \omega_N^{v}, \dotsc, \omega_N^{(N-1)v})^\tp$,
wobei $\sp{\vec{x}, \vec{y}} := \vec{x}^\ast \vec{y}$ mit
$\vec{x}^\ast := \overline{\vec{x}^\tp}$.

\linie

\textbf{diskrete 2D-\name{Fourier}-Transformation}:
Sei $g := (g_{m,n})_{m,n=0}^{M-1,N-1} \in \complex^{M \times N}$.\\
Dann heißt
$G_{u,v} := \sum_{m=0}^{M-1} \sum_{n=0}^{N-1} \omega_M^{-um} \omega_N^{-vn} g_{m,n}$ für
$u = 0, \dotsc, M - 1$, $v = 0, \dotsc, N - 1$
\begriff{diskrete 2D-\name{Fourier}-Transf. (2D-DFT)}.
$g_{m,n} = \frac{1}{MN} \sum_{u=0}^{M-1} \sum_{v=0}^{N-1} \omega_M^{um} \omega_N^{vn} G_{u,v}$
heißt \begriff{inverse 2D-DFT}.\\
Es gilt $G_{u,v} = \sqrt{MN} \sp{B_{u,v}, g}$
und $g_{m,n} = \frac{1}{\sqrt{MN}} \sp{B_{-m,-n}, G}$
mit den Basismatrizen\\
$B_{u,v} := \frac{1}{\sqrt{MN}} (\omega_M^0, \omega_M^u, \dotsc, \omega_M^{(M-1)u})^\tp
(\omega_N^0, \omega_N^v, \dotsc, \omega_N^{(N-1)v})$,\\
wobei $\sp{G, H} := \sum_{m=0}^{M-1} \sum_{n=0}^{N-1} \overline{g_{m,n}} h_{m,n}$.

\textbf{Eigenschaften}:
\begin{itemize}
    \item
    \vspace{-1mm}
    \emph{1D-Periodizität}:
    $G_{v+\ell N} = G_v$,
    $g_{n+\ell N} = g_n$
    für alle $\ell \in \integer$

    \item
    \vspace{-1mm}
    \emph{2D-Periodizität}:
    $G_{u+kM,v+\ell N} = G_{u,v}$,
    $g_{m+kM,n+\ell N} = g_{m,n}$
    für alle $k, \ell \in \integer$

    \item
    \vspace{-1mm}
    \emph{1D-DFT einer reellen Folge ist hermitesch}:
    $g_n \in \real \implies \overline{G_v} = G_{-v} = G_{N-v}$

    \item
    \vspace{-1mm}
    \emph{1D-DFT einer hermiteschen Folge ist reell}:
    $g_{N-n} = g_{-n} = \overline{g_n} \implies G_v \in \real$

    \item
    \vspace{-1mm}
    \emph{2D-Faltungssatz}:
    2D-DFT von $f_{m,n} = \sum_{m'=0}^{M-1} \sum_{n'=0}^{N-1} h_{m',n'} g_{m-m',n-n'}$
    ist $MN \cdot (H \cdot G)$
\end{itemize}

\pagebreak
