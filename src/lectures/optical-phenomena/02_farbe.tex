\section{%
    Farbe%
}

\begin{itemize}
    \item
    \textbf{Teilchenmodell}:
    Lichtstrahlen bestehen aus Teilchen (Photonen), die sich in geraden
    Bahnen ausbreiten (früher akzeptiertes Modell)
    
    \item
    \textbf{Wellenmodell}:
    Licht zeigt Eigenschaften einer Welle (Frequenz, Polarisation usw.) --
    später akzeptiertes Modell
    
    \item
    \textbf{Wellen-/Teilchendualismus}:
    Licht zeigt beide Eigenschaften, daher kann man beide Modelle zur
    Beschreibung benutzen (Quantenmechanik)
\end{itemize}
\linie
\begin{itemize}
    \item
    \textbf{Licht ist elektromagnetische Welle}:
    Licht breitet sich wie Wasserwelle in alle Richtungen aus,
    hat Wellenlänge $\lambda$, Frequenz $f = \nu$ und Geschwindigkeit $c$,
    unter Annahme gleichförmiger Bewegung gilt $c = \lambda \cdot \nu$.
    Licht versteht man heute als elektromagnetische Welle, im Spektrum besitzt
    sichtbares Licht Wellenlängen zwischen $\SI{300}{\nano\meter}$ und
    $\SI{700}{\nano\meter}$
    (kürzer: $\gamma$-Strahlung, UV, Röntgen,
    länger: Radiowellen, IR, Mikrowellen)
    
    \item
    \textbf{Brechzahl $n$}:
    gibt an, wie schnell Licht in einem bestimmten Stoff ist, d.\,h.
    $c' = \frac{c}{n}$, z.\,B. $c_\text{Wasser} = 1.3$
    
    \item
    \textbf{Übergang zwischen Stof"|fen}:
    z.\,B. von Luft in Wasser, nach EES bleibt Energie (Frequenz) erhalten,
    d.\,h. die Wellenlänge muss sich nach $\lambda' = \frac{\lambda}{n}$ ändern
    (da sich $c$ ändert) --
    doch Badehose im Wasser ändert nicht die Farbe, da die Frequenz des Lichts
    die wahrgenommene Farbe bestimmt
\end{itemize}
\linie
\begin{itemize}
    \item
    \textbf{Grund für Farbensehen}:
    Objekte können durch die zusätzliche Information besser identifiziert
    werden (bspw. Früchte)
    
    \item
    \textbf{Mensch besitzt drei Farbkanäle}:
    Zapfen sind farb-empfindlich, für Rot/Grün/Blau gibt es solche Sehzellen
    (daher haben TV/Monitor/Bayer-Sensor in Kameras solche Pixel)
    
    \item
    \textbf{additive Farbmischung}:
    verschiedene Farben entstehen durch Kombination von Licht verschiedener
    Wellenlängen (Rot + Grün = Gelb)
    
    \item
    \textbf{Empfindlichkeit der Zapfen}:
    für Blau sehr gering, für Grün mittel und Rot am stärksten --
    blaue Sehzellen weniger empfindlich, außerdem gibt es weniger blaue
    Sehzellen. \\
    Verteilung Rot/Grün ist stark individuell abhängig
    (daher unterschiedlicher Farbeindruck),
    Wellenlängen mit max. Empfindlichkeit für Rot/Grün liegen nah beieinander
    (Begründung mit Evolution: früher nur zwei Farbkanäle, dann Abspaltung) --
    Rot-Grün-Blindheit: 5\,\% der Männer (Gen liegt auf dem X-Chromosom),
    hier hat die Mutation nicht stattgefunden
    (scheint kein großer Nachteil für das Überleben zu sein)
    
    \item
    \textbf{nachts}:
    geht das Farbempfinden verloren, die Wahrnehmung wird eher blau-empfindlich
    
    \item
    \textbf{Tierwelt}:
    Insekten/Vögel sehen auch UV-Licht
    (Rabe ist für andere Raben weiß, da viel UV-Licht reflektiert wird),
    Schmetterlinge mit 16 Farbkanälen
\end{itemize}
\linie
\pagebreak
\begin{itemize}
    \item
    \textbf{Farbkreis}:
    es gibt Farben, die nicht im Spektrum sind, sondern eine Mischung aus
    anderen Farben (Newton),
    Farbkreis entsteht durch Farbband, das am Rand verklebt wird,
    zur Mitte wird es heller, in der Mitte ist weiß,
    Farbmischung kann mittels Vektoraddition erfolgen,
    um Weiß zu erhalten, können alle Farben oder nur zwei Komplementärfarben
    (gegenüberliegend) gemischt werden --
    Komplementärfarben generieren die stärkste visuelle Spannung,
    wichtig: Warm-/Kalt-Kontrast (z.\,B. Wandfarbe),
    LED-Taschenlampen scheinen weiß, weil sie zwei Peaks bei
    Komplementärfarben haben
    
    \item
    \textbf{Metamerie}:
    Objekte haben unterschiedliche Farben, da sie Licht von verschiedenen
    Wellenlängen absorbieren und nur Licht von bestimmten Wellenlängen streuen,
    allerdings kann ein weißes Objekt auch rot erscheinen, wenn es nur
    mit rotem Licht bestrahlt wird, dies nennt sich Metamerie
    (gleicher Farbeindruck trotz unterschiedlicher spektraler
    Zusammensetzung) -- z.\,B. Einkaufen im Laden (Neonlampen) im Gegensatz zu
    Tageslicht
\end{itemize}
\linie
\begin{itemize}
    \item
    \textbf{subtraktive Farbmischung}:
    Grundfarben Cyan/Magenta/Gelb, z.\,B. Mischung von Cyan und Gelb:
    türkises Farbpigment absorbiert Rot (Komplementärfarbe von Cyan) und
    gelbes Farbpigment absorbiert Blau (Komplementärfarbe von Gelb), also
    bleibt Grün übrig, daher der Name
    (Farben werden aus dem Lichtstrahl entfernt) --
    Drucken/Malen
    
    \item
    \textbf{Drucker}: CMYK, K für Schwarz (key), zusätzlicher Kontrast,
    kleine Punkte zur Farberzeugung,
    Druckerverschwörung (yellow dots) ist wahr
    
    \item
    \textbf{warum mischt man nicht alles aus Grundfarben}:
    Farbpigmente absorbieren nicht nur Komplementärfarbe, sondern auch
    etwas mehr, d.\,h. zu viele Anteile werden absorbiert, besser:
    Verwendung von andersfarbigen Farbpigmenten
    (außerdem z.\,B. bei Drucker: zu teuer)
\end{itemize}
\linie
\begin{itemize}
    \item
    \textbf{Absorption}:
    Photonen regen Elektronen in höheres Energieniveau an, beim Zurückfallen
    kann neues Photon emittiert (Glas, daher
    langsame Ausbreitungsgeschwindigkeit) oder in andere Energieformen
    umgesetzt werden (Wärme, chemische Energie bei Fotosynthese oder
    Ladungstrennung bei Fotodioden), in jedem Fall wird Photon vernichtet,
    d.\,h. Farbe, Energie eines Photons beträgt $E = h \cdot \nu$ mit
    dem \name{Planck}schen Wirkungsquantuum $h$
    
    \item
    \textbf{Farbe von Atomen/Molekülen}:
    die meisten Atome absorbieren stark im IR- und UV-Bereich, z.\,B.
    Wasser ($\text{H}_2 \text{O}$) hat breite Resonanz im Roten, daher ist
    Wasser leicht bläulich, die Tiefe beeinflusst die Farbstärke
    
    \item
    \textbf{Edelsteine}:
    Farbe stark von Stoff (oder Kristallstruktur) abhängig, z.\,B.
    Ersetzen jedes 100. Atoms eines Diamanten durch ein Bor-Atom führt zu
    tiefem Blau, oder Ersetzen jedes 100. Aluminium-Atoms eines weißen Saphirs
    durch ein Chrom-Ion führt zu rotem Rubin
    
    \item
    \textbf{Blautopf nach Regen}:
    erscheint türkis wg. Sedimenten in Suspension, analog bei Stränden
    (Licht muss kleine Strecke durch Wasser laufen)
\end{itemize}
\linie
\begin{itemize}
    \item
    \textbf{psychologische Grundfarben}:
    Rot/Gelb/Grün/Blau,
    Grund liegt in der neuronalen Verschaltung in der Netzhaut
    
    \item
    \textbf{gelbe Sonnenbrillen}:
    bringt wahrscheinlich nichts für Kontrastverbesserung, aber
    bringt etwas bei hellen Objekten auf gelbem/bläulichem Hintergrund
    oder bei vielen kleinen blauen Strukturen,
    in jedem Fall ist der Ef"|fekt stark subjektiv,
    beachtet werden muss auch die psychologische Wirkung auf den Brillenträger
\end{itemize}
\linie
\pagebreak
\begin{itemize}
    \item
    \textbf{Farbkonstanz}:
    Objekte werden auch bei spektral veränderter Beleuchtung in der korrekten
    Farbe wahrgenommen (Gehirn führt Weißabgleich durch), z.\,B.
    ein Würfel mit farbigen, schattierten Flächen,
    oder Beleuchtung von Leinwand mit rotem Licht und Erzeugung eines
    Schattens kann dazu führen, dass Schatten grün oder blau erscheint
    (es muss allerdings noch kritisches weißes Licht vorhanden sein)
    
    \item
    \textbf{Retinex-Farbtheorie (Edwin \name{Land}, 1971)}:
    einzelne Farbkanäle werden getrennt betrachtet und jeder für sich normiert
    (z.\,B. in der Form $(R, G, B)$ hat Leinwand die Farbe $(50, 10, 10)$ und
    der Schatten $(10, 10, 10)$, anschließende Normierung auf
    $(1, 1, 1)$ bzw. $(0.2, 1, 1)$, d.\,h. Schatten erscheint farbig)
\end{itemize}
\linie
\begin{itemize}
    \item
    \textbf{Reflexion bei Metallen}:
    viele freie Elektronen vorhanden, schwingen mit elektromagnetischen
    Wellen mit, schwingende Ladungen strahlen selbst elektromagnetische
    Welle ab, Summe beider Wellen $0$ in Lichtrichtung, maximal entgegen
    Lichtrichtung, \\
    Elektronen können nur bis zur sog. Plasmafrequenz folgen, bei Gold/Kupfer
    liegt diese Frequenz im sichtbaren Spektralbereich, sodass blau z.\,B.
    kaum reflektiert wird, \\
    sehr dünne Metallfilme sind durchsichtig ab einer gewissen Grenzfrequenz
    (Nutzung bei Raumanzügen, Sonnenschutzfolien und Skibrillen)
\end{itemize}
\linie
\begin{itemize}
    \item
    \textbf{Chamäleon}:
    Haut mit Chromatophoren (Farbzellen mit Pigmenten) überdeckt,
    können von Muskel klein/groß gemacht werden, Musterzeugung zur
    Tarnung/Kommunikation
    
    \item
    \textbf{Theromochromismus}:
    Erhitzung bewirkt Änderung der Energieniveaus und somit Farbänderung,
    z.\,B. Stimmungsringe (mood rings) besitzen mehrere Farbschichten
    
    \item
    \textbf{Leukofarbstof"|fe}:
    Temperatur- führt zu Phasenänderung und damit zu Änderung des
    Absorptionsspektrums (Ladestandsanzeigen bei Batterien
    oder bedruckte Tassen)
    
    \item
    \textbf{Fotochromismus}:
    Licht bewirkt Farbänderung (z.\,B. bedrucktes T-Shirt mit Farbänderung)
\end{itemize}
\linie
\begin{itemize}
    \item
    \textbf{Pulver, Schaum, Schnee}:
    weiß, da sie wenig Licht absorbieren (sehr kleine Teilchen) und
    viele Oberflächen aufweisen (sehr viele Teilchen), geringe Absorption wegen
    des kleines Volumens ist im Vergleich zur massenhaften Reflektion an
    den Grenzflächen vernachlässigbar, sodass in alle Richtungen viel
    Licht reflektiert wird
\end{itemize}
\linie
\begin{itemize}
    \item
    \textbf{spiegelnder Monitor}:
    im Vorteil, wenn alle Lichtquellen so positioniert sind, dass Licht nicht
    in Betrachterrichtung reflektiert wird,
    bei nicht-spiegelndem Monitor ist die Oberfläche aufgeraut,
    im Vorteil, wenn Licht direkt zum Betrachter reflektiert werden würde,
    da die Lichtintensität sich aufteilt
    
    \item
    \textbf{Usambara-Ef"|fekt}:
    bläuliches Cola-Glas wirkt am Rand deutlich violett
    (unterschiedliche Materialdicke am Rand im Vergleich zur Mitte)
\end{itemize}

\pagebreak
