\section{%
    Brechung%
}

\begin{itemize}
    \item
    \textbf{Brechung}:
    Änderung der Ausbreitungsrichtung von Licht
    beim Durchgang von Licht durch eine Grenfläche zwischen zwei
    (dielektrischen) Medien

    \item
    \textbf{\name{Fermat}sches Prinzip}:
    Licht verläuft zwischen zwei Punkten so, dass die Reisezeit für ein
    Lichtteilchen extremal (maximal oder minimal) wird
    (kompliziert: bei kontinuierlicher Brechungszahlverteilung beträgt die
    Reisezeit $\delta t = \frac{1}{c} \int_A^B n(x, y, z) \ds$)

    \item
    \textbf{geradlinige Ausbreitung von Licht}:
    kann einfach durch Fermatsches Prinzip erklärt werden
    (kompliziert wird es erst, wenn Start- und Zielpunkt des Lichts in
    unterschiedlichen Medien liegen, d.\,h. Medien mit unterschiedlichem $c$)

    \item
    \textbf{Brechzahl}:
    im Vakuum ist $c_0 = \SI{3e8}{\meter\per\second}$, in Medien
    wird die Ausbreitungsgeschwindigkeit $c$ durch die Brechzahl $n$
    beschrieben: $c = \frac{c_0}{n}$

    \item
    \textbf{Brechungsgesetz}:
    $n \sin \phi = n' \sin \phi'$, wobei $\phi$ und $\varphi'$ den Winkel
    zwischen Lichtstrahl und Senkrechte zur Grenzfläche durch
    Übergangspunkt bezeichnen
    (kann mit Fermatschem Prinzip hergeleitet werden)

    \item
    \textbf{Totalreflexion}:
    im Brechungsgesetz geht man von $n' = 1$ für Luft und
    $\phi' = 90^\circ$ als Einfallswinkel aus,
    man erhält die Formel $\sin \phi = \frac{1}{n}$
    für den Ausfallswinkel (z.\,B. $n = 1.5$, $\phi = 42^\circ$ für Glas),
    somit gilt für größere Einfallswinkel als $42^\circ$ in Glas,
    dass der Ausfallswinkel größer als $90^\circ$ wäre (nicht möglich), d.\,h.
    Totalreflexion tritt ein (mit dem Reflexionsgesetz
    Einfallswinkel = Ausfallswinkel), 100\,\% Reflexion
    (im Gegensatz zu metallischem Spiegel), keine Verlauste,
    Anwendung z.\,B. Glasfaser

    \item
    \textbf{Prisma}:
    die Brechzahl ist nicht nur vom Medium abhängig, sondern auch von der
    Frequenz des Lichts (auch von Temperatur, Dichte usw.),
    daher ergeben sich beim Prisma je nach Farbanteil unterschiedliche
    Brechzahlen und somit unterschiedliche Ablenkungswinkel, sodass
    eine Farbaufspaltung erfolgt
\end{itemize}
\linie
\begin{itemize}
    \item
    \textbf{Bildanhebung}:
    Fisch im Wasser von oben betrachtet liegt aufgrund Brechung
    tiefer als er scheint (daher beim Harpunieren unter den Fisch zielen),
    klare Gewässer erscheinen weniger tief als sie in Wirklichkeit sind,
    funktioniert auch bei vertikalen Grenzflächen,
    ähnlich findet Brechung von Sternenlicht an Grenzfläche
    Weltraum/Atmosphäre statt

    \item
    \textbf{Sonne am Horizont}:
    Licht der Sonne wird durch die Atmosphäre gebrochen und läuft daher
    auf einer gebogenen Bahn, z.\,B. Sonne beim Sonnenuntergang
    noch sichtbar, obwohl sie bereits schon unterhalb des Horizonts steht,
    bei Sonnenaufgang und -untergang ist der Ef"|fekt besonders stark,
    da das Licht durch viel Erdatmosphäre läuft,
    daher erscheint die Sonne abgeflacht
    (Licht vom unteren Rand muss durch mehr Atmosphäre laufen)
\end{itemize}
\linie
\begin{itemize}
    \item
    \textbf{Linse}:
    funktioniert mit Brechung

    \item
    \textbf{3D-Sehen}:
    basiert auf zwei Augen,
    Augen müssen die Lichtstrahlen empfangen, die für sie bestimmt sind
\end{itemize}
\linie
\begin{itemize}
    \item
    \textbf{Sternfunkeln}:
    Licht durchwandert die Atmosphäre, die nicht homogen ist,
    sondern zeitliche und räumliche Dichte- und somit auch
    Brechzahlschwankungen aufweist, daher wird Stern mal heller und mal dunkler

    \item
    \textbf{Flimmern bei heißer Luft}:
    ähnlich verhält sich das Flimmern bei heißer Luft
    (hohe Brechzahlschwankungen)
\end{itemize}
\linie
\pagebreak
\begin{itemize}
    \item
    \textbf{Retroreflexion}:
    um Schatten des eigenen Kopfes (oder Kamera) auf taubenetzter Wiese
    gibt es eine Aufhellung ("`Heiligenschein"'),
    Grund liegt in Retroreflexion (paralleles Sonnenlicht wird durch viele
    kleine Tautröpfchen fokussiert, trifft nahe Brennpunkt auf Grashalm, wird
    reflektiert und geht denselben Weg zurück)

    \item
    \textbf{Katzenaugen}:
    hinter der Netzhaut liegt bei Katzen eine reflektierende Schicht, sodass
    das Licht nochmals die Sinneszellen passiert und somit die Empfindlichkeit
    fast verdoppelt wird, restliches Licht geht wieder durch die Linse,
    wird durch sie parallelisiert und kann von außen gesehen werden

    \item
    \textbf{rote Augen bei Menschen}:
    Blitz nahe bei Objektiv, große Pupillen (Dunkelheit)

    \item
    \textbf{Aureole}:
    gekräuselte Wasseroberflächen mit kleinen Wellen, Strahlenkranz um eigenen
    Schatten sichtbar (Retroreflexion des Sonnenlichts an Streuern im Wasser,
    zurücklaufendes Licht ist zwar parallel, aufgrund der Perspektive laufen
    sie im zweidimensionalen Bild jedoch auf einen Fluchtpunkt, den
    Antisolarpunkt, zu)
\end{itemize}
\linie
\begin{itemize}
    \item
    \textbf{Mirage Inferior/Superior}:
    Luftspiegelungen,
    gespiegeltes Bild liegt unterhalb/ober"-halb des Objekts

    \item
    \textbf{Mirage Inferior (highway mirage, Wüstenmirage)}:
    warme Luft über Straßenbelag hat geringere Dichte und somit geringere
    Brechzahl, Totalreflexion, sieht aus wie nasse Fläche auf der Straße

    \item
    \textbf{Mirage Superior}:
    warme Luftschicht oberhalb einer Schicht mit kalter Temperatur,
    kommt häufig über kalten Oberflächen vor, daher vor allem in polaren
    Regionen (das erklärt den Namen Eismirage)

    \item
    \textbf{Hillingar-Ef"|fekt}:
    Objekte unterhalb des Horizonts können sichtbar werden

    \item
    \textbf{Fata Morgana}:
    wie Mirage Superior, bloß mit komplizierteren Temperaturverteilungen,
    führt zu komplexen, verzerrten Spiegelbildern wie aus dem Boden ragende
    Türme (towering, castles in the sky)

    \item
    \textbf{laterale Mirage}:
    wie Mirage Inferior, nur mit senkrechter erhitzter Schicht
    (bei konstantem Temperaturgradient ergibt sich ein parabolischer Lichtweg)
\end{itemize}
\linie
\begin{itemize}
    \item
    \textbf{grünes Leuchten}:
    grüner Blitz am Ende des Sonnenuntergangs,
    kurzwelligere Anteile des Sonnenlichts werden aufgrund der
    Wellenlängenabhängigkeit der Brechung stärker angehoben, jedoch
    wird Licht stärker weggestreut und das Auge hat für Blau eine geringe
    Empfindlichkeit, daher ist grünes Licht dominant
    (das auch stärker angehoben wird als bspw. rotes und gelbes Licht),
    zusätzlich müssen allerdings noch Luftspiegelungen in der Luft den
    Ef"|fekt vergrößern, da eigentlich vom Auge nicht auflösbar
\end{itemize}
\linie
\pagebreak
\begin{itemize}
    \item
    \textbf{Regenbogen}:
    Lichtstrahl trifft auf Wassertropfen,
    wird gebrochen, trifft auf die Rückseite,
    wird reflektiert (keine Totalreflexion) und beim Austritt
    erneut gebrochen, blauer Strahl wird um $40^\circ$ und roter Strahl
    um $42^\circ$ abgelenkt,
    geometrische Bogenform hat Kreismittelpunkt im Antisolarpunkt,
    roter Rand außen, blauer Rand innen,
    nicht im Sommer sichtbar, wenn Sonne höher als $42^\circ$ steht
    (außer von großer Höhe)

    \item
    \textbf{roter Regenbogen}:
    bei Sonnenuntergang, da nur rotes Licht vorhanden

    \item
    \textbf{Sekundärbogen}:
    Licht wird zweimal reflektiert,
    Strahl um $51^\circ$ (rotes Licht) abgelenkt, schwächerer Sekundärbogen hat
    Rot innen und Blau außen (nur 43\,\% der Intensität des Hauptbogens)

    \item
    \textbf{Alexanders Band}:
    dunkles Band zwischen Haupt- und Nebenbogen
    (Licht wird beim Hauptbogen in den Innenbereich gelenkt und beim
    Sekundärbogen in den Außenbereich)

    \item
    \textbf{Mondbogen}:
    es taugen auch andere Lichtquellen,
    aber Mondbogen ist sehr selten

    \item
    \textbf{Supernumeraries}:
    Nebenbögen aufgrund der Beugung

    \item
    \textbf{Reflexionsbögen}: zwei Möglichkeiten \\
    1. Regenbogen wird von spiegelnder Flächer (Wasser) reflektiert,
    erscheint an anderer Stelle wie andere gespiegelte Objekte, da
    Regenbogen kein Objekt (nur Phänomen) ist, \\
    2. Sonnenlicht wird reflektiert, bevor es auf den Wassertropfen trifft

    \item
    \textbf{Nebelbogen}:
    Regenbogen in einer Nebelwand, weitgehend farblos, da
    Nebeltröpfchen viel kleiner als Regentropfen sind,
    Beugungsef"|fekte viel stärker, starke Farbaufspaltung,
    aber Stärke von Tröpfchengröße abhängig und im Nebel befinden sich
    viele Tröpfchen unterschiedlicher Größe,
    in der Summe daher Mischung vieler Farben und daher weißer Bogen
\end{itemize}

\pagebreak
