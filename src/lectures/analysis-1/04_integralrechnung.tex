\chapter{%
    Zur Integralrechnung von Funktionen einer Variablen%
}

\section{%
    Das \textsc{Riemann}-Integral%
}

Gegeben sei eine Funktion $f: [a,b] \rightarrow \mathbb{R}$, wobei $a \le b$.

Durch $a = x_0 \le x_1 \le \cdots \le x_{n-1} \le x_n = b$ wird das Intervall
$[a,b]$ zerlegt, die Menge $\delta = \{x_k\}_{k=0}^n = \{x_0, \ldots, x_n\}$
heißt \textbf{Zerlegung} von $[a,b]$.

$\Delta x_k = x_k - x_{k-1}$ ist die Länge, $\Delta_k = [x_{k-1}, x_k]$ das
Intervall des $k$-ten Teilstücks.\\
$\lambda(\delta) = \max_{k=1,\ldots,n} \Delta x_k$ bezeichnet den
\textbf{Rang der Zerlegung} (Länge des längsten Teilstücks).

Für jedes $\Delta_k$ kann man eine Stützstelle
$\xi_k \in \Delta_k = [x_{k-1}, x_k]$ wählen ($k = 1, \ldots, n$). \\
$\xi = \{\xi_k\}_{k=1}^n = \{\xi_1, \ldots, \xi_n\}$ bezeichnet einen
\textbf{Satz von Stützstellen} für die Zerlegung $\delta$.

$\mathfrak{S}(f, \delta, \xi) = \sum_{k=1}^n f(\xi_k) \Delta x_k$ heißt dann
\textbf{\textsc{Riemann}-Summe} von $f$ bzgl. der Zerlegung $\delta$ und dem
Satz von Stützstellen $\xi$.

\linie

$f: [a,b] \rightarrow \mathbb{R}$ heißt \textbf{\textsc{Riemann}-integrierbar},
falls es ein $I \in \mathbb{R}$ gibt mit
$I = \lim_{\lambda(\delta) \to 0} \mathfrak{S}(f, \delta, \xi)$ \\
$\overset{\text{def.}}{\Leftrightarrow}\;
\forall_{\varepsilon > 0}
\exists_{\eta > 0}
\forall_{\text{Zerlegungen } \delta,\; \lambda(\delta) < \eta}
\forall_{\text{Stützstellen } \xi \text{ zu } \delta}\;
|I - \mathfrak{S}(f, \delta, \xi)| < \varepsilon$ \\
$\Leftrightarrow\;
\forall_{\varepsilon > 0}
\exists_{\eta > 0}
\forall_{\delta', \delta'',\; \lambda(\delta') < \eta,\;
\lambda(\delta'') < \eta}
\forall_{\xi' = \xi'(\delta'),\; \xi'' = \xi''(\delta'')}\;
|\mathfrak{S}(f, \delta', \xi') - \mathfrak{S}(f, \delta'', \xi'')|
< \varepsilon$.

\textbf{alternative Definition}:
Eine Folge von Zerlegungen $\{\delta_k\}_{k \in \mathbb{N}}$ heißt
\emph{ausgezeichnet}, falls \\
$\lambda(\delta) \to 0$ für $k \to \infty$.
Sei $\xi = \xi(\delta_k)$ ein beliebiger Satz von Stützstellen zu
$\delta_k$. \\
Falls $\mathfrak{S}(f, \delta_k, \xi(\delta_k))$ für $k \to \infty$
immer einen Grenzwert $I$ besitzt und dieser Grenzwert unabhängig von der Wahl
der $\delta_k$ und $\xi(\delta_k)$ ist, so nennt man $f$ Riemann-integrierbar.

In jedem Fall schreibt man dann $\int_a^b f(x)\dx := I$.

Es gilt $\int_a^a f(x)\dx = 0$
\quad und \quad $\int_b^a f(x)\dx := -\int_a^b f(x)\dx$ für $a < b$ \\
(das Riemann-Integral ist \emph{gerichtet}).

\textbf{komplexwertige Funktionen}:
Eine Funktion $f: [a,b] \rightarrow \mathbb{C}$ ist Riemann-integrierbar \\
$\overset{\text{def.}}{\Leftrightarrow}\;$ $\Re f, \Im f$ sind
Riemann-integrierbar, \quad
$\int_a^b f(x)\dx := \int_a^b \Re(f(x))\dx + i \cdot \int_a^b \Im(f(x))\dx$.

\textbf{vektorwertige Funktionen}:
Eine Funktion $f: [a,b] \rightarrow \mathbb{K}^m$ ist Riemann-integrierbar \\
$\overset{\text{def.}}{\Leftrightarrow}\;$ $\pi_j(f(x))$ ist
Riemann-integrierbar für $j = 1, \ldots, m$, \quad
$\pi_j\left(\int_a^b f(x)\dx\right) := \int_a^b \pi_j(f(x))\dx$.

$R[a,b]$ ist die \textbf{Menge der Riemann-integrierbaren Funktionen}
$f: [a,b] \rightarrow \mathbb{R}$.

\textbf{R.-integr. Funktionen sind beschränkt}:
Sei $f \in R[a,b]$. \qquad
Dann ist $f$ beschränkt.

\linie

\textbf{Stetigkeitsmodul}:
Seien $f: [a,b] \rightarrow \mathbb{R}$ sowie $E \subset [a,b]$. \\
$\omega(f, E) = \sup_{x', x'' \in E} |f(x') - f(x'')|$
heißt Stetigkeitsmodul von $f$ auf $E$.

\textbf{Satz}:
Seien $f: [a,b] \rightarrow \mathbb{R}$ beschränkt und
$\lim_{\lambda(\delta) \to 0} \sum_{k=1}^n \omega(f, \Delta_k) \Delta x_k = 0$,
d.\,h. \\
$\forall_{\varepsilon > 0} \exists_{\eta > 0}
\forall_{\delta,\; \lambda(\delta) < \eta}\;
\sum_{k=1}^n \omega(f, \Delta_k) \Delta x_k < \varepsilon$. \qquad
Dann ist $f \in R[a,b]$.

\textbf{stetige Funktionen}:
Stetige Funktionen $f \in C([a,b])$ sind Riemann-integrierbar. \\
Ist eine Funktion bis auf endlich viele Punkte stetig, so ist sie
Riemann-integrierbar. \\
Verändert man eine Riemann-integrierbare Funktionen in nur einem Punkt,
so ist sie immer noch Riemann-integrierbar und das Integral ist dasselbe.

\linie

\pagebreak

\textbf{obere/untere \textsc{Darboux}-Summe}:
Seien $f: [a,b] \rightarrow \mathbb{R}$ beschränkt und \\
$a = x_0 < x_1 < \cdots < x_{n-1} < x_n = b$ mit
$\Delta_k = [x_{k-1}, x_k]$. \\
Außerdem sei $m_k = \inf_{x \in \Delta_k} f(x)$ und
$M_k = \sup_{x \in \Delta_k} f(x)$. \\
Dann heißt $s(f, \delta) = \sum_{k=1}^n m_k \Delta x_k$ untere bzw.
$S(f, \delta) = \sum_{k=1}^n M_k \Delta x_k$ obere Darboux-Summe. \\
Es gilt $s(f, \delta) \le \mathfrak{S}(f, \delta, \xi) \le S(f, \delta)$
für jeden Satz von Stützstellen $\xi$.

\textbf{Konvergenz der \textsc{Darboux}-Summe}:
Sei $f: [a,b] \rightarrow \mathbb{R}$ beschränkt.
Dann ist \\
$f \in R[a,b] \;\Leftrightarrow\;
\exists I = \lim_{\lambda(\delta) \to 0} s(f, \delta) =
\lim_{\lambda(\delta) \to 0} S(f, \delta)$ \qquad
$(I = \int_a^b f(x)\dx)$.

\textbf{Satz}:
Sei $f: [a,b] \rightarrow \mathbb{R}$ beschränkt.
Dann ist \\
$f \in R[a,b] \;\Leftrightarrow\;
\lim_{\lambda(\delta) \to 0} \sum_{k=1}^n \omega(f, \Delta_k) \Delta x_k = 0$.

\linie

\textbf{\textsc{Lebesgue}-Maß}:
Eine Menge $E \subset \mathbb{R}$ besitzt das Lebesgue-Maß $0$, falls \\
$\forall_{\varepsilon > 0}
\exists_{\{I_k(\varepsilon)\}_{k \in \mathbb{N}}}$ \quad
1) $E \subset \bigcup_{k \in \mathbb{N}} I_k(\varepsilon)$ \quad und \quad
2) $\sup_{n \in \mathbb{N}} \left(\sum_{k=1}^n |I_k(\varepsilon)|\right) <
\varepsilon$, \\
wobei $I_k = [\alpha_k, \beta_k] \subset \mathbb{R}$ mit $k \in \mathbb{N}$
und $|I_k| = \beta_k - \alpha_k$.

\textbf{fast überall}:
Eine Aussageform $H(x)$ ist fast überall wahr, falls es eine Menge
$E \subset [a,b]$ mit Lebesgue-Maß 0 gibt, sodass $H(x)$ wahr ist auf
$[a,b] \setminus E$.

\textbf{\textsc{Lebesgue}-Kriterium zur Riemann-Integrierbarkeit}:
Sei $f: [a,b] \rightarrow \mathbb{R}$ beschränkt. \\
Dann ist $f \in R[a,b] \;\Leftrightarrow\;$
$f$ ist fast überall auf $[a,b]$ stetig.

\textbf{monotone Funktionen}:
Beschränkte und monotone Funktionen sind Riemann-integrierbar.

\textbf{Struktur von $R[a,b]$}:
Seien $f, g: [a,b] \rightarrow \mathbb{R}$,
$\alpha \in \mathbb{R}$,
$f, g \in R[a,b]$ und
$[c,d] \subset [a,b]$. \\
Dann ist auch $f + g,\quad \alpha \cdot f,\quad |f|,\quad f \cdot g \in R[a,b]$
sowie $f|_{[c,d]} \in R[c,d]$.

\section{%
    Eigenschaften des \textsc{Riemann}-Integrals%
}

\textbf{Satz (Linearität)}:
Seien $f, g \in R[a,b]$ und $\alpha, \beta \in \mathbb{R}$. \\
Dann ist $\alpha f + \beta g \in R[a,b]$ und
$\int_a^b (\alpha f + \beta g)(x)\dx =
\alpha \int_a^b f(x)\dx + \beta \int_a^b g(x)\dx$.

\textbf{Satz (Additivität bzgl. Integrationsbereich)}: \\
$f \in R[a,b] \;\Leftrightarrow\;
f|_{[a,c]} \in R[a,c] \land f|_{[c,b]} \in R[c,b]$ \\
und $\int_a^b f(x)\dx = \int_a^c f(x)\dx + \int_c^b f(x)\dx$
für $c \in \left]a,b\right[$. \\
\emph{(Satz gilt mit $\int_b^a f(x)\dx = -\int_a^b f(x)\dx$ unabhängig von
$c \in \left]a,b\right[$!)}

\textbf{Satz (Monotonie des Riemann-Integrals)}:
Seien $f_1, f_2 \in R[a,b]$ mit $f_1(x) \le f_2(x)$ für alle $x \in [a,b]$,
wobei $a < b$. \qquad
Dann ist $\int_a^b f_1(x)\dx \le \int_a^b f_2(x)\dx$.

\textbf{Spezialfall}:
Sei $f: [a,b] \rightarrow \mathbb{R}$ mit $m \le f(x) \le M$ für alle
$x \in [a,b]$, wobei $a < b$. \\
Dann ist $m \cdot (b - a) \le \int_a^b f(x)\dx \le M \cdot (b - a)$.

\textbf{Spezialfall}:
$\left|\int_a^b f(x)\dx\right| \le (b - a) \cdot \sup_{x \in [a,b]} |f(x)|$

\textbf{Spezialfall}:
$\left|\int_a^b f(x)\dx\right| \le \int_a^b |f(x)|\dx$

\pagebreak

\section{%
    Die Formel von \textsc{Newton}-\textsc{Leibniz}%
}

\textbf{Satz von \textsc{Newton}-\textsc{Leibniz}}:
Seien $F: [a,b] \rightarrow \mathbb{R}$ stetig,
dif"|fb. in $\left]a,b\right[$ und \\
$\dot{F}(x) =$ {\footnotesize $\begin{cases}F'(x) & x \in \left]a,b\right[ \\
0 & x = a \lor x = b\end{cases}$} \;mit $\dot{F}(x) \in R[a,b]$. \qquad\qquad
Dann ist $\int_a^b \dot{F}(x)\dx = F(b) - F(a)$.

Der Satz lässt sich für Funktionen $F: [a,b] \rightarrow \mathbb{K}^n$
verallgemeinern (komponentenweise).

\textbf{Stammfunktion}:
$F: [a,b] \rightarrow \mathbb{K}^n$ ist eine Stammfunktion von
$f: [a,b] \rightarrow \mathbb{K}^n$, falls \\
$F$ stetig auf $[a,b]$, \qquad
$F$ dif"|fb. in $\left]a,b\right[$ \qquad und \qquad
$F'(x) = f(x)$ für alle $x \in \left]a,b\right[$.

Existiert zu $f$ eine Stammfunktion $F$, so ist diese bis auf eine Konstante
eindeutig bestimmt.

\textbf{Hauptsatz der Dif"|ferential- und Integralrechnung}:
Sei $f: [a,b] \rightarrow \mathbb{R}$ mit $f \in R[a,b]$ und es existiere
eine Stammfunktion $F$ zu $f$. \qquad\qquad
Dann ist $\int_a^b f(x)\dx = F(b) - F(a)$.

\linie

\textbf{Satz}:
Jede stetige Funktion $f: [a,b] \rightarrow \mathbb{R}$ besitzt eine
Stammfunktion $F$ der Form \\
$F(x) =$ {\footnotesize $\begin{cases}C & x = a \\
\int_a^x f(t)\dt + C & x \in \left]a,b\right]\end{cases}$}.

Allerdings besitzt nicht jede Funktion $f \in R[a,b]$ eine Stammfunktion!
Beispiele sind monotone Funktionen mit Sprungstellen.
Sie können nach \textsc{Darboux} keine Ableitung einer anderen
Funktion darstellen.
Auch bedeutet die Existenz nicht, dass man sie
explizit hinschreiben kann.

\linie

\textbf{partielle Integration}:
Seien $f, g: [a,b] \rightarrow \mathbb{R}$ stetig auf $[a,b]$,
dif"|fb. in $\left]a,b\right[$ sowie \\
$f'g, fg' \in R[a,b]$. \qquad\qquad
Dann ist $\int_a^b f'g\dx = fg|_a^b - \int_a^b fg'\dx$.

\section{%
    Zur Integration rationaler Funktionen%
}

Wir betrachten rationale Funktionen
$R(x) =$ {\large $\frac{P_m(x)}{Q_n(x)}$} $ = $
{\large $\frac{a_m x^m \;+\; \cdots \;+\; a_1 x \;+\; a_0}
{b_m b^m \;+\; \cdots \;+\; b_1 x \;+\; b_0}$} mit $a_m, b_n \not= 0$.

\textbf{Spezialfälle}: \\
$Q_n(x) = 1$: \qquad $\int P_m(x)\dx = $ {\large $\frac{a_m}{m + 1}$}
$x^{m+1} + \cdots + $ {\large $\frac{a_1}{2}$} $x^2 + a_0 x + C$ \\
$P_m(x) = 1$, $Q_n(x) = (x - a)^n$: \qquad
$\int$ {\large $\frac{1}{(x - a)^n}$} $ = $
{\footnotesize $\begin{cases}\ln |x - a| + C & \quad n = 1 \\
\frac{(x - a)^{1-n}}{1 - n} + C & \quad n \ge 2 \end{cases}$}

\textbf{Polynomdivision}:
Seien $P_m(x)$ und $Q_n(x)$ zwei Polynome mit $m \ge n \ge 1$.
Dann existieren eindeutig bestimmte Polynome $S_{m-n}(x)$ und
$T_\ell(x)$ mit $\ell < n$, sodass
{\large $\frac{P_m(x)}{Q_n(x)}$}
$= S_{m-n}(x) \;+$ {\large $\frac{T_\ell(x)}{Q_n(x)}$}. \\
$S_{m-n}(x)$ und $T_\ell(x)$ kann man durch \emph{Polynomdivision} bestimmen.

\textbf{Satz}:
Seien $P_m(x)$, $Q_n(x)$ Polynome mit $m < n$,
$Q_n(x) = \prod_{i=1}^\ell (x - a_i)^{\kappa_i}$ und
$\sum_{i=1}^\ell \kappa_i = n$. \\
Dann gibt es eindeutig bestimmte Koef"|fizienten $A_{ir}$, sodass
{\large $\frac{P_m(x)}{Q_n(x)}$} $= \sum_{i=1}^\ell \sum_{r=1}^{\kappa_i}$
{\large $\frac{A_{ir}}{(x - a_i)^r}$}.

\linie

\textbf{Bestimmung der Koef"|fizienten}:
$\frac{x^2 + 1}{x(x+1)(x-1)} = \frac{A_{11}}{x} + \frac{A_{21}}{x + 1} +
\frac{A_{31}}{x - 1}$
\begin{itemize}
   \item \emph{Ausmultiplizieren und Koef"|fizientenvergleich}: \\
   $\Rightarrow\; x^2 + 0x + 1 = A_{11}(x + 1)(x - 1) + A_{21}x(x - 1) +
   A_{31}x(x + 1)$ \\
   $\Leftrightarrow\; x^2 + 0x + 1 = (A_{11} + A_{21} + A_{31})x^2 +
   (-A_{21} + A_{31})x + (-A_{11}) \cdot 1$, \qquad
   LGS lösen

   \item \emph{Hand auf"|legen}:
   $x_1 = 0$, $x_2 = -1$, $x_3 = 1$,
   $A_{11} = \frac{x_1^2 + 1}{(x_1 + 1)(x_1 - 1)}$,
   $A_{21} = \frac{x_2^2 + 1}{x_2 (x_2 - 1)}$,
   $A_{31} = \frac{x_3^2 + 1}{x_3 (x_3 + 1)}$ \\
   Man setzt in die linke Seite immer eine Nullstelle ein, während
   man den zur Nullstelle zugehörigen Faktor im Nenner "`zudeckt"'.
   Nachteil: Bei mehrfachen Nullstellen kann man nur den Koef"|fizienten mit
   dem höchsten Exponenten ermitteln.
   Empfohlen wird eine gemischte Anwendung beider Methoden mit
   "`Hand auf"|legen"' zuerst und dann LGS lösen.
\end{itemize}

\section{%
    Die Mittelwertsätze der Integralrechnung%
}

\textbf{1. Mittelwertsatz der Integralrechnung}: \\
Seien $f, g: [a,b] \rightarrow \mathbb{R}$ stetig und $g(x) \ge 0$
für $x \in [a,b]$. \\
Dann gibt es ein $\xi \in [a,b]$, sodass
$\int_a^b f(x)g(x)\dx = f(\xi) \cdot \int_a^b g(x)\dx$.

\textbf{Spezialfall ($g(x) = 1$)}:
Sei $f: [a,b] \rightarrow \mathbb{R}$ stetig. \\
Dann gibt es ein $\xi \in [a,b]$, sodass
$\int_a^b f(x)\dx = f(\xi) \cdot (b-a)$.

\linie

\emph{Lemma}: Sei $g \in R[a,b]$.
Dann ist $G: [a,b] \rightarrow \mathbb{R}$, $G(x) = \int_a^\xi g(x)\dx$ stetig.

\textbf{2. Mittelwertsatz der Integralrechnung}: \\
Seien $f: [a,b] \rightarrow \mathbb{R}$ monoton fallend, $f \ge 0$
und $g \in R[a,b]$. \\
Dann gibt es ein $\xi \in [a,b]$, sodass
$\int_a^b f(x)g(x)\dx = f(a) \cdot \int_a^\xi g(x)\dx$.

\textbf{Spezialfall ($g(x) = 1$)}:
Sei $f: [a,b] \rightarrow \mathbb{R}$ monoton fallend und $f \ge 0$. \\
Dann gibt es ein $\xi \in [a,b]$, sodass
$\int_a^b f(x)\dx = f(a) \cdot (\xi - a)$.

Analog lässt sich der Satz für $f\!\!\uparrow$, $f \ge 0$ und $g \in R[a,b]$
formulieren \\
($\exists_{\xi \in [a,b]}\; \int_a^b f(x)g(x)\dx =
f(b) \cdot \int_\xi^b g(x)\dx$).

\linie

\textbf{Satz (Erweiterung des 2. MWS)}:
Sei $f$ monoton und beschränkt sowie $g \in R[a,b]$. \\
Dann gibt es ein $\xi \in [a,b]$, sodass
$\int_a^b f(x)g(x) = f(a) \int_a^\xi g(x)\dx + f(b) \int_\xi^b g(x)\dx$.

Die Mittelwertsätze gelten i.\,A. nicht für komplex- oder vektorwertige
Funktionen (beim komponentenweisen Anwenden können die $\xi$ unterschiedlich
sein).

\section{%
    Zur Substitution der Integrationsvariablen%
}

\textbf{Satz}:
Seien $f: [a,b] \rightarrow \mathbb{K}^n$ und
$\psi: [\alpha,\beta] \rightarrow [a,b]$ stetig, $\psi$ dif"|fb. in
$\left]\alpha,\beta\right[$ und $\psi'$ stetig in $\left]\alpha,\beta\right[$,
wobei $\psi(\alpha) = a$ und $\psi(\beta) = b$.
Außerdem existieren die Grenzwerte
$\lim_{t \to \alpha} \psi'(t) \in \mathbb{R}$ und
$\lim_{t \to \beta} \psi'(t) \in \mathbb{R}$,
d.\,h. $\psi'$ lässt sich in den Randpunkten stetig fortsetzen. \\
Dann ist $\int_a^b f(x)\dx = \int_\alpha^\beta f(\psi(t)) \psi'(t)\dt$.

\section{%
    Das Restglied in der Formel von \textsc{Taylor}%
}

\textbf{Formel von \textsc{Taylor} (Wiederholung)}:
Sei $f: X \subset \mathbb{K} \rightarrow \mathbb{K}^n$ mit $X$ of"|fen,
wobei $\overline{x_0 x} \in X$ ($x = x_0 + h$) und $f$ in $x_0$ $m$-fach
dif"|fb. \\
Dann ist $f(x_0 + h) = f(x_0) + \sum_{k=1}^m$
{\large $\frac{f^{(k)}(x_0)}{k!}$}$h^k +
r_m(x_0, h)$ mit $r_m(x_0, h) = o(h^m)$ für $h \to 0$.

\textbf{Satz}: Sei $f$ zusätzlich in allen Punkten von $\overline{x_0 x}$
$(m+1)$-fach stetig dif"|fb. \\
Dann ist $r_m(x_0, h) =$ {\large $\frac{h^{m+1}}{m!}$} $\cdot
\int_0^1 f^{(m+1)}(x_0 + th) (1 - t)^m \dt$.

\textbf{Folgerung}:
$\Vert r_m(x_0, h) \Vert \le$ {\large $\frac{|h|^{m+1}}{(m + 1)!}$} $\cdot
\sup_{y \in \overline{x_0 x}} \Vert f^{(m+1)}(y) \Vert$

\textbf{Spezialfall}:
Für Funktionen $f: X \subset \mathbb{R} \rightarrow \mathbb{R}$ gilt
$r_m(x_0, h) =$ {\large $\frac{f^{(m+1)}(y)}{(m + 1)!}$}$h^{m+1}$
für einen bestimmten Punkt $y \in \overline{x_0 x}$.

\pagebreak

\section{%
    Interpolationsformel von \textsc{Lagrange}%
}

Gegeben sei eine Fkt. $f: [a,b] \rightarrow \mathbb{R}$ und eine Zerlegung
$a' = x_0 < x_1 < \cdots < x_{n-1} < x_n = b'$ mit $a < a' < b' < b$.
Gesucht wird ein Polynom $P_n(x)$ mit $P(x_k) = f(x_k)$ für $k = 0, \ldots, n$,
wobei $\deg P_n \le n$.

Eine Lösung existiert in der Form $P_n(x) = \sum_{k=0}^n f(x_k)q_k(x)$
mit $q_k(x_\ell) = 0$ für $\ell \not= k$ und
$q_k(x_\ell) = 1$ für $\ell = k$. \\
Die $q_k$ sind Polynome vom Grad $\le n$: \quad
$q_k(x) =$
{\large $\frac{(x - x_0) \cdots (x - x_{k-1}) (x - x_{k+1}) \cdots (x - x_n)}
{(x_k - x_0) \cdots (x_k - x_{k-1}) (x_k - x_{k+1}) \cdots (x_k - x_n)}$}.

\textbf{Satz (Fehlerabschätzung)}:
Sei $f$ $(n+1)$-mal stetig dif"|fb. \\
Dann gilt $\forall_{x \in [a',b']} \exists_{\eta_x \in [a',b']}\;
f(x) - P_n(x) =$ {\large $\frac{f^{(n+1)}(\eta_x)}{(n + 1)!}$}
$(x - x_0) \cdots (x - x_n)$.

\section{%
    Anwendungen der Dif"|ferential- und Integralrechnung%
}

\subsection{%
    Länge und Krümmung einer Kurve%
}

Seien $\varphi: [a,b] \rightarrow \mathbb{R}^n$ und
$\Gamma_\varphi = \varphi([a,b])$.

\textbf{einfache Kurve}: Sei $\varphi$ stetig und injektiv.
Dann erzeugt $\varphi$ die \emph{einfache Kurve} $\Gamma_\varphi$.

\textbf{geschlossene Kurve}: Sei $\varphi$ stetig,
$\varphi|_{\left[a,b\right[}$ injektiv und $\varphi(a) = \varphi(b)$. \\
Dann erzeugt $\varphi$ die \emph{geschlossene Kurve} $\Gamma_\varphi$.

\textbf{\textsc{Jordan}sche Kurve}: Einfache/geschlossene Kurven
werden \emph{Jordansche Kurven} genannt.

\textbf{\textsc{Jordan}sche Kurve der Klasse $C^p$}: $\varphi$ erzeuge eine
Jordansche Kurve $\Gamma_\varphi$. \\
Ist zusätzlich $\varphi$ $p$-fach stetig dif"|fb., $\varphi^{(k)}$ für
$k = 0, \ldots, p$ stetig auf $[a,b]$ fortsetzbar sowie $\varphi'(t) \not= 0$
für alle $t \in [a,b]$, so erzeugt $\varphi$ eine
\emph{Jordansche Kurve der Klasse $C^p$}.

\linie

$\varphi: [a,b] \rightarrow \mathbb{R}^n$ erzeuge die Jordansche Kurve
$\Gamma_\varphi$.
Für jede Zerlegung $\delta = \{x_k\}_{k=0}^m$ von $[a,b]$ kann man die
\textbf{Länge $\ell^\delta$ des zugehörigen Polygonzugs} definieren als
$\ell^\delta = \sum_{k=1}^m \Vert \varphi(x_k) - \varphi(x_{k-1}) \Vert$. \\
Für eine Zerlegung $\delta'$ mit $\delta \subset \delta'$ gilt
$\ell^\delta \le \ell^{\delta'}$ (aufgrund Dreiecksungleichung).

Die Kurve $\Gamma_\varphi$ ist \textbf{rektifizierbar}, falls
$L(\Gamma_\varphi) = \sup_\delta \ell^\delta$ endlich ist. \\
$L(\Gamma_\varphi) = \sup_\delta \ell^\delta$ heißt dann die
\textbf{Bogenlänge} der Kurve $\Gamma_\varphi$.

\textbf{Satz}: $\varphi: [a,b] \rightarrow \mathbb{R}^n$ erzeuge die
Jordansche Kurve $\Gamma_\varphi$ der Klasse $C^1$. \\
Dann ist $\Gamma_\varphi$ rektifizierbar und
$L(\Gamma_\varphi) = \int_a^b \Vert \varphi'(t) \Vert \dt$.

\linie

\textbf{Kanon. Parametrisierung}:
$\varphi: [a,b] \rightarrow \mathbb{R}^n$ erzeuge die
Jordansche Kurve $\Gamma_\varphi$ der Klasse $C^1$. \\
Sei $S: [a,b] \rightarrow [0, L(\Gamma_\varphi)]$ mit
$S(t) = L(\Gamma_t) = L(\varphi([a,t])) =
\int_a^t \Vert \varphi'(\tau) \Vert d \tau$. \\
Es gilt $S'(t) = \Vert \varphi'(t) \Vert > 0$ und daher
$S\!\!\upuparrows$, $S$ stetig. \\
Also ist $S$ bijektiv mit der stetigen Umkehrfunktion
$S^{-1}: [0, L(\Gamma_\varphi)] \rightarrow [a,b]$. \\
$r: [0, L(\Gamma_\varphi)] \rightarrow \mathbb{R}^n$,
$r(s) = \varphi(S^{-1}(s))$ ist dann eine neue Parametrisierung und wird
\textbf{kanonische Parametrisierung} genannt.
Das Kurvenstück zu $r|_{[0,s]}$ besitzt die Bogenlänge $s$. \\
$r$ ist stetig dif"|fb. und $\Gamma_r$ ist eine Jordansche Kurve der Klasse
$C^1$. \\
$\tau: \left]0, L(\Gamma_\varphi)\right[ \rightarrow \mathbb{R}^n$,
$\tau(s) = r'(s) =$ {\large $\frac{\varphi'(t)}{\Vert \varphi'(t) \Vert}$}
ist der \textbf{Tangentialvektor} im Punkt $s$ (Länge $1$).

\linie

\textbf{Krümmung}: Gilt zusätzlich $\varphi \in C^2$, so ist
$\kappa(s) = \tau'(s)$ der \textbf{Krümmungsvektor}, \\
$K(s) = \Vert \kappa(s) \Vert$ die \textbf{Krümmung} und
$\rho(s) =$ {\large $\frac{1}{K(s)}$} der \textbf{Krümmungsradius}.

\textbf{Lemma}: $\kappa(s) \;\bot\; \tau(s)$

\pagebreak

\textbf{Krümmungsvektor}:
$\kappa(s) =$ {\large $\frac{\varphi''(t) \cdot \Vert \varphi'(t) \Vert^2 -
\varphi'(t) \cdot \langle \varphi'(t), \varphi''(t) \rangle}
{\Vert \varphi'(t) \Vert^4}$} ist der Krümmungsvektor an $\Gamma_\varphi$
im Punkt $r(s) = \varphi(t)$ ($s = S(t)$).

\textbf{Krümmung einer Kurve im $\mathbb{R}^3$}:
$\kappa(s) =$ {\large $\frac{[\varphi'(t), [\varphi''(t), \varphi'(t)]]}
{\Vert \varphi'(t) \Vert^4}$}, \quad
$K(s) =$ {\large $\frac{\Vert [\varphi''(t), \varphi'(t)] \Vert}
{\Vert \varphi'(t) \Vert^3}$}

\subsection{%
    Flächen und Volumina%
}

Seien $X \subset \mathbb{R}^2$, $P', P''$ Vielecke mit
$P' \subset X \subset P''$ und $A(P')$ bzw. $A(P'')$ der Flächeninhalt
von $P'$ bzw. $P''$.
Dann heißen $S_\ast = \sup_{P' \subset X} A(P')$ bzw.
$S^\ast = \inf_{P'' \supset X} A(P'')$
\textbf{innerer bzw. äußerer Flächeninhalt} von $X$.

Eine Menge $X \subset \mathbb{R}^2$ heißt \textbf{quadrierbar}, falls
$S_\ast = S^\ast =: A(X)$.

\textbf{Lemma}: Sind $X_1, X_2 \subset \mathbb{R}^2$ quadrierbar mit
$X_1 \cap X_2 = \emptyset$, so ist auch $X_1 \cup X_2$ quadrierbar und
$A(X_1 \cup X_2) = A(X_1) + A(X_2)$.

\linie

\textbf{Fläche unter einer Kurve}:
Seien $f: [a,b] \rightarrow \left[0, +\infty\right[$ stetig und \\
$X = \{(x, y) \in \mathbb{R}^2 \;|\; x \in [a,b],\; 0 \le y \le f(x)\}$. \\
Dann ist $X$ quadrierbar und $A(X) = \int_a^b f(x)\dx$.

\textbf{Fläche mit Polarkoordinaten}:
Seien $f: [\alpha,\beta] \subset [0,2\pi] \rightarrow \left[0,+\infty\right[$
stetig, $(r, \varphi)$ Polarkoordinaten in $\mathbb{R}^2$ und
$X = \{(r, \varphi) \;|\;
\varphi \in [\alpha,\beta],\; 0 \le r \le f(\varphi)\}$. \\
Dann ist $X$ quadrierbar und
$A(X) = \frac{1}{2} \int_\alpha^\beta f^2(\varphi) d\varphi$.

\textbf{Fläche zwischen zwei Kurven}:
Seien $f_1, f_2: [a,b] \rightarrow \mathbb{R}$ stetig mit $f_1(x) \le f_2(x)$
für alle $x \in [a,b]$ und $X = \{(x,y) \in \mathbb{R}^2 \;|\; x \in [a,b],\;
f_1(x) \le y \le f_2(x)\}$. \\
Dann ist $X$ quadrierbar und $A(X) = \int_a^b (f_2(x) - f_1(x))\dx$.

Man schreibt auch $A(X) = \int_a^b (f_2(x) - f_1(x))\dx
= \int_a^b f_2(x)\dx + \int_b^a f_1(x)\dx = -\oint y\dx = \oint x\dy$
(für $\oint$ bzw. $-\oint$ kann man auch einen Pfeil gegen den bzw. im
Uhrzeigersinn schreiben).

\linie

\textbf{Volumen}:
beliebige Körper: $V := \int_a^b A(x)\dx$ ($A(x_0)$ ist die Querschnittsfläche
bei $x = x_0$) \\
Rotationskörper: $A(x) = \pi f^2(x)$, $V = \pi \int_a^b f^2(x)\dx$

\textbf{Oberfläche von Rotationskörpern}:
$F = 2\pi \int_a^b f(x) \sqrt{1 + (f'(x))^2}\dx$

\linie

\textbf{Schwerpunkt einer Kurve}:
Sei ein System von Massepunkten $\{x_i, y_i\}$ mit den Massen $m_i$ gegeben.
Dann liegt der Schwerpunkt $S$ bei
$x_S :=$ {\large $\frac{\sum m_i x_i}{\sum m_i} = \frac{M_x}{M}$} und
$y_S :=$ {\large $\frac{\sum m_i y_i}{\sum m_i} = \frac{M_y}{M}$}. \\
Überträgt man das auf eine Jordansche Kurve $\Gamma$ der Klasse $C^1$
(die Masse eines Kurvenstücks soll proportional
zu dessen Länge sein), wobei $r(s) = (x(s), y(s))$ die kanonische
Parametrisierung ist, so definiert man $x_S :=$
{\large $\frac{\int_0^{L(\Gamma)} x(s)\ds}{L(\Gamma)} = \frac{M_y}{M}$}
sowie $y_S :=$
{\large $\frac{\int_0^{L(\Gamma)} y(s)\ds}{L(\Gamma)} = \frac{M_x}{M}$}.

\textbf{1. \textsc{Guldin}sche Regel}:
$2\pi y_s \cdot L(\Gamma) = 2\pi \int_0^{L(\Gamma)} y\ds$, wobei
$2\pi y_s$ die Weglänge des Schwerpunkts bei Rotation um die $x$-Achse
und $2\pi \int_0^{L(\Gamma)} y\ds$ die Oberfläche des Rotationskörpers ist.

\linie

\textbf{Schwerpunkt einer Fläche}:
Sei $f: [a,b] \rightarrow \mathbb{R}$ mit $f \ge 0$ eine Funktion,
dann definiert man den Schwerpunkt $S$ mit
$x_S :=$ {\large $\frac{\int_a^b xf(x)\dx}{A(X)}$} und
$y_S :=$ {\large $\frac{\frac{1}{2} \int_a^b f^2(x)\dx}{A(X)}$}, wobei
$A(X) = \int_a^b f(x)\dx$ der Flächeninhalt des betrachteten Gebiets $X$
(Fläche zwischen der Kurve von $f$ und der $x$-Achse).

\textbf{2. \textsc{Guldin}sche Regel}:
$2\pi y_s \cdot A(X) = \pi \int_a^b f^2(x)\dx$, wobei
$2\pi y_s$ die Weglänge des Schwerpunkts bei Rotation um die $x$-Achse
und $\pi \int_a^b f^2(x)\dx$ das Volumen
des Rotationskörpers ist.

\section{%
    Interpolationsformeln und numerische Integration%
}

Seien $f: [a,b] \rightarrow \mathbb{R}$ und $\delta = \{x_k\}_{k=0}^n$
eine \emph{äquidistante Zerlegung}, also $x_0 = a$, $x_n = b$ und
$x_k = a + kh$ mit $h =$ {\large $\frac{b - a}{n}$}.
Wähle $\xi = \{\xi_k\}_{k=1}^n$ mit
$\xi_k =$ {\large $\frac{x_{k-1} + x_k}{2}$} als äquidistanten
Satz von Stützstellen.

\linie

\textbf{die Rechteckformel (stückweise Approximation mit $P_0$)}: \\
Auf jedem $\Delta_k$ wird $f$ durch das konstante Polynom
$P_0(x) = f(\xi_k)$ für $k = 1, \ldots, n$ approximiert. \\
Dann wird die Rechteckformel durch
$\int_a^b f(x)\dx = \sum_{k=1}^n \int_{x_{k-1}}^{x_k} f(x)\dx
\approx \sum_{k=1}^n \int_{x_{k-1}}^{x_k} P_0(x)\dx
= \sum_{k=1}^n f(\xi_k) h =$
{\large $\frac{b - a}{n}$} $\sum_{k=1}^n f(\xi_k) := I_R$ hergeleitet.

\textbf{Fehlerabschätzung}: Ist $f \in C^2([a,b])$, so ist
$\left|\int_a^b f(x)\dx - I_R\right| \le$
{\large $\frac{(b - a)^3}{24n^2}$}
$\max_{x \in [a,b]} |f''(x)|$, \\
d.\,h. der Fehler verhält sich wie $\mathcal{O}(\frac{1}{n^2})$ für
$n \to \infty$.

\linie

\textbf{die Trapezformel (stückweise Approximation mit $P_1$)}: \\
Auf jedem $\Delta_k$ wird $f$ durch das Lagrange-Polynom ersten Grades \\
$P_1(x) =$ {\large $\frac{x - x_k}{x_{k-1} - x_k}$} $f(x_{k-1}) \;+$
{\large $\frac{x - x_{k-1}}{x_k - x_{k-1}}$} $f(x_k)$ approximiert
($P_1$ geht durch $x_{k-1}$ und $x_k$). \\
Dann wird die Trapezformel durch
$\int_a^b f(x)\dx = \sum_{k=1}^n \int_{x_{k-1}}^{x_k} f(x)\dx
\approx \sum_{k=1}^n \int_{x_{k-1}}^{x_k} P_1(x)\dx
= \sum_{k=1}^n$ {\large $\frac{f(x_{k-1}) + f(x_k)}{2}$} $\cdot\; h
=$ {\large $\frac{b - a}{n}$} $\cdot\; \Big(${\large $\frac{f(a) + f(b)}{2}$}
$+\; f(x_1) + \cdots + f(x_{n-1})\Big) =: I_T$ hergeleitet.

\textbf{Fehlerabschätzung}: Ist $f \in C^2([a,b])$, so ist
$\left|\int_a^b f(x)\dx - I_T\right| \le$
{\large $\frac{(b - a)^3}{12n^2}$}
$\max_{x \in [a,b]} |f''(x)|$, \\
d.\,h. der Fehler verhält sich wie $\mathcal{O}(\frac{1}{n^2})$ für
$n \to \infty$.

\linie

\textbf{die \textsc{Simpson}sche Regel (stückweise Approximation mit $P_2$)}:
\\
Auf jedem $\Delta_k$ wird $f$ durch das Lagrange-Polynom zweiten Grades \\
$P_2(x) =$ {\large $\frac{(x - \xi_k)(x - x_k)}
{(x_{k-1} - \xi_k)(x_{k-1} - x_k)}$} $f(x_{k-1}) \;+$
{\large $\frac{(x - x_{k-1})(x - x_k)}
{(\xi_k - x_{k-1})(\xi_k - x_k)}$} $f(\xi_k) \;+$
{\large $\frac{(x - x_{k-1})(x - \xi_k)}
{(x_k - x_{k-1})(x_k - \xi_k)}$} $f(x_k)$ \\
approximiert ($P_2$ geht durch $x_{k-1}$, $\xi_k$ und $x_k$). \\
Dann wird die Simponsche Regel durch
$\int_a^b f(x)\dx = \sum_{k=1}^n \int_{x_{k-1}}^{x_k} f(x)\dx
\approx \sum_{k=1}^n \int_{x_{k-1}}^{x_k} P_2(x)\dx$\\
$= \sum_{k=1}^n (f(x_{k-1}) + 4f(\xi_k) + f(x_k)) \;\cdot$
{\large $\frac{h}{6}$} $=$ \\
{\large $\frac{b - a}{6n}$} $\cdot \Big(f(a) + f(b) +
2 \cdot \big(f(x_1) + \cdots + f(x_{n-1})\big) +
4 \cdot \big(f(\xi_1) + \cdots + f(\xi_n)\big)\Big) =: I_S$ hergeleitet.

\textbf{Fehlerabschätzung}: Ist $f \in C^4([a,b])$, so ist
$\left|\int_a^b f(x)\dx - I_S\right| \le$
{\large $\frac{(b - a)^5}{2880n^4}$}
$\max_{x \in [a,b]} |f^{(4)}(x)|$, \\
d.\,h. der Fehler verhält sich wie $\mathcal{O}(\frac{1}{n^4})$ für
$n \to \infty$.

\pagebreak
