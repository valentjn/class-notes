\chapter{%
    Symmetrische Verschlüsselungsverfahren%
}

\section{%
    Definitionen%
}

\textbf{kryptografisches Verschlüsselungsverfahren}:
Ein \begriff{(kryptografisches) Verschlüsselungsver"-fahren}
(auch \begriff{kryptografisches System} oder \begriff{Kryptosystem})
$(P, C, K, (c_k)_{k \in K}, (d_k)_{k \in K})$ wird definiert durch endliche Mengen
\begin{itemize}
    \item
    $P$ (\begriff{Klartexte}),

    \item
    $C$ (\begriff{Geheimtexte}) und

    \item
    $K$ (\begriff{Schlüssel})
\end{itemize}
sowie durch Funktionen für jeden Schlüssel $k \in K$ mit
\begin{itemize}
    \item
    $c_k\colon P \rightarrow C$ (\begriff{Codierungs-/Verschlüsselungsfunktion}) und

    \item
    $d_k\colon C \rightarrow P$ (\begriff{Decodierungs-/Entschlüsselungsfunktion}),
\end{itemize}
sodass $\forall_{k \in K} \exists_{\ell \in K} \forall_{x \in P}\; d_\ell(c_k(x)) = x$
(\begriff{Korrektheit des Verfahrens}).

Aus der Korrektheit folgt für alle $k \in K$, dass $c_k$ injektiv ist
(für $x_1, x_2 \in P$ mit $c_k(x_1) = c_k(x_2)$ gilt
$x_1 = d_\ell(c_k(x_1)) = d_\ell(c_k(x_2)) = x_2$).

\textbf{symmetrisches Verfahren}:
Das Verfahren heißt \begriff{symmetrisch} (oder \begriff{Private-Key-Verfahren}), falls
in obiger Definition $\ell = k$ oder $\ell$ sich leicht aus $k$ berechnen lässt.

In diesem Kapitel werden nur symmetrische Verfahren betrachtet.

\linie

\textbf{kryptografisches Szenario}:
Das typische Szenario bei symmetrischen Verfahren ist wie folgt.
Alice will Bob eine Nachricht schicken und Eve (\emph{eavesdropper}) will mithören.
\begin{enumerate}
    \item
    Alice und Bob wählen einen gemeinsamen Schlüssel $k \in K$.
    Dieser ist beiden vor der Übertragung schon bekannt oder wird über einen
    sicheren Kanal übermittelt.

    \item
    Alice verschlüsselt $x \in P$ durch $y = c_k(x)$ und sendet $y$ an Bob.

    \item
    Bob empfängt $y$ und entschlüsselt $y$ durch $d_k(y) = d_k(c_k(x)) = x$.
\end{enumerate}
Eve, die mithören will, kann nur $y$ abfangen (nicht den Schlüssel $k$).
Mit dieser Information startet sie ihre \begriff{Kryptanalyse}.

\pagebreak

\section{%
    Kryptanalyse%
}

\textbf{\name{Kerckhoffs}' Prinzip}:
Die Sicherheit eines Verfahrens darf nur auf der Geheimhaltung des Schlüssels basieren,
nicht aber auf der Geheimhaltung des Verfahrens.
Angreifer kennen also zumindest das verwendete kryptografische System.

\linie

\textbf{Kompromittierung von kryptografischer Kommunikation}:
\begin{itemize}
    \item
    \begriff{Bruch des Schlüssels}:
    Eve kann nicht nur Nachrichten entschlüsseln und lesen, sondern auch
    Fehlnachrichten verschlüsseln und verschicken

    \item
    \begriff{globale Lösung}:
    funktionierendes Entschlüsselungsverfahren ohne Kenntnis des Schlüssels\\
    (Eve kann keine Fehlnachrichten verschlüsseln)

    \item
    \begriff{lokale Lösung}:
    nur eine einzelne Nachricht entschlüsselt

    \item
    \begriff{Informationsgewinn}:
    partielle Informationen über Schlüssel oder Klartext
\end{itemize}

\linie

\textbf{Angriffsszenarien}:
\begin{itemize}
    \item
    \begriff{ausschließlich Geheimtext (ciphertext-only)}:
    Eve kennt nur Geheimtexte

    \item
    \begriff{bekannter Klartext (known-plaintext)}:
    Eve kennt Klartexte und zugehörige Geheimtexte

    \item
    \begriff{gewählter Klartext (chosen-plaintext)}:
    Eve kann Klartexte verschlüsseln lassen

    \item
    \begriff{gewählter Geheimtext (chosen-ciphertext)}:
    Eve kann Geheimtexte entschlüsseln lassen
\end{itemize}

\linie

\textbf{Sicherheitsstufen eines Verfahrens}:
\begin{itemize}
    \item
    \begriff{perfekte/absolute Sicherheit}:
    Entschlüsselung beweisbar unmöglich

    \item
    \begriff{Berechnungssicherheit}:
    Entschlüsselung beweisbar für die Praxis zu aufwändig

    \item
    \begriff{relative Berechnungssicherheit}:
    Entschlüsselung mindestens so schwer wie die Lösung eines als schwierig geltenden Problems

    \item
    \begriff{pragmatische Sicherheit}:
    trotz intensiver Suche keine ef"|fiziente Methode zur Entschlüsselung bekannt
\end{itemize}
In der Praxis tritt bei der klassischen Kryptografie die pragmatische Sicherheit am häufigsten auf.
Allerdings können Verfahren schwieriger zu brechen scheinen, als sie es sind, und
außerdem könnten geheime Hintertüren in Verfahren eingebaut sein.

\pagebreak

\section{%
    Blockchif"|fren%
}

\textbf{Blockchif"|fre}:
Seien $n \in \natural$ und $\Sigma$ ein Alphabet (eine endliche Menge).
Eine \begriff{Blockchif"|fre} mit \begriff{Blocklänge} $n$
ist ein symmetrisches Verschlüsselungsverfahren mit $P = C := \Sigma^n$.

Bei einer Blockchif"|fre wird zur Verschlüsselung eines beliebigen Texts $w \in \Sigma^\ast$ der
Text in Blöcke der Länge $n$ aufgeteilt.
Anschließend wird jeder Block mit demselben Schlüssel verschlüsselt.

\linie

\textbf{Lemma (Blockchif"|fren sind Permutationen)}:
Codierungsfunktionen einer Blockchif"|fre sind Permutationen von $\Sigma^n$
und die Decodierungsfunktionen sind die inversen Permutationen.

\begin{Beweis}
    Für alle $k \in K$ ist $c_k\colon \Sigma^n \rightarrow \Sigma^n$
    injektiv und damit bijektiv (\emph{pigeonhole principle}).
    Weil das Verfahren symmetrisch ist, gilt die Korrektheit in der Form
    $\forall_{k \in K} \forall_{x \in P}\; d_k(c_k(x)) = x$, für
    alle $k \in K$ ist also $d_k = c_k^{-1}$ bijektiv.
\end{Beweis}

Es wäre unpraktikabel, als Codierungsfunktion $c_k$ einer Blockchif"|fre jede Permutation
zuzulassen (d.\,h. $K$ gleich der Menge der Permutationen von $\Sigma^n$ und
$c_\sigma := \sigma$ für $\sigma \in K$),
denn um einen Schlüssel zu speichern, benötigt man Platz $\Omega(|\Sigma|^n)$.
Daher schränkt man die Anzahl der möglichen Permutationen ein,
wofür es im Wesentlichen drei Möglichkeiten gibt
(mono- und polyalphabetische Substitution sowie Permutationschif"|fre).

\section{%
    Monoalphabetische Substitution%
}

\textbf{monoalphabetische Verschiebung}:
Sei $P = C = K := \integer/26\integer$ (Alphabet $\Sigma := \{\code{a}, \dotsc, \code{z}\}$
wird mit $\integer/26\integer$ identifiziert).
Dann ist die \begriff{monoalphabetische Verschiebung} eine Blockchif"|fre mit Blocklänge $1$,
die definiert ist durch $c_k(x) := x + k$ und $d_k(x) := x - k$ für $k \in K$.

\textbf{\name{Caesar}-Verfahren}:
monoalphabetische Verschiebung mit Schlüssel $k = 3$

\textbf{ROT13}:
monoalphabetische Verschiebung mit Schlüssel $k = 13$ (Spezialfall, da $c_k = d_k$)

\textbf{Angriff}:
\begriff{Brute-Force} (alle Möglichkeiten durchprobieren),
da der Schlüsselraum nur aus $|\Sigma|$ vielen Schlüsseln besteht.

\linie

\textbf{monoalphabetische Substitution}:
Seien $P = C := \Sigma$ und
$K := \{\sigma \;|\; \sigma \text{ Permutation von } \Sigma\}$.\\
Dann ist die \begriff{monoalphabetische Substitution} eine Blockchif"|fre mit Blocklänge $1$,
die definiert ist durch
$c_\sigma := \sigma$ und $d_\sigma := \sigma^{-1}$ für $\sigma \in K$.

Die monoalphabetische Verschiebung ist eine monoalphabetische Substitution
($\Sigma := \integer/26\integer$ und für $\sigma = k \in \integer/26\integer$ ist
$c_\sigma = \smallpmatrix{0 & 1 & \cdots & 25 \\ k & k + 1 & \cdots & k - 1}$).

\textbf{Angriff (Häufigkeitsanalyse)}:
Die Größe des Schlüsselraums ist $|K| = |\Sigma|!$, für $n = 26$ gilt also
$|K| = 26! \approx 10^{26}$.
Brute-Force ist bei solchen Größenordnungen nicht praktikabel.
Stattdessen führt man eine \begriff{Häufigkeitsanalyse} durch.
Dazu nutzt man aus, dass bei Klartexten in natürlichen Sprachen bestimmte
Zeichen aus $\Sigma$ häufiger vorkommen als andere.
%In der deutschen Sprache kommt z.\,B. das \code{e} am häufigsten vor.
Da die monoalphabetische Substitution diese Häufigkeiten invariant lässt, kann man dem
häufigsten Geheimtext-Zeichen das \code{e} zuordnen und evtl. dem zweithäufigsten das \code{n}.
Dies lässt sich auch für häufig auftretende Doppellaute durchführen (\code{en}, \code{er}, \dots).
Weitere Zuordnungen können aus dem Kontext erschlossen werden
(es gibt nur bestimmte Kombinationen von Klartext-Zeichen zu sinnvollen Wörtern).
Eine Abhilfe gegen diese überflüssigen Kontextinformationen kann Datenkompression sein
(um den Angriff zu erschweren).

\pagebreak

\section{%
    Polyalphabetische Substitution%
}

\textbf{polyalphabetische Substitution}:
Eine \begriff{polyalphabetische Substitution} ist ein Blockchif"|fre mit Blocklänge $n$,
bei dem jeder Klartext-Buchstabe einzeln permutiert wird, d.\,h.\\
$P = C := \Sigma^n$ und $K := \{\sigma \;|\; \sigma \text{ Permutation von } \Sigma\}^n$ sowie\\
$c_\sigma(x) := \sigma_1(x_1) \dotsb \sigma_n(x_n)$ und
$d_\sigma(x) := \sigma^{-1}_1(x_1) \dotsb \sigma^{-1}_n(x_n)$ für $\sigma \in K$.

\textbf{\name{Vigenère}-Verfahren}:
Seien $P = C := \Sigma^n$ und $K := \Sigma^d$ für ein festes $d \in \natural$.\\
Dann ist das \begriff{\name{Vigenère}-Verfahren mit Schlüssellänge $d$} eine
Blockchif"|fre mit Blocklänge $n$, die definiert ist durch
$c_k(x) := \widetilde{c}(x_1, k_{0 \bmod d}) \widetilde{c}(x_2, k_{1 \bmod d}) \cdots
\widetilde{c}(x_n, k_{(n-1) \bmod d})$ für $x =: x_1 \cdots x_n$ und
$k =: k_0 \dotsb k_{d-1} \in K$, wobei
$\widetilde{c}(a, k_i) := a + k_i$ für $a \in \Sigma$ und $i = 0, \dotsc, d - 1$\\
(es wird angenommen, dass $\Sigma$ geordnet ist, d.\,h.
$\widetilde{c}(\cdot, k_i)$ ist die Verschiebung um $k_i$).

Vigenère-Verfahren sind polyalph. Subst.en
(die Permutationen sind einfache Verschiebungen).
Monoalph. Verschiebungen sind Vigenère-Verfahren mit Schlüssellänge $1$,
allerdings sind monoalph. Subst.en i.\,A. keine Vigenère-Verfahren.
Polyalph. Subst.en über $\Sigma$ können als monoalph. Subst.en über
$\Sigma' := \Sigma^n$ angesehen werden.

Vorteile des Vigenère-Verfahrens gegenüber der monoalphabetischen Substitution sind
eine gleichmäßigere Verteilung der Geheimtext-Zeichen und Elimination von Doppellauten.

\textbf{Angriff}:
Gleiche Klartextstellen werden gleich verschlüsselt, wenn der Abstand zwischen ihnen ein Vielfaches
der Schlüssellänge $d$ ist.
Kennt man daher die Schlüssellänge $d$ (oder auch nur ein Vielfaches von ihr),
dann kann man den Geheimtext $y$ in $d$ Spalten schreiben
(die $i$-te Spalte enthält die Zeichen $y_j$ mit $j \equiv i \mod d$).
Jede einzelne Spalte wurde durch eine eine monoalphabetische Verschiebung verschlüsselt und kann
dementsprechend mit Brute-Force oder Häufigkeitsanalyse leicht geknackt werden.

\linie

Um die Schlüssellänge $d$ oder ein Vielfaches dieser herauszufinden, muss man etwas arbeiten.

\textbf{Koinzidenz-Index}:
Seien $\Sigma$ ein Alphabet, $n \in \natural$ und $x, x' \in \Sigma^n$.\\
Dann heißt $\kappa(x, x') := \frac{1}{n} \sum_{i=1}^n \delta_{x_i,x_i'}$
\begriff{Koinzidenz-Index} von $x$ und $x'$.

Sind $x$ und $x'$ aus einem gleichverteilten Zufallsexperiment entstanden, dann gilt
$\EE[\kappa] = \frac{1}{|\Sigma|}$.
Für $|\Sigma| = 26$ gilt $\EE[\kappa] = \SI{3.8}{\percent}$.
Allerdings erhält man experimentell $\EE[\kappa] \approx \SI{7}{\percent}$ für
natürlich erzeugte Texte.
Wenn man sich zunutze macht, dass der Koinzidenz-Index zweier Klartexte bei
polyalphabetischer Substitution mit demselben Schlüssel
nicht ändert, d.\,h. $\kappa(x, x') = \kappa(c_k(x), c_k(x'))$,
dann lässt sich die Schlüssellänge einfach bestimmen.

Seien $y \in \Sigma^n$ ein Geheimtext und $\kappa^k := \kappa(y, \sigma^k(y))$ für
$k \in \natural$, wobei $\sigma^k(y)$ die zyklische Verschiebung von $y$ um $k$ Zeichen nach links
ist.
Für $k \in \natural$ mit $d \teilt k$ werden $x$ und $\sigma^k(x)$ mit demselben Schlüssel
Vigenère-verschlüsselt (bis auf das Ende), d.\,h. $\kappa^k$ wird für diese $k$ bei
ca. \SI{7}{\percent} liegen und für andere bei deutlich weniger.

Der Angriff durch das Koinzidenz-Kriterium ist für allgemeine polyalphabetische Substitutionen
durchführbar und damit sind diese Verfahren leicht angreifbar.
Trotzdem sind sie heute noch relevant
(z.\,B. kommerzielle Software, US-Mobiltelefone, oberflächlicher Schutz).

\pagebreak

\section{%
    Perfekte Sicherheit%
}

\textbf{Wahrscheinlichkeiten}:
Mit $\Pr(x)$ und $\Pr(y)$ werden die \begriff{Wahrscheinlichkeiten} bezeichnet,
dass Eve den Klartext $x$ bzw. den Geheimtext $y = c_k(x)$ für ein $k \in K$ abfängt.
Die Wahrscheinlichkeit, dass der Schlüssel $k \in K$ verwendet wird, wird mit $\Pr(k)$
bezeichnet.\\
$\Pr(y \;|\; x)$ und $\Pr(x \;|\; y)$ bezeichnen die Wahrscheinlichkeiten, dass $y$ abgefangen
wird unter der Bedingung, dass der zugehörige Klartext $x$ ist, bzw. umgekehrt.

$\Pr(y)$ wird durch $\Pr(x)$ und $\Pr(k)$ induziert.
In der Praxis gilt meist, dass $x$ und $k$ unabhängig sind (d.\,h. $\Pr(x, k) = \Pr(x) \Pr(k)$
für alle $x \in P$ und $k \in K$).
OBdA kann man annehmen, dass $\Pr(x) > 0$ und $\Pr(y) > 0$ für alle
$x \in P$ und $y \in C$ (sonst schränkt man $P$ und $C$ ein).

\textbf{perfekte Sicherheit}:
Ein Verschlüsselungsverfahren hat \begriff{perfekte Sicherheit}, falls\\
$\Pr(x \;|\; y) = \Pr(x)$ für alle $x \in P$ und $y \in C$.

Das heißt, dass die Kenntnis von $y$ keine Information über den Klartext $x$ bringt.

\linie

\textbf{Satz (notwendige Bedingung für perfekte Sicherheit)}:\\
Ist ein Verschlüsselungsverfahren perfekt sicher, so gilt
$|P| \le |C| \le |K|$.

\begin{Beweis}
    Für alle $k \in K$ ist $c_k\colon P \rightarrow C$ injektiv, d.\,h. $|P| \le |C|$.

    Sei $x \in P$ fest.
    Dann gibt es für jedes $y \in C$ ein $k_y \in K$ mit $c_{k_y}(x) = y$.
    Sonst gäbe es ein $y \in C$ mit $c_k(x) \not= y$ für alle $k \in K$,
    also $\Pr(x \;|\; y) = 0 < \Pr(x)$, ein Widerspruch zur perfekten Sicherheit.
    Aus $k_{y_1} = k_{y_2}$ folgt $y_1 = c_{k_{y_1}}(x) = c_{k_{y_2}}(x) = y_2$,
    d.\,h. $|C| \le |K|$.
\end{Beweis}

\linie

\textbf{Satz (Charakterisierung von perfekt sicheren Verfahren)}:
Seien  $|P| = |C| = |K|$ und Schlüssel- und Klartext-Wahrscheinlichkeiten voneinander unabhängig.
Dann ist das Verfahren perfekt sicher genau dann, wenn
$\forall_{k \in K}\; \Pr(k) = \frac{1}{|K|}$ und
$\forall_{x \in P} \forall_{y \in C} \exists!_{k_{x,y} \in K}\; c_{k_{xy}}(x) = y$ gilt.

\begin{Beweis}
    "`$\Rightarrow$"':
    Sei das Verfahren perfekt sicher.
    Wie im Beweis des vorherigen Satzes gibt es für alle $x \in P$ und $y \in C$ ein
    $k_{x,y} \in K$ mit $c_{k_{x,y}}(x) = y$,
    wobei $k_{x,y_1} \not= k_{x,y_2}$ für $y_1 \not= y_2$.
    Die Abbildung $f_x\colon C \rightarrow K$, $y \mapsto k_{x,y}$ ist für alle $x \in P$
    injektiv und wegen $|C| = |K|$ daher bijektiv, weswegen $k_{x,y} \in K$ eindeutig bestimmt ist.

    Sei nun $y \in C$ fest.
    Mit dem Satz von Bayes gilt $\Pr(x) = \Pr(x \;|\; y) = \frac{\Pr(y \;|\; x) \Pr(x)}{\Pr(y)}$,
    also $\Pr(y \;|\; x) = \Pr(y)$.
    Allerdings ist $\Pr(y \;|\; x) = \Pr(k_{xy})$, d.\,h.
    $\Pr(k_{xy}) = \Pr(y)$ für alle $x \in P$.
    Wie eben ist $g_y\colon P \rightarrow K$, $x \mapsto k_{x,y}$ bijektiv, d.\,h.
    $\Pr(k) = \Pr(y)$ ist für alle $k \in K$ gleich.

    "`$\Leftarrow$"':
    Es gilt $\Pr(y) = \sum_{x \in P} \Pr(y \;|\; x) \Pr(x)$
    (Gesetz der totalen Wahrscheinlichkeit).
    Weil die Schlüssel $k_{xy}$ eindeutig sind, gilt
    $\Pr(y \;|\; x) = \Pr(k_{xy}) = \frac{1}{|K|}$, also
    $\Pr(y) = \sum_{x \in P} \frac{1}{|K|} \Pr(x) = \frac{1}{|K|}$ wegen $|P| = |K|$.
    Damit gilt $\Pr(x \;|\; y) = \frac{\Pr(y \;|\; x) \Pr(x)}{\Pr(y)} = \Pr(x)$
    (perfekte Sicherheit).
\end{Beweis}

\pagebreak

\section{%
    One-Time-Pad%
}

\textbf{One-Time-Pad}:
Seien $P = C = K := \{0, 1\}^n$.
Dann ist das \begriff{(\name{Vernam}-)One-Time-Pad} eine Blockchif"|fre mit Blocklänge $n$,
die definiert ist durch $c_k(x) = x \xor k$ und $d_k(y) = y \xor k$ für $k \in \{0, 1\}^n$.

Dabei ist "`$\xor$"' das bitweise XOR
(Addition der Argumente, wenn man sie als Elemente von $(\integer/2\integer)^n$ auf"|fasst).
One-Time-Pads sind also genau die Vigenère-Verfahren über dem Alphabet $\{0, 1\}$
mit Schlüssellänge $n$.

One-Time-Pads sind perfekt sichere Verfahren im obigen Sinne.
Jeder Klartext ist möglich, wenn man den Schlüssel nicht kennt.
Wenn Eve z.\,B. die Nachricht \code{xsvii} abfängt, kann der Klartext \code{sonne} lauten
(mit dem Schlüssel \code{feive}) oder aber \code{regen} (mit dem Schlüssel \code{gopev}).
Dies gilt aber nur, wenn derselbe Schlüssel nur einmal verwendet wird:
Sind $y_1 = c_k(x_1)$ und $y_2 = c_k(x_2)$ zwei verschiedene abgefangene Nachrichten,
die mit demselben Schlüssel verschlüsselt wurden, so gilt
$y_1 \xor y_2 = (x_1 \xor k) \xor (x_2 \xor k) = x_1 \xor x_2$.
Wenn $x_1$ und $x_2$ natürliche Texte sind, kommen verschiedene Zeichen in
$x_1 \xor x_2$ stark unterschiedlich häufig vor und man kann eine Häufigkeitsanalyse starten.
Außerdem sind One-Time-Pads anfällig gegenüber Known-Plaintext-Attacken:
Die Kenntnis eines einzigen Klartext-Geheimtext-Paars $(x, c_k(x))$ reicht aus, um mit
$x \xor c_k(x) = x \xor (x \xor k) = k$ den Schlüssel herauszufinden.

One-Time-Pads sind aus verschiedenen Gründen unpraktikabel
(langer Schlüssel, zufällige Erzeugung nicht-trivial).

\section{%
    Data Encryption Standard (DES)%
}

DES ist eine symmetrische binäre Blockchif"|fre
(Blocklänge von 64 Bit und eff. Schlüssellänge von 56 Bit), wurde 1977 eingeführt und 2001 durch
AES abgelöst, weil es als nicht mehr sicher genug galt.
Allerdings wird DES immer noch verwendet, Triple-DES gilt nach wie vor als sicher.
Im Wesentlichen sind zwar nur Brute-Force-Angrif"|fe bekannt,
diese enden aber heutzutage innerhalb eines Tages erfolgreich.

DES sollte gewisse Designziele erfüllen,
wie hohe Sicherheit,
vollständige Spezifikation,
gute Verständlichkeit,
Einhaltung von Kerkhoffs' Prinzip,
Verfügbarkeit für alle Benutzer,
Vielseitigkeit,
Ef"|fizienz und
Eignung für Hardware-Implementationen.

\textbf{Data Encryption Standard (DES)}:
Sei $\BB := \{0, 1\}$.\\
Der \begriff{Data Encryption Standard (DES)} ist eine Blockchif"|fre mit Blocklänge 64,
wobei $y := c_k(x)$ für $x \in \BB^{64}$ und $k \in \BB^{64}$ wie folgt definiert ist:
\begin{enumerate}
    \item
    Wende eine Initialpermutation $\IP\colon \BB^{64} \rightarrow \BB^{64}$ auf $x$ an
    (siehe unten) und teile das Ergebnis auf, d.\,h.
    $L_0 R_0 := \IP(x)$ mit $L_0, R_0 \in \BB^{32}$.

    \item
    Berechne sukzessive für $i = 1, \dotsc, 16$ die Wörter $L_i, R_i \in \BB^{32}$ durch\\
    $L_i := R_{i-1}$ und $R_i := L_{i-1} \xor f(R_{i-1}, k_i)$,
    wobei die Rundenfunktion $f\colon \BB^{32} \times \BB^{48} \rightarrow \BB^{32}$
    und der Rundenschlüssel $k_i \in \BB^{48}$ weiter unten definiert sind.

    \item
    Berechne $y := \IP^{-1}(R_{16} L_{16})$.
\end{enumerate}

\linie
\pagebreak

\textbf{Initialpermutation}:
Die \begriff{Initialpermutation} $\IP\colon \BB^{64} \rightarrow \BB^{64}$ ist statisch
definiert.\\
Ein Ausschnitt lautet $\IP(x_1 \cdots x_{64}) := x_{58} x_{50} x_{42} \cdots x_{15} x_7$.\\
Ein Ausschnitt der Inversen ist $\IP(x_1 \cdots x_{64}) := x_{40} x_8 x_{48} \cdots x_{57} x_{25}$.

\textbf{Rundenfunktion}:
Die \begriff{Rundenfunktion} (oder \begriff{interne Blockchif"|fre})
$f\colon \BB^{32} \times \BB^{48} \rightarrow \BB^{32}$,\\
$(A,J) \mapsto f(A,J)$
ist wie folgt definiert:
\begin{enumerate}
    \item
    Expandiere $A$ auf 48 Bit mittels der \begriff{Expandierungsfunktion}
    $E\colon \BB^{32} \rightarrow \BB^{48}$,
    wobei alle 32 Bit in $E(A)$ auftauchen und 16 der Bits an bestimmten Stellen wiederholt werden
    (ebenfalls statisch).

    \item
    Berechne $B := E(A) \xor J$ und zerlege das Ergebnis in $B =: B_1 \cdots B_8$ mit
    $B_i \in \BB^6$.

    \item
    Transformiere $B_i$ in $C_i := S_i(B_i) \in \BB^4$ mittels den sog. \begriff{S-Boxen}
    $S_i\colon \BB^6 \rightarrow \BB^4$
    (statisch).

    \item
    Bilde $C := C_1 \cdots C_8 \in \BB^{32}$ und wende eine statische Bitpermutation an:
    $f(A, J) := P(C)$.
\end{enumerate}
Die S-Boxen sind dabei als Tabellen der Größe $4 \times 16$ gegeben.
Für die Berechnung von\\
$C_i = S_i(B_i)$ sei $B_i =: b_{i,1} \cdots b_{i,6}$ mit
$b_{i,j} \in \BB$.
Dann wählt man die Zeile $(b_1 b_6)_2$ und die Spalte $(b_2 b_3 b_4 b_5)_2$ und die Zahl an dieser
Stelle in Binärdarstellung ist $C_i$
(die Nummerierung beginnt jeweils bei $0$).
Die S-Boxen (engl. \emph{substitution box}) sind nicht-lineare Komponenten, die verhindern,
dass die Verschlüsselung nur die Lösung eines LGS ist.

\textbf{Rundenschlüssel}:\\
Die \begriff{Rundenschlüssel} $k_1, \dotsc, k_{16} \in \BB^{48}$ werden aus
$k \in \BB^{64}$ wie folgt berechnet:
\begin{enumerate}
    \item
    Wähle bestimmte 56 Bit aus $k$ mittels der statischen Abbildung
    $\PC_1\colon \BB^{64} \rightarrow \BB^{56}$.

    \item
    Zerlege das Ergebnis in $C_0 D_0 := \PC_1(k)$ mit $C_0, D_0 \in \BB^{28}$.

    \item
    Berechne für $i = 1, \dotsc, 16$ die Wörter $C_i, D_i \in \BB^{28}$ durch
    $C_i := \sigma_i(C_{i-1})$ und\\
    $D_i := \sigma_i(D_{i-1})$, wobei $\sigma_i$ die zyklische Linksverschiebung um ein Bit
    für $i \in \{1, 2, 9, 16\}$ und um zwei Bit sonst ist.

    \item
    Wähle bestimmte 48 Bit aus $C_i D_i$ mittels der statischen Abbildung
    $\PC_2\colon \BB^{56} \rightarrow \BB^{48}$ und setze
    $k_i := \PC_2(C_i D_i)$.
\end{enumerate}

\linie

\textbf{DES-Entschlüsselung}:
Ein Geheimtext $y \in \BB^{64}$ wird entschlüsselt,
indem man den gleichen Algortihmus wie bei der Verschlüsselung anwendet,
nur muss man dieses Mal die Rundenschlüssel in umgekehrter Reihenfolge $k_{16}, \dotsc, k_1$
benutzen.

Diese einfache Art der Entschlüsselung ist günstig für die Hardware-Implementierung
(spart Schaltungsaufwand).

\textbf{Satz (Korrektheit der Entschlüsselung)}:
Für den so erhaltenen Text $z \in \BB^{64}$ gilt $z = x$.

\begin{Beweis}
    Bei der Entschlüsselung von $y \in \BB^{64}$ berechnet man zuerst
    $L_0' R_0' := \IP(y)$\\
    $= \IP(\IP^{-1}(R_{16} L_{16})) = R_{16} L_{16}$ mit
    $L_0', R_0' \in \BB^{32}$.
    Es gilt also $L_0' = R_{16}$ und $R_0' = L_{16}$.
    Induktiv beweist man nun $L_i' = R_{16-i}$ und $R_i' = L_{16-i}$ für $i = 0, \dotsc, 16$.
    Daraus folgt dann die Behauptung mit
    $z = \IP^{-1}(R_{16}' L_{16}') = \IP^{-1}(L_0 R_0) = \IP^{-1}(\IP(x)) = x$.

    Angenommen, es gilt $L_{i-1}' = R_{17-i}$ und $R_{i-1}' = L_{17-i}$ für ein
    $i \in \{1, \dotsc, 16\}$.\\
    Dann ist $L_i' := R_{i-1}' = L_{17-i} = R_{16-i}$ und
    $R_i' := L_{i-1}' \xor f(R_{i-1}', k_{17-i}) =
    R_{17-i} \xor f(L_{17-i}, k_{17-i})$\\
    $= (L_{16-i} \xor f(R_{16-i}, k_{17-i})) \xor f(L_{17-i}, k_{17-i})
    = (L_{16-i} \xor f(L_{17-i}, k_{17-i})) \xor f(L_{17-i}, k_{17-i}) = L_{16-i}$,
    weil die Rundenschlüssel in umgekehrter Reihenfolge benutzt werden.
\end{Beweis}

\pagebreak

\section{%
    Mehrfachverschlüsselung%
}

DES gilt wegen der kurzen ef"|fektiven Schlüssellänge von 56 Bit als unsicher.
Man kann das Problem beheben, in dem man DES mehrfach anwendet.

\textbf{Triple-DES}:
\begriff{Triple-DES} ist eine Blockchif"|fre mit Blocklänge 64, wobei man
$y := c_k(x)$ für $k := k_1 k_2 \in \BB^{128}$ mit $k_1, k_2 \in \BB^{64}$ wie folgt berechnet:
$y = \DES_{k_1}(\DES^{-1}_{k_2}(\DES_{k_1}(x)))$
(dabei bezeichnet $\DES_{k_i}$ die Anwendung von DES mit dem Schlüssel $k_i$).

Ef"|fektiv verwendet also nun eine Schlüssellänge von 112 Bit, was heutzutage als sicher gilt.
DES ist nicht unter Komposition abgeschlossen, d.\,h.
$\exists_{k_1, k_2 \in \BB^{64}} \forall_{k \in \BB^{64}}\;
\DES_{k_2} \circ \DES_{k_1} \not= \DES_k$
(es gibt zwei Schlüssel, sodass die zweifache Anwendung der Verschlüsselung nicht durch
eine einfache Anwendung mit irgendeinem Schlüssel durchführbar ist).
Man kann alternativ auch drei unabhängige Schlüssel $k_i \in \BB^{64}$ verwenden, womit man auf
eine eff. Schlüssellänge von 168 Bit kommt.

Diese Art von \begriff{Mehrfachverschlüsselung} ist auch bei anderen Verfahren und mehreren
Stufen einsetzbar.

\section{%
    Betriebsmodi von Blockchif"|fren%
}

Bei der Definition von Blockchif"|fren wurde implizit der ECB-Modus verwendet,
weil jeder Block mit demselben Schlüssel verschlüsselt wird.
Allerdings kann man Blockchif"|fren auch anders anwenden, um andere Eigenschaften zu erhalten.

Im Folgenden sei $P = C := \BB^n$, $K := \BB^m$ und
$c_k, d_k$ für $k \in K$ die (De-)Codierungsfunktionen einer Blockchif"|fre mit Blocklänge $n$.

\subsection{%
    ECB-Modus%
}

\textbf{ECB-Modus}:
Der \begriff{ECB-Modus} (engl. \emph{electronic codebook}) ist wie folgt definiert.
\begin{itemize}
    \item
    Sei $x = x_1 x_2 \cdots$ mit $x_i \in \BB^n$ ein Klartext und $k \in K$ ein Schlüssel.\\
    Dann ist der entsprechende Geheimtext $y := y_1 y_2 \cdots$ mit $y_i := c_k(x_i) \in \BB^n$.

    \item
    Sei $y = y_1 y_2 \cdots$ mit $y_i \in \BB^n$ ein Geheimtext und $k \in K$ ein Schlüssel.\\
    Dann ist der entsprechende Geheimtext $x := x_1 x_2 \cdots$ mit $x_i := d_k(y_i) \in \BB^n$.
\end{itemize}
\begin{align*}
    \begin{array}{c}
        \xymatrix@R=5mm@C=5mm{
            &
            x_1\ar[d]&
            &
            x_2\ar[d]\\
            %
            k\ar[r]&
            *++[F-:<5mm>]{c_k}\ar[d]&
            k\ar[r]&
            *++[F-:<5mm>]{c_k}\ar[d]&
            \cdots\\
            %
            &
            y_1&
            &
            y_2
        }\\
        \\
        \text{\emph{Verschlüsselung}}
    \end{array}
    &&
    \begin{array}{c}
        \xymatrix@R=5mm@C=5mm{
            &
            y_1\ar[d]&
            &
            y_2\ar[d]\\
            %
            k\ar[r]&
            *++[F-:<5mm>]{d_k}\ar[d]&
            k\ar[r]&
            *++[F-:<5mm>]{d_k}\ar[d]&
            \cdots\\
            %
            &
            x_1&
            &
            x_2
        }\\
        \\
        \text{\emph{Entschlüsselung}}
    \end{array}
\end{align*}

\linie

\textbf{Nachteile}:
Gleiche Klartextblöcke $x_i = x_j$ für $i \not= j$ werden zu gleichen Geheimtextblöcken
verschlüsselt.
Damit übertragen sich Regelmäßigkeiten vom Klar- in den Geheimtext,
was Angreifern Informationen über den Klartext liefert.
Außerdem könnte der Angreifer einfach Geheimtext-Blöcke einfügen, die mit demselben
Schlüssel codiert wurden (Chosen-Plaintext-Attacke),
Geheimtext-Blöcke unbemerkt unterschlagen oder permutieren.
Aus diesen Gründen ist der ECB-Modus nicht sicher und für lange Texte ungeeignet.

\pagebreak

\subsection{%
    CBC-Modus%
}

\textbf{CBC-Modus}:
Der \begriff{CBC-Modus} (engl. \emph{cipher-block chaining}) ist wie folgt definiert.\\
Zunächst wählt man einen festen Initialisierungsblock $y_0 \in \BB^n$.
\begin{itemize}
    \item
    Sei $x = x_1 x_2 \cdots$ mit $x_i \in \BB^n$ ein Klartext und $k \in K$ ein Schlüssel.\\
    Dann ist der entsprechende Geheimtext $y := y_1 y_2 \cdots$ mit
    $y_i := c_k(y_{i-1} \xor x_i) \in \BB^n$.

    \item
    Sei $y = y_1 y_2 \cdots$ mit $y_i \in \BB^n$ ein Geheimtext und $k \in K$ ein Schlüssel.\\
    Dann ist der entsprechende Geheimtext $x := x_1 x_2 \cdots$ mit
    $x_i := y_{i-1} \xor d_k(y_i) \in \BB^n$.
\end{itemize}
\begin{align*}
    \begin{array}{c}
        \xymatrix@R=5mm@C=5mm{
            &
            x_1\ar[d]&
            &
            &
            x_2\ar[d]\\
            %
            y_0\ar[r]&
            \bigxor\ar[d]&
            *=0{}\ar[rr]^{y_1}&
            &
            \bigxor\ar[d]&
            *=0{}\ar[r]^{y_2}&
            \cdots\\
            %
            k\ar[r]&
            *++[F-:<5mm>]{c_k}\ar[dd]&
            &
            k\ar[r]&
            *++[F-:<5mm>]{c_k}\ar[dd]\\
            %
            &
            *=0{\smallbullet}\ar@{-}[r]&
            *=0{}\ar@{-}[uu]&
            &
            *=0{\smallbullet}\ar@{-}[r]&
            *=0{}\ar@{-}[uu]
            \\
            %
            &
            y_1&
            &
            &
            y_2
        }\\
        \\
        \text{\emph{Verschlüsselung}}
    \end{array}
    &&
    \begin{array}{c}
        \xymatrix@R=5mm@C=5mm{
            &
            y_1\ar[dd]&
            &
            &
            y_2\ar[dd]\\
            %
            &
            *=0{\smallbullet}\ar@{-}[r]&
            *=0{}\ar@{-}[dd]&
            &
            *=0{\smallbullet}\ar@{-}[r]&
            *=0{}\ar@{-}[dd]
            \\
            %
            k\ar[r]&
            *++[F-:<5mm>]{d_k}\ar[d]&
            &
            k\ar[r]&
            *++[F-:<5mm>]{d_k}\ar[d]\\
            %
            y_0\ar[r]&
            \bigxor\ar[d]&
            *=0{}\ar[rr]^{y_1}&
            &
            \bigxor\ar[d]&
            *=0{}\ar[r]^{y_2}&
            \cdots\\
            %
            &
            x_1&
            &
            &
            x_2
        }\\
        \\
        \text{\emph{Entschlüsselung}}
    \end{array}
\end{align*}

\linie

\textbf{Vorteile}:
Gleiche Texte werden nicht gleich verschlüsselt, wenn man einen anderen Initialisierungsblock $y_0$
verwendet (der aber beiden Kommunikationspartner bekannt sein muss, wenn $x_1$ decodiert werden
soll).
Im Gegensatz zum ECB-Modus werden gleiche Klartext-Blöcke $x_i = x_j$ mit $i \not= j$
nicht gleich verschlüsselt.
Daher kann Eve aus Mustern im Geheimtext keine Schlüsse über den Klartext ziehen,
außerdem können Blöcke nicht mehr geändert, eingefügt, unterschlagen oder permutiert werden,
ohne dass das Bob merken würde (die Entschlüsselung funktioniert dann nicht mehr).

\textbf{Nachteile}:
Zum Entschlüsseln eines Geheimtext-Blocks benötigt man den vorherigen Geheim"-text-Block,
was beim unzuverlässigen Datenkanälen wie Funk problematisch sein kann.
Allerdings wirkt sich ein fehlerhaft übertragener Geheimtext-Block nur auf den nächsten Block aus,
weswegen Übertragungsfehler nicht die komplette Kommunikation unterbinden
(daher muss Bob $y_0$ nur kennen, wenn er $y_1$ entschlüsseln will).
Ein weiterer Nachteil ist die Ef"|fizienz, da für die Entschlüsselung von $x_i$ auf die
vollständige Übertragung von $y_i$ und $y_{i-1}$ gewartet werden muss
(schlecht bei großen Blockgrößen).

\pagebreak

\subsection{%
    CFB-Modus%
}

\textbf{CFB-Modus}:
Sei $r \in \natural$ mit $r \le n$.
Der \begriff{CFB-Modus} (engl. \emph{cipher feedback}) ist wie folgt definiert:
Zunächst wählt man einen festen Initialisierungsblock $I_1 \in \BB^n$.
\begin{itemize}
    \item
    Sei $x \in \BB^\ast$ ein Klartext und $k \in K$ ein Schlüssel.\\
    Unterteile $x$ in $x = x_1 x_2 \cdots$ mit $x_i \in \BB^r$.
    Berechne nun sukzessive für $i \ge 1$ die Wörter $z_i, I_i \in \BB^n$ und $y_i \in \BB^r$
    durch $z_i := c_k(I_i)$,
    $y_i := x_i \xor \pi_r(z_i)$ und
    $I_{i+1} := \sigma_r(I_i \leftarrow y_i)$.\\
    Dann ist der entsprechende Geheimtext $y := y_1 y_2 \cdots$.

    \item
    Sei $y \in \BB^\ast$ ein Geheimtext und $k \in K$ ein Schlüssel.\\
    Unterteile $y$ in $y = y_1 y_2 \cdots$ mit $y_i \in \BB^r$.
    Berechne nun sukzessive für $i \ge 1$ die Wörter $z_i, I_i \in \BB^n$ und $x_i \in \BB^r$
    durch $z_i := c_k(I_i)$,
    $x_i := y_i \xor \pi_r(z_i)$ und
    $I_{i+1} := \sigma_r(I_i \leftarrow y_i)$.\\
    Dann ist der entsprechende Klartext $x := x_1 x_2 \cdots$.
\end{itemize}
Dabei ist $\pi_r\colon \BB^n \rightarrow \BB^r$ die Projektion auf die oberen $r$ Bit und
$\sigma_r(I_i \leftarrow y_i)$ eine Verschiebung um $r$ Bit ist,
wobei zunächst die oberen $r$ Bit von $I_i$ gelöscht, $I_i$ um $r$ nach links
verschoben und anschließend $y_i$ in die unteren $r$ Bit eingefügt wird.
\begin{align*}
    \begin{array}{ccc}
        \xymatrix@R=4mm@C=4mm{
            &
            &
            *=0{}\ar@{-}[rrr]&
            &
            &
            *=0{}\ar[d]^{I_1}&
            *=0{}\ar[rr]^{I_2}&
            &
            \cdots\\
            %
            &
            I_1\ar[dd]_{I_1}
            &
            &
            *=0{}\ar[rr]^{y_1}&
            &
            *++[F-:<5mm>]{\sigma_r}\ar[dd]_{I_2}&
            &
            *=0{}\ar[r]^{y_2}&
            \cdots\\
            %
            &
            *=0{\smallbullet}\ar@{-}[r]&
            *=0{}\ar@{-}[uu]&
            &
            &
            *=0{\smallbullet}\ar@{-}[r]&
            *=0{}\ar@{-}[uu]\\
            %
            k\ar[r]&
            *++[F-:<5mm>]{c_k}\ar[d]&
            &
            &
            k\ar[r]&
            *++[F-:<5mm>]{c_k}\ar[d]\\
            %
            &
            *++[F-:<5mm>]{\pi_r}\ar[d]&
            &
            &
            &
            *++[F-:<5mm>]{\pi_r}\ar[d]\\
            %
            x_1\ar[r]&
            \bigxor\ar[dd]&
            &
            &
            x_2\ar[r]&
            \bigxor\ar[dd]\\
            %
            &
            *=0{\smallbullet}\ar@{-}[rr]&
            &
            *=0{}\ar@{-}[uuuuu]&
            &
            *=0{\smallbullet}\ar@{-}[rr]&
            &
            *=0{}\ar@{-}[uuuuu]\\
            %
            &
            y_1&
            &
            &
            &
            y_2
        }
        &
        &
        \xymatrix@R=5mm@C=4mm{
            &
            &
            *=0{}\ar@{-}[rr]&
            &
            *=0{}\ar[d]^{I_1}&
            *=0{}\ar[rr]^{I_2}&
            &
            \cdots\\
            %
            &
            I_1\ar[dd]_{I_1}
            &
            &
            y_1\ar[r]&
            *++[F-:<5mm>]{\sigma_r}\ar[dd]_{I_2}&
            &
            y_2\ar[r]&
            \cdots\\
            %
            &
            *=0{\smallbullet}\ar@{-}[r]&
            *=0{}\ar@{-}[uu]&
            &
            *=0{\smallbullet}\ar@{-}[r]&
            *=0{}\ar@{-}[uu]\\
            %
            k\ar[r]&
            *++[F-:<5mm>]{c_k}\ar[d]&
            &
            k\ar[r]&
            *++[F-:<5mm>]{c_k}\ar[d]\\
            %
            &
            *++[F-:<5mm>]{\pi_r}\ar[d]&
            &
            &
            *++[F-:<5mm>]{\pi_r}\ar[d]\\
            %
            y_1\ar[r]&
            \bigxor\ar[d]&
            &
            y_2\ar[r]&
            \bigxor\ar[d]\\
            %
            &
            x_1&
            &
            &
            x_2
        }\\
        &&\\
        \text{\emph{Verschlüsselung}}
        &&
        \text{\emph{Entschlüsselung}}
    \end{array}
\end{align*}
Es ist kein Fehler, dass jeweils nur $c_k$ verwendet wird, weil die Blockchif"|fre nur zum
Ermitteln der Rundenschlüsseln $z_i$ eingesetzt werden.

\linie

\textbf{Vorteile}:
%Im Gegensatz zum CBC-Modus reicht die Kenntnis von $y_i$ aus, um $x_i$ zu berechnen.
Der Folge von $I_i$ und $z_i$ können von Bob simultan zu Alice errechnet werden,
was ein klarer Ef"|fizienzvorteil ist, wenn $r \ll n$
(nur $r$ Bit müssen übertragen werden).

\textbf{Nachteile}:
Übertragungsfehler wirken sich gravierender aus als beim CBC-Modus.
Wenn $y_\ell$ fehlerhaft ist, können $x_\ell, \dotsc, x_{\ell+\lceil\frac{n}{r}\rceil}$ nicht
entschlüsselt werden
(bis der Übertragungsfehler aus dem Schieberegister "`herausgeschoben"' wurde).
Außerdem können nur symmetrische Verfahren beim CFB-Modus angewendet werden.

\pagebreak

\subsection{%
    OFB-Modus%
}

\textbf{OFB-Modus}:
Sei $r \in \natural$ mit $r \le n$.
Der \begriff{OFB-Modus} (engl. \emph{output feedback}) ist wie folgt definiert:
Zunächst wählt man einen festen Initialisierungsblock $z_0 \in \BB^n$.
\begin{itemize}
    \item
    Sei $x \in \BB^\ast$ ein Klartext und $k \in K$ ein Schlüssel.\\
    Unterteile $x$ in $x = x_1 x_2 \cdots$ mit $x_i \in \BB^r$.
    Berechne nun sukzessive für $i \ge 1$ die Wörter $z_i \in \BB^n$ und $y_i \in \BB^r$
    durch $z_i := c_k(z_{i-1})$ und
    $y_i := x_i \xor \pi_r(z_i)$.\\
    Dann ist der entsprechende Geheimtext $y := y_1 y_2 \cdots$.

    \item
    Sei $y \in \BB^\ast$ ein Geheimtext und $k \in K$ ein Schlüssel.\\
    Unterteile $y$ in $y = y_1 y_2 \cdots$ mit $y_i \in \BB^r$.
    Berechne nun sukzessive für $i \ge 1$ die Wörter $z_i \in \BB^n$ und $x_i \in \BB^r$
    durch $z_i := c_k(z_{i-1})$ und
    $x_i := y_i \xor \pi_r(z_i)$.\\
    Dann ist der entsprechende Klartext $x := x_1 x_2 \cdots$.
\end{itemize}
Dabei ist $\pi_r\colon \BB^n \rightarrow \BB^r$ wie oben die Projektion auf die oberen $r$ Bit.
\begin{align*}
    \begin{array}{c}
        \xymatrix@R=5mm@C=5mm{
            &
            z_0\ar[d]&
            *=0{}\ar@{-}[rr]&
            &
            *=0{}\ar[d]_{z_1}&
            *=0{}\ar[r]^{z_2}&
            \cdots\\
            %
            k\ar[r]&
            *++[F-:<5mm>]{c_k}\ar[dd]_{z_1}&
            &
            k\ar[r]&
            *++[F-:<5mm>]{c_k}\ar[dd]_{z_2}\\
            %
            &
            *=0{\smallbullet}\ar@{-}[r]&
            *=0{}\ar@{-}[uu]&
            &
            *=0{\smallbullet}\ar@{-}[r]&
            *=0{}\ar@{-}[uu]\\
            %
            &
            *++[F-:<5mm>]{\pi_r}\ar[d]&
            &
            &
            *++[F-:<5mm>]{\pi_r}\ar[d]\\
            %
            x_1\ar[r]&
            \bigxor\ar[d]&
            &
            x_2\ar[r]&
            \bigxor\ar[d]\\
            %
            &
            y_1&
            &
            &
            y_2
        }\\
        \\
        \text{\emph{Verschlüsselung}}
    \end{array}
    &&
    \begin{array}{c}
        \xymatrix@R=5mm@C=5mm{
            &
            z_0\ar[d]&
            *=0{}\ar@{-}[rr]&
            &
            *=0{}\ar[d]_{z_1}&
            *=0{}\ar[r]^{z_2}&
            \cdots\\
            %
            k\ar[r]&
            *++[F-:<5mm>]{c_k}\ar[dd]_{z_1}&
            &
            k\ar[r]&
            *++[F-:<5mm>]{c_k}\ar[dd]_{z_2}\\
            %
            &
            *=0{\smallbullet}\ar@{-}[r]&
            *=0{}\ar@{-}[uu]&
            &
            *=0{\smallbullet}\ar@{-}[r]&
            *=0{}\ar@{-}[uu]\\
            %
            &
            *++[F-:<5mm>]{\pi_r}\ar[d]&
            &
            &
            *++[F-:<5mm>]{\pi_r}\ar[d]\\
            %
            y_1\ar[r]&
            \bigxor\ar[d]&
            &
            y_2\ar[r]&
            \bigxor\ar[d]\\
            %
            &
            x_1&
            &
            &
            x_2
        }\\
        \\
        \text{\emph{Entschlüsselung}}
    \end{array}
\end{align*}
Es handelt sich also um ein spezielles One-Time-Pad mit einem Schlüssel,
der nur von $k$ und $z_0$ abhängt.

\linie

\textbf{Vorteile}:
Die $z_i$ können schon vor der Kommunikation berechnen werden, wenn $z_0$ bekannt ist,
weil sie nicht von $x_i$ oder $y_i$ abhängen.
Die Geheimtexte $y_i$ hängen nur von der Position ab (also nur von $x_i$).
Damit wirken sich Übertragungsfehler nur lokal aus, d.\,h. bei einem Fehler in $y_i$ kann nur
$x_i$ nicht entschlüsselt werden.

\textbf{Nachteile}:
Texte sind wieder leichter manipulierbar als bei CFB oder CBC.
Soll bei einer zweiten Kommunikation mit Klartext-Blöcken $x_i'$
derselbe Schlüssel $k \in K$ verwendet werden,
so muss $z_0$ geändert werden.
Sonst ergibt sich die selbe Folge von $z_i$ und daher derselbe One-Time-Pad-Schlüssel,
wie oben erläutert lässt sich damit z.\,B. $x_i \xor x_i'$ ermitteln und,
wenn $x_i'$ bekannt ist, daher sogar $x_i$ selbst (Known-Plaintext-Attacke).
Weitere Nachteile sind, dass $z_0$ auf jeden Fall vorher vereinbart sein muss und
dass nur symmetrische Verfahren verwendet werden können.

\pagebreak
